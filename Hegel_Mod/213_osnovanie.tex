\subsubsection{Третья глава. Основание}
{\em Сущность определяет самоё себя как основание}.

Подобно тому, как {\em ничто} сначала находится в
простом непосредственном единстве с {\em бытием}, так и
здесь простое тождество сущности сначала находится в непосредственном
единстве с ее абсолютной отрицательностью. Сущность есть только эта ее
отрицательность, которая есть чистая рефлексия. Она есть эта чистая
отрицательность, как возвращение бытия в себя; таким образом, она
{\em определена в себе} или для нас как основание, в
котором разрешается бытие. Но эта определенность не положена
{\em ею же самой}; или иначе говоря, сущность не есть
основание, именно поскольку она не положила сама эту свою определенность.
Но ее рефлексия состоит в том, чтобы {\em положить}
себя как то, что она есть {\em в себе, положить}, как
отрицательное, и {\em определить} себя. Положительное и
отрицательное составляют то существенное определение, в которое она исчезла
как в свое отрицание. Эти самостоятельные рефлексивные определения снимают
себя, и погрузившееся в основание определение есть истинное определение
сущности.

Поэтому {\em основание} само есть
{\em одно из рефлексивных определений} сущности. Однако
—~последнее из них, правильнее говоря, лишь то определение, что оно есть
снятое определение. Рефлексивное определение, погружаясь в основание,
получает свое истинное значение, состоящее в том, что оно есть абсолютное
самоотталкивание себя в само себя, а именно, что та положенность, которая
присуща сущности, имеет бытие лишь как снятая положенность, и что,
наоборот, лишь снимающая себя положенность есть положенность сущности.
Сущность, определяя себя как основание, определяет себя как
не-определенное, и лишь снятие ее определенности есть ее процесс
определения. — В этой определенности, как снимающей самоё себя, она есть не
сущность, проистекающая из другого, а такая сущность, которая в своей
отрицательности тождественна с собой.

Поскольку от определения, как первого, непосредственного, идут дальше к
основанию (через природу самого определения, которое через себя погружается
в основание), то основание есть ближайшим образом определенное через то
первое. Однако этот процесс определения, с одной стороны, как снятие
процесса определения, есть лишь восстановленное, очищенное или
обнаружившееся тождество сущности, которое рефлексивное определение есть
{\em в себе}; с другой же стороны, это отрицающее
движение, как процесс определения, и есть впервые полагание той
рефлексивной определенности, которая представлялась непосредственной, но
которая на самом деле лишь положена исключающей самоё себя рефлексией
основания, и притом положена лишь как нечто положенное или снятое. — Таким
образом, сущность, определяя себя как основание, приходит лишь из себя.
Следовательно, как {\em основание}, она
{\em полагает себя как сущность}, и ее процесс
определения в том именно и состоит, что она полагает себя как сущность. Это
полагание есть рефлексия сущности, каковая рефлексия в своем
{\em процессе определения снимает} самое себя и есть, с
первой стороны, {\em полагание}, а, с последней,
полагание {\em сущности}, тем самым и то и другое в
одном действии.

Рефлексия есть {\em чистое опосредствование вообще},
основание есть {\em реальное опосредствование} сущности
с собой. Первая, движение ничто через ничто обратно к самому себе, есть
свечение {\em себя} в некотором
{\em другом}; но так как противоположность в этой
рефлексии еще не обладает никакой самостоятельностью, то ни то первое,
просвечивающее, не есть некоторое положительное, ни то
{\em другое}, в котором оно просвечивает, не есть
некоторое отрицательное. Оба суть, собственно говоря, субстраты только силы
воображения; они еще не суть соотносящиеся; с самими собою. Чистое
опосредствование есть лишь {\em чистое соотношение} без
соотносящихся. Хотя определяющая рефлексия полагает такие определения,
которые тождественны с собой, однако эти определения вместе с тем суть
только {\em определенные соотношения}. Напротив,
основание есть реальное опосредствование, потому что оно содержит в себе
рефлексию, как снятую рефлексию; оно есть возвращающаяся в себя через свое
небытие и {\em полагающая себя} сущность. По этому
моменту снятой рефлексии положенное получает определение
{\em непосредственности}, чего-то такого, что вне
соотношения или своей видимости тождественно с собой. Это непосредственное
есть восстановленное через сущность {\em бытие}, — то
небытие рефлексии, через которое сущность опосредствует себя. В себя
сущность возвращается обратно как отрицающая; она, таким образом, в своем
возвращении в себя сообщает себе определенность, которая именно поэтому
есть тождественное с собою отрицательное, снятая положенность, и тем самым
вместе с тем также нечто сущее как тождество сущности с собой, как
основание.

Основание есть, {\em во-первых},
{\em абсолютное основание}, в котором сущность
ближайшим образом есть вообще {\em основа} для того
соотношения, которое конституируется категорией основания; но ближе оно
определяет себя как {\em форму} и
{\em материю} и сообщает себе некоторое
{\em содержание}.

{\em Во-вторых}, оно есть
{\em определенное основание} как основание некоторого
определенного содержания; поскольку соотношение основания в своей
реализации становится вообще внешним себе, оно переходит в
{\em обусловливающее} опосредствование.

{\em В-третьих}, основание предполагает условие; но
условие в такой же степени предполагает основание; безусловное есть их
единство, {\em суть в себе} (die Sache an sich),
которая через опосредствование обусловливающего соотношения переходит в
существование.

{\centering
Примечание
[Начало основания]
\par}

Основание получило выражение, подобно другим рефлексивным определениям, в
особом предложении: {\em все имеет свое достаточное
основание}. — Смысл этого предложения состоит вообще не в чем ином, как в
том, что то, что {\em есть}, следует рассматривать не
как {\em сущее непосредственное}, а как
{\em положенное}; нельзя останавливаться на
непосредственном наличном бытии или на определенности вообще, а следует от
этого наличного бытия идти назад к его основанию, в каковой рефлексии оно
имеется как снятое и в своем в-себе-и-для-себя-бытии. В предложении об
основании высказывается, следовательно, существенность рефлексии в себя в
противоположность голому бытию. — Что основание
{\em достаточно} —~добавлять это, собственно говоря,
совершенно излишне, ибо это разумеется само собой; то, для чего основание
было бы недостаточным, не имело бы никакого основания, а между тем ведь все
должно иметь свое основание. Но {\em Лейбниц}, сердцу
которого был особенно близок принцип достаточного основания и который даже
сделал его основоположением всей своей философии, соединял с ним более
глубокий смысл и более важное понятие, чем это обыкновенно делают, когда
останавливаются лишь на непосредственном выражении, хотя предложение
следует признать важным также и в этом смысле, а именно потому, что бытие
как таковое в его непосредственности объявляется здесь неистинным и по
существу некоторым положенным, основание же признается истинным
непосредственным. Но Лейбниц преимущественно противопоставлял достаточность
основания причинности в ее строгом смысле как механическому способу
действия. Так как последний есть вообще внешняя, по своему содержанию
ограниченная одной определенностью деятельность, то положенные им
определения вступают в соединение {\em внешне} и
{\em случайно}; частичные определения постигаются через
свои причины, но {\em соотношение} этих частичных
определений, составляющее существенное содержание какого-либо
существования, не содержится в механических причинах. Это соотношение,
целое, как существенное единство, заключается лишь в
{\em понятии}, в {\em цели}. Для
этого единства механические причины недостаточны, так как в основании их не
лежит цель как единство определений. Лейбниц поэтому понимал под
достаточным основанием такое основание, которое было бы достаточно также и
для этого единства и поэтому обнимало бы собою не просто причины, но и
{\em целевые причины}. Однако это определение основания
сюда еще не относится; {\em телеологическое} основание
есть достояние {\em понятия} и опосредствования через
понятие, каковое опосредствование есть разум.

\paragraph[А. \ Абсолютное основание]{А. \ Абсолютное основание}
\subparagraph[а) \ Форма и сущность]{а) \ Форма и сущность}
Рефлексивное определение, поскольку оно
возвращается в основание, есть некоторое первое, некоторое непосредственное
наличное бытие вообще, с которого начинают. Но наличное бытие еще имеет
лишь значение положенности и по существу
{\em предполагает} некоторое основание в том смысле,
что оно его скорее не {\em полагает}, что это полагание
есть снятие самого себя, а непосредственное есть скорее положенное,
основание же —~неположенное. Как оказалось, это предполагание представляет
собою полагание, бьющее рикошетом по полагающему; как снятая
определенность, основание есть не нечто неопределенное, а определенная
через самоё себя сущность, однако {\em определенная}
как {\em неопределенная} или как снятая положенность.
{\em Оно есть сущность, которая в своей отрицательности
тождественна с собою}.

{\em Определенность} сущности как основание становится
тем самым двоякой —~определенностью {\em основания} и
определенностью {\em обоснованного}. Она есть,
{\em во-первых}, сущность как основание
{\em определенная} быть сущностью, противостоящей
положенности как {\em неположенность}.
{\em Во-вторых}, она есть обоснованное,
непосредственное, которое, однако, не есть в себе и для себя, —
положенность как положенность. Последняя тем самым равным образом
тождественна с собою, но она есть тождественность отрицательного с собою.
Тождественное с собою отрицательное и тождественное с собою положительное
есть теперь {\em одно и то же тождество}. Ибо основание
есть тождество с собой положительного или даже положенности; обоснованное
есть положенность как положенность, но эта его рефлексия в себя есть
тождество основания. — Это простое тождество, следовательно, само не есть
основание, ибо основание есть сущность,
{\em положенная} как неположенное,
{\em противостоящее положенности}. Оно есть, как
единство этого определенного тождества (основания) и отрицательного
тождества (обоснованного), сущность вообще, отличная от ее
{\em опосредствования}.

Это опосредствование, сравненное с предшествующими рефлексиями, из которых
оно проистекает, не есть, во-первых, чистая рефлексия, каковая не отлична
от сущности и еще не заключает в себе отрицательного и тем самым также и
самостоятельности определений. В основании же, как снятой рефлексии, эти
определения имеют устойчивое наличие. — Оно не есть также и определяющая
рефлексия, определения которой обладают существенной самостоятельностью;
ибо эта последняя пошла ко дну в основании, в единстве которого они суть
лишь положенные. — Это опосредствование основания есть поэтому единство
чистой и определяющей рефлексии; определения этого опосредствования или
положенное имеет устойчивое наличие, и наоборот, устойчивое наличие этих
определений есть некоторое положенное. Так как это их устойчивое наличие
само есть некоторое положенное или обладает определенностью, то они тем
самым отличны от их простого тождества и составляют
{\em форму, противостоящую} сущности.

Сущность {\em обладает} некоторой формой и ее
определениями. Лишь как основание, она впервые обладает прочной
непосредственностью или есть {\em субстрат}. Сущность
как таковая едина со своей рефлексией и есть неразличимо само движение
рефлексии. Поэтому нельзя сказать, что это движение рефлексии проделывается
{\em сущностью}; она также и не есть то, с чего
рефлексия начинает, как с первого. Это обстоятельство затрудняет вообще
изложение рефлексии; ибо нельзя собственно сказать, что
{\em сущность} возвращается сама в себя, что
{\em сущность} светит в себя, так как она не существует
{\em до} своего движения или
{\em в} последнем, и это движение не имеет основы, на
(an) которой оно протекало бы. Некое соотнесенное выступает впервые лишь в
основании по моменту снятой рефлексии. Сущность же, как соотнесенный
субстрат, есть определенная сущность; в силу этой положенности она по
существу имеет в себе форму. — Напротив, определения формы суть теперь
такие определения, которые находятся как бы {\em на
сущности}; последняя {\em лежит в их основании}, как
нечто неопределенное, которое в своем определении безразлично к ним; они
имеют в ней свою рефлексию в себя. Рефлексивные определения должны были
иметь свое устойчивое наличие в них самих и быть самостоятельными; но их
самостоятельность есть их разложение; таким образом, они обладают этой
самостоятельностью в чем-то другом; но это разложение само есть это
тождество с собой или то основание устойчивого наличия, которое они себе
сообщают.

К форме принадлежит вообще все {\em определенное}; оно
есть определение формы, поскольку оно есть некоторое положенное и тем самым
отличное от {\em того}, формой
{\em чего} оно служит (определенность как
{\em качество} едина со своим субстратом, бытием; бытие
есть то непосредственно определенное, которое еще не отлично от своей
определенности или, иначе говоря, которое в ней еще не рефлектировано в
себя, равно как последняя есть поэтому некоторое сущее, еще не некоторое
положенное). — Формальными определениями сущности, по своей более детальной
определенности, служат далее, как рефлексивные определенности,
рассмотренные выше моменты рефлексии —~{\em тождество}
и {\em различие}; последнее отчасти как разность,
отчасти как {\em противоположность}. Но, далее, сюда
принадлежит также и {\em соотношение основания},
поскольку хотя оно и есть снятое рефлексивное определение, но благодаря ему
сущность вместе с тем выступает как нечто положенное. Напротив, к форме не
относится то тождество, которое основание имеет внутри себя, а именно не
относится то обстоятельство, что положенность, как снятая, и положенность
как таковая, — основание и обоснованное —~есть единая рефлексия,
составляющая {\em сущность} как
{\em простую основу}, которая есть
{\em устойчивое наличие} формы. Но это устойчивое
наличие {\em положено} в основании; или, иначе говоря,
эта сущность сама \ по существу выступает как определенная; тем самым она
также опять-таки есть момент соотношения основания и формы. — В том-то и
состоит абсолютное взаимоотношение формы и сущности, что последняя есть
простое единство основания и обоснованного, но в этом единстве как раз сама
является определенной или есть отрицательное и отличает себя как основу от
формы, но таким образом сама становится вместе с тем основанием и моментом
формы.

Поэтому форма есть завершенное целое рефлексии; она содержит в себе также и
то определение этой рефлексии, которое состоит в том, что она есть снятая;
поэтому она, будучи единством своего процесса определения, вместе с тем
также {\em соотнесена} со своей снятостью, соотнесена
{\em с некоторым другим}, которое само не есть форма,
но {\em на котором} она находится. Как
{\em существенная}, соотносящаяся с самой собой
отрицательность, противостоящая этому простому отрицательному, она есть
нечто {\em полагающее} и
{\em определяющее}; напротив, простая сущность есть
неопределенная и {\em недеятельная} основа, в которой
определения формы имеют устойчивое наличие или рефлексию в себя. — Внешняя
рефлексия обыкновенно застревает в этом различении сущности и формы;
различение это необходимо, но само это различение есть их единство, равно
как это основное единство есть отталкивающая себя от себя и обращающая себя
в положенность сущность. Форма есть та самая абсолютная отрицательность или
то самое отрицательное абсолютное тождество с собой, в силу которого как
раз сущность есть не бытие, а сущность. Это тождество, взятое абстрактно,
есть сущность, противостоящая форме; равно как отрицательность, взятая
абстрактно как положенность, есть отдельное определение формы. Но
определение, как оно обнаружилось, есть в своей истине тотальная,
соотносящаяся с собой отрицательность, которая тем самым как такое
тождество, есть в себе самой простая сущность. Поэтому форма имеет в своем
собственном тождестве сущность, равно как сущность имеет в своей
отрицательной природе абсолютную форму. Нельзя, стало быть, задавать
вопрос, {\em каким образом форма привходит к сущности},
ибо она есть лишь свечение последней в себя самоё, ее собственная,
имманентная ей рефлексия. Точно так же и форма в ней самой есть
возвращающаяся в себя рефлексия или тождественная сущность; в своем
процессе определения она обращает определение в положенность как
положенность. — Она, следовательно, не определяет сущность, как будто бы
она была поистине чем-то предположенным, отдельным от сущности; ибо таким
образом она есть несущественное, безостановочно идущее ко дну рефлексивное
определение; тем самым она, таким образом, сама есть скорее основание
своего снятия или тождественное соотношение своих определений. «Форма
определяет сущность» означает, следовательно, что форма в ее различении
сама снимает это различение и есть тождество с собой, которое есть сущность
как устойчивое наличие определения; она есть то противоречие, что она в
своей положенности снята и в этой снятости имеет устойчивое наличие, — и
тем самым она есть основание, как тождественная с собой в своей
определяемости или отрицаемости сущность.

Эти различия формы и сущности суть поэтому лишь
{\em моменты} самого простого соотношения формы. Но мы
должны их рассмотреть ближе и фиксировать. Определяющая форма соотносится с
собой как снятая положенность; тем самым она соотносится со своим
тождеством как с некоторым другим. Она полагает себя как снятую; тем самым
она {\em пред}полагает свое тождество; сущность есть по
этому моменту то неопределенное, для которого форма есть некоторое другое.
Таким образом, она не есть сущность, которая в ней самой есть абсолютная
рефлексия, а она {\em определена} как бесформенное
тождество; она есть {\em материя}.

\subparagraph[b) \ Форма и материя]{b) \ Форма и материя}
1. Сущность становится материей, когда ее
рефлексия определяет себя так, что она относится к сущности, как к
бесформенному неопределенному. Материя есть, следовательно, простое
лишенное различий тождество, которое есть сущность, с тем определением, что
она есть другое формы. Она поэтому есть собственная
{\em основа} или субстрат формы, так как она составляет
рефлексию в себя определений формы или то самостоятельное, с которым они
соотносятся как со своим положительным устойчивым наличием.

Если абстрагироваться от всех определений, от всякой формы какого-нибудь
нечто, то в результате остается неопределенная материя. Материя есть некое
безоговорочно {\em абстрактное}. (— Материю нельзя ни
видеть, ни ощупывать и~т.~д. — то, что мы видим или ощупываем, есть
некоторая {\em определенная материя}, т.~е. некоторое
единство материи и формы). Однако это абстрагирование, из которого
проистекает материя, не есть лишь {\em внешнее}
устранение и упразднение формы, но, как было показано, форма через самоё
себя низводит себя к этому простому тождеству.

Далее, форма {\em предполагает} некоторую материю, с
которой она соотносится. Но это не значит, что обе они
{\em оказываются} рядом одна с другой внешне и
случайно; ни материя, ни форма не самобытны или, говоря другим языком, не
{\em вечны}. Материя есть нечто безразличное к форме,
но это безразличие есть {\em определенность} тождества
с собой, в которую форма возвращается как в свою основу. Форма
{\em предполагает} материю именно потому, что она
полагает себя, как снятое, и тем самым соотносится с этим своим тождеством
как с чем-то другим. Равным образом, и наоборот, форма предполагается
материей; ибо последняя не есть простая сущность, которая сама
непосредственно есть абсолютная рефлексия, но сущность, определенная как
положительное, а именно, как то, что имеет бытие лишь как снятое отрицание.
— Но, с другой стороны, так как форма полагает себя как материю лишь
поскольку она сама себя снимает и тем самым
{\em предполагает} материю, то материя также
определена, как {\em лишенное основания} устойчивое
наличие. Равным образом, материя не определена как основание формы; но так
как материя полагает себя как абстрактное тождество снятого определения
формы, то материя не есть тождество в смысле основания, и постольку форма
по отношению к ней лишена основания. Тем самым форма и материя определены,
и та и другая, не быть положенными друг другом, не быть основанием друг
друга. Материя есть скорее тождество основания и обоснованного как основа,
противостоящая этому соотношению формы. Это их общее определение
безразличия есть определение материи как таковой, и составляет также и
взаимоотношение их обеих. Равным образом, и определение формы, состоящее в
том, что она есть соотношение их, как различенных, есть также и другой
момент их взаимного отношения. — Материя как то, что определено быть
безразличным, есть {\em пассивное}, в противоположность
форме как {\em деятельному}. Последняя, как соотносящее
с собою отрицательное, есть противоречие внутри себя самого, есть нечто
разлагающееся, отталкивающее себя от себя и определяющее себя. Она
соотносится с материей и {\em положена} так, чтобы
соотноситься с этим своим устойчивым наличием как с чем-то другим. Материя,
напротив, положена так, чтобы соотноситься только с собой самой и быть
безразличной к другому; но {\em в себе} она соотносится
с формой, ибо она содержит в себе снятую отрицательность и есть материя
лишь через это определение. Она соотносится с формой, как с некоторым
{\em другим} лишь потому, что форма в ней не положена,
что она есть форма лишь {\em в себе}. Она заключает в
себе форму в скрытом виде и есть абсолютная восприимчивость к форме лишь
потому, что абсолютно обладает внутри себя последней, что это есть ее
в-себе-сущее определение. Поэтому {\em материя должна
иметь форму}, а {\em форма} должна
{\em материализоваться}, сообщить себе в материи
тождество с собой, или, иначе говоря, устойчивое наличие.

2. Поэтому форма определяет материю, а материя определяется формой. — Так
как форма сама есть абсолютное тождество с собой и, следовательно, содержит
в себе материю и так как равным образом материя в ее чистой абстрактности
или абсолютной отрицательности обладает формой в себе самой, то действие
формы на материю и определяемость последней первою есть скорее лишь
{\em снятие видимости их безразличия} и различенности.
Это соотношение процесса определения есть, таким образом, опосредствование
каждой из них с собой через ее собственное небытие; но оба эти
опосредствования суть одно и то же движение и восстановление их
первоначального тождества: овнутренение их отчуждения (die Erinnerung ihrer
Entäusserung).

{\em Во-первых}, форма и материя взаимно
{\em предполагают} одна другую. Как выяснилось, это
значит, что единое существенное единство есть отрицательное соотношение с
собою самим; поэтому оно раздваивается на существенное тождество,
определенное как безразличная основа, и на существенное различие или
отрицательность, как определяющую форму. Это единство сущности и формы,
противополагающихся друг другу как форма и материя, есть
{\em абсолютное основание}, которое
{\em определяет} себя. Поскольку оно делает себя
некоторым разным, соотношение, в силу лежащего в основании тождества
разных, становится взаимным предположением.

{\em Во-вторых}, форма как самостоятельная есть помимо
этого снимающее само себя противоречие; но она также и положена как
таковое; ибо она одновременно и самостоятельна и существенно соотнесена с
некоторым другим; тем самым она снимает себя. Так как она сама двустороння,
то и это ее снятие имеет две стороны. {\em Во-первых},
она снимает {\em свою самостоятельность}, обращает себя
во что-то {\em положенное}, во что-то такое, что
находится на некотором другом, и это ее другое есть материя.
{\em Во-вторых}, она снимает свою определенность по
отношению к материи, свое соотношение с последней, тем самым свою
{\em положенность}, и этим сообщает себе
{\em устойчивое наличие}. Так как она снимает свою
положенность, то эта ее рефлексия есть ее собственное тождество, в которое
она переходит; но так как она вместе с тем отчуждает это тождество и
противополагает его себе как материю, то сказанная рефлексия положенности в
себя выступает как соединение с материей, в которой она получает устойчивое
наличие; она, следовательно, сливается в этом соединении и с материей как с
{\em некоторым другим} (по той своей первой стороне, по
которой она обращает себя во что-то положенное), и в этой материи также и
{\em со своим собственным тождеством}.

Итак, та {\em деятельность формы}, через которую
определяется материя, состоит в отрицательном отношении формы к самой себе.
Но и наоборот, она тем самым относится отрицательно также и к материи;
однако эта определяемость материи есть в той же мере собственное движение
самой формы. Последняя свободна от материи, но она снимает эту свою
самостоятельность; но ее самостоятельность и есть сама материя, ибо в
последней она имеет свое существенное тождество. Поскольку она, таким
образом, обращает себя в положенность, то это равносильно тому, что она
обращает материю в нечто определенное. — Но, рассматриваемое с другой
стороны, собственное тождество формы вместе с тем отчуждает себя, и материя
есть ее другое; постольку материя оказывается также и не определенной, в
результате того, что форма снимает свою собственную самостоятельность. Но
материя самостоятельна лишь по отношению к форме; поскольку отрицательное
снимает себя, снимается также и положительное. Следовательно, так как форма
снимает себя, то отпадает также и определенность материи, которой она
обладает по отношению к форме, определенность, состоящая в том, чтобы быть
неопределенным устойчивым наличием.

То, что выступает как {\em деятельность формы}, есть,
далее, в такой же мере также и {\em собственное
движение} самой {\em материи}.
{\em В-себе}{}-сущее определение или долженствование
материи есть ее абсолютная отрицательность. Через последнюю материя
соотносится безоговорочно не только с формой как с некоторым другим, но это
внешнее есть та форма, которую сама материя содержит в себе, как скрытую.
Материя есть то же самое противоречие в себе, какое содержится и в форме, и
это их противоречие, как и его разрешение, есть лишь одно противоречие. Но
материя противоречива внутри себя самой, так как она как неопределенное
тождество с собой есть вместе с тем абсолютная отрицательность; она поэтому
снимает себя в себе самой, и ее тождество распадается в ее отрицательности,
а последняя получает в нем свое устойчивое наличие. Следовательно, так как
материя определяется формой, как чем-то внешним, то первая этим достигает
своего определения, и внешний характер отношения состоит как для формы, так
и для материи, в том, что каждая из них или, правильнее, их первоначальное
единство есть в своем полагании вместе с тем нечто
{\em предполагающее}, вследствие чего соотношение с
собой есть вместе с тем соотношение с собой, как со снятым, или соотношение
со своим другим.

{\em В-третьих}, через это движение формы и материи их
первоначальное единство, с одной стороны, восстановлено, а, с другой, есть
теперь положенное. Материя столь же определяет сама себя, сколь этот
процесс определения есть для нее внешнее действие формы; и наоборот, форма
столь же определяет лишь себя или имеет определяемую ею материю в ней
самой, сколь в своем процессе определения относится к некоторому другому; и
то, и другое (действие формы и движение материи) есть одно и то же, только
первое есть действие, т.~е. отрицательность, как положенная, а второе
—~движение или становление, отрицательность как
{\em в-себе-сущее} определение. В результате получается
поэтому единство в-себе-бытия и положенности. Материя как таковая
определена или необходимо имеет некоторую форму, а форма есть безоговорочно
материальная, устойчиво наличная форма.

3. Форма, поскольку она предполагает некоторую материю как свое другое,
{\em конечна}. Она есть не основание, но лишь нечто
деятельное. Равным образом, и материя, поскольку она предполагает форму как
свое небытие, есть {\em конечная} материя; она столь же
мало есть основание своего единства с формой, а есть лишь основа для формы.
Но как эта конечная материя, так и конечная форма не имеют истины; каждая
соотносится с другой или, иначе говоря, лишь их единство есть их истина. В
это единство возвращаются оба эти определения и снимают в нем свою
самостоятельность; тем самым оно оказывается их основанием. Материя есть
поэтому лишь постольку основание определения своей формы, поскольку она
есть не материя как материя, а абсолютное единство сущности и формы; равным
образом, форма есть основание устойчивого наличия своих определений лишь
постольку, поскольку она есть то же самое единое единство. Но это единое
единство как абсолютная отрицательность, и, определеннее, как исключающее
единство, есть в своей рефлексии предполагающее; иначе говоря, в процессе
полагания одним и тем же действием оказывается сохранение в единстве себя,
как положенного, и отталкивание себя от самого себя; —~одним и тем же
действием оказывается соотнесение себя с собою как с собою и соотнесение
себя с собою как с некоторым другим. Или, иными словами, определяемость
материи формой есть опосредствование сущности как основания с собой в
некотором единстве через себя само и через отрицание себя самого.

Оформленная материя или обладающая устойчивым наличием форма есть теперь не
только вышеуказанное абсолютное единство основания с собой, но также и
{\em положенное} единство. Именно в рассмотренном
движении абсолютное основание изобразило свои моменты вместе с тем, как
снимающие себя и тем самым как положенные. Или, иначе говоря,
восстановленное вновь единство в своем слиянии с собой вместе с тем также и
оттолкнуло себя от самого себя и определило себя; ибо их единство, как
получившееся через отрицание, есть также и отрицательное единство. Поэтому
оно есть единство формы и материи как их основа, но как их
{\em определенная основа}, которая есть оформленная
материя, но которая вместе с тем безразлична к форме и материи как к снятым
и несущественным. Оно есть {\em содержание}.

\subparagraph[с) \ Форма и содержание]{с) \ Форма и содержание}
Форма, во-первых, противостоит сущности; таким
образом, она есть вообще соотношение основания, и ее определениями служат
основание и обоснованное. Затем она противостоит материи; таким образом,
она есть определяющая рефлексия, и ее определениями служат само
рефлексивное определение и его устойчивое наличие. Наконец, она
противостоит содержанию; таким образом, ее определениями служат опять-таки
она сама и материя. То, что было раньше тождественным с собой (сперва
основание, затем устойчивое наличие вообще и, наконец, материя), вступает
под власть формы и есть снова одно из ее определений.

Содержание имеет, {\em во-первых}, некоторую форму и
некоторую материю, принадлежащие ему и существенные для него; оно есть их
единство. Но так как это единство есть вместе с тем
{\em определенное} или
{\em положенное} единство, то содержание противостоит
форме; последняя составляет {\em положенность} и есть
по отношению к содержанию несущественное. Поэтому содержание безразлично в
форме; последняя объемлет собою как форму как таковую, так и материю; и
содержание имеет, стало быть, форму и материю, основу которых оно
составляет и которые суть для него голая положенность.

Содержание, {\em во-вторых}, есть то, что тождественно в
форме и материи, так что последние суть как бы лишь безразличные внешние
определения. Они суть положенность вообще, которая однако в содержании
возвратилась в свое единство или в свое основание. Тождество содержания с
самим собой есть поэтому, с одной стороны, указанное безразличное к форме
тождество, а с другой, оно есть тождество
{\em основания}. Основание ближайшим образом исчезло в
содержании; но содержание есть вместе с тем отрицательная рефлексия
определений формы внутрь себя; его единство, которое ближайшим образом лишь
безразлично к форме, есть поэтому также и формальное единство или
{\em соотношение основания} как таковое. Поэтому
содержание имеет последнее своей {\em существенной}
формой и, наоборот, {\em основание} имеет некоторое
содержание.

Содержанием основания, следовательно, служит возвратившееся в свое единство
с собой основание; основание есть ближайшим образом сущность, которая
тождественна с собой в своей положенности; как разная и безразличная к
своей положенности она есть неопределенная материя; но как содержание она
есть вместе с тем оформленное тождество, и эта форма становится
соотношением основания потому, что определения ее противоположности
положены в содержании также и как подвергшиеся отрицанию. — Содержание,
далее, {\em определено} в себе самом не только подобно
материи, как безразличное вообще, но и как оформленная материя, так что
определения формы обладают материальным, безразличным устойчивым наличием.
С одной стороны, содержание есть существенное тождество основания с собой в
его положенности, а с другой стороны, оно есть положенное тождество в
противоположность соотношению основания; эта положенность, которая имеется
в этом тождестве, как определение формы, противостоит свободной
положенности, т.~е. форме, как целому соотношению основания и
обоснованного; последняя форма есть тотальная возвращающаяся в себя
положенность; первая же есть поэтому лишь положенность как
непосредственная, — {\em определенность} как таковая.

Основание тем самым вообще обратило себя в определенное основание, и сама
определенность теперь двояка: она есть, во-первых, определенность формы и,
во-вторых, определенность содержания. Первая есть определенность основания,
состоящая в том, что оно вообще внешне содержанию, которое безразлично к
этому соотношению. Вторая есть определенность того содержания, которым
обладает основание.

\paragraph[В. \ Определенное основание]{В. \ Определенное основание}
\subparagraph[а) \ Формальное основание]{а) \ Формальное основание}
Основание имеет некоторое определенное
содержание. Определенность содержания есть, как выяснилось,
{\em основа} для формы, простая
{\em непосредственность}, противостоящая
{\em опосредствованию} формы. Основание есть
отрицательно соотносящееся с собою тождество, которое этим делает себя
{\em положенностью}; тождество это соотносится
отрицательно с {\em собой}, будучи в этой своей
отрицательности тождественным с собою; это тождество есть основа или
содержание, которое, таким образом, составляет безразличное или
положительное единство соотношения основания и есть его
{\em опосредствующее}.

В этом содержании исчезла ближайшим образом определенность основания и
обоснованного по отношению друг к другу. Но опосредствование есть, далее,
{\em отрицательное} единство. Отрицательное, как
имеющее место в этой безразличной основе, есть ее
{\em непосредственная определенность}, через которую
основание имеет определенное содержание. Но затем отрицательное есть
отрицательное соотношение формы с самой собой. Положенное, с одной стороны,
снимает само себя и возвращается в свое основание; основание же, как
существенная самостоятельность, соотносится отрицательно с самим собой и
обращает себя в положенное. Это отрицательное опосредствование основания и
обоснованного есть своеобразное опосредствование формы как таковой,
{\em формальное опосредствование}. Обе стороны формы,
так как одна переходит в другую, тем самым полагают себя теперь сообща
{\em в одном и том же тождестве} как снятые; этим они
его (это тождество) вместе с тем {\em предполагают}.
Оно есть определенное содержание, с которым, следовательно, формальное
опосредствование соотносится через себя само как с положительным
опосредствующим. Это содержание есть тождественное в них обеих, и,
поскольку они различны, но при этом каждая в своем различии есть
соотношение с другой, это содержание есть их устойчивое наличие, устойчивое
наличие {\em каждой} как само
{\em целое}.

Отсюда следует, что в определенном основании имеется следующее:
{\em во-первых}, некоторое определенное
{\em содержание} рассматривается с
{\em двух сторон}, один раз, поскольку оно положено как
{\em основание}, и другой раз, поскольку оно положено
как {\em обоснованное}. Само содержание безразлично к
этой форме; в обоих случаях оно есть вообще лишь одно определение.
{\em Во-вторых}, само основание есть в такой же мере
момент формы, как и положенное им; это есть их
{\em тождество по форме}. Безразлично, какое из этих
двух определений делают первым, безразлично, переходить ли от одного как
положенного к другому как к основанию или от одного как основания к другому
как к положенному. Обоснованное, рассматриваемое само по себе, есть снятие
себя самого; тем самым оно делает себя, с одной стороны, положенным и есть
вместе с тем полагание основания. То же самое движение есть и основание как
таковое; оно делает себя положенным и благодаря этому становится основанием
чего-то, т.~е. она имеется в этом движении и как положенное и как
основание, и притом лишь в этом движении оно впервые становится основанием.
Основанием того факта, что есть основание, служит положенное, и, наоборот,
этим самым основание оказывается чем-то положенным. Опосредствование
начинается столько же от одного, сколько и от другого; каждая сторона есть
столь же основание, сколь и положенное, и каждая есть все опосредствование
или вся форма. — Эта вся форма сама, далее, есть как нечто тождественное с
собою, {\em основа} тех определений, которые составляют
обе стороны, — основание и обоснованное; форма и содержание сами суть,
таким образом, одно и то же тождество.

В силу этого тождества основания и обоснованного как по содержанию, так и по
форме, основание есть {\em достаточное} (при
ограничении достаточности этим отношением); {\em нет
ничего в основании}, чего нет в {\em обоснованном},
точно {\em так же, как нет ничего в обоснованном, чего
нет в основании}. Когда спрашивают, что служит основанием, то желают
получить {\em то же самое} определение, которое
составляет собою {\em содержание, вдвойне}: во-первых,
в форме положенного и, во-вторых, в форме рефлектированного в себя
наличного бытия, существенности.

Итак, поскольку в определенном основании основание и обоснованное суть вся
форма, и их содержание, хотя и определенное, есть одно и то же, то
основание в обеих его сторонах еще не определено реально, они не имеют
разного содержания; определенность есть пока что лишь простая, еще не
перешедшая на эти стороны определенность; определенное основание имеется
пока что лишь в своей чистой форме, имеется лишь
{\em формальное основание}. — Так как содержание есть
лишь эта простая определенность, не имеющая в самой себе формы соотношения
основания, то она есть тождественное с собой содержание, безразличное к
форме, и последняя для него внешня; оно есть некое другое, чем она.

{\centering
Примечание
[Формальный способ объяснения из тавтологических оснований]
\par}

Если рефлексия об определенных основаниях придерживается той формы
основания, которая получилась здесь, то указание основания остается голым
формализмом и пустой тавтологией, выражающей в форме рефлексии в себя,
существенности, то же самое содержание, которое уже имеется в форме
непосредственного, наличного бытия, рассматриваемого как положенное. Такое
указание оснований сопровождается поэтому такой же пустотой, как и
высказывания, делаемые сообразно предложению о тождестве. Науки, особенно
физические, преисполнены этого рода тавтологиями, которые как бы составляют
прерогативу науки. — Например, как на основание движения планет вокруг
солнца, указывают на силу взаимного {\em притяжения}
Земли и Солнца. Этим не высказывается по содержанию ничего другого, кроме
того, что уже заключается в феномене, т.~е. в соотношении этих тел друг с
другом в их движении, но только это высказывается в форме рефлектированного
в себя определения —~силы. Если затем задают вопрос, что за силу
представляет собою эта притягивающая сила, то получается ответ, что она
есть сила, производящая движение Земли вокрук Солнца, т.~е. она имеет
совершенно то же самое содержание, как и то наличное бытие, основанием
которого она должна быть; соотношение Земли и Солнца в их движении есть
тождественная основа основания и обоснованного. — Если какая-либо форма
кристаллизации объясняется тем, что основанием ее служит особое взаимное
расположение молекул, то ведь налично-сущая кристаллизация и есть именно
само то расположение, которое объявляется основанием. В обычной жизни эти
этиологии~\pagenote{Слово «этиология»
(от греческого «aitia» —~причина, начало, основание) означает учение о
причинах, указание причин или оснований для тех или иных явлений.},
которые составляют привилегию наук, считаются тем, что они суть, —
тавтологическими пустыми речами. Если на вопрос, почему такой-то человек
едет в город, указывается то основание, что в городе находится влекущая его
туда притягательная сила, то этого рода ответ, санкционированный в науках,
считается нелепой пошлостью. — Лейбниц упрекал
{\em ньютонову} силу притяжения в том, что она есть
такое же скрытое качество, как те, которыми пользовались для объяснения
схоластики. Ей следовало бы сделать скорее противоположный упрек, а именно,
что она есть {\em слишком известное} качество, ибо в
ней нет никакого другого содержания, кроме самого явления. — Этот способ
объяснения нравится именно своей большой ясностью и понятностью, ибо что
может быть яснее и понятнее указания, например, на то, что растение имеет
свое основание в некоторой растительной, т.~е. производящей растения силе.
— {\em Скрытым} качеством эта сила могла бы быть
названа лишь в том смысле, что основание должно было бы иметь
{\em иное содержание}, чем объясняемое им, а между тем
этого содержания не дают; служащая для объяснения сила есть, конечно,
скрытое основание постольку, поскольку требуемое основание
{\em не} указывается. Посредством такого формализма
нечто объясняется столь же мало, как мало познается природа какого-нибудь
растения, если я скажу, что оно есть растение; при всей ясности такого
предложения или того предложения, что растение имеет свое основание в
производящей растения
силе~\pagenote{Лассон считает,
что придаточное предложение «что растение имеет свое основание в
производящей растения силе» попало сюда по ошибке и должно быть поставлено
двумя строчками выше, после слов «я скажу, что оно есть растение».
Стилистически такая перестановка улучшает конструкцию всей этой фразы, но
логический смысл заставляет предпочесть тот текст, какой дается в издании
Глокнера. С этого текста и сделан перевод этой фразы с добавлением слов
«того предложения» перед приведенным выше придаточным предложением.},
этот способ объяснения в силу изложенного может быть назван весьма
{\em скрытым}.

{\em Во-вторых}, что касается
{\em формы}, то при этом способе объяснения встречаются
оба {\em противоположные направления соотношения
основания}, не приводя к познанию их определенного отношения. Основание
есть, с одной стороны, основание, как рефлектированное в себя определение
содержания того наличного бытия, которое оно обосновывает, а с другой
стороны, оно есть положенное. Оно есть то, из чего должно быть понято
наличное бытие; но на самом деле, {\em наоборот, от
последнего умозаключают к первому}, и основание понимается из наличного
бытия. А именно, главная задача этой рефлексии состоит в том, чтобы из
наличного бытия найти основания, т.~е. превратить непосредственное наличное
бытие в форму рефлектированности: основание, вместо того чтобы быть в себе
и для себя и самостоятельным, оказывается тем самым скорее чем-то
положенным и производным. А так как оно посредством такого способа
объяснения прилаживается к явлению, в его определения основываются на
последнем, то последнее, конечно, совершенно гладко и по попутному ветру
вытекает из своего основания. Но познание от этого не движется с места; оно
продолжает вращаться в формальном различии, которое перевертывается
наоборот и упраздняется самим этим приемом объяснения. Поэтому одно из
главных затруднений при погружении в науки, в которых господствует этот
прием, обусловливается именно этим извращением ситуации, состоящим в том,
что предпосылается как основание то, что в действительности есть
производное, и, переходя к следствиям, впервые находят в них в
действительности основание того, что должно было бы быть их основанием.
Изложение начинается с оснований, их выставляют наобум, как принципы и
первые понятия; они суть простые определения, сами по себе не обладающие
никакой необходимостью; последующее должно быть основано на них. Поэтому,
если кто желает проникнуть в такого рода науки, тот должен начать с того,
чтобы вбить себе в голову эти основания; для разума это —~тягостная задача,
так как он должен при этом признать основой то, что лишено основания. Лучше
всех преуспевает в этом деле тот, кто, не долго раздумывая, соглашается
признать принципы как данные и отныне пользуется ими как основными
правилами своего рассудка. Без этого метода нельзя найти начал, и столь же
мало возможно без него дальнейшее продвижение. Однако последнее
затрудняется тем, что в нем обнаруживается обратный толчок метода, который
в последующем хочет обнаружить производное, между тем как на самом деле это
производное содержит в себе впервые основания для тех предпосылок. Далее,
так как последующее оказывается тем наличным бытием, из которого было
выведено основание, то это отношение, в котором здесь выступает явление,
создает недоверие к его изображению; ибо явление оказывается выраженным не
в своей непосредственности, а как доказательство основания. Но так как
последнее опять-таки выводится из первого, то является требование увидеть
явление в его непосредственности, чтобы получить возможность, исходя из
него, дать оценку основания. Поэтому при таком изложении, в котором
собственно обосновывающее выступает как производное, остается неизвестным,
как обстоит дело и с основанием и с явлением. Эта неизвестность
увеличивается еще вследствие того, — в особенности, если изложение не
строго последовательно, а более {\em добросовестно}, —
что в явлении повсюду обнаруживаются следы и обстоятельства, указывающие на
многое и часто совсем другое, чем то, что содержится в принципах. Наконец,
путаница получается еще бóльшая, когда смешивают рефлектированные и лишь
гипотетические определения с непосредственными определениями самого
явления, когда их излагают так, как будто они взяты из непосредственного
опыта. Таким образом, многие, приступающие к этим наукам с искренней верой,
могут держаться того мнения, будто молекулы, пустые промежутки,
центробежная сила, эфир, отдельные световые лучи, электрическая,
магнетическая {\em материя} и еще множество тому
подобного суть вещи или отношения, которые, судя по тому, что о них
говорится, как о непосредственных определениях наличного бытия, на самом
деле даны {\em в восприятии}. Они служат первыми
основаниями для другого; о них говорят, как о действительности, и их
применяют с полным доверием; их добросовестно признают действительностью
раньше, чем начинают сознавать, что они суть, собственно говоря,
определения, полученные путем умозаключения из того, что они должны
обосновать, — выведенные некритической рефлексией гипотезы и измышления. На
самом деле те, кто так поступает, оказываются вращающимися в каком-то
заколдованном круге, в котором определения наличного бытия и определения
рефлексии, основание и обоснованное, феномены и фантомы перемешаны вместе в
нераздельном сообществе и пользуются одинаковым почетом.

При формальном характере этого способа объяснения из оснований мы вместе с
тем опять, несмотря на все эти объяснения при помощи хорошо известных сил и
материй, слышим разговоры о том, что мы {\em не знаем
внутренней сущности} самих этих сил и материй. В этом можно усмотреть лишь
признание того обстоятельства, что такое обоснование само совершенно не
удовлетворено собой, что оно само требует чего-то совсем иного, чем таких
оснований. Но в таком случае не видать только, к чему служит этот труд
такого «объяснения», почему не ищут этого иного или по крайней мере не
оставляют в стороне того «объяснения» и не останавливаются на
констатировании простых фактов.

\subparagraph[b) \ Реальное основание]{b) \ Реальное основание}
Определенность основания, как оказалось, есть, с
одной стороны, определенность {\em основы} или
определение содержания, а с другой стороны, она есть инобытие в самом
{\em соотношении основания}, а именно различность его
содержания и формы; соотношение основания и обоснованного проходит как
внешняя форма по содержанию, безразличному к этим определениям. — Но на
самом деле оба они не внешни друг другу; ибо содержание состоит в том, что
оно есть {\em тождество основания} с самим собой
{\em в обоснованном и обоснованного в основании}.
Оказалось, что сторона основания сама есть положенное, а сторона
обоснованного сама есть основание: каждая в ней самой есть это тождество
целого. Но так как они вместе с тем принадлежат к форме и составляют ее
определенное различие, то тождеством целого с собою каждая оказывается
{\em в своей собственной
определенности}~\pagenote{т.~е. по-своему, по-разному.}.
Каждая, стало быть, имеет {\em разнящееся} от другой
{\em содержание}. — Или, если рассматривать их со
стороны содержания, так как оно оказывается тождеством с собою обеих
сторон, как сторон {\em соотношения основания}, то
содержание оказывается по существу, имеющим в нем самом это различие формы,
и как основание оно представляет собою некое другое, чем как обоснованное.

Но тем, что основание и обоснованное имеют разное содержание, соотношение
основания перестало быть формальным: возвращение в основание и выхождение
из него к положенному уже не есть тавтология;
{\em основание} реализовано. Поэтому, когда спрашивают
об основании, то в качестве основания требуют, собственно говоря,
некоторого другого определения содержания, чем то, об основании которого
спрашивают.

Это соотношение определяет себя теперь далее. А именно, поскольку две его
стороны представляют собою разное содержание, они безразличны друг к другу;
каждая есть непосредственное, тождественное с собой определение. Далее,
будучи соотнесены друг с другом как основание и обоснованное, основание
есть рефлектированное внутрь себя в другом как в своей положенности; таким
образом, то содержание, которым обладает сторона основания, имеется также и
в обоснованном; последнее как положенное имеет лишь в основании свое
тождество с собой и свое устойчивое наличие. Но кроме этого содержания
основания, обоснованное отныне имеет еще также и свое своеобразное
содержание и, значит, есть {\em единство двоякого}
содержания. Это единство как единство разных есть, правда, их отрицательное
единство, но так как они суть взаимно безразличные определения содержания,
то оно есть лишь их пустое, в себе самом бессодержательное соотношение, а
не их опосредствование —~некоторое {\em одно} или
{\em нечто}, как их внешнее сочетание.

Таким образом, в реальном соотношении основания имеется двоякое:
{\em во-первых}, то определение содержания, которое
есть основание и которое непрерывно продолжает себя само в положенности,
так что оно составляет то, что просто тождественно в основании и в
обоснованном; таким образом, обоснованное содержит основание полностью
внутри себя; их соотношение есть лишенная различий существенная сплошность
(Gediegenheit). Поэтому то, что в обоснованном еще присоединяется к этой
простой {\em сущности}, есть лишь некоторая
несущественная форма, внешние определения содержания, которые как таковые
свободны от основания и составляют непосредственное многообразие. Для этого
несущественного то существенное не есть, стало быть, основание, равно как
оно не есть основание для {\em соотношения} обоих в
обоснованном. Это —~некоторое положительно тождественное, которое
имманентно обоснованному, но не полагает себя в нем ни в каком различии
формы, а как соотносящееся с собою самим содержание есть безразличная
положительная {\em основа}.
{\em Во-вторых}, то, что в данном нечто сочетается с
этой основой, есть некоторое безразличное содержание, но как несущественная
сторона. Главным является {\em соотношение} основы и
несущественного многообразия. Но это соотношение, так как соотносящиеся
определения суть безразличное содержание, также {\em не
есть основание}; правда, одна сторона этого соотношения определена как
существенное, а другая —~лишь как несущественное или положенное содержание,
но как соотносящееся с собой содержание эта форма внешня обеим им. То
{\em одно} в данном {\em нечто},
которое составляет их соотношение, есть поэтому не соотношение формы, а
лишь некоторая внешняя связь, которая не содержит в себе несущественного
многообразного содержания как {\em положенного}; оно,
следовательно, есть равным образом лишь {\em основа}.

Таким образом, основание, поскольку оно определяет себя как реальное,
распадается из-за разности содержания, составляющей его реальность, на
внешние определения. Оба соотношения
—~{\em существенное содержание} как простое
{\em непосредственное тождество} основания и
обоснованного, а затем {\em нечто} как соотношение
различного содержания суть {\em две равные основы};
тождественная с собой форма основания, заключающаяся в том, что одно и то
же имеется один раз как существенное, а другой раз как положенное, исчезла;
соотношение основания стало, таким образом,
{\em внешним} самому себе.

Поэтому теперь оказывается, что некоторое внешнее обоснование приводит в
связь разные содержания в определяет, какое из них есть основание и какое
то, что положено последним; в самих же этих содержаниях указанного
определения нет. Реальное основание есть поэтому
{\em соотношение с другим —~}с одной стороны,
соотношение содержания с другим содержанием, а, с другой стороны,
соотношение самого соотношения основания (формы) с другим, а именно с
некоторым {\em непосредственным}, не положенным им.

{\centering
Примечание
[Формальный способ объяснения из некоторого, отличного от обоснованного,
основания]
\par}

Формальное соотношение основания заключает в себе лишь одно содержание и для
основания и для обоснованного; в этом тождестве заключается его
необходимость, но вместе с тем и его тавтологичность. Реальное основание
заключает в себе различные содержания; но при этом появляется случайный и
внешний характер соотношения основания. С одной стороны, то, что
рассматривается как существенное и потому как определение основания, не
есть основание для других, связанных с ним определений. С другой стороны,
остается также неопределенным, какое из многих определений содержания
некоторой конкретной вещи должно быть принято за существенное и за
основание; поэтому выбор между ними оказывается свободным. Так, что
касается первого замечания, то основанием, например, дома служит его
фундамент; то, что делает последний основанием, есть присущая чувственной
материи {\em тяжесть}, которая совершенно тождественна
как в основании, так и в обоснованном им доме. То обстоятельство, что в
тяжелой материи имеется такое различие, как различие между фундаментом и
отличной от него модификацией, вследствие которой она образует жилище,
совершенно безразлично для самой тяжелой материи; соотношение ее с другими
определениями содержания дома—с целью, устройством последнего и~т.~д. — ей
внешне; поэтому она есть, правда, их основа, но не их основание. Тяжесть
есть настолько же основание того, что дом стоит, насколько и того, что
камень падает; камень имеет внутри себя это основание, тяжесть; но то
обстоятельство, что он имеет то дальнейшее определение содержания,
вследствие которого он не только есть нечто тяжелое, но и камень, — это
внешне для тяжести; далее, что его сначала отдалили от того тела, на
которое он падает, это положено через некоторое другое, равно как время и
пространство и их соотношение, движение, суть некоторое другое содержание,
чем тяжесть, и их (как обыкновенно говорят) можно представлять себе без
нее; они, следовательно, не положены ею по существу. — Тяжесть также есть
основание того, что брошенное тело совершает движение, противоположное
падению. — Из разности определений, основанием которых служит тяжесть,
явствует, что требуется вместе с тем нечто другое, делающее ее основанием
того или другого определения.

Если о {\em природе} говорится, что она есть
{\em основание мира}, то, что называется природой, с
одной стороны, есть {\em то же самое}, что и мир, и мир
есть не что иное, как сама природа. Но они также и разны, так что природа
есть в большей мере неопределенная или, по крайней мере, определенная лишь
в общих различиях, которые суть законы, тождественная с собой сущность
мира; и для того чтобы природа стала миром, к ней извне присоединяются еще
многообразные определения. Но последние имеют свое основание не в природе
как таковой; она, напротив, безразлична к ним, как к случайностям. — Так же
обстоит дело, когда определяют {\em бога}, как
{\em основание природы}. Как основание он есть ее
сущность, природа содержит ее в себе и тождественна с нею; но она обладает
еще дальнейшим многообразием, которое отличается от самого основания; она
есть то {\em третье}, в котором связаны оба эти разные;
сказанное основание не есть основание ни разнящегося от него многообразия,
ни своей связи с ним. Поэтому природа не познается из бога как из
основания, ибо при указанном понимании он был бы лишь ее всеобщей
сущностью, не содержащей ее как определенной сущности и природы.

Вследствие этой разности содержания основания или, собственно говоря, основы
и того, что связано с нею в обоснованием, указание реальных оснований
становится, следовательно, таким же формализмом, как и формальное
основание. В последнем тождественное с собой содержание безразлично к
форме; в реальном основании имеет место то же самое. Вследствие этого
оказывается далее, что оно не содержит в себе самом указания, какое из
многообразных определений должно считаться существенным.
{\em Нечто} есть некоторое
{\em конкретное}, состоящее из таких многообразных
определений, которые оказываются в нем одинаково постоянными и
пребывающими. Поэтому одно из них с точно таким же правом, как и другое,
может быть определено как основание, а именно, как то
{\em существенное} определение, сравнительно с которым
прочие суть тогда лишь некоторое положенное. К этому присоединяется то, о
чем мы упомянули выше, а именно, что если имеется некоторое определение,
которое в одном случае принимается за основание другого, то отсюда не
следует, что в другом случае или вообще это другое положено вместе с
первым. — {\em Наказание}, например, имеет
многообразные определения, а именно, оно есть возмездие, оно есть, далее,
устрашающий пример, оно есть провозглашенная законом угроза для острастки,
а также есть нечто, служащее для вразумления и исправления преступника.
Каждое из этих разных определений рассматривалось, как
{\em основание наказания}, ибо каждое есть существенное
определение, и вследствие этого прочие, как отличные от него, определяются
по отношению к нему лишь как случайные. Но то из них, которое принимается
за основание, еще не есть все наказание как таковое; это конкретное
содержит в себе также и те другие определения, которые в нем лишь соединены
с первым, не имея в нем своего основания. — Или возьмем другой пример:
какое-либо {\em должностное лицо} обладает служебной
сноровкой, состоит как индивидуум в родстве, имеет те или иные знакомства,
обладает особенным характером, имел такие-то возможности или случаи
отличиться и~т.~д. Каждое из этих свойств может быть или считаться
основанием того, что он занимает эту должность; они суть различные
содержания, соединенные в некотором третьем; форма, по которой они
определены по отношению друг к другу, одно, как существенное, а другие, как
положенные, внешня этому третьему. Каждое из названных свойств существенно
для данного должностного лица, так как последнее через эти свойства есть
тот определенный индивидуум, которым оно является; поскольку должность
может рассматриваться как внешнее, положенное определение, каждое из этих
свойств можно определять по отношению к ней как ее основание, но и,
наоборот, \ можно смотреть на те свойства как на положенные, а на должность
—~как на их основание. Как они относятся между, собой
{\em действительно}, т.~е. в отдельном случае, есть
определение, внешнее для самого соотношения основания и для самого
содержания; форму основания и обоснованного им сообщает некое третье.

Вообще говоря, каждое наличное бытие может таким образом иметь разнообразные
основания; каждое из определений его содержания, как тождественное с собой,
пронизывает собою конкретное целое и может поэтому рассматриваться как
существенное; тем разнообразным {\em точкам зрения},
т.~е. определениям, которые лежат {\em вне }самой вещи,
открыт поэтому бесконечный простор ввиду случайности способа связи. —
Поэтому случайно также и то, имеет ли некоторое основание то или иное
{\em следствие}. Например, моральные мотивы суть
{\em существенные определения} нравственной природы, но
то, что из них проистекает, есть вместе с тем некоторое отличное от них
внешнее, которое и проистекает и не проистекает из них; лишь через
некоторое третье оно прибавляется в ним. Точнее говоря, это следует
понимать так, что для морального определения,
{\em если} оно есть основание,
{\em не} случайно то, что оно имеет некоторое следствие
или обоснованное, но для него вообще случайно, делают ли его или не делают
основанием. Но так как опять-таки содержание, составляющее следствие
морального определения, если последнее сделано основанием, имеет характер
внешности, то это содержание может быть непосредственно упразднено другой
внешностью. Таким образом, из того или другого морального мотива может как
произойти, так и не произойти некоторый поступок. И, наоборот, тот или иной
поступок может иметь разнообразные основания; как нечто конкретное, он
содержит в себе многообразные существенные определения, каждое из которых
может поэтому быть принято за основание. — Отыскивание и указание
оснований, в чем преимущественно состоит
{\em рассуждение}, есть поэтому бесконечное шатание из
стороны в сторону, не приводящее ни к какому окончательному определению;
относительно всего и каждого можно указать одно или несколько хороших
оснований, равно как и относительно его противоположного, и может иметься
множество оснований, без того чтобы из них что-нибудь следовало. То, что
Сократ и Платон называют {\em софистикой}, есть не что
иное, как рассуждение из оснований; Платон противопоставляет ему
рассмотрение идеи, т.~е. сути в себе и для себя самой или в ее
{\em понятии}. Основания почерпываются лишь из
{\em существенных} определений содержания, отношений и
точек зрения, которые в каждой вещи имеются во множестве, равно как и в ее
противоположности; каждое из этих определений в своей форме существенности
значимо ровно столько же, сколько и другое; так как оно не объемлет всей
вещи, то оно есть одностороннее основание, причем другие стороны вещи имеют
в свою очередь другие особенные основания, ни одно из которых не
исчерпывает вещи, составляющей их {\em соединение} и
содержащей их все вместе; ни одно из них не есть
{\em достаточное} основание, т.~е. понятие.

\subparagraph[с) \ Полное основание]{с) \ Полное основание}
1. В реальном основании основание как содержание
и основание как соотношение суть лишь {\em основы}.
Первое лишь {\em положено} как существенное и как
основание; соотношение же есть то {\em нечто}, которое
составляет обоснованное и которое представляет собою неопределенный
субстрат разного содержания, его связь, которая есть не его собственная
рефлексия, а внешняя и, стало быть, только положенная. Реальное соотношение
основания есть поэтому скорее основание как снятое; оно, стало быть,
составляет скорее сторону {\em обоснованного} или
{\em положенности}. Но как положенность, само основание
возвратилось теперь в свое основание; оно есть теперь некоторое
обоснованное, имеющее {\em другое основание}. Последнее
определяет себя вследствие этого так, что оно,
{\em во-первых}, есть нечто
{\em тождественное} с реальным основанием как со своим
обоснованным; обе стороны имеют по этому определению одно и то же
содержание; два определения содержания и их связь в нечто равным образом
находятся в новом основании. Но, {\em во-вторых}, новое
основание, в которое перешла, сняв себя, та лишь положенная внешняя связь,
есть как ее рефлексия внутрь себя {\em абсолютное
соотношение} двух определений содержания.

Вследствие того, что реальное основание само возвратилось в свое основание,
в нем восстанавливается тождество основания и обоснованного, или формальное
основание. Возникшее соотношение основания есть поэтому
{\em полное} соотношение, содержащее в себе вместе и
формальное и реальное основание и опосредствующее друг с другом те
определения содержания, которые в реальном основании непосредственны по
отношению друг к другу.

2. Тем самым соотношение основания определилось ближе следующим образом.
{\em Во-первых}, нечто имеет некоторое основание; оно
содержит в себе {\em то определение содержания},
которое есть {\em основание}, и еще некоторое
{\em второе} определение как
{\em положенное} им. Но как безразличное содержание
первое не есть в себе самом основание, а второе не есть в себе самом
обоснованное первым, но это {\em соотношение} имеется в
непосредственности содержания как снятое или положенное и как таковое имеет
свое основание в каком-то {\em другом} соотношении. Это
второе соотношение, как различенное лишь по форме, имеет то же самое
содержание, что и первое, а именно, оба определения содержания, но есть их
{\em непосредственная} связь. Однако, так как
связываемое имеет вообще разное содержание и, стало быть, есть взаимно
безразличное определение, то эта связь не есть их поистине абсолютное
соотношение, в силу которого одно из определений было бы чем-то
тождественным с собой в положенности, а другое —~лишь этой положенностью
того же самого тождественного. Дело обстоит не так, а иначе: некоторое
нечто служит их носителем и составляет их не рефлектированное, а лишь
непосредственное соотношение, которое, поэтому, есть лишь релятивное
основание, основание по отношению к их связи в другом нечто.
{\em Оба нечто} суть, таким образом, те два различные
соотношения содержания, которые у нас получились. Они находятся между собой
в тождественном формальном соотношении основания; \ они суть одно и то же
{\em все содержание}, а именно, те два определения
содержания и их соотношение; они различаются лишь способом этого
соотношения, которое в одном есть непосредственное соотношение, а в другом
—~положенное, вследствие чего одно отличается от другого лишь
{\em по форме}, как основание и обоснованное. —
{\em Во-вторых}, это соотношение основания не только
формально, но и реально. Формальное основание, как оказалось, переходит в
реальное; моменты формы рефлектируют себя в себя самих; они суть некоторое
самостоятельное содержание, и соотношение основания также содержит в себе
некоторое особое {\em содержание} как
{\em основание} и некоторое
{\em особое} содержание как
{\em обоснованное}. Содержание образует собою,
во-первых, {\em непосредственное} тождество обеих
сторон формального основания так, что они имеют одно и то же содержание. Но
оно имеет также форму в нем самом и есть, таким образом, удвоенное
{\em содержание}, относящееся, как основание и
обоснованное. Одно из двух определений содержания вышеуказанных двух нечто
определено поэтому не только как общее им по внешнему сравнению, но как их
тождественный субстрат и основа их соотношения. По отношению к другому
определению содержания оно есть существенное определение и основание этого
другого определения, как положенного, а именно, положенного в том нечто,
соотношение которого есть обоснованное соотношение. В первом нечто, которое
есть соотношение основания, это второе определение содержания оказывается
тоже непосредственным и {\em в себе} связанным с первым
определением содержания. Второе же нечто содержит
{\em в себе} лишь одно определение, как то, в чем оно
непосредственно тождественно с первым нечто, другое же определение оно
содержит, как положенное в нем. Первое определение содержания есть
основание этого положенного определения благодаря тому, что оно в первом
нечто {\em первоначально} связано с другим определением
содержания.

{\em Соотношение основания} в определениях содержания
второго нечто, таким образом, {\em опосредствовано}
первым в-себе-сущим соотношением первого нечто. Умозаключение здесь таково:
так как в одном нечто определение {\em B} связано в
себе с определением {\em A}, то во втором нечто,
которому присуще непосредственно лишь определение
{\em A}, с последним связано также и
{\em B}. Во втором нечто опосредствовано не только это
второе определение, но также и то, что его непосредственное определение
есть основание, а именно, через первоначальное соотношение этого
определения с {\em B} в первом нечто. Это соотношение
есть тем самым основание основания {\em A}, и
{\em все} соотношение основания имеется во втором нечто
как то, что положено или обосновано.

3. Реальное основание оказывается {\em внешней} себе
{\em рефлексией} основания; полное опосредствование
последнего есть восстановление его тождества с собою. Но так как последнее
вследствие этого приобрело вместе с тем внешний характер реального
основания, то формальное соотношение основания есть в этом единстве его
самого и реального основания столь же полагающее себя, сколь и
{\em снимающее} себя основание; соотношение основания
опосредствует себя с собою {\em через свое отрицание}.
Основание есть, во-первых, как {\em первоначальное
}{\em соотношение}, соотношение непосредственных
определений содержания. Соотношение основания как существенная форма имеет
своими сторонами такие определения, которые суть снятые или моменты.
Поэтому как форма {\em непосредственных} определений
оно есть тождественное с собой соотношение вместе с тем как соотношение
своего {\em отрицания}; тем самым оно есть основание не
в себе и для себя самого, а как соотношение со
{\em снятым} соотношением основания. — Во-вторых,
снятое соотношение или то непосредственное, которое и в первоначальном и в
положенном соотношении есть тождественная {\em основа},
есть реальное основание равным образом не в себе и для себя самого, а есть
основание, поскольку это положено через указанную первоначальную связь.

Соотношение основания в его целостности есть тем самым по существу
{\em предполагающая} рефлексия; формальное основание
предполагает {\em непосредственное} определение
содержания, а это определение как реальное основание предполагает форму.
Основание есть, следовательно, форма как непосредственная связь; но таким
образом, что она отталкивает себя от себя самой и скорее предполагает
непосредственность, соотносится в ней с собой как с чем-то другим. Это
непосредственное есть определение содержания, простое
{\em основание}; но последнее как таковое, т.~е. как
основание, столь же и оттолкнуто от себя и соотносится с собой равным
образом, как с чем-то другим. — Таким образом, целостное соотношение
основания определило себя в {\em обусловливающее
опосредствование.}

\paragraph[С. \ Условие]{С. \ Условие}
\subparagraph[а)
\ Относительно{}-безусловное]{а) \ Относительно-безусловное}
Основание есть непосредственное, а обоснованное
— опосредствованное. Но основание есть полагающая рефлексия; как таковая
оно делает себя положенностью и есть предполагающая рефлексия; таким
образом, оно соотносится с собой, как со снятым, как с некоторым
непосредственным, которым оно само опосредствовано. Это опосредствование
как движение вперед от непосредственного в основанию не есть некоторая
внешняя рефлексия, а, как оказалось, собственное действие основания, или,
что то же самое, соотношение основания как рефлексия в тождество с собою
есть столь же существенным образом отчуждающая себя рефлексия. То
непосредственное, с которым основание соотносится, как со своей
существенной предпосылкой, есть {\em условие}; поэтому
реальное основание по существу обусловлено. Содержащаяся в нем
определенность есть инобытие его самого.

Условие есть, следовательно, {\em во-первых},
непосредственное, многообразное наличное бытие.
{\em Во-вторых}, это наличное бытие соотнесено с
некоторым другим, с чем-то таким, что является основанием не этого
наличного бытия, а в другом отношении; ибо само наличное бытие
непосредственно и не имеет основания. По указанному соотношению наличное
бытие есть некоторое {\em положенное}; непосредственное
наличное бытие должно быть условием не для себя, а для другого. Но вместе с
тем то обстоятельство, что оно есть условие для другого, само есть лишь
некоторое положенное; что оно есть некоторое положенное, это снято в его
непосредственности, и какое-либо {\em наличное бытие
безразлично к тому факту, что оно есть условие}.
{\em В-третьих}, условие есть непосредственное таким
образом, что оно составляет {\em предпосылку}
основания. Условие есть в этом определении возвратившееся в тождество с
собою формальное соотношение основания и тем самым есть
{\em содержание} последнего. Но содержание как таковое
есть лишь безразличное единство основания, как облеченного в форму; без
формы нет содержания. Оно еще до некоторой степени освобождается от нее,
поскольку соотношение основания становится в
{\em полном} основании некоторым по отношению к своему
тождеству внешним соотношением, благодаря чему содержание получает
непосредственность. Поэтому, поскольку условие есть то, в чем соотношение
основания имеет свое {\em тождество} с собой, оно
составляет содержание основания; но так как оно безразлично к этой форме,
то оно есть ее содержание лишь {\em в себе}, есть нечто
такое, что лишь {\em должно} стать содержанием и, стало
быть, составляет {\em материал} для основания.
Положенное как условие, наличное бытие имеет по второму моменту то
определение, что оно утрачивает свою безразличную непосредственность и
становится моментом некоторого другого. В силу своей непосредственности оно
безразлично к этому соотношению; но поскольку оно вступает в последнее, оно
образует собой {\em в-себе-бытие} основания и есть для
него {\em безусловное}. Чтобы быть условием, оно имеет
свою предпосылку в основании и само обусловлено; но это определение ему
внешне.

2. Нечто есть не в силу своего условия; его условие не представляет собой
его основания. Условие есть для основания момент безусловной
непосредственности, но само оно не есть то движение и полагание, которое
соотносится с собой отрицательно и делает себя положенностью. Поэтому
условию противостоит {\em соотношение основания}. Нечто
имеет кроме своего условия также и основание. — Последнее есть пустое
движение рефлексии, так как оно (движение) имеет непосредственность как
свою предпосылку вне себя. Но оно есть вся форма и самостоятельный процесс
опосредствования; ибо условие не есть его основание. Так как этот процесс
опосредствования соотносится как полагание с собой, то он по этой своей
стороне равным образом есть некоторое непосредственное и
{\em безусловное}; он, правда, предполагает себя, но
как отчужденное или снятое полагание; напротив, то, что он есть по своему
определению, он есть в себе и для себя самого. — Поскольку, таким образом,
соотношение основания есть самостоятельное соотношение с собой и имеет
тождество рефлексии в себе самом, оно имеет некоторое
{\em собственное содержание}, противостоящее содержанию
условия. Первое есть содержание основания и потому по существу оформлено;
напротив, содержание условия есть лишь непосредственный материал, которому
соотношение с основанием одновременно в такой же мере внешне, в какой он
составляет также его (основания) в-себе-бытие; тем самым он есть смесь
самостоятельного содержания, не имеющего никакого соотношения с содержанием
определения основания, и такого содержания, которое входит в это
определение и как его материал должно стать его моментом.

3. Обе стороны целого, {\em условие и основание}, суть,
следовательно, с одной стороны, взаимно
{\em безразличные и безусловные}: первое как
несоотнесенное, которому внешне то соотношение, в котором оно есть условие;
второе как соотношение или форма, для которой определенное наличное бытие
условия есть лишь материал, лишь нечто пассивное, форма которого, присущая
ему особо, несущественна. Далее, обе эти стороны суть также и
{\em опосредствованные}. Условие есть
{\em в-себе-бытие} основания; условие есть в такой мере
существенный момент соотношения основания, что составляет простое его
тождество с собою. Но и это снято; это в-себе-бытие есть лишь некоторое
положенное; непосредственное наличное бытие безразлично к тому, чтобы быть
условием. Что условие есть для основания
{\em в-себе-бытие}, это составляет, следовательно, ту
сторону условия, по которой оно есть опосредствованное. Равным образом
соотношение основания имеет в своей самостоятельности также и некоторую
предпосылку и имеет свое в-себе-бытие вне себя. — Стало быть, каждая из
обеих сторон есть {\em противоречие} безразличной
непосредственности и существенного опосредствования, — есть то и другое в
{\em одном} соотношении; или, иначе говоря, каждая из
обеих сторон есть противоречие самостоятельного устойчивого наличия и того
определения, что она есть лишь момент.

\subparagraph[b) \ Абсолютное безусловное]{b) \ Абсолютное безусловное}
Оба относительно-безусловных ближайшим образом
светят каждое в другое: условие, как непосредственное, светит в формальное
соотношение основания, а это соотношение —~в непосредственное наличное
бытие как в свою положенность; но каждое из них вне этого мерцания в нем
его другого самостоятельно и имеет свое собственное содержание.

Прежде всего {\em условие} есть непосредственное
наличное бытие; форма этого наличного бытия имеет два момента:
{\em положенность}, по которой оно как условие есть
материал и момент основания, и {\em в-себе-бытие}, по
которому оно составляет существенность основания или его простую рефлексию
в себя. Обе стороны формы внешни непосредственному наличному бытию; ибо оно
есть снятое соотношение основания. — Но,
{\em во-первых}, наличное бытие в себе самом состоит
лишь в том, что оно снимает себя в своей непосредственности и идет ко дну,
погружается в основание (zu Grunde gehen мы чаще всего переводим словами
«идет ко дну, погружается в основание», чтобы передать те два смысла,
которые Гегель вкладывает в это выражение: гибель и гибель именно в
основании. — Перев.). {\em Бытие} есть вообще лишь
{\em становление} сущностью; его существенная природа
состоит в том, чтобы обратить себя в положенное и в тождество, которое есть
непосредственное через отрицание себя. Следовательно, такие определения
формы, как положенность и тождественное с собой в-себе-бытие, — та форма, в
силу которой непосредственное наличное бытие есть условие, — эти
определения поэтому не внешни ему, а оно есть сама эта рефлексия.
{\em Во-вторых}, как условие бытие теперь также и
положено как то, что оно есть по существу, а именно, как момент и, стало
быть, момент некоторого другого, и вместе с тем, как в-себе-бытие (равным
образом некоторого другого); но оно есть {\em в себе}
лишь через отрицание себя, а именно, через основание и через его снимающую
себя и, стало быть, предполагающую рефлексию; в-себе-бытие бытия есть,
следовательно, лишь некоторое положенное. Это в-себе-бытие условия имеет
две стороны: одна заключается в том, что оно есть существенность условия
как существенность основания, а другая —~в том, что оно есть
непосредственность его наличного бытия. Или, правильнее сказать, обе эти
стороны суть одно и то же. Наличное бытие есть нечто непосредственное, но
эта непосредственность есть по существу опосредствованное, а именно,
опосредствованное снимающим само себя основанием. Как такая
опосредствованная снимающим себя процессом опосредствования
непосредственность оно есть одновременно и в-себе-бытие основания и его
безусловное; но само это в-себе-бытие есть вместе с тем в свою очередь
равным образом лишь момент или положенность, ибо оно опосредствовано. —
Условие есть поэтому вся форма соотношения основания; условие есть
предположенное в-себе-бытие основания, но тем самым оно само есть некоторая
положенность, и его непосредственность состоит в том, что оно превращает
себя в положенность и, стало быть, так отталкивает себя от самого себя, что
оно столь же идет ко дну, погружается в основание, сколь и есть основание,
превращающее себя в положенность, а, значит, также и в обоснованное; и оба
суть одно и то же.

Равным образом, в обусловленном основании в-себе-бытие есть не только
свечение в нем некоторого другого. Это основание есть самостоятельная,
т.~е. соотносящаяся с собой рефлексия полагания и, следовательно, нечто
тождественное с собою, или, иначе говоря, оно есть в себе самом свое
в-себе-бытие и свое содержание. Но вместе с тем оно есть предполагающая
рефлексия; оно соотносится с самим собой отрицательно и противополагает
себе свое в-себе-бытие как другое ему, и условие как по своему моменту
в-себе-бытия, так и по моменту непосредственного наличного бытия есть
собственный момент соотношения основания; непосредственное наличное бытие
по существу имеет бытие лишь через свое основание и есть момент его как
предполагания. Поэтому основание есть равным образом и само целое.

Таким образом, имеется вообще лишь {\em одно} целое
{\em формы}, но равным образом и лишь
{\em одно} целое {\em содержания}.
Ибо своеобразное содержание условия есть существенное содержание лишь
постольку, поскольку оно (содержание) есть тождество рефлексии с собою в
форме или, иначе говоря, поскольку оно как это непосредственное наличное
бытие в нем самом есть соотношение основания. Это наличное бытие есть,
далее, условие лишь в силу предполагающей рефлексии основания; оно есть
тождество последнего с самим собою или его содержание, которому основание
противополагает себя. Поэтому наличное бытие не есть только бесформенный
материал для соотношения основания, а ввиду того, что наличное бытие имеет
в себе самом эту форму, оно есть оформленная материя, и как то, что вместе
с тем в своем тождестве с формой есть безразличное к ней, оно есть
содержание. Оно есть, наконец, то же самое содержание, которым обладает
основание, ибо оно есть содержание именно как то, что тождественно с собой
в соотношении формы.

Обе стороны целого, условие и основание, суть, следовательно, единое
существенное единство и как содержание и как форма. Они переходят друг в
друга через себя самих, или, иными словами, так как они суть рефлексии, то
они полагают сами себя как снятые, соотносят себя с этим своим, отрицанием
и {\em взаимно предполагают себя}. Но это есть вместе с
тем лишь одна и та же рефлексия обоих, в потому их предполагание также есть
лишь одно и то же пред- полагание: взаимность этого предполагания
переходит, собственно говоря, в то, что они предполагают единое свое
тождество как свое устойчивое наличие и свою основу. Последняя (одно и то
же содержание и единство формы обоих) есть {\em истинно
безусловное; мыслимая вещь в себе самой}. — Условие, как выяснилось выше,
есть лишь относительно-безусловное. Поэтому само условие обыкновенно
рассматривают, как нечто обусловленное, и спрашивают о некотором новом
условии, чем вводится обычный {\em бесконечный
прогресс} от одного условия к другому. Но почему же по поводу одного
условия спрашивают о некотором новом условии, т.~е. почему первое
признается обусловленным? Потому что оно есть некоторое конечное наличное
бытие. Но это есть дальнейшее определение условия, не заключающееся в его
понятии. Однако условие как таковое потому есть некоторое обусловленное,
что оно есть положенное в-себе-бытие. Оно поэтому снято в абсолютно
безусловном.

Последнее содержит в себе теперь обе стороны, условие и основание, как свои
моменты; оно есть то единство, в которое они возвратились. Обе они вместе
образуют форму или положенность этого безусловного. Безусловная мыслимая
вещь есть условие обеих, но абсолютное, т.~е. такое условие, которое само
есть основание. — Как {\em основание} же оно есть
отрицательное тождество, которое растолкнуло себя, раскололо себя на эти
два момента: {\em во-первых}, приняло вид снятого
соотношения основания, непосредственного, лишенного единства, внешнего
самому себе многообразия, соотносящегося с основанием, как с чем-то чуждым
ему, и составляющего вместе с тем его в-себе-бытие;
{\em во-вторых}, приняло вид внутренней простой формы,
которая есть основание, но соотносится с тождественным с собой
непосредственным, как с другим, и определяет его как условие, т.~е.
определяет это свое «в себе» как свой собственный момент. — Эти две стороны
{\em предполагают} целокупность так, что она есть
полагающая их. Обратно, так как они {\em предполагают}
целокупность, то кажется, что последняя опять-таки обусловлена ими и что
мыслимая вещь возникает из ее условия и из ее основания. Но поскольку обе
эти стороны оказались чем-то тождественным, то исчезает соотношение условия
и основания; последние понижаются {\em до видимости};
абсолютно безусловное в своем движении полагания и предполагаем есть лишь
такое движение, в котором эта {\em видимость}
снимается. Действие мыслимой вещи состоит в том, чтобы обусловливать себя и
противопоставлять себе свои условия как основание; а ее соотношение как
соотношение условий и основания есть излучение видимости
{\em в себя}, а ее отношение к ним есть
{\em ее слияние с самою собою}.

\subparagraph[с) \ Выход мыслимой вещи в существование]{с) \ Выход мыслимой
вещи в существование}
Абсолютно-безусловное есть абсолютное,
тождественное со своим условием основание, непосредственная мыслимая вещь
как истинно-существенная. Как {\em основание} она
соотносится отрицательно с собою самой, делает себя положенностью, но такой
положенностью, которая есть полная рефлексия в обеих ее сторонах и
тождественное в них с собою соотношение формы, т.~е. так, как получилось
выше ее понятие. Поэтому, указанная положенность есть,
{\em во-первых}, снятое основание, мыслимая вещь как
лишенное рефлексии непосредственное; это —~сторона условий. Последняя есть
{\em целокупность} определений мыслимой вещи, — сама
мыслимая вещь, но отброшенная во внешность бытия; это —~восстановленный
круг бытия. В условии сущность отпускает единство своей рефлексии в-себя
как некоторую непосредственность, которая, однако, имеет теперь определение
быть {\em обусловливающей} предпосылкой и составлять по
существу лишь одну из ее (сущности) сторон. — Условия суть поэтому все
содержание мыслимой вещи, так как они суть безусловное в форме
бесформенного бытия. Но в силу этой формы они имеют еще и другой вид, чем
определения содержания, как это последнее есть в мыслимой вещи как таковой.
Они являются лишенным единства многообразием, смешаны с вне-существенным и
с другими обстоятельствами, не принадлежащими к кругу того наличного бытия,
которое образует условия этой {\em определенной}
мыслимой вещи.— Для абсолютной неограниченной мыслимой вещи условием служит
сама {\em сфера бытия}. Основание, возвращающееся в
себя, полагает ее как первую непосредственность, с которой оно соотносится,
как со своим безусловным. Эта непосредственность как снятая рефлексия есть
{\em рефлексия} в элементе бытия, которое, стало быть,
как таковое развивается в некоторое целое; форма как определенность бытия
разрастается дальше и выступает, таким образом, как некоторое
многообразное, отличное от рефлексивного определения и безразличное к нему
содержание. Несущественное, которым сфера бытия обладает в ней самой и
которое она, поскольку она есть условие, сбрасывает с себя, есть та
определенность непосредственности, в которую погружено единство формы. Это
единство формы как соотношение бытия есть в последнем ближайшим образом
{\em становление}, переход одной определенности бытия в
другую. Но становление бытия есть, далее, становление сущностью и
возвращение в основание. Поэтому в действительности дело обстоит не так,
что наличное бытие, образующее условия, определяется как условие и
употребляется как материал чем-то другим, а так, что само оно превращает
себя через себя само в момент некоторого другого. — Далее, его становление
не есть начинание с себя, как с поистине первого и непосредственного, а его
непосредственность есть лишь предположенное, и движение его становления
есть дело самой рефлексии. Истина наличного бытия состоит поэтому в том,
что оно есть условие; его непосредственность имеет место лишь через
рефлексию соотношения основания, полагающую себя самое как снятую.
Становление есть тем самым, подобно непосредственности, лишь видимость
безусловного, поскольку последнее предполагает себя само и имеет в этом
предполагании свою форму; непосредственность бытия есть поэтому по существу
лишь {\em момент} формы.

Другой стороной этого свечения безусловного видимостью служит соотношение
основания как таковое, определенное по отношению к непосредственности
условий и содержания как форма. Но она есть форма абсолютной мыслимой вещи,
которая (мыслимая вещь) в себе самой содержит единство своей формы с собою
самой или свое {\em содержание} и которая, поскольку
она определяет последнее как условие, в самом этом полагали и снимает его
разность и обращает его в момент; равно как и, наоборот, она как лишенная
сущности форма сообщает себе в этом тождестве с собою непосредственность
устойчивого наличия. Рефлексия основания снимает непосредственность условий
и соотносит их, делает их моментами в единстве мыслимой вещи; но условия
суть нечто предположенное самой безусловною мыслимою вещью, и, таким
образом, она тем самым снимает свое собственное полагание; или, иначе
говоря, тем самым ее полагание непосредственно само делает себя вместе с
тем также и {\em становлением}. — То и другое есть
поэтому единое единство; движение условий в них самих есть становление,
возвращение в основание и полагание основания; но основание как положенное,
т.~е. как снятое, есть непосредственное. Основание соотносится с самим
собой отрицательно, делает себя положенностью и обосновывает условия; но
тем, что, таким образом, непосредственное наличное бытие определяется как
нечто положенное, основание снимает его и делает себя только теперь
основанием. — Таким образом, эта рефлексия есть опосредствование
безусловной мыслимой вещи с собой через ее отрицание. Или, лучше сказать,
рефлексия безусловного есть, во-первых, предполагание, но это снятие ее
самой есть непосредственно определяющее полагание; во-вторых, она при этом
есть непосредственно снятие предположенного и определяемость из себя; тем
самым эта определяемость есть опять-таки снятие полагания и представляет
собою становление в себе самом. В этом становлении исчезло опосредствование
как возвращение к себе через отрицание; оно есть простая светящая
видимостью внутрь себя рефлексия и лишенное основания абсолютное
становление. Движение мыслимой вещи, состоящее в том, что она,
{\em с одной стороны}, полагается через ее
{\em условия}, а с другой —~через ее основание, есть
лишь {\em исчезание видимости опосредствования}.
Становление мыслимой вещи положенною есть, стало быть,
{\em выступление}, простой выход в
{\em существование}, чистое движение ее к себе самой.

{\em Если имеются налицо все условия какой-нибудь
мыслимой вещи}, то она вступает в существование: мыслимая вещь
{\em имеет бытие ранее, чем она существует}, а именно
она имеет бытие, во-первых, как {\em сущность} или как
безусловное; во-вторых, она обладает {\em наличным
бытием} или определена, и определена именно рассмотренным выше двояким
образом: с одной стороны, в своих условиях, а с другой —~в своем основании.
В первых она сообщила себе форму внешнего, лишенного основания бытия, так
как она как абсолютная рефлексия есть отрицательное соотношение с собой и
обращает себя в свою предпосылку, в свое предположение. Это предположенное
безусловное есть поэтому лишенное основания непосредственное, бытие
которого состоит лишь в том, что оно налично как лишенное основания. Таким
образом, если все условия мыслимой вещи налицо, т.~е. если положена ее
тотальность как лишенная основания непосредственность, то это рассеянное
многообразие {\em углубляется} в нем самом. — Вся
мыслимая вещь должна наличествовать в ее условиях, или, иначе говоря, для
ее существования требуются все условия, ибо {\em все}
они составляют рефлексию; или, еще иначе, наличное бытие, поскольку оно
есть условие, определено формой; его определения суть поэтому определения
рефлексии и вместе с каждым из них существенным образом доложены и другие.
— Получение условиями {\em внутреннего характера} есть
ближайшим образом опускание ко дну непосредственного наличного бытия и
становление основания. Но тем самым основание есть положенное, т.~е. оно в
той же мере, в какой оно есть основание, также и снято как основание, и
есть непосредственное бытие. Если, следовательно, налицо все условия
мыслимой вещи, то они снимают себя как непосредственное наличное бытие и
как предпосылку, и одинаковым образом снимает себя также и основание.
Основание оказывается лишь видимостью, которая непосредственно исчезает;
это выступление есть тем самым тавтологическое движение мыслимой вещи к
себе, и ее опосредствование через условия и через основание есть исчезание
как условий, так и основания. Выступление в существование потому столь
непосредственно, что оно опосредствовано лишь через исчезание
опосредствования.

{\em Мыслимая вещь проистекает, из основания}. Она не
обосновывается или не полагается им так, что основание еще остается под
нею, а полагание есть движение основания вовне, выход его в себе самому и
простое его исчезание. Через {\em соединение} с
условиями оно получает внешнюю непосредственность и момент бытия. Но оно
получает их не как нечто внешнее и не через некоторое внешнее соотношение,
а как основание оно делает себя положенностью, его простая существенность
сливается с собою в положенности и есть в этом снятии себя самого исчезание
его различия от его положенности и, стало быть, простая существенная
непосредственность. Основание, следовательно, не остается после обоснования
как нечто отличное от обоснованного, а истина процесса обоснования состоит
в том, что основание в нем соединяется с самим собою, и, стало быть, его
рефлексия в другое есть его рефлексия в себя само. Поэтому мыслимая вещь
точно так же, как она есть {\em безусловное}, есть
равным образом и {\em лишенное основания} и выступает
из основания, лишь поскольку последнее опустилось
{\em ко дну}, поскольку его уже нет, — выступает из
того, что не имеет основания, т.~е. из собственной существенной
отрицательности или чистой формы.

Эта опосредствованная основанием и условием и ставшая через снятие
опосредствования тождественною с собою непосредственность есть
{\em существование}.


\bigskip

