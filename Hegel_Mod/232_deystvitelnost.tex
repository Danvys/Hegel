\subsubsection{Вторая глава. Действительность}
Абсолютное есть единство внутреннего и внешнего
как {\em первое},
{\em в-себе-сущее} единство. Развертывание
представлялось {\em внешней} рефлексией, которая со
своей стороны имеет непосредственное как некоторое преднайденное, но вместе
с тем есть его движение и соотношение с абсолютным и как таковое приводит
это преднайденное обратно в абсолютное и определяет его как
{\em голый вид и способ}. Но этот вид и способ есть
определение самого абсолютного, а именно, его
{\em первое тождество} или его
{\em лишь в-себе-сущее единство}. И притом этой
рефлексией не только полагается то первое в-себе-бытие, как лишенное
сущности определение, но так как она есть отрицательное соотношение с
собой, то лишь через нее возникает сказанный модус. Эта рефлексия, как
снимающая самое себя в своих определениях, и вообще, как возвращающееся в
себя движение, есть впервые истинно абсолютное тождество и вместе с тем она
есть процесс определения абсолютного или его модальность. Модус есть
поэтому внешность абсолютного, но также лишь как его рефлексия в себя; или,
иначе говоря, он есть {\em собственное проявление}
абсолютного, так что это проявление во вне есть его рефлексия-в-себя и тем
самым его в себе-и-для-себя-бытие.

Таким образом, как проявление того обстоятельства, что абсолютное не
представляет собою ничего другого и не имеет никакого другого содержания,
кроме того, что оно есть проявление себя, абсолютное есть
{\em абсолютная форма}.

{\em Действительность} следует понимать как эту
рефлектированную абсолютность. {\em Бытие} еще не
действительно: оно есть первая непосредственность; его рефлексия есть
поэтому становление и {\em переход в другое}; или,
иначе говоря, его непосредственность не есть в-себе-и-для-себя-бытие.
Действительность стоит выше также и
{\em существования}. Последнее есть, правда, такая
непосредственность, которая произошла из основания и условий или, иначе
говоря, из \ сущности и ее рефлексии. Оно поэтому
{\em в себе} есть то же самое, что и действительность,
есть {\em реальная рефлексия}, но еще не есть
{\em положенное} единство рефлексии и
непосредственности. Существование переходит поэтому в
{\em явление}, развивая ту рефлексию, которую оно
содержит в себе. Оно есть пошедшее ко дну основание; его определение есть
восстановление последнего; таким образом, оно становится существенным
отношением, и его последняя рефлексия состоит в том, что его
непосредственность положена как рефлексия в себя, и обратно; это единство,
в котором существование или непосредственность, и в-себе-бытие, основание
или рефлектированное суть всецело моменты, и есть
{\em действительность}. Действительное есть поэтому
{\em проявление}; оно не вовлекается своей внешностью в
сферу {\em изменения}, оно не есть также и свечение
своей видимостью в {\em некотором другом}, а оно
проявляет себя; это означает, что оно есть в своей внешности
{\em оно же само}, и что лишь
{\em в ней}, а именно, лишь как отличающее себя от себя
и определяющее себя движение, оно есть {\em оно же
само}.

В действительности, как в этой абсолютной форме, моменты суть лишь снятые
или формальные, еще не реализованные; их разность принадлежит, таким
образом, прежде всего к внешней рефлексии и не определена как содержание.

Действительность, как представляющая собою
{\em непосредственное} формальное единство внутреннего
и внешнего, тем самым обладает определением
{\em непосредственности}, противостоящим определению
рефлексии в себя; или, иначе сказать, она есть некоторая
{\em действительность, противостоящая некоторой
возможности}. {\em Соотношение} их обоих друг с другом
есть {\em третье} определение —~действительное,
определенное также и как рефлектированное в себя бытие, и последнее,
определенное вместе с тем, и как непосредственно существующее. Это третье
есть {\em необходимость}.

Но {\em ближайшим образом}, поскольку действительное и
возможное суть {\em формальные различия}, их
соотношение также только {\em формально} и состоит лишь
в том, что как одно, так и другое есть
{\em положенность}, или, иначе сказать, состоит в
{\em случайности}.

А тем самым, что в случайности действительное, равно как и возможное, есть
{\em положенность}, они получили в них определение;
благодаря этому возникает, {\em во-вторых, реальная
действительность}, а тем самым возникают также реальная
{\em возможность} и
{\em относительная необходимость}.

Рефлексия в себя относительной необходимости дает,
{\em в-третьих, абсолютную необходимость}, которая есть
абсолютная {\em возможность} и
{\em действительность}.

\paragraph[А. \ Случайность или формальная действительность, формальная
возможность и формальная необходимость]{А. \ Случайность или формальная
действительность, формальная возможность и формальная необходимость}
1. Действительность формальна, поскольку она,
как первая действительность, есть лишь непосредственная, нерефлектированная
действительность, и тем самым ей присуще лишь это определение формы, но не
тотальность формы. Таким образом, она не есть ничего более, как некоторое
{\em бытие} или {\em существование}
вообще. Но так как она {\em по существу} есть не голое
непосредственное существование, а формальное единство в-себе-бытия или
внутренности и внешности, то она содержит в себе непосредственно
{\em в-себе-бытие} или
{\em возможность}. {\em Что
действительно, то возможно.}

2. Эта возможность есть рефлектированная в себя действительность. Но сама
эта первая {\em рефлектированность} есть равным образом
нечто формальное и тем самым вообще лишь
{\em определение тождества} с собою или в-себе-бытия
вообще.

Но так как определение есть здесь {\em тотальность
формы}, то это в-себе-бытие (определенное, как
{\em снятое} или как находящееся по существу лишь в
соотношении в действительностью, как отрицательное последней)
{\em положено} как отрицательное. Возможность содержит
в себе поэтому два момента; {\em во-первых}, тот
{\em положительный} момент, что она есть
рефлектированность в себя самое; но так как эта рефлектированность
низведена в абсолютной форме до значения момента, то рефлектированность в
себя уже не признается {\em сущностью}, а имеет,
{\em во-вторых},
{\em отрицательное} значение, по которому возможность
есть нечто недостаточное, указует на некоторое другое, на действительность,
и восполняет себя в ней.

По первой, только положительной стороне возможность есть, следовательно,
голое формальное определение {\em тождества с собой}
или форма существенности. Таким образом, она есть лишенное отношения,
неопределенное вместилище всего вообще. — В смысле этой формальной
возможности {\em возможно все, что не противоречит
себе}; царство возможности представляет собой поэтому безграничное
многообразие. Но каждое многообразное {\em определено
внутри себя и в противоположность другому} и содержит в себе отрицание;
безразличная {\em разность} переходит вообще в
{\em противоположение}; противоположение же есть
противоречие. Поэтому {\em все} есть также и нечто
противоречивое и потому {\em невозможное}.

Поэтому это чисто формальное высказывание о чем-либо, что
{\em оно возможно}, столь же плоско и пусто, как и
предложение о противоречии, и каждое вобранное в это высказывание
содержание $-А$ возможно —~означает не более того, что $А$
есть $А$. Поскольку не пытаются развить содержание, оно имеет форму
{\em простоты}; лишь через его разложение на его
определения в нем появляется {\em различие}. Пока
держатся за простую форму, содержание остается чем-то тождественным с собой
и потому некоторой {\em возможностью}. Но этим так же
{\em ничего} не сказано, как и формальным тождественным
предложением.

Однако возможное содержит в себе нечто большее, чем чисто тождественное
предложение. Возможное есть {\em рефлектированная
рефлектированность в себя} или тождество всецело как
{\em момент} тотальности; тем самым оно определено
также и как то, что {\em не есть в себе}; оно имеет
поэтому второе определение, согласно которому оно есть
{\em лишь} возможное и
{\em долженствование} тотальности формы. Возможность
без этого долженствования есть {\em существенность} как
таковая; но в абсолютной форме содержится, что сама сущность есть лишь
момент и без бытия не имеет своей истины. Возможность есть эта голая
существенность, {\em положенная} так, что она есть лишь
момент и не соответствует абсолютной форме. Она есть в-себе-бытие,
определенное как лишь некоторое {\em положенное} или
равным образом как то, что {\em не есть}
{\em в себе}. — Поэтому возможность есть в себе самой
также и противоречие, или, иначе говоря, она есть
{\em невозможность}.

Это выражается ближайшим образом так: возможность, как
{\em положенное снятым определение формы}, обладает
некоторым {\em содержанием} вообще. Это содержание, как
возможное, есть некоторое в-себе-бытие, которое вместе с тем есть нечто
снятое или некоторое {\em инобытие}. Но так как оно
есть только нечто возможное, то равным образом
{\em возможно} также и {\em другое}
содержание, в том числе и противоположное первому. $A$ {\em есть}
$А$; точно так же и $-А$ есть $-А$. Каждое из этих двух
предложений выражает возможность своего определения содержания. Но как
такие тождественные предложения они безразличны друг к другу; вместе с
одним из них {\em не положено}, что имеет место и
другое. Возможность есть сравнивающее соотношение их обоих; в том своем
определении, согласно которому она есть некоторая рефлексия тотальности,
она содержит то обстоятельство, что возможно также и противоположное.
Поэтому она есть соотносящее основание того, что
{\em так как} $А=А$, то и
$-A=-A$; в возможном $А$ содержится также и
возможное не $-А$, и самое это соотношение определяет и то и другое
как возможные.

Но как то соотношение, что в одном возможном содержится также и его другое,
возможность есть противоречие, которое снимает себя. Так как она по своему
определению есть нечто рефлектированное и, как оказалось, снимающее себя
рефлектированное, то она, стало быть, есть также и непосредственное, и тем
самым становится {\em действительностью}.

3. Эта действительность есть не первая, а рефлектированная действительность,
{\em положенная как единство} себя самой и возможности.
Действительное как таковое возможно; оно находится в непосредственном
положительном тождестве с возможностью; но последняя определила себя как
{\em только} возможность; тем самым и действительное
определено как {\em нечто только возможное}. И
непосредственно потому, что возможность
{\em непосредственно} содержится в действительности,
она содержится в последней как снятая, как {\em только}
возможность. Наоборот, действительность, находящаяся в единстве с
возможностью, есть лишь снятая непосредственность; или, иначе сказать,
именно потому, что формальная действительность есть лишь
{\em непосредственная}, первая действительность, она
есть лишь момент, лишь снятая действительность или лишь
{\em возможность}.

Тем самым вместе с тем выражено более детально определение, указывающее, в
какой мере {\em возможность есть действительность}. А
именно, возможность еще не есть {\em всяческая}
действительность, — о реальной и абсолютной действительности еще не было
речи, — возможность есть лишь та действительность, которая встретилась нам
сначала, а именно, формальная, определившая себя так, что она есть
{\em лишь} возможность. Речь идет, стало быть, о
формальной действительности, которая есть лишь
{\em бытие} или {\em существование}
вообще. Все возможное имеет поэтому вообще некоторое
{\em бытие} или некоторое
{\em существование}.

Это единство возможности и действительности есть
{\em случайность}. — Случайное есть некоторое
действительное, определенное вместе с тем лишь как возможное, другое или
противоположность которого также имеет бытие. Эта действительность есть
поэтому голое бытие или существование, но положенное в своей истине,
заключающейся в том, что оно имеет значение некоторой положенности или
возможности. Наоборот, возможность, как {\em рефлексия
в себя} или {\em в-себе-бытие}, положена как
положенность; то, что возможно, есть некоторое действительное в этом смысле
действительности; оно имеет лишь значение случайной действительности; оно
само есть нечто случайное.

Случайное представляет нам поэтому две стороны;
{\em во-первых}, поскольку оно
{\em непосредственно} содержит в себе возможность или,
что то же самое, поскольку возможность в нем снята, оно
{\em не есть ни положенность}, ни опосредствованное, а
есть {\em непосредственная} действительность; оно не
имеет {\em основания}. — Так как эта непосредственная
действительность присуща также и возможному, то и оно определено столь же
как действительное, сколь и как случайное, и равным образом есть нечто
{\em лишенное основания}.

Но случайное есть, {\em во-вторых}, действительное как
некоторое {\em лишь} возможное или как некоторое
положенное; точно так же и возможное, как формальное в-себе-бытие, есть
лишь положенность. Тем самым и то и другое не есть в себе и для себя
самого, а имеет свою истинную рефлексию-в-себя в некотором другом,
{\em или, иначе сказать, оно имеет некоторое
основание}.

Случайное, следовательно, не имеет основания потому, что оно случайно; и оно
точно так же имеет некоторое основание, потому что оно случайно.

Оно есть {\em положенное}, неопосредствованное
{\em превращение} друг в друга внутреннего и внешнего
или рефлектированности-в-себя и бытия, —
{\em положенное} через то, что и возможность и
действительность, каждая в себе самой, обладают этим определением, — через
то, что они суть моменты абсолютной формы. — Таким образом,
действительность в ее {\em непосредственном} единстве с
возможностью есть лишь существование и определена как такое лишенное
основания, которое есть {\em только нечто положенное}
или {\em только} возможное; или, если возьмем ее, как
рефлектированную и определенную в
{\em противоположность} к возможности, то она отделена
от возможности, от рефлектированности-в-себя, и тем самым столь же
непосредственно есть также {\em лишь} возможное. —
Равным образом, возможность, как {\em простое}
в-себе-бытие, есть некоторое непосредственное,
{\em лишь} некоторое сущее вообще; или, если возьмем ее
{\em противопоставленною} действительности, то она
равным образом есть лишенное действительности в-себе-бытие,
{\em лишь} некоторое возможное, но именно потому она
опять-таки есть лишь некоторое нерефлектированное в себя существование
вообще.

Это {\em абсолютное беспокойство становления} обоих
определений есть {\em случайность}. Но именно потому,
что каждое из них непосредственно превращается в противоположное, оно в
последнем столь же безоговорочно {\em сливается с самим
собой}, и это {\em тождество} каждого из них в другом
есть {\em необходимость}.

Необходимое есть некоторое {\em действительное}; таким
образом, оно, как непосредственное, есть {\em лишенное
основания}; однако в такой же мере оно имеет свою действительность
{\em через некоторое другое} или в своем основании, но
есть вместе с тем положенность этого основания и его рефлексия в себя;
возможность необходимого есть снятая возможность. Следовательно, случайное
необходимо потому, что действительное определено как возможное, и тем самым
его непосредственность снята, оттолкнута так, что она раскалывается на
{\em основание} или
{\em в-себе-бытие} и на
{\em основанное}, а также потому, что эта его
{\em возможность, соотношение основания} безоговорочно
снята и положена как бытие. Необходимое {\em есть}, и
это сущее {\em само} есть
{\em необходимое}. Вместе с тем оно есть в себе; эта
рефлексия в себя есть некоторое {\em другое}, чем та
непосредственность бытия; и необходимость сущего есть
{\em некоторое другое}. Само сущее не есть, таким
образом, необходимое; но это в-себе-бытие само есть лишь положенность; оно
снято и само непосредственно. Таким образом, действительность в отличенном
от нее определении, т.~е. в возможности, оказывается тождественной с самой
собой. Как это тождество, она есть необходимость.

\paragraph[В. \ Относительная необходимость или реальная действительность,
реальная возможность и реальная необходимость]{В. \ Относительная
необходимость или реальная действительность, реальная возможность и
реальная необходимость}
1. Та необходимость, которая у нас получилась,
{\em формальна}, так как ее моменты формальны, а
именно, суть простые определения, которые представляют собой тотальностъ,
лишь как непосредственное единство или как непосредственное превращение
одного в другое, и тем самым не имеют образа самостоятельности. — В этой
формальной необходимости единство поэтому ближайшим образом просто и
безразлично в своим различиям. Как
{\em непосредственное} единство определений формы, эта
необходимость есть {\em действительность}, но такая
действительность, которая (так как ее единство теперь
{\em определено как безразличное} к
{\em различию} определений формы, а именно, ее самой и
возможности) имеет некоторое {\em содержание}.
Последнее, как безразличное тождество, содержит в себе также и форму как
безразличную, т.~е. как просто {\em разные}
определения, и есть {\em многообразное} содержание
вообще. Эта действительность есть {\em реальная
действительность}.

Реальная действительность {\em как таковая} есть
ближайшим образом вещь со многими свойствами, существующий мир; но она не
есть то существование, которое разрешается в явление, а, как
действительность, она есть вместе с тем в-себе-бытие и рефлексия в себя;
она сохраняется в многообразии голого существования; ее внешность есть
внутреннее отношение лишь к {\em себе самой}. То, что
действительно, {\em может действовать}; свою
действительность нечто возвещает {\em через то, что оно
порождает}. Его отношение к другому есть проявление
{\em себя}: это отношение не есть ни переход —~таким
образом сущее нечто соотносит себя с другим, — ни некоторое явление —~таким
образом вещь имеет бытие лишь в отношении к другим вещам и хотя и
представляет собою некоторое самостоятельное, однако имеет свою рефлексию в
себя, свою определенную существенность в некотором другом самостоятельном.

Реальная действительность теперь также имеет
{\em возможность} непосредственно
{\em в ней же самой}. Она содержит в себе момент
в-себе-бытия; но, как пока что лишь
{\em непосредственное} единство, она находится
{\em в одном} из определений формы и тем самым отлична,
как сущее, от в-себе-бытия или возможности.

2. Эта возможность, как в-себе-бытие {\em реальной}
действительности, сама есть {\em реальная возможность},
есть ближайшим образом {\em содержательное}
в-себе-бытие. — Формальная возможность есть рефлексия в себя лишь как
абстрактное тождество, заключающееся в том, что нечто не противоречит себе
внутри себя. Но поскольку входят в рассмотрение определений, обстоятельств,
условий некоторой мыслимой вещи, чтобы из этого познать ее возможность, то
уже больше не задерживаются на формальной возможности, а рассматривают ее
(вещи) реальную возможность.

Эта реальная возможность сама есть {\em непосредственное
существование}, но уже не потому, что возможность как таковая, как
формальный момент, есть непосредственно своя собственная противоположность,
некоторая нерефлектированная действительность; а именно потому, что она
есть {\em реальная} возможность, она сразу же имеет это
определение в себе самой. Реальная возможность некоторой мыслимой вещи есть
поэтому налично сущее многообразие относящихся к ней обстоятельств.

Следовательно, хотя это многообразие наличного бытия и есть как возможность,
так и действительность, все же их тождество есть пока что лишь
{\em содержание}, безразличное к этим определениям
формы; они составляют поэтому форму, {\em определенную}
в противоположность их тождеству. — Или, иначе говоря,
{\em непосредственная} реальная действительность именно
потому, что она непосредственна, определена в противоположность к ее
возможности; как эта определенная и тем самым рефлектированная
действительность, она есть {\em реальная возможность}.
Последняя есть теперь, правда, положенное {\em целое}
формы, но формы в ее определенности, а именно, действительности как
формальной или непосредственной, и равным образом возможности как
абстрактного в-себе-бытия. Эта действительность, составляющая возможность
некоторой мыслимой вещи, есть поэтому не своя
{\em собственная возможность}, а в-себе-бытие
некоторого {\em другого} действительного; сама она есть
такая действительность, которая должна быть снята, возможность как
{\em лишь} возможность.— Таким образом, реальная
возможность составляет {\em целокупность условий},
некоторую нерефлектированную в себя, рассеянную действительность, которая,
однако, определена так, что она есть в-себе-бытие некоторого другого и
должна возвратиться в себя.

То, что реально возможно, есть, следовательно, по своему
{\em в-себе-бытию} некоторое формальное тождественное,
которое по своему {\em простому} определению содержания
не противоречит себе; но также и по своим развитым и различенным
обстоятельствам и по всему тому, с чем оно находится в связи, оно, как
тождественное с собой, должно не противоречить себе. Но,
{\em во-вторых}, так как оно многообразно внутри себя и
находится с другим в многообразной связи, разность же переходит в себе
самой в противоположение, то оно есть нечто противоречивое. Если речь идет
о некоторой возможности и следует обнаружить ее противоречие, то нужно
только держаться многообразия, которое она заключает в себе как содержание
или как свое обусловленное существование, из чего легко отыскать ее
противоречие. — Но это не есть противоречие сравнения, а многообразное
существование {\em в самом себе} состоит в том, что оно
снимает себя и идет ко дну, и, стало быть, существенным образом имеет в
себе же самом то определение, что оно есть {\em лишь
некоторое возможное}. — Когда все условия некоторой мыслимой вещи полностью
имеются налицо, тогда она вступает в действительность; полнота условий есть
тотальность как тотальность содержания, и {\em сама
мыслимая вещь} есть это содержание, определенное так, что оно есть столь же
некоторое действительное, сколь и возможное. В сфере обусловленного
основания условия имеют форму —~именно основание или для-себя-сущую
рефлексию —~{\em вне себя}, и эта рефлексия соотносит
их так, что они становятся моментами мыслимой вещи, и порождает
{\em в них} существование. Здесь же, напротив,
непосредственная действительность определена к тому, чтобы быть условием,
не некоторою предполагающею рефлексией, а оказывается положенным, что сама
она есть возможность.

В снимающей себя реальной возможности снимается теперь нечто двоякое, ибо
она сама есть двоякое: и действительность и возможность. (1)
Действительность есть формальная действительность, или некоторое
существование, которое выступало как самостоятельное непосредственное
существование и становится через свое снятие рефлектированным бытием,
моментом некоторого другого, получая тем самым
{\em в-себе-бытие}. (2) Указанное существование было
также определено как {\em возможность} или как
{\em в-себе-бытие}, но некоторого другого.
Следовательно, когда оно себя снимает, то снимается также и это
в-себе-бытие и последнее переходит в
{\em действительность}. — Это движение снимающей самое
себя реальной возможности порождает, следовательно,
{\em те же самые уже имеющиеся налицо моменты}, но так,
что каждый вырастает из другого; поэтому оно в этом отрицании и есть не
{\em переход}, а {\em слияние с
самим собой}. — Согласно формальной возможности, если какое-нибудь нечто
было возможно, то в силу этого было возможно также и не
{\em оно само}, а его {\em другое}.
Реальная возможность уже не имеет противостоящим себе
{\em такого другого}, ибо она реальна, поскольку она
сама есть также и действительность. Следовательно, когда
{\em непосредственное существование} этой возможности,
круг условий, снимает себя, то она превращает себя в то
{\em в-себе-бытие}, которое сама она уже представляет
собою, а именно, как {\em в-себе-бытие} некоторого
другого. А так как и, обратно, ее момент в-себе-бытия этим вместе с тем
также и снимает себя, то она становится действительностью, следовательно,
моментом, которым она равным образом уже является сама. — Исчезает,
следовательно, то обстоятельство, что действительность была определена как
возможность или в-себе-бытие некоторого {\em другого}
и, наоборот, возможность была определена как такая действительность,
которая {\em не есть та действительность}, возможность
которой она есть.

3. {\em Отрицание} реальной возможности есть, стало
быть, {\em свое тождество} с собой; поскольку оно,
таким образом, в своем снимании есть отскок этого снимания внутрь самого
себя, постольку оно (это отрицание) есть {\em реальная
необходимость}.

То, что необходимо, {\em не} может быть
{\em иначе}; но то, что вообще
{\em возможно}, вполне может быть иным; ибо возможность
есть то в-себе-бытие, которое представляет собою только положенность и
потому есть по существу инобытие. Формальная возможность есть это тождество
как переход в безоговорочно иное; реальная же возможность ввиду того, что
она содержит в себе другой момент, действительность, уже сама есть
необходимость. Поэтому то, что реально возможно, не может уже быть другим,
чем оно само; при таких-то условиях и обстоятельствах не может
воспоследовать нечто другое. Реальная возможность и необходимость различны
поэтому лишь {\em кажущимся образом}; необходимость
есть то {\em тождество}, которое не
{\em становится} только теперь, а уже
{\em пред-положено} и лежит в основании. Реальная
необходимость есть поэтому {\em содержательное}
соотношение; ибо содержание есть то в-себе-сущее тождество, которое
безразлично к различиям формы.

Но эта необходимость вместе с тем {\em относительна}. —
А именно, она имеет некоторую {\em предпосылку}, с
которой она начинает, она имеет свой {\em исходный
пункт} в {\em случайном}. Реально действительное как
таковое есть именно {\em определенное} действительное и
имеет ближайшим образом свою {\em определенность}, как
{\em непосредственное} бытие, в том, что оно есть
многообразие существующих обстоятельств; но это непосредственное бытие, как
определенность, есть также и {\em отрицательное} себя,
есть в-себе-бытие или возможность; таким образом, оно есть реальная
возможность. Как это единство обоих моментов, она есть тотальность формы,
но {\em еще внешняя} {\em себе}
тотальность; она есть единство возможности и действительности таким
образом, что: 1)~многообразное существование есть
{\em непосредственным} или
{\em положительным образом} возможность, есть некоторое
возможное (тождественное с собой вообще) потому, что оно (существование)
есть некоторое действительное; 2)~поскольку эта возможность существования
положена, она определена как {\em лишь} возможность,
как непосредственное превращение действительности в свою противоположность,
или, иначе сказать, как {\em случайность}. Поэтому
указанная возможность, которая, будучи условием, заключает в себе
непосредственную действительность, есть лишь в-себе-бытие как возможность
некоторого {\em другого}. В силу того, что, как
показано, это инобытие снимает себя и эта положенность сама оказывается
положенной, реальная возможность становится, правда, необходимостью; однако
последняя тем самым начинает с вышеуказанного еще не рефлектированного в
себя единства возможного и действительного: это
{\em пред-полагание} и
{\em возвращающееся} {\em в себя
движение} еще раздельны, или, иначе сказать,
{\em необходимость еще не определила себя из самой себя
как случайность}.

Относительность реальной необходимости проявляется в
{\em содержании} таким образом, что оно есть пока что
лишь безразличное к форме тождество и поэтому отлично от нее и есть
некоторое {\em определенное}
{\em содержание} вообще. Реально необходимое есть
поэтому какая-либо ограниченная действительность, которая из-за этой
ограниченности и есть в другом отношении лишь
{\em случайное}.

Тем самым {\em реальная необходимость} есть на самом
деле {\em в себе} также и
{\em случайность}. — Это проявляется ближайшим образом
в том, что хотя реально необходимое и есть нечто необходимое
{\em по форме}, оно по содержанию все же есть нечто
ограниченное и вследствие этого содержания носит отчасти характер
случайного. Однако и в форме реальной необходимости также содержится
случайность; ибо, как оказалось, реальная возможность лишь
{\em в себе} есть необходимое, положена же она как
{\em инобытие} действительности и возможности по
отношению друг к другу. Реальная необходимость содержит поэтому в себе
случайность; она есть возвращение в себя из того беспокойного
{\em инобытия} действительности и возможности по
отношению друг к другу, но не есть возвращение из самой себя к себе.

Следовательно, {\em в себе} здесь имеется единство
необходимости и случайности; это единство должно быть названо
{\em абсолютной действительностью}.

\paragraph[С. \ Абсолютная необходимость]{С. \ Абсолютная необходимость}
Реальная необходимость есть
{\em определенная} необходимость; формальная
необходимость еще не заключает в себе никакого содержания и определенности.
{\em Определенность} необходимости состоит в том, что
она содержит в себе свое отрицание, случайность. Такой она получилась.

Но эта определенность {\em в ее первой простоте} есть
действительность; {\em определенная} необходимость есть
поэтому непосредственно {\em действительная
необходимость}. Эта действительность, {\em которая сама
как таковая необходима} потому именно, что она содержит в себе
необходимость как свое {\em в-себе-бытие}, есть
{\em абсолютная действительность}, действительность,
которая уже больше не может быть иной, чем она есть, ибо ее
{\em в-себе-бытие} есть не возможность, а сама
необходимость.

Но эта {\em действительность}, так как она положена
такой, что она есть {\em абсолютная}, т.~е.
{\em сама есть единство себя и возможности}, тем самым
есть лишь некоторое {\em пустое} определение, или,
иначе говоря, она есть {\em случайность}. — Эта
{\em пустота} ее определения делают ее
{\em голой возможностью}, чем-то таким, что в такой же
мере может быть также и {\em иным}, чем оно есть, и что
можно определить как возможное. Но сама эта возможность есть
{\em абсолютная возможность}; ибо она как раз есть
возможность быть определенной и как возможность и как действительность. Тем
самым, что она есть это безразличие к самой себе, она положена как пустое,
{\em случайное} определение.

Таким образом, в реальной необходимости случайность содержится не только
{\em в себе}, а последняя также и
{\em становится} в первой; но это
{\em становление}, как внешность, само есть лишь
{\em в-себе-бытие} этой необходимости, так как оно есть
лишь некоторая {\em непосредственная определяемость}.
Но оно не только таково, а есть {\em ее собственное}
становление, или, иначе сказать, то
{\em пред-положение}, которое она имела, есть ее
собственное полагание. Ибо, как реальная необходимость, она есть снятость
действительности и возможности, и наоборот; поскольку она есть это
{\em простое превращение} одного из этих моментов в
другой, она есть также их простое {\em положительное
единство}, так как каждый из них, как оказалось,
{\em сливается} в другом {\em лишь
с самим собой}. Но таким образом она есть
{\em действительность}, однако такая действительность,
которая имеет бытие лишь как это простое слияние формы с самой собой.
Отрицательное полагание ею этих моментов тем самым само есть
{\em пред-полагание}, или, иначе говоря, полагание
{\em ее самой}, {\em как снятой}
или полагание {\em непосредственности}.

Но именно этим указанная действительность определена как отрицательное; она
есть слияние с собой как выхождение из той действительности, которая была
реальной возможностью; таким образом, эта новая действительность возникает
лишь из своего в-себе-бытия, из {\em отрицания ее
самой}. — Тем самым она вместе с тем непосредственно определена как
{\em возможность}, как нечто
{\em опосредствованное} через ее отрицание. Однако эта
возможность том самым есть непосредственно не что иное, как
{\em тот процесс опосредствования}, в котором
в-себе-бытие (т.~е. она сама) и непосредственность, оба суть одинаковым
образом {\em положенность}. — Таким образом,
необходимость есть в такой же мере снятие этой положенности или полагание
{\em непосредственности} и
{\em в-себе-бытия}, в какой она тем самым есть
{\em процесс определения} этого снятия как
{\em положенности}. Поэтому
{\em она же сама} определяет себя как
{\em случайность}, она сама отталкивает себя от себя в
своем бытии, в этом самом отталкивании лишь возвратилась в себя, и в этом
возвращении как в своем бытии, оттолкнула себя от себя самой.

Таким образом, {\em форма} в своей реализации пронизала
собой все свои различия и сделала себя прозрачной и, как
{\em абсолютная необходимость}, есть лишь это простое
{\em тождество бытия с самим собой в своем отрицании}
или в {\em сущности}. — Само различие между
{\em содержанием} и формой также исчезло; ибо это
единство возможности в действительности и наоборот есть
{\em форма}, безразличная к себе самой в своей
определенности или в положенности —~та
{\em содержательная мыслимая вещь}, в которой форма
необходимости протекала внешним образом. Но, таким образом, она есть это
{\em рефлектированное} тождество обоих определений как
{\em безразличное} к ним, есть тем самым формальное
определение {\em в-себе-бытия} в противоположность
{\em положенности}, и эта возможность составляет
ограниченность того содержания, которым обладала реальная необходимость.
Растворение же этого различия есть абсолютная необходимость, содержание
которой составляет это пронизывающее себя в ней различие.

Абсолютная необходимость есть, следовательно, та истина, в которую
возвращаются действительность и возможность вообще, равно как и формальная
и реальная необходимость. — Абсолютная необходимость, как оказалось, есть
бытие, которое в своем отрицании, в сущности, соотносится с собой и есть
бытие. Она есть в такой же мере простая непосредственность или
{\em чистое бытие}, в какой и простая рефлексия в себя
или {\em чистая сущность}. Она состоит в том, что и то
и другое есть одно и то же. — Безоговорочно необходимое
{\em есть} лишь потому, что оно
{\em есть}; оно не имеет помимо этого ни условия, ни
основания. — Но оно есть также и чистая {\em сущность};
его {\em бытие} есть простая рефлексия в себя; оно
есть, {\em потому что} оно есть. Как рефлексия, оно
имеет основание и условие, но имеет основанием и условием лишь себя. Оно
есть в-себе-бытие, но его в-себе-бытие есть его непосредственность; его
возможность есть его действительность. — {\em Оно,
следовательно, есть, потому что оно есть}; как
{\em слияние} бытия с собой, оно есть сущность; но так
как это простое есть равным образом и непосредственная простота, то оно
есть {\em бытие}.

Абсолютная необходимость есть, таким образом,
{\em рефлексия или форма абсолютного}, единство бытия и
сущности, простая непосредственность, которая есть абсолютная
отрицательность. Поэтому имеющиеся в ней различия,
{\em с одной стороны}, суть не рефлективные
определения, а {\em сущее многообразие}, различенная
действительность, имеющая вид взаимного противоположения самостоятельных
иных по отношению друг к другу. {\em С}
{\em другой стороны}, так как их соотношение есть
абсолютное тождество, то она есть {\em абсолютное
превращение} их действительности в их возможность и их возможности в
действительность. — Абсолютная необходимость поэтому
{\em слепа}~\label{bkm:Ref474669669}\pageref{bkm:bm93}.
С одной стороны, различенные в ней моменты, определенные как
действительность и как возможность, имеют вид
{\em рефлексии-в-себя} как
{\em бытия}; они поэтому оба имеют бытие как
{\em свободные действительности}, из которых
{\em ни одна не светится в другой}, ни одна не хочет
обнаруживать в себе самой следа своего соотношения с другою; будучи
обоснована внутри себя, каждая из них есть нечто необходимое в себе самой.
Необходимость, как {\em сущность}, заперта в этом
{\em бытии}; взаимное соприкосновений этих
действительностей представляется поэтому пустой внешностью;
действительность {\em одного в другом} есть
{\em лишь} возможность,
{\em случайность}. Ибо бытие положено как абсолютно
необходимое, как опосредствование с собой, которое есть абсолютное
отрицание опосредствования через другое, или как бытие, которое
тождественно лишь с бытием; поэтому некоторое
{\em другое}, имеющее действительность в
{\em бытии}, определено как безоговорочно
{\em лишь возможное}, как пустая положенность.

Но [с другой стороны] эта {\em случайность} есть скорее
абсолютная необходимость; она есть {\em сущность} тех
свободных, необходимых в себе действительностей. Эта сущность
{\em боится света}, так как в сказанных
действительностях нет {\em свечения}, нет отражения
(Reflex), так как они обоснованы исключительно лишь внутри себя, отделаны
сами по себе, являют себя лишь {\em самим себе}, — так
как они суть лишь {\em бытие}. — Но их
{\em сущность} прорвется в них и откроет, что такое
{\em она} и что такое {\em они}.
{\em Простота} их бытия, их самодовления, есть
абсолютная отрицательность; это —~{\em свобода} их
непосредственности, лишенной всякого свечения. Это отрицательное
прорывается в них, так как благодаря этой своей сущности бытие есть
противоречие с самим собой, а именно, противоречие этому бытию в форме
бытия, следовательно, как такое {\em отрицание} этих
действительностей, которое {\em абсолютно разнится} от
их бытия, как их {\em ничто}, как столь же
{\em свободное} в отношении их
{\em инобытие}, сколь свободно их бытие. — Тем не менее
наличие в них этого инобытия не могло не быть признано. В своей покоящейся
на себе структуре они безразличны к форме, суть некоторое
{\em содержание} и тем самым
{\em различенные} действительности и некоторое
{\em определенное} содержание. Последнее есть
{\em клеймо}, которое наложила на них необходимость,
когда она (которая есть в своем {\em определении}
абсолютное возвращение в себя самое) свободно отпустила их как абсолютно
действительные, — клеймо, на которое она ссылается, как на свидетеля своего
права, и по которому они схватываются и уничтожаются. Это обнаружение того,
что {\em определенность} представляет собою поистине (а
именно обнаружение того обстоятельства, что она представляет собою
отрицательное соотношение с самой собою), есть
{\em слепое} уничтожение в инобытии; пробивающееся
здесь {\em свечение} или
{\em рефлексия} выступает в
{\em сущих} как {\em становление}
или {\em переход} бытия в ничто. Но и обратно,
{\em бытие} есть также и
{\em сущность}, и {\em становление}
есть {\em рефлексия} или
{\em свечение}. Таким образом, внешность есть их
собственная внутренность, и их соотношение есть абсолютное тождество, а
{\em переход} действительного в возможное, бытия в
ничто есть {\em слияние с самим собой}; случайность
есть абсолютная необходимость, она сама есть пред-полагание тех первых
абсолютных действительностей.

Это {\em тождество бытия} в его отрицании
{\em с самим собой} есть
{\em субстанция}. Оно есть это единство как в его
{\em отрицании} или как в
{\em случайности}; таким образом, оно есть
{\em субстанция как отношение к себе самому}.
{\em Слепой} переход необходимости есть скорее
{\em собственное развертывание} абсолютного, его
движение внутри себя, так что абсолютное в своем отчуждении скорее
показывает себя само.

