\clearpage\setcounter{page}{1}\subsection[Третий
отделДействительность]{\textstyleHeadiiFirstLine{Третий отдел}\newline
Действительность}
\hypertarget{Toc478978758}{}Действительность есть
{\em единство сущности и существования}; в ней имеет
свою истину {\em лишенная образа сущность} и
{\em лишенное устойчивости} явление, или, иначе
сказать, неопределенное устойчивое наличие и не имеющее упора многообразие.
{\em Существование} есть, правда, происшедшая из
основания непосредственность, но оно еще не положило в нем (в
существовании) формы; когда оно себя определяет и формирует, оно есть
{\em явление}, а когда это устойчивое наличие,
определенное лишь как рефлексия в другое, развивается далее, превращаясь в
рефлексию внутрь себя, оно становится {\em двумя
мирами}, двумя {\em тотальностями содержания}, из
которых одна определена как {\em рефлектированная в
себя}, а другая —~как {\em рефлектированная в другое}.
Существенное же отношение представляет собой их
{\em формальное соотношение}, завершением которого
служит {\em отношение внутреннего и внешнего},
заключающееся в том, что {\em содержание} обоих есть
лишь единая {\em тождественная основа} и точно так же
—~лишь единое {\em тождество формы}. — Благодаря тому,
что получилось \ также и это тождество в отношении формы, формальное
определение их разности оказывается снятым, и
{\em положено}, что они суть
{\em единая} абсолютная тотальность.

Это единство внутреннего и внешнего есть {\em абсолютная
действительность}. Но эта действительность есть
{\em ближайшим образом абсолютное} как таковое,
поскольку она положена как такое единство, в котором форма сняла себя и
обратила себя в {\em пустое или внешнее различие} между
внешним и внутренним. {\em Рефлексия} ведет себя по
отношению к этому абсолютному как {\em внешняя}
рефлексия, которая скорее лишь рассматривает его, чем есть его собственное
движение. Поскольку же она по существу есть это движение, она выступает как
его отрицательное возвращение в себя.

{\em Во-вторых},
{\em действительность} в собственном смысле.
{\em Действительность},
{\em возможность} и
{\em необходимость} составляют
{\em формальные моменты} абсолютного или его рефлексию.

{\em В-третьих}, единство абсолютного и его рефлексии
есть {\em абсолютное отношение} или, вернее, абсолютное
как отношение к себе самому, — {\em субстанция}.

\subsubsection[Первая главаАбсолютное]{\textstyleHeadiiiFirstLine{Первая
глава}\newline
Абсолютное}
\hypertarget{Toc478978759}{}Простое сплошное тождество абсолютного
неопределенно, или, вернее, в этом тождестве разрешилась всякая
определенность {\em сущности} и
{\em существования} или как
{\em бытия} вообще, так и
{\em рефлексии}. Постольку
{\em процесс определения} того,
{\em что есть абсолютное}, оказывается отрицательным, и
само абсолютное выступает лишь как отрицание всех предикатов и как пустота.
Но так как оно есть также и полагание всех предикатов, то оно выступает как
формальнейшее противоречие. Поскольку то подвергание отрицанию и это
полагание принадлежат к области {\em внешней
рефлексии}, постольку перед нами некая формальная несистематическая
диалектика, которая без особого труда там и сям подбирает разнообразные
определения и с такой же легкостью, с одной стороны, показывает их
конечность и голую относительность, а, с другой стороны (так как абсолютное
предносится ей как тотальность), высказывает также и присущность ему всех
определений, не будучи в состоянии возвести эти полагания и отрицания в
истинное единство. — Между тем нужно изобразить, что такое абсолютное;
однако это изображение не может быть ни процессом определения, ни внешней
рефлексией, посредством которой возникали бы определения этого абсолютного,
а оно есть {\em развертывание} (Auslegung) и притом
{\em собственное} развертывание абсолютного и лишь
{\em показывание того, что оно есть}.

\paragraph[А. \ Развертывание абсолютного]{А. \ Развертывание абсолютного}
\hypertarget{Toc478978760}{}Абсолютное не есть ни только
{\em бытие}, ни также только
{\em сущность}. Бытие есть первая нерефлектированная
непосредственность, а сущность —~рефлектированная непосредственность;
каждое из них есть, далее, тотальность в себе самом, но определенная
тотальность. В сущности бытие выступает как
{\em существование}, и соотношение сущности и бытия
доразвилось до отношения {\em внутреннего} и
{\em внешнего}. {\em Внутреннее}
есть {\em сущность}, однако, как такая
{\em тотальность}, которая имеет существенно то
определение, что она {\em соотнесена с бытием} и есть
непосредственно {\em бытие}.
{\em Внешнее} есть {\em бытие}, но
с тем существенным определением, что оно
{\em соотнесено с рефлексией} и непосредственно
представляет собою столь же безотносительное тождество с сущностью. Само
абсолютное есть абсолютное единство обоих; оно есть то, что составляет
вообще {\em основание} существенного отношения, которое
как отношение только еще не возвратилось в это свое тождество и основание
которого еще не {\em положено}.

Из этого вытекает, что определение абсолютного состоит в том, что это
абсолютное есть {\em абсолютная форма}, но вместе с тем
не как такое тождество, моменты которого суть лишь простые определенности,
а как такое тождество, каждый момент которого в себе самом есть
{\em тотальность} и тем самым, как безразличный к
форме, представляет собою полное {\em содержание}
целого. Но и обратно, абсолютное есть абсолютное содержание таким образом,
что содержание, которое как таковое есть безразличное многообразие,
заключает в себе отрицательное соотношение формы, в силу чего его
многообразие есть лишь {\em одно} сплошное тождество.

Следовательно, тождество абсолютного есть абсолютное тождество вследствие
того, что каждая из его частей сама есть целое, или, иначе сказать, каждая
определенность есть тотальность, т.~е., что определенность вообще стала
безоговорочно прозрачной видимостью, некоторым
{\em исчезнувшим в своей положенности} различием.
{\em Сущность, существование, в-себе-сущий мир, целое,
части, сила} —~все эти рефлектированные определения кажутся представлению в
себе и для себя значимым, истинным бытием; абсолютное же есть по отношению
к ним основание, в котором они потонули. — Так как в абсолютном форма есть
лишь простое тождество с собой, то абсолютное не
{\em определяет} себя; ибо определение есть различие
формы, которое ближайшим образом и признается за таковое. Но так как
абсолютное вместе с тем содержит в себе всякое различие и определение формы
вообще, или, иначе сказать, так как оно само есть абсолютная форма и
рефлексия, то в нем должна выступать также и
{\em разность содержания}. Но само абсолютное есть
{\em абсолютное тождество}; это есть его
{\em определение}, так как всякое многообразие
в-себе-сущего и являющегося миров, или, иначе сказать, внутренней и внешней
тотальности в нем снято. — В нем самом нет никакого
{\em становления}, ибо оно не есть бытие; оно не есть
также и {\em рефлектирующий}, себя процесс определения,
ибо оно не есть определяющая себя лишь внутри себя сущность; оно также не
есть {\em проявление себя во вне}, ибо оно есть
тождество внутреннего и внешнего. — Но таким образом движение рефлексии
{\em противостоит} его абсолютному тождеству. Это
движение снято в последнем, и таким образом это движение есть лишь его
{\em внутреннее}; но тем самым оно
{\em внешне} ему. — Движение рефлексии поэтому состоит
ближайшим образом лишь в том, что оно упраздняет в абсолютном свое
действие. Оно есть {\em лежащее позади} абсолютного
потустороннее, состоящее из многообразных различий и определений и их
движения; оно поэтому есть, правда, вбирание их, но вместе с тем и их
гибель; таким образом, оно есть то {\em отрицательное
развертывание} абсолютного, которое было упомянуто выше. — В своем истинном
изображении это развертывание есть полученное доселе всё целое логического
движения сферы {\em бытия и сущности}, содержание
которых не было нахватано извне как нечто данное и случайное, а равно и не
было погружено внешней ему рефлексией в бездну абсолютного, но определило
себя в нем в силу своей внутренней необходимости и (как собственное
{\em становление} бытия и как
{\em рефлексия} сущности) возвратилось в абсолютное как
в свое основание.

Но само это развертывание имеет вместе с тем и
{\em положительную} сторону, поскольку, именно,
конечное тем, что оно идет ко дну, погружается в основание, доказывает свою
природу, состоящую в том, что оно соотнесено с абсолютным, или, иначе
сказать, содержит абсолютное в себе самом. Но эта сторона есть не столько
положительное развертывание самого абсолютного, сколько, скорее,
развертывание {\em определений}, показывающее, что они
имеют абсолютное не только своей бездной, но также и своим
{\em основанием}, или, иначе сказать, что то, что им,
т.~е. видимости, сообщает устойчивое наличие, есть само
{\em абсолютное}. — Видимость есть не
{\em ничто}, а она есть рефлексия,
{\em соотношение} с абсолютным; или, иначе сказать, она
{\em есть} видимость, поскольку
{\em в ней светится абсолютное}. Таким образом, это
положительное развертывание еще задерживает конечное перед его
исчезновением и рассматривает его как некоторое выражение и отображение
абсолютного. Но прозрачность конечного, через которое проглядывает лишь
абсолютное, кончается полным исчезновением; ибо нет ничего в конечном, что
могло бы сохранить для него какое-нибудь отличие от абсолютного; конечное
есть среда, поглощаемая тем, что светится через нее.

Это положительное развертывание абсолютного само есть поэтому лишь некоторое
излучение видимости; ибо то истинно положительное, которое содержат в себе
оно и развертываемое содержание, есть само абсолютное. Все встречающиеся
дальнейшие определения, т.~е. та форма, в которой абсолютное светится,
представляет собою нечто лишенное всякого значения, нечто такое, что
подбирается развертыванием {\em извне} и в чем оно
приобретает {\em некоторое начало} для своего делания.
Такого рода определение имеет в абсолютном не свое начало, а только
{\em свой конец}. Поэтому рассматриваемое развертывание
есть, правда, абсолютное делание благодаря своему
{\em соотношению} с абсолютным, в которое оно
{\em возвращается}, но не есть абсолютное делание по
своему {\em исходному пункту}, который есть внешнее
абсолютному определение.

На самом же деле развертывание абсолютного есть его
{\em собственное} дело, и дело,
{\em начинающееся у него самого}, равно как и
{\em приходящее к нему самому}. Абсолютное, взятое лишь
как абсолютное тождество, оказывается {\em определенно}
абсолютным, а именно определенным как
{\em тождественное}; как таковое оно
{\em положено} рефлексией против противоположения и
многообразия; или, иначе сказать, оно есть лишь
{\em отрицательное} рефлексии и вообще процесса
определения. — Поэтому не только указанное развертывание абсолютного есть
нечто несовершенное, но и также само это
{\em абсолютное}, к которому лишь
{\em приходят}. Или, иначе говоря, то абсолютное,
которое есть лишь {\em абсолютное тождество}, есть лишь
{\em абсолютное некоторой внешней рефлексии}. Оно есть
поэтому не абсолютно-абсолютное, а абсолютное в некоторой определенности
или, другими словами, оно есть {\em атрибут}.

Но абсолютное есть атрибут не только потому, что оно есть
{\em предмет} некоторой внешней рефлексии и, стало
быть, нечто определенное через последнюю. — Или, иначе сказать, рефлексия
не только {\em внешня} ему, а она,
{\em непосредственно} потому, что она ему
{\em внешня}, также и
{\em внутрення} ему. Абсолютное есть абсолютное лишь
потому, что оно есть не абстрактное тождество, а тождество бытия и сущности
или внутреннего и внешнего. Следовательно, сама абсолютная форма и
заставляет его светиться видимостью внутри себя и определяет его в атрибут.

\paragraph[В. \ Абсолютный атрибут]{В. \ Абсолютный атрибут}
\hypertarget{Toc478978761}{}Выражение
«{\em абсолютно-абсолютное}», которое мы употребили
выше, обозначает абсолютное, {\em возвратившееся в
своей форме} в себя, или, иначе сказать, такое абсолютное, форма которого
одинакова с его содержанием. Атрибут есть лишь
{\em относительно абсолютное}, некоторая связь, не
означающая ничего другого, кроме абсолютного в некотором
{\em определении формы}. А именно, форма сначала, до ее
завершенного развертывания, пока что {\em только
внутрення}, или, что то же самое, {\em только внешня},
и вообще есть сначала {\em определенная} \ форма или
отрицание вообще. Но так как она вместе с тем есть форма абсолютного, то
атрибут составляет все содержание абсолютного; он есть та тотальность,
которая раньше представлялась некоторым {\em миром} или
одной из {\em сторон существенного отношения}, каждая
из которых сама есть целое. Но рассмотренные выше два мира, являющийся и
в-себе-и-для-себя-сущий, должны были каждый в своей сущности быть
{\em противоположными} друг другу. Одна сторона
существенного отношения была, правда, одинакова с другой, целое оказалось
тем же самым, что и части, проявление силы во вне тем же самым содержанием,
что и сама сила, и вообще внешнее тем же самым, что и внутреннее. Но вместе
с тем каждая из этих сторон должна была еще иметь свое собственное
{\em непосредственное} устойчивое наличие —~одна
сторона как сущая, а другая как рефлектированная непосредственность.
Напротив, в абсолютном эти различенные непосредственности низведены на
степень видимости, и та {\em тотальность}, которую
представляет собой атрибут, {\em положена как его
истинное и единственное устойчивое наличие; определение} же, в котором он
есть, положено как нечто несущественное.

Абсолютное есть атрибут потому, что оно, как простое абсолютное тождество,
имеет бытие в определении тождества; к определению вообще можно теперь
присоединить другие определения, например, также и то определение, что
существуют {\em многие} атрибуты. Но так как абсолютное
тождество имеет лишь то значение, что не только все определения сняты, но
что оно также есть та рефлексия, которая сняла самое себя, то в нем все
определения {\em положены как снятые}. Или, иначе
сказать, тотальность положена как абсолютная тотальность, или, еще иначе,
атрибут имеет абсолютное своим содержанием и устойчивым наличием; его
определение формы, через которое он есть атрибут, также поэтому положено
непосредственно как голая видимость, — отрицательное как отрицательное.
Положительная видимость, которую развертывание сообщает себе через атрибут,
беря конечное в его ограниченности не как некоторое
в-себе-и-для-себя-сущее, а растворяя его устойчивое наличие в абсолютное и
растягивая его до атрибута, снимает само то обстоятельство, что он есть
атрибут; это развертывание погружает его и свое различающее делание
{\em в простое абсолютное}.

Но рефлексия, возвращаясь таким образом из своего различения лишь к
{\em тождеству} абсолютного, не вышла вместе с тем из
своей внешности и не пришла к истинному абсолютному. Она достигла лишь
неопределенного, абстрактного тождества, т.~е. того, которое пребывает в
{\em определенности} тождества. — Или, иными словами,
когда рефлексия, как {\em внутренняя} форма, определяет
абсолютное в атрибут, то этот процесс определения еще разнится от
внешности; внутреннее определение не пронизывает абсолютного; его
проявление во вне состоит в том, что оно, как некоторая голая положенность,
исчезает в абсолютном.

Следовательно, та форма, — возьмем ли мы ее как внешнюю или как внутреннюю,
— через которую абсолютное есть атрибут, вместе с тем положена так, что она
есть некоторое в самом себе ничтожное, внешняя видимость, или голый
{\em вид и способ}.

\paragraph[С. \ Модус абсолютного]{С. \ Модус абсолютного}
\hypertarget{Toc478978762}{}Атрибут есть,
{\em во-первых}, абсолютное в простом
{\em тождестве} с собой.
{\em Во-вторых}, он есть
{\em отрицание}, и последнее,
{\em как} отрицание, есть формальная рефлексия в себя.
Эти две стороны образуют ближайшим образом два
{\em крайних термина} атрибута,
{\em средний термин} которых образует он сам, так как
он есть и абсолютное и определенность. — Второй из этих крайних терминов
есть {\em отрицательное} как
{\em отрицательное}, есть
{\em внешняя} абсолютному рефлексия. — Или, иначе
говоря, поскольку он берется как {\em внутреннее}
абсолютного, и полагание им себя как модуса есть его
{\em собственное} определение, то модус есть
вне-себя-бытие абсолютного, потеря себя в изменчивости и случайности бытия,
его совершившийся переход в противоположное без
{\em возвращения} в себя, —лишенное тотальности
многообразие форм и определений содержания.

Но модус, {\em внешность} абсолютного, есть не только
это, но и внешность, положенная как внешность, голый
{\em вид \ и способ}, тем самым видимость как видимость
или {\em рефлексия формы в себя} и, стало быть, то
{\em тождество} с собой,
{\em которое есть абсолютное}. Следовательно, на самом
деле только в модусе абсолютное впервые положено как абсолютное тождество;
оно есть то, что оно {\em есть}, а именно, тождество с
собой, лишь как соотносящаяся с собой отрицательность, как
{\em излучение видимости}, положенное
{\em как излучение видимости}.

Поэтому, поскольку {\em развертывание} абсолютного
начинает с его абсолютного тождества и переходит к атрибуту, а от
последнего к модусу, то оно тем самым прошло полностью свои моменты. Но,
{\em во-первых}, оно этим есть не только некоторое
отрицательное отношение к этим определениям, а это
{\em его дело} есть само то
{\em рефлектирующее движение}, в качестве которого
{\em абсолютное единственно только и есть поистине
абсолютное тождество}. — {\em Во-вторых}, оно при этом
имеет дело не только с {\em внешним}, и модус не есть
только наиболее внешняя внешность, а так как он есть видимость как
видимость, то он есть возвращение в себя, та сама себя растворяющая
рефлексия, в качестве которой абсолютное есть абсолютное бытие. —
{\em В-третьих}, развертывающая рефлексия кажется
начинающей со своих собственных определений и с внешнего, кажется
подбирающей модусы или определения атрибута, как
{\em преднайденные} где-то вне абсолютного, и ее дело
кажется состоящим в том, что она лишь приводит их обратно в неразличенное
тождество. На самом же деле она имеет в самом абсолютном ту определенность,
с которой она начинает. Ибо абсолютное, как
{\em первое} неразличенное тождество, само есть лишь
{\em определенное абсолютное} или атрибут, так как оно
есть неподвижное, еще не рефлектированное абсолютное. Эта
{\em определенность}, так как она есть определенность,
принадлежит к рефлектирующему движению; лишь через последнее абсолютное
определенно как {\em первое тождественное}, и точно так
же лишь через него оно обладает абсолютной формой и представляет собою не
просто нечто {\em равное} самому себе, а нечто такое,
что само {\em полагает} себя
{\em равным} самому себе.

Истинное значение модуса заключается поэтому в том, что он есть
рефлектирующее, собственное движение абсолютного, некоторый
{\em процесс определения}, но не такой, через который
абсолютное становилось бы чем-то {\em другим}, а
процесс определения только того, что оно уже
{\em есть}; прозрачная внешность, которая есть
{\em показывание} его самого; некоторое движение из
себя во вне, но таким образом, что это во-вне направленное бытие есть в
такой же мере и сама внутренность, и тем самым представляет собою вместе с
тем такое полагание, которое есть все только положенность, но абсолютное
бытие.

Поэтому если задают вопрос, в чем состоит
{\em содержание} развертывания (что именно нам
показывает абсолютное?), то нужно сказать, что различие формы и содержания
и без того растворено в абсолютном. Или, иначе сказать, содержание
абсолютного и состоит как раз в том, что оно
{\em проявляет себя}. Абсолютное есть абсолютная форма,
которая, как раздвоение себя, безоговорочно тождественна с собой, —
отрицательное {\em как} отрицательное, или, иначе
сказать, такое отрицательное, которое сливается с собой и лишь таким путем
есть абсолютное тождество с собой, которое также и
{\em безразлично к своим различиям} или есть абсолютное
{\em содержание}; содержание есть поэтому лишь самое
это развертывание.

Абсолютное, как это носящее само себя движение развертывания, как
{\em вид} и {\em способ}, который
есть его абсолютное тождество с самим собой, есть проявление во вне не
некоторого внутреннего и не по отношению к некоторому другому, а имеет
бытие лишь как абсолютное проявление себя для самого себя; оно, таким
образом, есть {\em действительность}.

{\centering
\hypertarget{Toc478978763}{}\textstyleHeadviiFirstLine{Примечание}\newline
~\newline
[Философия Спинозы и Лейбница]
\par}

Понятию абсолютного и отношению рефлексии к последнему, как оно здесь
изображено, соответствует {\em понятие спинозовской
субстанции}. {\em Спинозизм} есть неудовлетворительная
философия потому, что {\em рефлексия} и ее
многообразный процесс определения есть в нем некоторое
{\em внешнее мышление}. — Субстанция есть в этой
системе {\em единая субстанция}, единая нераздельная
тотальность; нет ни одной определенности, которая не содержалась бы и не
была бы растворена в этом абсолютном; и немалое значение имеет то
обстоятельство, что все то, что естественному представлению или
определяющему рассудку кажется самостоятельным и предносится ему как
таковое, целиком низведено в указанном необходимом понятии на степень голой
{\em положенности}. —
{\em «Определенность есть отрицание»} —~таков
абсолютный принцип спинозовской
философии~\textstyleEndnodeLink{(\ref{bkm:Ref474669620}}\label{bkm:bm91}\textstyleEndnodeLink{)};
это истинное и простое усмотрение служит обоснованием абсолютного единства
субстанции. Но Спиноза останавливается на
{\em отрицании} как
{\em определенности} или качестве; он не идет дальше,
не переходит к познанию его как абсолютного, т.~е.
{\em себя самого отрицающего отрицания}; тем самым
{\em спинозовская субстанция не содержит в себе сама
абсолютной формы}; и познание этой субстанции не есть имманентное познание.
Правда, субстанция есть абсолютное единство
{\em мышления} и бытия или протяжения; она,
следовательно, содержит в себе само мышление, но лишь в его
{\em единстве} с протяжением, т.~е. содержит его не как
{\em отделяющее} себя от протяжения, тем самым вообще
не как процесс определения и формирования, а также не как возвращающееся в
себя и начинающееся из самого себя движение. Этой субстанции отчасти
недостает вследствие этого принцип {\em личности}
—~недостаток, который главным образом вызывал возмущение против
спинозовской системы, — отчасти же это познание есть внешняя рефлексия,
которая постигает и выводит то, что выступает как конечное, —
определенность атрибута и модус, равно как и вообще самоё себя, — не из
субстанции, а действует, как внешний рассудок, подбирает определения, как
{\em данные}, и {\em сводит} их к
абсолютному, вместо того, чтобы начинать с абсолютного.

Понятия, которые {\em Спиноза} дает о субстанции, суть
понятия {\em причины самой себя}, — что она есть то,
{\em сущность чего заключает в себе существование}, —
что понятие абсолютного {\em не нуждается в понятии
чего-либо другого}, из которого оно должно было бы быть образовано. Эти
понятия, как бы глубоки и правильны они ни были, представляют собою
дефиниции, {\em непосредственно} принимаемые в науке с
самого начала. Математика и другие подчиненные науки необходимо должны
начинать некоторыми {\em предпосылками}, составляющими
их стихию и положительную основу. Но абсолютное не может быть некоторым
первым, непосредственным, а есть существенно {\em его
результат}.

Вслед за дефиницией абсолютного у Спинозы
{\em выступает} далее
{\em дефиниция атрибута}, и последний определяется им,
как то, каким образом {\em рассудок постигает сущность
этого
абсолютного}~\textstyleEndnodeLink{(\ref{bkm:Ref474669634}}\label{bkm:bm92}\textstyleEndnodeLink{)}.
Помимо того, что {\em рассудок} принимается Спинозой
как нечто позднейшее но своей природе, чем атрибут (ибо Спиноза определяет
его как модус), этой дефиницией атрибут—определение, как определение
абсолютного, — делается у Спинозы {\em зависимым от
чего-то другого}, от рассудка, и это другое выступает по отношению к
субстанции внешним и непосредственным образом.

Атрибуты Спиноза определяет, далее, как
{\em бесконечные} и притом бесконечные также и в смысле
{\em бесконечного множества}. Правда, в дальнейшем мы
встречаем лишь {\em два} атрибута
—~{\em мышление и протяжение}, — и нам не показывают,
каким образом эта бесконечная множественность необходимо сводится лишь к
противоположности и притом к этой \ определенной противоположности двух
атрибутов, мышления и протяжения. — Эти два атрибута взяты поэтому
{\em эмпирически}. Мышление и бытие представляют
абсолютное в некоторой детерминации; само же абсолютное есть их абсолютное
единство, так что они суть лишь несущественные формы, порядок вещей есть
тот же, что и порядок представлений или мыслей, и единое абсолютное только
внешней рефлексией, некоторым модусом рассматривается, под этими двумя
определениями, то как тотальность представлений, то как тотальность вещей и
их изменений. Подобно тому, как эта внешняя рефлексия делает указанное
различие, точно так же она и погружает это различив обратно в абсолютное
тождество. Но все это движение совершается вне абсолютного. Правда, само
это последнее есть также и {\em мышление}, и постольку
это движение происходит лишь в абсолютном; но, как мы уже заметили, оно
имеется в абсолютном лишь как единство с протяжением; и тем самым не как то
движение, которое до существу есть также и момент противоположения. —
Спиноза предъявляет к мышлению возвышенное требование
{\em рассматривать} все {\em под
образом вечности}, sub specie aeterni, т.~е. как оно есть в абсолютном. Но
в этом абсолютном, которое есть лишь неподвижное тождество, атрибут, как и
модус, находится лишь как {\em исчезающий}, а не как
{\em становящийся}, так что тем самым и указанное
исчезание берет свое положительное начало лишь извне.

{\em Третье} определение,
{\em модус}, есть у Спинозы
{\em состояние} (affectio) субстанции, определенная
определенность, то, что {\em есть в некотором другом} и
постигается {\em через это другое}. Атрибуты имеют,
собственно говоря, своим определением: лишь неопределенную разность; каждый
{\em должен} выражать тотальность субстанции и
постигаться из себя самого; но поскольку он есть абсолютное как
определенное абсолютное, он содержит в себе инобытие и не может быть
постигнут только {\em из себя самого}. Поэтому
определение атрибута положено впервые, собственно говоря, только в модусе.
Это третье, далее, остается голым модусом; с одной стороны, последний есть
непосредственно {\em данное}, а, с другой стороны, его
ничтожность не познается, как рефлексия в себя. — Спинозовское
развертывание абсолютного поэтому есть, правда, постольку
{\em полное}, поскольку оно начинает с абсолютного,
затем переходит к атрибуту и кончает модусом; но эти три определения лишь
перечисляются {\em одно за другим} без внутренней
последовательности развития, и третье определение не есть отрицание
{\em как} отрицание, не есть отрицательно соотносящееся
с собою отрицание, вследствие чего оно {\em в себе
самом} было бы возвращением в первое тождество, а это последнее было бы
истинным тождеством. Здесь поэтому недостает необходимости движения
абсолютного к несущественности, равно как и ее разрешения самой по себе в
тождество; или, иначе говоря, недостает становления как тождества, так и
его определений.

Подобным же образом в восточном представлении об
{\em эманации} абсолютное есть озаряющий сам себя свет.
Однако он не только озаряет себя, но и {\em истекает}
из себя. Его истечения суть {\em удаления} от его
непомутненной ясности; дальнейшие порождения менее совершенны, чем
предшествующие, из которых они возникают. Истечение понимается лишь как
некоторое {\em происшествие}, становление —~лишь как
нарастающая утрата. Таким образом, бытие все больше и больше помутняется, и
ночь, отрицательное, есть последнее в линии эманаций, которое уже не
возвращается в первый свет.

Отсутствие {\em рефлексии} {\em в
себя}, характеризующее спинозовское развертывание абсолютного, равно как и
учение об эманации, восполнено в понятии
{\em лейбницевской монады}. — Односторонности одного
философского принципа обыкновенно противопоставляет себя противоположная
односторонность, и, как во всем, оказывается наличной тотальность по
\ крайней мере, как некоторая {\em разбросанная
полнота}. — {\em Монада} есть лишь
{\em одно}, рефлектированное в себя отрицательное; она
есть тотальность содержания мира; разное многообразное в ней не только
исчезло, но и {\em сохранено} отрицательным образом
(спинозовская субстанция есть единство всякого содержания; но это
многообразное содержание мира наличествует как таковое не в ней, а во
внешней для нее рефлексии). Поэтому монада есть по существу
{\em представляющая}; но в ней, хотя она и конечна, нет
никакой {\em пассивности}, а изменения и определения в
ней суть проявление ее в ней же самой. Она есть
{\em энтелехия}; открывание себя есть ее собственное
дело. — При этом монада также и {\em определена},
{\em отлична от других}; определенность получает место
в особенном содержании и в способе проявления. Поэтому монада есть
тотальность только {\em в себе}, по своей
{\em субстанции, а не в своем проявлении}. Это
{\em ограничение} монады необходимо имеет место не в
{\em полагающей самое себя} или
{\em представляющей} монаде, а в ее
{\em в-себе-бытии}, или, иначе говоря, есть абсолютная
граница, {\em предопределение}, \ которое положено
другим существом. Далее, так как ограниченные существа имеют бытие лишь как
соотносящееся с другими ограниченными существам, монада же есть вместе с
тем замкнутое в себе абсолютное, то {\em гармония} этих
ограничений, а именно, соотношение монад друг с другом, имеет место вне их,
и равным образом предустановлена другим существом или
{\em в себе}.

Ясно, что принцип {\em рефлексии в себя}, составляющий
основное определение {\em монады}, устраняет, правда,
инобытие и вообще воздействие извне, и изменения монады суть ее собственное
полагание, но что, с другой стороны, пассивность, определяемость, через
другое превращается лишь в абсолютный предел, в предел
{\em в-себе-бытия}. {\em Лейбниц}
приписывает монадам {\em известную} завершенность
внутри себя, {\em некоторого рода} самостоятельность;
они суть {\em сотворенные} существа. — При ближайшем
рассмотрении их ограниченности оказывается, что, согласно даваемому
Лейбницем изображению, присущее им проявление самих себя есть
{\em тотальность формы}. В высшей степени важно то
понятие, по которому изменения монады представляются как действия, не
имеющие в себе пассивности, как {\em проявления} ее
самой, и выдвигается, как существенный, принцип рефлексии в себя или
{\em индивидуации}. Необходимым допущенном является
затем допущение конечности, как состоящей в том, что содержание или
{\em субстанция} оказывается
{\em отличной от формы} и что, далее, субстанция
выступает как ограниченная, форма как бесконечная. Но теперь следовало бы в
понятии {\em абсолютной монады} выявить не только
абсолютное единство формы и содержания, но также и природу рефлексии,
заключающуюся в том, что последняя, как соотносящаяся с самой собой
отрицательность, отталкивает себя от себя, вследствие чего абсолютная
монада есть полагающая и творящая. Правда, в лейбницевской системе \ мы
находим также и тот дальнейший вывод, что {\em бог}
есть {\em источник существования и сущности монад},
т.~е. что указанные абсолютные пределы во в-себе-бытии монад суть не
в-себе-и-для-себя-сущие, а исчезают в абсолютном. Но в этих определениях
выступают лишь обычные представления, которые Лейбниц оставляет без
философского развития и не возвышает до уровня спекулятивных понятий. Таким
образом, принцип индивидуации не получает своего более глубокого развития;
понятия о различении разных конечных монад и об их отношении к их
абсолютному не проистекают из самого этого существа или проистекают не
абсолютным образом, а принадлежат резонирующей, догматической рефлексии и
поэтому недозрели до внутренней связности.

