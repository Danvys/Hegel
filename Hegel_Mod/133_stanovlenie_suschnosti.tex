\section[А. Абсолютная индифференция (неразличенность)]{А. Абсолютная индифференция (неразличенность)}

Бытие есть абстрактное безразличие, для
обозначения которого, поскольку оно само по себе должно быть мыслимо как
бытие, было употреблено выражение
«индифференция»~\pagenote{Выражение
«индифференция» (неразличенность, безразличие) было употреблено Шеллингом в
его вышедшей в 1801 г. работе «Изложение моей системы философии» для
обозначения «абсолютного тождества» субъекта и объекта. «Совершенная
неразличенность (totale Indifferenz) субъективного и объективного» —~таково
было основное понятие этой фазы философского развития Шеллинга. В
дальнейшем Гегель подробно рассматривает шеллинговскую концепцию
«индифференции как обратного отношения между ее факторами» и подвергает ее
имманентной критике, вскрывая ее «всестороннюю противоречивость». По учению
Шеллинга абсолютное представляет собою неподвижное тождество, полнейшую
неразличенность, совершенное безразличие двух факторов: субъективного и
объективного. Всякое же дифференцирование состоит лишь в количественном
перевесе одного из этих двух факторов над другим, причем «вечным основанием
и опорою всех количественных различий субъективного и объективного служит
их совершенная индифференция, составляющая форму абсолютного тождества,
форму их бесконечного бытия» (см. {\em Куно Фишер}, Шеллинг, его жизнь,
сочинения и учение, пер. H. Лocского, Спб. 1905, стр. 590). О шеллинговой
системе «абсолютного тождества» Гегель упоминал также и выше, на
стр.~\pageref{bkm:bm69a}. Ср. прим.~\ref{bkm:Ref474665876} к этому месту.}
и в котором еще нет какого бы то ни было рода определенности. Чистое
количество есть индифференция, как способное ко всяким определениям, но так,
что последние внешни ему, а оно само по себе не имеет никакой связи с ними.
Но та индифференция, которая может быть названа абсолютной, есть та
неразличенность, которая {\em через отрицание} всех
определенностей бытия —~качества и количества и их сперва непосредственного
единства —~меры —~{\em опосредствует себя с собою} в
простое единство. Определенность имеется в ней лишь как состояние,~т.~е.
как некоторое {\em качественно внешнее}, имеющее
индифференцию своим {\em субстратом}.

Но то, что мы таким образом определили как качественно внешнее, есть лишь
исчезающее; как таковое внешнее по отношению к бытию, качественное как
противоположность самого себя есть лишь то, что упраздняет себя.
Определенность еще положена таким образом в субстрате лишь как некоторое
пустое различение. Но именно это пустое различение есть сама
неразличенность (индифференция) как результат. И притом последняя есть,
таким образом, конкретное, опосредствованное в самом себе с собою через
отрицание всех определений бытия. Как это опосредствование, она содержит в
себе отрицание и отношение; и то, что называлось состоянием, есть ее
имманентное, соотносящееся с собой различение; именно внешность и ее
исчезание и делает единство бытия индифференцией и имеется, стало быть,
{\em внутри} последней, которая тем самым перестает
быть только субстратам и только абстрактной {\em в самой себе}.

\section[В. Индифференция как обратное отношение ее факторов]{В. Индифференция как обратное отношение ее факторов}
\pagenote{См. предыдущее примечание.}
Теперь следует посмотреть, каким
образом положено это определение индифференции в ней самой и тем самым,
каким образом она положена как {\em для-себя-сущая}.

1. Сведéние считавшихся первоначально самостоятельными отношений меры к
простым состояниям обосновывает {\em единый} их
{\em субстрат}; последний есть их продолжение одного в
другом и, стало быть, то нераздельное самостоятельное, которое
{\em целиком} наличествует в своих различиях. Для
образования этих различий имеются налицо содержащиеся в нем определения,
качество и количество, и весь вопрос лишь в том, как они в нем положены. Но
это определено тем, что субстрат есть опосредствование ближайшим образом
как результат и {\em в себе}, но это опосредствование,
таким образом, еще не положено {\em в нем} как таковое,
вследствие чего он ближайшим образом есть субстрат и, что касается
определенности, выступает как {\em индифференция}.

Поэтому различие есть в последней по существу прежде всего лишь
количественное, внешнее, и имеются два различных определенных количества
одного и того же субстрата, который, таким образом, есть их
{\em сумма} и следовательно сам определен как
определенное количество. Но индифференция есть такого рода прочная мера,
в-себе-сущая абсолютная граница, лишь в
{\em соотношении} с указанными различиями, так что она
не есть в самой себе определенное количество и ни в каком виде не
противостоит как сумма или же как показатель другим суммам или
индифференциям. В индифференцию входит лишь абстрактная определенность; оба
определенные количества, чтобы быть положенными; в ней как моменты, должны
быть изменчивыми, безразличными, большими или меньшими одно относительно
другого. Но ограниченные прочной границей своей суммы, они вместе с тем
относятся друг к другу не внешним образом, а отрицательно, что образует
связывающее их качественное определение. Поэтому они находятся в
{\em обратном отношении} друг к другу. От
рассмотренного ранее формального обратного отношения это отношение
отличается тем, что здесь целое есть реальный субстрат и что каждый из
обоих членов положен так, чтобы самому быть {\em в себе} этим целым.

Далее, по указанной качественной определенности различие оказывается
различием {\em двух качеств}, из которых одно
упраздняется другим, но которые, как удерживаемые в одном единстве и
составляющие его, неотделимы друг от друга. Сам субстрат как индифференция
равным образом есть в себе единство обоих качеств; поэтому каждая из сторон
отношения также содержит в себе их обоих и отличается от другой стороны
лишь тем, что в ней имеется больше одного качества и меньше другого, или
наоборот: одно качество в одной стороне, а другое —~в другой оказывается
лишь {\em преобладающим} благодаря своему определенному
количеству.

Каждая из сторон есть, следовательно, в себе же самой обратное отношение;
это отношение, как формальное, снова появляется в различенных сторонах.
Таким образом, сами эти стороны продолжаются друг в друга также и по своим
качественным определениям; каждое из качеств относится в ином к самому
себе и имеется в каждой из обеих сторон, но только в разных определенных
количествах. Их количественное различие есть та индифференция, в
соответствии с которой они продолжаются друг в друга, и эта их
продолжаемость представляет собою тождественность качеств в каждом из обоих
единств. — Стороны же, каждая из которых содержит в себе полноту этих
определений и, стало быть, самое индифференцию, вместе с тем положены, таким
образом, одна относительно другой как самостоятельные.

2. Как эта индифференция, бытие есть теперь определенность меры уже не в
непосредственности последней, а в только что вскрытом нами развитом виде.
Оно есть {\em индифференция}, поскольку мера есть
{\em в себе} полнота определений бытия, растворившихся,
чтобы стать этим единством. И точно так же оно есть
{\em наличное бытие} как целостность положенной
реализации, в которой самые моменты суть в-себе-сущая целостность
индифференции, несомые ею как их единством. Но так как единство фиксируется
лишь как {\em индифференция} и тем самым лишь
{\em в себе}, а моменты еще не определены как
{\em сущие-для-себя},~т.~е. еще не снимаются
{\em в них} самих и {\em друг через
друга}, чтобы стать единством, то тем самым имеется вообще, как развитая
определенность, {\em безразличие} его самого (этого
единства) {\em к себе}.

Это, таким образом, нераздельное самостоятельное мы должны рассмотреть
ближе. Оно имманентно во всех своих определениях и остается в них в
единстве с собою, непомутненным ими, но $\alpha$)
как остающееся целостностью {\em в себе}, имеет
определенности, которые в ней сняты, лишь
{\em выступающими} в ней без основания.
«{\em В себе}» индифференции и это ее
{\em наличное бытие} не связаны между собою;
определенности обнаруживаются в ней непосредственным образом; она находится
целиком в каждой из них; их различие, стало быть, положено ближайшим
образом как упраздненное (ein aufgehobener), следовательно, как
{\em количественное}; но именно поэтому оно положено не
как отталкивание ее от себя самой, она положена не как самоопределяющаяся,
а лишь как {\em внешним образом} определенная и
определяемая.

$\beta $) Оба момента находятся в обратном
количественном отношении; это —~хождение взад и вперед вдоль величины, но
хождение, определенное не индифференцией, которая как раз и представляет
собою безразличие такого движения взад и вперед, а, стало быть, лишь извне.
Тем самым категория индифференции указывает на некоторое другое, которое
находится вне ее и в котором лежит то, что ее определяет.
{\em Абсолютное} как индифференция страдает с этой
стороны вторым недостатком {\em количественной} формы,
заключающимся в том, что определенность различия не определяется самим
абсолютным, — первый недостаток состоит в том, что различия в нем лишь
вообще {\em выступают},~т.~е. что полагание их есть
нечто непосредственное, а не его опосредствование с самим собой.

$\gamma$) Количественная определенность моментов,
которые теперь суть {\em стороны} отношения, составляет
способ их {\em существования} (Bestehens); их
{\em наличное бытие} изъемлется вследствие этого
безразличия из перехода качественного. Но они имеют отличное от этого их
наличного бытия свое {\em в себе} сущее существование в
том, что они {\em в себе} суть сама индифференция и что
каждая из них сама есть единство обоих {\em качеств},
на которые раскалывается качественный момент. Различие этих двух сторон
ограничивается тем, что одно качество положено в одной стороне в большем
количестве, а в другой —~в меньшем количестве, другое же качество
соответственно с этим наоборот. Таким образом каждая сторона есть в себе
самой целостность индифференции. — Каждое из обоих качеств, взятое отдельно,
само по себе, равным образом остается той же самой суммой, которую собою
представляет индифференция. Оно продолжается из одной стороны отношения в
другую, и его не ограничивает та количественная граница, которая при этом
полагается в нем. Тут определения приходят в непосредственную
противоположность друг к другу, которая развивается в противоречие, что мы
теперь и должны рассмотреть.

3. А именно, каждое качество вступает {\em внутри}
каждой стороны в соотношение с другим качеством, и притом таким образом,
что, как мы уже установили, и это соотношение должно быть лишь
количественным различием. Если оба качества самостоятельны, если, скажем
примерно, они взяты как независимые друг от друга чувственные материи, то
вся определенность индифференции распадается; ее единство и целостность были
бы в таком случае пустыми названиями. Но они, наоборот, определены вместе с
тем так, что они охватываются единым единством, что они нераздельны и что
каждое из них имеет смысл и реальность только в этом одном качественном
соотношении с другим. Но именно {\em потому, что их
количественность безоговорочно есть количественность именно такой
качественной природы, каждое из них простирается лишь столь далеко, как и
другое}. Поскольку они различались бы просто как определенные количества,
одно из них выходило бы за границы другого и обладало бы в этом своем
избытке некоторым безразличным наличным бытием, которым не обладало бы
другое. Но в их качественном соотношении каждое из них есть лишь постольку,
поскольку другое есть. — Из этого следует, что они находятся
{\em в равновесии}, что, насколько увеличивается или
уменьшается одно, настолько же возрастает или убавляется также и другое,
именно в том же отношении.

На основании их {\em качественного} соотношения дело не
может поэтому дойти ни до какого {\em количественного}
различия и ни до какого {\em избытка} одного качества.
Избыток, которым {\em один} из соотносящихся моментов
превышал бы {\em другой}, был бы лишь неудерживающимся
определением или, иначе говоря, этот избыток {\em был
бы лишь опять-таки самим другим}; но при этом равенстве обоих нет ни одного
из них, ибо их наличное бытие должно было покоиться лишь на неравенстве их
определенных количеств. — Каждый из этих якобы факторов одинаково исчезает,
примем ли мы, что он {\em превышает} другой, или
примем, что он {\em равен} ему. Это исчезновение
выступает в таком виде, что, если мы исходим из количественного
представления, то нарушается равновесие и один фактор принимается бóльшим,
чем другой; таким образом, получается упразднение качества другого фактора
и его неспособность удержаться; первый фактор становится преобладающим,
другой убывает с возрастающей скоростью и побеждается первым, и этот
первый, следовательно, делается единственным самостоятельным; но тем самым
имеется уже не два специфических [нечто], не два фактора, а лишь одно
целое.

Это единство, положенное, таким образом, как целостность процесса
определения, взятая так, как она здесь определилась,~т.~е. как
индифференция, есть всестороннее противоречие; оно, стало быть, должно быть
{\em положено} таким образом, что оно, как это
снимающее само себя противоречие, определено в для-себя-сущую
самостоятельность, имеющую своей истиной и результатом уже не только лишь
индифферентное, а имманентно в нем самом отрицательное, абсолютное единство,
которое есть {\em сущность}.


\subsubsection[Примечание О центростремительной и центробежной силе]
{Примечание О центростремительной и центробежной силе}

{\em Отношение} некоторого целого, долженствующего иметь
свою определенность в различии величин качественно определенных друг
относительно друга факторов, находит употребление в применении к
эллиптическому движению небесных тел. В этом примере мы видим ближайшим
образом лишь два качества, находящиеся в обратном отношении друг к другу, а
не две стороны, из которых каждая сама была бы единством обоих качеств и их
обратным отношением. Эмпирическая основа здесь прочна, но при этом упускают
из вида то последствие, к которому приводит внесенная в нее теория, а
именно, что здесь или теория разрушает лежащий в основании факт или, если,
как это подобает, будут крепко держаться этого факта, обнаруживается
пустота теории в сопоставлении с фактом. Игнорирование этого последствия
оставляет спокойно стоять рядом факт и противоречащую ему теорию. — Простой
факт состоит в том, что в эллиптическом движении небесных тел их скорость
возрастает по мере их приближения к перигелию и уменьшается по мере их
приближения к афелию. Количественная сторона этого факта неустанными
старательными наблюдениями точно определена, и самый факт приведен далее к
своему простому закону и формуле, и тем самым сделано все, что поистине
можно требовать от теории. Но это казалось рефлектирующему рассудку
недостаточным. Для так называемого объяснения явления и его закона
принимаются {\em центростремительная} и
{\em центробежная} силы как качественные моменты
криволинейного движения. Их качественное различие состоит в
противоположности их направлений, а с количественной стороны —~в том, что,
поскольку они определены как неравные, когда одна увеличивается, другая
уменьшается и наоборот, и, далее, также и в том, что их отношение снова
переворачивается, а именно, что после того как в продолжение некоторого
времени центростремительная сила увеличивалась, а центробежная уменьшалась,
наступает в известной точке момент, когда центростремительная сила начинает
уменьшаться, а центробежная, напротив,
увеличиваться~\pagenote{В первом издании
«Науки логики» (1812) здесь стояла еще следующая фраза, выпущенная Гегелем
в 1831 г., когда он готовил второе издание: «Этот вопрос я осветил в моей
более ранней диссертации, где я доказал несостоятельность этого различения
и построенных на нем объяснений». Имеется в виду «Философская диссертация
об орбитах планет» (1801) Возможно, что Гегель выпустил эту ссылку на свою
диссертацию потому, что в ней, между прочим, доказывалось, что между
Юпитером и Марсом не может быть никаких планет, между тем как еще в 1801 г.
была открыта малая планета Церера, расположенная как раз между Юпитером и
Марсом.}.
Но этому представлению противоречит отношение друг к другу существенно
качественной определенности этих сил. Эта качественная определенность никак
не допускает их разделения; каждая из них имеет значение лишь в связи с
другой; следовательно, поскольку одна имела бы избыток перед другой,
постольку она не имела бы никакого соотношения к этой другой и ее не было
бы. — При предположении, что одна из них сначала больше другой, если она
как бóльшая находилась бы в соотношении с меньшей, то имеет место то, о чем
мы сказали выше, а именно, что она получила бы абсолютный перевес, а другая
исчезла бы; последняя положена как исчезающая, лишенная опоры, и в этом
определении ничего не меняет то обстоятельство, что исчезновение происходит
лишь постепенно, и столь же мало меняет в нем то обстоятельство, что
{\em сколько} она теряет в величине,
{\em столько же} прибавляется к величине первой; эта
первая гибнет вместе с другой, так как то, что она есть, она есть лишь
постольку, поскольку имеется другая. Ведь это очень простое соображение,
что если, например, как это утверждают, центростремительная сила тела
увеличивается по мере его приближения к перигелию, а центробежная,
напротив, настолько же уменьшается, то последняя уже
{\em была бы более не в состоянии} вырвать его у первой
и снова отдалить его от центрального тела; напротив, раз первая сила
однажды получила абсолютный перевес, то другая сила побеждена, и тело будет
двигаться с возрастающей скоростью по направлению к своему центральному
телу. И, наоборот, если бесконечно близко к афелию центробежная сила
преобладает, то столь же противоречиво, чтобы она в самом афелии была
побеждена более слабой силой. — Ясно, далее, что только некоторая
{\em чужая сила} могла бы вызвать
{\em этот переворот}; это значит, что то ускоряющаяся,
то замедляющаяся скорость движения {\em не} может быть
{\em познана} из принятого определения указанных
факторов или, как обыкновенно выражаются, {\em не}
может {\em быть объяснена} этими факторами, а между тем
их существование было допущено именно для того, чтобы объяснить это
различие. Последовательно вытекающий вывод, что одно или другое направление
движения исчезает и, стало быть, исчезает и само эллиптическое движение,
вообще игнорируется и остается скрытым ввиду того несомненного факта, что
движение это продолжается и из ускоренного переходит в замедляющееся.
Предположение, что в афелии слабость центростремительной силы превращается
в ее преобладание над центробежной силой и что обратное получается в
перигелии, {\em отчасти} означает то, что было развито
выше, а именно, что каждая из сторон обратного отношения есть в себе самой
все это обратное отношение целиком; ибо та сторона, которую образует
движение от афелия к перигелию, являющееся, по предположению, перевесом
центростремительной силы, содержит в себе согласно этому предположению еще
и центробежную силу, но убывающую по мере того, как первая увеличивается; в
таком же обратном отношении находится в той стороне, которую образует
замедляющееся движение, перевешивающая и становящаяся все более и более
перевешивающей центробежная сила к центростремительной, так что ни в одной
стороне никакая из этих сил не исчезает, а лишь становится все меньше и
меньше до того момента, когда она начинает превращаться во все более и
более преобладающую над другой. Стало быть, в каждой стороне этого движения
—~в этом и состоит недостаток рассматриваемого нами обратного отношения
—~возвращается одно из двух: либо каждая сила берется самостоятельно, сама
по себе, и, вместе с чисто {\em внешней} встречей их,
сочетанием в одно движение, как например, в параллелограме сил,
упраздняется единство понятия, природа вещи, либо же обе силы соотносятся
между собою качественно через понятие, и ни одна из них не может получить
безразличного самостоятельного существования относительно другой, которое
было бы ей сообщено некоторым ее избытком (форма интенсивности, так
называемое динамическое, здесь ничего не меняет, так как динамическое само
имеет свою определенность в определенном количестве и тем самым тоже может
обнаружить лишь столько силы,~т.~е. существует лишь настолько, сколько ему
противостоит противоположной силы). {\em Отчасти} же
этот поворот от перевеса силы к его противоположности означает чередование
качественных определений положительного и отрицательного; возрастание
одного есть столько же убывание другого. Нераздельная качественная связь
этих качественных противоположностей расщеплена рассматриваемой теорией в
некоторую {\em последовательность}; но вследствие этого
оказывается, что она не дает {\em объяснения} этого
чередования, равно как и главным образом не дает
{\em объяснения} самого этого расщепления. Видимость
единства, еще имеющаяся в возрастании одной силы и соответственном убывании
другой, здесь окончательно исчезает; нам указывают чисто
{\em внешнее} последование, находящееся лишь в
противоречии с необходимым выводом из данной связи, согласно которому,
поскольку одна сила сделалась перевешивающей, другая должна исчезнуть.

То же самое отношение применялось к притягательной и отталкивательной силам,
чтобы понять различие {\em плотности} тел. Таким же
образом полагали привлекать обратное отношение между чувствительностью и
раздражимостью, чтобы из того обстоятельства, что эти два фактора
{\em жизни} делаются неравными, постигнуть различные
определения целого, здоровья, равно как и различие родов живых существ.
Однако путаница и галиматья, к которым при некритическом употреблении этих
определений понятия привело это объяснение, которое, как полагали, должно
было сделаться натурфилософской основой физиологии, носологии и зоологии,
имели здесь своим следствием, что вскоре снова отказались от этого
формализма, который продолжают применять во всем его объеме главным образом
в науке физической астрономии.

Поскольку может казаться, что {\em абсолютная
индифференция} представляет собою основное определение
{\em спинозовской субстанции}, можно об этом еще
заметить, что она несомненно такова в том отношении, что в обеих полагаются
исчезнувшими все определения бытия, как и вообще всякое дальнейшее
конкретное различение мышления и протяжения и~т.~д. Вообще, если не
двинемся дальше абстракции, то совершенно безразлично, как выглядело, когда
оно существовало, то, что исчезло в эту пропасть. Но субстанция как
индифференция до известной степени уже связана с
{\em потребностью определить} ее и
{\em с принятием во внимание} этой потребности; она не
должна оставаться субстанцией Спинозы, единственным определением которой
служит то отрицательное определение, что все в нее исчезло,
поглощено\pagenote{\label{omnisnotel2}Это не совсем
так. Как уже было упомянуто в примечании \ref{omnisnote1}-м, Гегель
в значительной мере произвольно толковал философию Спинозы, приписывая ей
акосмизм (отрицание мира) и идеализм. В этом отношении он шел по стопам
Шеллинга, использовавшего некоторые элементы спинозизма для построения
своей идеалистической философии «абсолютного тождества». В действительности
же между материалистом Спинозой и объективным идеалистом Шеллингом огромная
принципиальная разница, и эта-то разница как раз и смазывается в
гегелевской трактовке спинозизма.}.
У Спинозы различие, атрибуты —~мышление и протяжение, — а затем также и
модусы, аффекты и все прочие детерминации, привходят совершенно
эмпирически; это различение имеет место в уме, который сам есть модус.
Атрибуты не находятся ни в какой {\em дальнейшей
определенной связи} (weitern Bestimmtheit) с субстанцией и друг с другом,
помимо той, что они выражают субстанцию всю целиком и их содержание,
порядок вещей как протяженных и как мыслимых, один и тот же. Но определяя
субстанцию как индифференцию, рефлексия подходит к
{\em различию}; оно теперь
{\em положено} как то, что оно у Спинозы есть в себе, а
именно как {\em внешнее} и тем самым, ближе, как
{\em количественное}. Таким образом, индифференция
остается, правда, в нем имманентной себе, как субстанция, но имманентной
себе абстрактно, лишь {\em в себе}; различие
{\em не} имманентно {\em ей}; как
количественное различие, оно, наоборот, есть противоположность
имманентности, и количественная индифференция есть скорее вне-себя-бытие
единства. Различие тем самым также и здесь не понимается качественно,
субстанция не определяется как различающая (дифференцирующая) сама себя, как
субъект. Ближайшим следствием в отношении самой категории индифференции
является то, что в ней распадается различие качественного и количественного
определений, как это оказалось в ходе развития индифференции; она есть
{\em разложение меры}, в которой эти два момента были
непосредственно положены как одно.

\section[С. Переход в сущность]{С. Переход в сущность}
Абсолютная индифференция есть последнее
определение {\em бытия}, прежде чем последнее перейдет
в {\em сущность}; но она не достигает сущности. Она
оказывается еще принадлежащей сфере {\em бытия}, так
как она, будучи определена как {\em безразличная},
имеет в себе различие еще как {\em внешнее},
количественное. Это —~ее {\em наличное бытие}, к
которому она вместе с тем находится в противоположности, заключающейся в
том, что она определена относительно него лишь как
{\em в-себе-сущая}, а не мыслится как
{\em для-себя-сущее} абсолютное. Или, иначе говоря,
только {\em внешней рефлексии} свойственно застревать в
воззрении, что специфические суть {\em в себе} или в
абсолютном {\em одно и то же}, что их различие есть
лишь безразличное различие, не есть различие в себе. Здесь еще нет того,
чтобы эта рефлексия была не {\em внешней} рефлексией
{\em мыслящего}, субъективного сознания, а собственным
определением различий указанного единства снимать себя, каковое единство
оказывается, таким образом, абсолютной отрицательностью, своим безразличием
{\em к самому себе}, безразличием как к своему
собственному безразличию, так и к инобытию.

Но это самоснимание определения индифференции уже получилось выше; это
определение оказалось в ходе развития его положенности противоречием со
всех сторон. {\em В себе} индифференция есть
целостность, в которой все определения бытия сняты и содержатся; таким
образом, она есть основа, но дана пока что лишь в
{\em одностороннем определении в-себе-бытия}, а тем
самым [выступающие в ней] различия, количественная разница и обратное
отношение факторов {\em внешни} в ней. Таким образом,
будучи противоречием между самой собою и своей определенностью, между своим
в-себе-сущим определением и своей положенной определенностью, она есть
отрицательная целостность, определенности которой сняли себя в себе самих и
тем самым сняли эту ее основную односторонность, ее в-себе-бытие.
Определение индифференции, положенное тем самым как то, что индифференция
есть на самом деле, представляет собою простое и бесконечное отрицательное
соотношение с собой, несовместимость себя с самим собою, отталкивание себя
от себя самого. Процесс определения и определяемость не есть ни переход, ни
внешнее изменение, ни {\em выступление} в ней
определений, а ее собственное соотнесение с собой, которое есть
отрицательность ее самой, ее в-себе-бытия.

Но определения, как такие оттолкнутые, теперь не принадлежат самим себе, не
выступают как нечто самостоятельное или внешнее, а даны (sind) как моменты;
они принадлежат, во-первых, {\em сущему в себе}
единству, не отпускаются им, а носимы им как субстратом и наполнены только
им; а, во-вторых, как определения, имманентные
{\em для-себя-сущему} единству, они имеют бытие лишь
через его отталкивание от себя. Вместо того чтобы быть
{\em сущими}, как это мы видели во всей сфере бытия,
они, теперь всецело суть только {\em положенные},
всецело имеют то определение и значение, что они
{\em соотнесены} со своим единством и, стало быть, что
каждое из них соотнесено со своим другим и отрицанием, — они отмечены этой
своей соотносительностью (Relativität).

Тем самым бытие вообще и как бытие или непосредственность различенных
определенностей, так и {\em в-себе-бытие} исчезли, и
единство есть бытие, {\em непосредственная
предположенная} целостность, так что оно есть это
{\em простое соотношение с собой},
{\em лишь как опосредствованное снятием этого
предположения}, и сама эта предположенность, само это непосредственное
бытие есть лишь момент его отталкивания, а изначальная самостоятельность и
тождество с собой даны лишь как {\em получающееся в
виде результата}, {\em бесконечное слияние с собой}.
Таким образом, бытие определилось в {\em сущность},
бытие, ставшее через снятие бытия простым, однородным с собой бытием.

