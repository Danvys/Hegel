\clearpage

Полное изменение, которое претерпел у нас за
последние лет двадцать пять характер философского мышления, более высокая
точка зрения относительно себя, которой в этот период времени достигло
самосознание духа, до сих пор еще оказали мало влияния на облик
{\em логики}.

То, что до этого промежутка времени носило название метафизики, подверглось,
так сказать, радикальному искоренению и исчезло из ряда наук. Где теперь мы
услышим или где теперь смеют еще раздаваться голоса прежней онтологии,
рациональной психологии, космологии или даже прежней естественной теологии?
Где теперь будут интересоваться такого рода исследованиями, как, например,
об имматериальности души, о механических и целевых причинах? Да и прежние
доказательства бытия божия излагаются лишь исторически или в целях
назидания и душевного ободрения. Это —~факт, что интерес, отчасти к
содержанию, отчасти к форме прежней метафизики, а отчасти к обоим вместе
утрачен. Сколь ни замечательно явление народа, для которого сделались
непригодными, например, наука его государственного права, его общие
убеждения, его нравственные привычки и добродетели, но столь же по меньшей
мере замечательное явление представляет собой народ, который утрачивает
свою метафизику, народ, среди которого дух, занимающийся своей чистой
сущностью, уже не имеет действительного существования.

\label{bkm:Ref474526580} Экзотерическое учение кантовской философии, гласящее, что
{\em рассудок не имеет права залетать дальше области
опыта} и что в противном случае способность познания становится
{\em теоретическим разумом}, который сам по себе
порождает только {\em химерические домыслы}, — это
учение доставило с научной стороны оправдание отказа от спекулятивного
мышления. На подмогу этому популярному учению шли вопли новейшей педагогики
(отзвук трудных времен, направляющих взор людей на непосредственные нужды),
разговоры о том, что подобно тому, как для познания опыт является первым и
главным, так и для достижения умелости в общественной и частной жизни
теоретическое понимание даже вредно, и существенным, единственно полезным
является упражнение и вообще практическое образование. Так как наука и
здравый человеческий смысл, таким образом, помогали друг другу в деле
уничтожения метафизики, то казалось, что их общими усилиями получилось
странное зрелище {\em образованного народа без
метафизики}, нечто вроде во всем прочем многообразно украшенного храма, но
без святого святых. Теология, которая в прежние времена была хранительницей
спекулятивных таинств и (правда, зависимой) метафизики, отказалась от этой
науки, заменив ее чувствованиями (Gefühle), практически-популярными
поучениями и учено-историческими сведениями. Этой перемене соответствует то
обстоятельство, что в другом месте исчезли те
{\em одинокие}, которые приносились в жертву своим
народом и удалялись им из мира, дабы существовали созерцание вечного и
жизнь, посвященная единственно только этому созерцанию не ради пользы, а
ради благодати. Это исчезновение, хотя оно и стоит в другой связи, может
быть рассматриваемо как явление, по существу дела тождественное с
вышеуказанным. Казалось, таким образом, что после изгнания этого мрака,
этого бесцветного занятия ушедшего в себя духа самим собою, существование
превратилось в светлый, радостный мир цветов, среди которых, как известно,
нет {\em черных}.

{\em Логика} испытала не столь печальную участь, как
метафизика. Правда, тот предрассудок, будто она
{\em научает мыслить}, в чем раньше видели приносимую
ею пользу и, стало быть, также и ее цель (это похоже на то, как если бы
сказали, что только благодаря изучению анатомии и физиологии мы впервые
научаемся переваривать пищу и двигаться), — этот предрассудок давно уже
исчез, и дух практицизма уготовлял ей не лучшую участь, чем ее сестре. Тем
не менее, вероятно ввиду приносимой ею некоторой формальной пользы, ей было
еще оставлено место среди наук, и ее даже сохранили в качестве предмета
публичного преподавания. Но этот лучший жребий касается только ее внешних
судеб, ибо ее облик и содержание остались такими же, какими они по давней
традиции передавались от поколения к поколению, причем, однако, в процессе
этой передачи ее содержание делалось все более и более тощим и скудным; в
ней еще не чувствуется того нового духа, восход которого сказался в науке
не менее, чем в действительности. Но нужно сказать раз навсегда, что тщетно
желание удержать формы прежнего образования, когда перестроилась
субстанциальная форма духа. Они представляют собою увядшие листья,
спадающие благодаря напору образовавшихся у их основания новых почек.

{\em Игнорирование} этой общей перемены начинает
постепенно исчезать также и в научной области. Незаметным образом даже сами
противники освоились с этими другими представлениями, усвоили их себе, и
если они все еще не приемлют источника этих представлений, лежащих в их
основании принципов, и возражают против них, то им зато приходится мириться
с выводами и они оказываются не в силах противиться влиянию последних.
Помимо того, что все более и более слабеет их отрицательное отношение к
указанным представлениям, этим противникам удается сообщить своим работам
положительное значение и содержание только благодаря тому, что сами они
начинают говорить на языке новых представлений.

С другой стороны, уже прошло, повидимому, время того брожения, с которого
начинается всякое творчество нового. В своей начальной стадии это
творчество относится фанатически враждебно к существующей широко
разветвленной систематизации прежнего принципа; оно отчасти также
опасается, что потеряется в пространных частностях, но отчасти пугается
труда, которого потребовала бы научная разработка, и, чувствуя потребность
в такой разработке, хватается сначала за пустой формализм. Требование,
чтобы содержание подверглось обработке и было развито, становится после
этого еще настоятельнее. В ходе развития той или иной эпохи, как и в ходе
развития отдельного человека, бывает период, когда дело идет главным
образом о приобретении и отстаивании принципа во всей его неразвитой
напряженности. Но более высокое требование состоит в том, чтобы этот
принцип развился в науку.

Но, что бы ни было уже сделано в других отношениях для сути и формы науки,
логическая наука, составляющая подлинную метафизику или чистую,
спекулятивную науку, еще находилась до сих пор в большом пренебрежении. Что
я разумею ближе под этой наукой и ее точкой зрения, это я указал
предварительно во {\em введении}. Необходимость снова
начать в этой науке с самого начала, характер самого предмета и отсутствие
таких подготовительных трудов, которые могли бы быть использованы для
предпринятого нами преобразования, — пусть все эти обстоятельства будут
приняты во внимание справедливыми судьями, если бы оказалось, что и
многолетняя работа не дала автору возможности сообщить этой попытке большее
совершенство. Существенно главным образом иметь в виду, что дело идет о
том, чтобы дать новое понятие научного рассмотрения. Философия, если она
должна быть наукой, как я на это указал в другом
месте\footnote{«Феноменология духа», Предисловие к первому изданию.
Подлинным развитием сказанного является познание метода, которое находит
себе место в самой
логике~\pagenote{Это примечание
прибавлено Гегелем в 1831~г., когда он подготовлял 2-е издание своей «Науки
логики».}.},
не может для этой цели заимствовать свой метод от подчиненной науки,
каковой является математика, и точно так же она не может успокаиваться на
категорических заверениях внутренней интуиции (Anschauung) или пользоваться
рассуждениями, опирающимися на основания, доставляемые внешней рефлексией.
Методом философии может быть лишь {\em движущаяся} в
научном познании {\em природа содержания}, причем
вместе с тем эта же {\em собственная рефлексия}
содержания впервые полагает и {\em порождает} само
{\em его} ({\em содержания})
{\em определение}.

{\em Рассудок определяет} и твердо держится за свои
определения; {\em разум} же отрицателен и
{\em диалектичен}, ибо он разрешает определения
рассудка в ничто; он {\em положителен}, ибо он
порождает {\em всеобщее} и постигает в нем особенное.
Подобно тому, как рассудок обычно понимается как нечто отдельное от разума
вообще, точно так же и диалектический разум обычно признается чем-то
отдельным от положительного разума. Но в своей истине разум есть
{\em дух}, который выше их обоих; он есть рассудочный
разум или разумный рассудок. Он есть отрицательное, то, что составляет
качество как диалектического разума, так и рассудка. Он отрицает простое, и
тем самым он полагает определенное различие, за которое держится рассудок.
Но вместе с тем он также и разлагает это различие, и тем самым он
диалектичен. Однако он не задерживается на этом нулевом результате: он
здесь вместе с тем выступает также и как положительный разум, и, таким
образом, он восстанавливает первоначальное простое, но как всеобщее,
которое конкретно внутри себя. Под последнее не просто подводится то или
другое данное особенное, а в вышеуказанном процессе определения и
разлагании этого определения уже определилось вместе с тем и особенное. Это
духовное движение, дающее себе в своей простоте свою определенность, а в
последней —~свое равенство с самим собою, это движение, представляющее
собою, стало быть, имманентное развитие понятия, есть абсолютный метод
познания и вместе с тем имманентная душа самого содержания. — Единственно
только на этом конструирующем сам себя пути философия, утверждаю я,
способна быть объективной, доказательной наукой. — Таким способом я
попытался в «{\em Феноменологии духа}» изобразить
{\em сознание}. Сознание есть дух, как конкретное
знание, притом находящееся в плену у внешности. Но движение форм этого
предмета, подобно развитию всякой природной и духовной жизни, покоится
только на природе {\em чистых сущностей}, составляющих
содержание логики. Сознание как являющийся дух, который освобождается на
проходимом им пути от своей непосредственности и сращенности с внешним,
становится чистым знанием, дающим себе в качестве предмета те самые
вышеуказанные чистые сущности, как они суть в себе и для себя. Они суть
чистые мысли, мыслящий свою сущность дух. Их самодвижение есть их духовная
жизнь и представляет собою то, что конституирует науку и изображением чего
она является.

Этим указано внутреннее отношение к логике той науки, которую я называю
{\em феноменологией духа}. Что же касается внешнего
отношения между ними, то я полагал, что за первой частью
«{\em Системы науки}»\footnote{ Бамберг и Вюрцбург в
издательстве Гебгарда 1807. Во втором издании, которое появится в свет на
ближайшей пасхе, это название будет исключено. Вместо указываемой далее
предполагавшейся второй части, которая должна была содержать в себе все
другие философские науки, я выпустил после этого в свет «Энциклопедию
философских наук», вышедшую в прошлом году третьим
изданием~\pagenote{Это примечание
прибавлено Гегелем в 1831~г., когда он подготовлял 2-е издание своей «Науки
логики». Упоминаемое в нем 2-е издание «Феноменологии духа» Гегель не успел
подготовить к печати, так как внезапно умер 14 ноября 1831 г. Исправления
текста «Феноменологии» доведены были Гегелем лишь до 37-й страницы
«Предисловия». С этими исправлениями «Феноменология» была издана учениками
Гегеля в качестве II тома Собрания его сочинений (в 1832~г.). «Энциклопедия
философских наук» впервые вышла в 1817~г.: третье, расширенное издание ее
появилось в 1830~г., за год до смерти Гегеля.}.},
содержащей в себе феноменологию, последует вторая часть, которая должна
была содержать в себе логику и обе реальные части философии, философию
природы и философию духа, так что этой частью заканчивалась бы система
науки. Но необходимость расширить объем логики, взятой сама по себе,
побудила меня выпустить ее в свет отдельно; она, таким образом, составляет
согласно этому расширенному плану первое продолжение «Феноменологии духа».
Позднее за нею воспоследует обработка двух вышеназванных реальных
философских наук. Этот первый том «Логики» содержит первую книгу
—~{\em учение о бытии}, а затем вторую книгу,
{\em учение о сущности}, как второй отдел первого тома;
второй же том будет содержать в себе {\em субъективную
логику}, или {\em учение о понятии}.

{\em Нюренберг}, 22 марта 1812 г.
\bigskip
