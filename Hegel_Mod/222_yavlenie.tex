\subsubsection{Вторая главаЯвление}
Существование есть та непосредственность бытия,
которою сущность снова сделала себя. Эта непосредственность есть
{\em в себе} рефлексия сущности в себя. Сущность как
существование выступила из своего основания, которое само перешло в
существование. Существование есть эта
{\em рефлектированная} непосредственность, поскольку
оно в самом себе есть абсолютная отрицательность. Оно теперь также и
положено как таковая, поскольку оно себя определило как
{\em явление}.

Явление есть поэтому ближайшим образом сущность в ее существовании; сущность
наличествует в нем непосредственно. Что оно есть не непосредственное, а
{\em рефлектированное} существование, это
обстоятельство составляет момент сущности в нем; или, иными словами,
существование, как {\em существенное} существование,
есть явление.

Нечто есть {\em лишь} явление в том смысле, что
существование как таковое есть лишь некоторое положенное, а не
в-себе-и-для-себя-сущее. Его существенность в том и состоит, что оно в
самом себе обладает отрицательностью рефлексии, природой сущности. Речь
идет не о какой-то чуждой, внешней рефлексии, к которой принадлежала бы
сущность и которая путем сравнения сущности с существованием объявляла бы
последнее явлением; а, как выяснилось, эта существенность существования,
заключающаяся в том, что оно есть явление, есть собственная истина
существования. Рефлексия, через которую оно таково, принадлежит ему самому.

Но если говорят, что нечто есть {\em только} явление, в
том смысле, как будто в сравнении с ним
{\em непосредственное существование} есть истина, то
нужно сказать, что скорее, наоборот, явление есть более высокая истина; ибо
оно есть существование, как оно существенно, тогда как непосредственное
существование есть еще лишенное сущности явление, так как оно заключает в
себе лишь один момент явления, именно, существование как непосредственное,
и не имеет еще в себе его (явления) отрицательной рефлексии. Если
{\em явление} называют
{\em лишенным сущности}, то о моменте его
отрицательности мыслят так, как будто непосредственное есть по сравнению с
ним положительное и истинное; но это непосредственное скорее, наоборот, еще
не содержит в себе существенной истины. Существование скорее перестает быть
лишенным сущности благодаря тому, что оно переходит в явление.

Сущность сначала {\em светится видимостью} в себе самой,
в своем простом тождестве; таким образом, она есть абстрактная рефлексия,
чистое движение ничто через ничто обратно к самому себе. Сущность
{\em является}; таким образом, она отныне есть
{\em реальная} видимость, так как моменты видимости
обладают существованием. Явление, как выяснилось, есть вещь как
отрицательное {\em опосредствование} себя с самим
собой; различия, которые содержатся в вещи, суть
{\em самостоятельные} материи, представляющие собою
противоречие, заключающееся в том, что они суть некоторое непосредственное
устойчивое наличие, и вместе с тем имеют свое устойчивое наличие лишь в
чужой самостоятельности, следовательно, в отрицании собственной
самостоятельности, и опять-таки именно поэтому также лишь в отрицании той
чужой самостоятельности или, иначе говоря, в отрицании их собственного
отрицания. Видимость есть то же самое опосредствование, но ее неустойчивые
моменты имеют в явлении вид непосредственной самостоятельности. Напротив,
непосредственная самостоятельность, присущая существованию, понижается, с
своей стороны, на степень момента. Явление есть поэтому единство видимости
и существования.

Явление теперь определяет себя ближе. Оно есть существенное существование;
существенность последнего отличается от него как несущественного, и эти две
стороны вступают в соотношение друг с другом. — Оно есть поэтому,
{\em во-первых}, простое тождество с собой, содержащее
в себе вместе с тем разные определения содержания; это тождество с собою
—~как оно само, так и соотношение входящих в него определений —~есть то,
что остается равным себе в смене явлений; это
—~{\em закон явления}.

Но, {\em во-вторых}, простой в своей разности закон
переходит в противоположность; существенное в явлении противопоставляется
самому явлению и в противоположность {\em являющемуся
миру} выступает {\em в-себе-сущий мир}.

{\em В-третьих}, эта противоположность возвращается в
свое основание; в-себе-сущее определено в явлении, и, наоборот, являющееся
определено, как вобранное в свое в-себе-бытие; явление становится
{\em отношением}.

\paragraph[А. \ Закон явления]{А. \ Закон явления}
\hypertarget{Toc478978746}{}1. Явление есть существующее, опосредствованное
своим {\em отрицанием}, которое составляет его
{\em устойчивое наличие}. Хотя это его отрицание есть
некоторое {\em другое} самостоятельное, однако
последнее столь же существенным образом есть некоторое снятое. Существующее
есть поэтому свое {\em возвращение} в само себя через
свое отрицание и через отрицание этого своего отрицания; оно,
следовательно, обладает {\em существенной
самостоятельностью}, равно как оно столь же непосредственно есть
безоговорочно {\em положенность}, имеющая своим
устойчивым наличием некоторое {\em основание} и
некоторое другое. — Следовательно, явление, во-первых, есть существование
вместе с его существенностью, положенность вместе с ее основанием; но это
основание есть {\em отрицание}, и то другое
самостоятельное, которое служит основанием перового, есть равным образом
лишь некоторая положенность. Или, иначе говоря, существующее, как
являющееся, рефлектировано в некоторое другое и имеет своим основанием это
другое, которое само заключается лишь в том, что оно рефлектировало в
другое. Присущая ему {\em существенная}
самостоятельность, так как она есть возвращение в себя само, есть ввиду
отрицательности моментов возвращение ничто через ничто к себе самому;
самостоятельность существующего есть поэтому лишь
{\em существенная видимость}. Связь обосновывающих друг
друга существующих состоит поэтому в указанном взаимном отрицании, в том,
что устойчивое наличие одного есть не устойчивое наличие другого, а его
положенность, каковое соотношение положенности единственно только и
составляет их устойчивое наличие. Основание имеется так, как оно есть в
своей истине, которая именно заключается в том, что оно есть такое первое,
которое есть лишь нечто {\em предположенное}.

Это составляет {\em отрицательную} сторону явления. Но в
этом отрицательном опосредствовании непосредственно, содержится
{\em положительное тождество} существующего с собой.
Ибо оно не есть {\em положенность},
{\em противостоящая} некоторому
{\em существенному основанию}, или, иначе говоря, оно
не есть {\em видимость в некотором самостоятельном}, а
есть {\em положенность}, соотносящаяся с некоторой
{\em положенностью}, или есть
{\em видимость} лишь в
{\em некоторой видимости}. Оно в этом своем отрицании
или в своем другом, которое само есть некоторое снятое,
{\em соотносится с самим собой}, есть, следовательно,
тождественная с собой или положительная существенность. — Это тождественное
не есть та {\em непосредственность}, которая присуща
существованию как таковому и которая есть лишь та несущественная сторона
его, по которой оно имеет свое устойчивое наличие в некотором другом.
Напротив, оно есть {\em существенное содержание}
явления, имеющее две стороны: во-первых, форму
{\em положенности} или внешней непосредственности и,
во-вторых, положенность как тождественность с собой. По первой стороне это
содержание есть некоторое наличное бытие, но как случайное, несущественное,
которое по своей непосредственности подвержено переходу, возникновению и
прехождению. По второй стороне оно есть простое определение содержания,
изъятое из-под власти этой смены явлений, — то, что
{\em пребывает} в последней.

Помимо того, что это содержание есть вообще
{\em простое} в преходящем, оно есть также и
{\em определенное}, {\em внутри
себя различное }содержание. Оно есть рефлексия в себя явления,
отрицательного наличного бытия, в нем, следовательно, существенным образом
содержится {\em определенность}. Но явление есть
{\em сущая} многообразная разность, мечущаяся в
несущественном многообразии; напротив, ее рефлектированное содержание есть
ее многообразие, сведенное к {\em простому различию}. А
именно, определенное существенное содержание, говоря точнее, не только
определено вообще, но как существенное в явлении представляет собою полную
определенность, — {\em одно} и его
{\em другое}. В явлении каждое из этих двух имеет свое
устойчивое наличие в другом таким образом, что оно вместе с тем есть лишь в
его {\em отсутствии устойчивого наличия}. Это
противоречие снимается, и его рефлексия в себя есть
{\em тождество} их обоюдного устойчивого наличия,
состоящее в том, что {\em положенность одного есть
также и положенность другого}. Они составляют
{\em единое} устойчивое наличие и вместе с тем
выступают как {\em разные}, безразличные друг к другу
содержания. Таким образом, в существенной стороне явления то
{\em отрицательное}, которое присуще несущественному
содержанию и которое состоит в том, что оно снимает себя, возвратилось в
тождество; это содержание есть безразличное
{\em устойчивое наличие}, которое есть не снятость, а
скорее {\em устойчивое наличие другого}.

Это единство есть {\em закон явления}.

2. Закон есть, следовательно, {\em положительное} в
опосредствовании являющегося. Явление есть ближайшим образом существование
как {\em отрицательное} опосредствование с собой, так
что существующее опосредствовано с собой через
{\em свое собственное отсутствие устойчивого наличия},
через некоторое другое, и опять-таки через
{\em отсутствие устойчивого наличия этого другого}.
Здесь мы имеем, {\em во-первых}, голое свечение и
исчезание обоих, несущественное явление, а,
{\em во-вторых}, также и
{\em пребывание} или {\em закон};
ибо {\em каждое} из рассматриваемых двух
{\em существует} в указанном снятии другого, и их
положенность, как их отрицательность, есть вместе с тем
{\em тождественная, положительная} положенность обоих.

Таким образом, это пребывающее устойчивое наличие, которое явление имеет в
законе, {\em во-первых}, противоположно той
{\em непосредственности} бытия, которой обладает
существование. Эта непосредственность есть, правда,
{\em в себе} рефлектированная непосредственность, а
именно, возвратившееся в себя основание; но теперь, в явлении, эта простая
непосредственность отлична от рефлектированной, тогда как в вещи они только
начинали отделяться друг от друга. Существующая вещь стала в своем
разложении этой противоположностью; {\em положительное}
в ее разложении есть указанное тождество с собой являющегося, как
положенного в своей другой положенности. —
{\em Во-вторых}, эта рефлектированная
непосредственность сама определена как
{\em положенность} по отношению к сущей
непосредственности существования. Эта положенность есть теперь существенное
и истинно положительное. Немецкое слово «Gesetz» заключает в себе также и
это определение (Gesetz—закон, gesetztes —~положенное, положительное. —
Перев.). В этой положенности заключается существенное
{\em соотношение} обеих сторон различия, содержащихся в
законе; они суть разные, непосредственные по отношению друг к другу
содержания и суть таковые как рефлексия принадлежащего явлению, исчезающего
содержания. Как существенная разность эти разные содержания суть простые,
соотносящиеся с собою определения содержания. Но в то же время ни одно из
них не непосредственно само по себе, а каждое из них есть по существу
{\em положенность}, или, иначе говоря,
{\em есть лишь постольку, поскольку есть другое}.

{\em В-третьих}, явление и закон имеют одно и то же
содержание. Закон есть {\em рефлексия} явления в
тождество с собой; таким образом, явление, как ничтожное
{\em непосредственное}, противостоит
{\em рефлектированному-в-себя}, и они различаются между
собой по этой форме. Но рефлексия явления, в силу которой эго различие
имеет бытие, есть также и существенное тождество самого явления и его
рефлексии, в чем вообще и состоит природа рефлексии; она есть то, что в
положенности тождественно с собой, и оказывается безразличной к указанному
различию, которое есть форма или положенность, есть, следовательно, такое
содержание, которое {\em непрерывно продолжается} из
явления в закон, — содержание и закона и явления.

Это содержание составляет тем самым {\em основу}
явления; закон есть сама эта основа; явление есть то же самое содержание,
но оно содержит в себе еще кое-что сверх того, а именно, несущественное
содержание своего непосредственного бытия. А именно, также и то определение
{\em формы}, которым явление как таковое различается от
закона, представляет собою некоторое {\em содержание},
и притом такое содержание, которое отлично от содержания закона. Ибо
существование, как непосредственность вообще, равным образом есть такая
себетождественность материи и формы, которая безразлична к своим
определениям формы и потому есть содержание; оно есть вещность с ее
свойствами и материями. Однако оно есть то содержание, самостоятельная
непосредственность которого вместе с тем есть лишь отсутствие устойчивого
наличия. Но его тождество с собой в этом его отсутствии устойчивого наличия
есть другое, существенное содержание. Это тождество, основа явления,
составляющая закон, есть его (явления) собственный момент; это та
положительная сторона существенности, в силу которой существование есть
явление.

Закон находится поэтому не по ту сторону явления, а непосредственно
{\em наличен} в нем; царство законов есть
{\em спокойное} отображение существующего или
являющегося мира. Но, правильнее сказать, что оба суть единая целостность,
и существующий мир сам есть царство законов которое, как простое
тождественное, вместе с тем тождественно с собой также и в положенности или
в разлагающей самое себя самостоятельности существования. Существование
возвращается в закон, как в свое основание; явление заключает в себе и то и
другое —~простое основание и то разлагающее движение являющейся вселенной,
существенностью которого служит основание.

3. Закон есть, следовательно, {\em существенное}
явление; он есть рефлексия явления в себя в его положенности,
{\em тождественное} содержание себя и несущественного
существования. {\em Во-первых}, это тождество закона с
его существованием есть пока что только
{\em непосредственное}, простое тождество, и закон
безразличен к своему существованию; явление обладает еще другим
содержанием, кроме содержания закона. Первое содержание есть, правда,
несущественное содержание и возвращение во второе; но для закона оно есть
некое первое, неположенное им; оно поэтому как содержание
{\em связано} с законом
{\em внешним образом}. Явление есть некоторое множество
дальнейших определений, которые принадлежат «этому» или конкретному и не
содержался в законе, а определены чем-то другим. —
{\em Во-вторых}, то, что явление содержит в себе
разнящегося от закона, определило себя как некоторое положительное или как
некоторое другое {\em содержание}; но оно по существу
есть отрицательное; явлению присущи форма и ее движение как таковые.
Царство законов есть {\em спокойное} содержание
явления; последнее есть то же самое содержание, но изображающееся в
беспокойной смене и как рефлексия в другое. Явление есть закон как
отрицательное, безоговорочно изменяющееся существование,
{\em движение} перехода в противоположное, снятия себя
и возвращения в единство. Эту сторону беспокойной формы или отрицательности
закон не содержит в себе; явление есть поэтому по сравнению с законом
целокупность, ибо оно содержит в себе закон, но также и еще кое-что, а
именно, момент движущей самое себя формы. — Этот недостаток имеется,
{\em в-третьих}, в законе таким образом, что его
содержание есть пока что лишь {\em разное} и тем самым
безразличное в себе; поэтому тождество его сторон друг с другом есть пока
что лишь {\em непосредственное} и тем самым
{\em внутреннее} или, иначе говоря, еще не необходимое
тождество. В законе соединены два определения содержания как существенные
(например, в законе движения падающих тел величина пространства и величина
времени; пройденные пространства относятся между собой, как квадраты
протекших времен); они просто {\em соединены}; это
соотношение есть пока что лишь непосредственное соотношение. Поэтому оно
также пока что есть лишь {\em положенное} соотношение,
как и вообще в явлении непосредственное получило значение положенности.
Существенным единством обеих сторон закона была бы их отрицательность, а
именно то обстоятельство, что одна сторона содержит в ней же самой свою
другую; но это существенное единство еще не выступило в законе. — (Так,
например, в понятии пройденного при падении тела пространства не
содержится, что ему соответствует время как квадрат. Так как падение есть
чувственное движение, то оно есть соотношение времени и пространства; но,
во-первых, в самом определении времени, — т.~е. как последнее берется
согласно представлению о нем, — не содержится, что оно соотносится с
пространством, и наоборот; говорят, что можно очень хорошо представлять
себе время без пространства и пространство без времени; одно,
следовательно, присоединяется внешним образом к другому, каковое внешнее
соотношение есть движение. Во-вторых, безразличный характер носит также и
дальнейшее определение, касающееся того, в каком количественном отношении
находятся друг к другу в движении пространство и время. Закон этого
соотношения познается из опыта; постольку он лишь
{\em непосредственен}; он требует еще
{\em доказательства}, т.~е. опосредствования для
познания, что закон не только {\em имеет место}, но и
{\em необходим}; этого доказательства и его объективной
необходимости закон как таковой не содержит в себе.) —~Закон есть поэтому
лишь {\em положительная} существенность явления, а не
его отрицательная существенность, по которой определения содержания суть
моменты формы, переходят как таковые в свое другое и в себе самих суть
вместе с тем не они, а свое другое. Следовательно, в законе положенность
одной стороны его есть, правда, положенность другой; но их содержание
безразлично к этому соотношению, оно не заключает в самом себе этой
положенности. Закон есть поэтому, правда, существенная форма, но еще не
реальная форма, рефлектированная в свои стороны как содержание.

\paragraph[В. \ Являющийся и в{}-себе{}-сущий мир]{В. \ Являющийся и
в-себе-сущий мир}
\hypertarget{Toc478978747}{}1. Существующий мир поднимается спокойно на
высоту царства законов; ничтожное содержание его многообразного наличного
бытия имеет свое устойчивое наличие в некотором другом; его устойчивое
наличие есть поэтому его разложение. Но в этом другом являющееся также и
сливается с {\em самим собою}; таким образом, явление в
своих сменах есть также и некоторое пребывание, и его положенность есть
закон. Закон есть это простое тождество явления с собой; он есть поэтому
основа, а не основание последнего; ибо он есть не отрицательное единство
явления, а как его простое тождество непосредственное (как абстрактное)
единство, {\em наряду} с которым поэтому имеет место
{\em также} и другое содержание явления. Содержание
есть «это», оно связно внутри себя или, иначе говоря, имеет свою
отрицательную рефлексию внутри самого себя. Оно рефлектировано в нечто
другое; это другое само есть существование явления; являющиеся вещи имеют
свои основания и условия в других являющихся вещах.

Но на самом деле закон есть также и {\em другое явления
как такового} и отрицательная рефлексия последнего как рефлексия в его
другое. Содержание явления, отличное от содержания закона, есть такое
существующее, которое своим основанием имеет свою отрицательность, или,
иначе говоря, которое рефлектировано в свое небытие. Но это
{\em другое}, которое тоже есть некоторое
{\em существующее}, есть равным образом такое
существующее, рефлектированное в свое небытие; оно есть, следовательно,
{\em то же самое}, и являющееся на самом деле
{\em рефлектировано} тут не в другое, а
{\em в себя}; именно эта рефлексия положенности в себя
и есть закон. Но как являющееся, положенность эта по существу
{\em рефлектирована в свое небытие}, или, иначе говоря,
ее тождество само есть по существу также ее отрицательность и ее другое.
Рефлексия явления в себя, т.~е. закон, есть, следовательно, также не только
тождественная основа явления, а явление имеет в законе свою
противоположность, и закон есть его отрицательное единство.

Благодаря этому изменилось определение закона в нем самом. Ближайшим образом
он есть лишь некоторое разное содержание и формальная рефлексия
положенности в себя, так что положенность одной его стороны есть
положенность другой стороны. Но так как он есть также и отрицательная
рефлексия в себя, то его стороны относятся друг к другу не только как
разные, но и как отрицательно соотносящиеся друг с другом. — Или, иначе
говоря, если рассматривать закон только сам по себе, то стороны его
содержания суть безразличные друг к другу; но вместе с тем они суть снятые
через свое тождество; положенность одной стороны есть положенность
{\em другой}; стало быть, устойчивое наличие каждой из
них есть также и {\em отсутствие устойчивого наличия ее
самой}. Эта положенность одной из них в другой есть их отрицательное
единство, и каждая из них есть {\em не только
положенность себя, но также и положенность другой}, или, иначе говоря,
каждая есть сама это отрицательное единство. То положительное тождество,
которое они имеют в законе как таковом, есть пока что лишь их
{\em внутреннее} единство, нуждающееся в
{\em доказательстве} и
{\em опосредствовании}, потому что это отрицательное
единство еще не положено в них. Но так как разные стороны закона теперь
определены как такие стороны, которые разны в их отрицательном единстве
или, иначе говоря, как такие, из которых каждая содержит свое другое в себе
самой и вместе с тем, как нечто самостоятельное, отталкивает от себя это
свое инобытие, то тождество закона есть теперь также
{\em положенное} и {\em реальное}
тождество.

Таким образом, закон тем самым приобрел также и недостававший ему момент
отрицательной формы своих сторон, момент, который ранее того принадлежал
еще к явлению; существование, стало быть, полностью возвратилось в себя и
рефлектировало себя в свое абсолютное, в-себе-и-для-себя-сущее инобытие.
То, что ранее было законом, поэтому уже больше не есть лишь
{\em одна} из сторон того целого, другой стороной
которого было явление как таковое, но само есть целое. Оно есть
существенная тотальность явления, так что оно теперь содержит в себе также
и момент несущественности, который раньше принадлежал к явлению, но
содержит его, как рефлектировавшую в-себе-сущую несущественность, т.~е. как
{\em существенную отрицательность}. — Как
непосредственное содержание, закон {\em определен}
вообще, отличается от других законов, и их существует не поддающееся
определению множество. Но так как теперь он имеет существенную
отрицательность в нем самом, то он уже больше не содержит в себе такого
лишь безразличного, случайного определения содержания, а его содержание
есть вообще всякая определенность в существенном соотношении, делающем себя
тотальностью. Таким образом, рефлектированное в себя явление есть тетерь
некоторый {\em мир}, который
{\em возвышается} над
{\em являющимся миром}, как
{\em сущий в себе и для себя}.

Царство законов содержит в себе лишь простое, неизменное, но разнообразное
содержание существующего мира. Но так как оно теперь есть тотальная
рефлексия последнего, то оно содержит в себе также и момент несущественного
многообразия существующего мира. Этот момент изменчивости и изменения, как
рефлектированный в себя, существенный, есть абсолютная отрицательность или
форма вообще как таковая; однако в сущем в себе и для себя мире моменты
этой формы обладают реальностью самостоятельного, но рефлектированного
существования, равно как и наоборот, эта рефлектированная самостоятельность
обладает теперь формой в себе самой, и тем самым ее содержание есть не
просто многообразное, а и существенно связное внутри себя.

Этот в-себе-и-для-себя-сущий мир называется также
{\em сверхчувственным миром}, поскольку существующий
мир определяется как {\em чувственный}, т.~е. как
такой, который есть для созерцания, для непосредственного отношения к нему
сознания. — Сверхчувственный мир тоже обладает непосредственностью,
существованием, но рефлектированным, существенным существованием.
{\em Сущность} не имеет еще никакого наличного бытия,
но она {\em есть}, и в более глубоком смысле, чем
бытие; {\em вещь} есть начало рефлектированного
существования; она есть некоторая непосредственность, которая еще не
{\em положена} как существенная или рефлектированная;
она, однако, по правде не есть некоторое {\em сущее}
непосредственное. Лишь как вещи другого, сверхчувственного мира, вещи
впервые положены, во-первых, как истинные существования, и, во-вторых, как
истинное по сравнению с сущим; в них признается, что имеется отличное от
непосредственного бытия бытие, которое и есть истинное существование. С
одной стороны, в этом определении преодолено чувственное представление,
приписывающее существование лишь непосредственному бытию, служащему
объектом чувства и созерцания; но, с другой стороны, в нем преодолевается
также и бессознательная рефлексия, которая, хотя и имеет представление о
{\em вещах, силах, внутреннем}, и~т.~д., но не знает,
что такие определения суть не чувственные или сущие непосредственности, а
рефлектированные существования.

2. В-себе-и-для-себя-сущий мир есть целокупность существования; вне его нет
ничего другого. Но так как он в нем самом есть абсолютная отрицательность
или форма, то его рефлексия в себя есть
{\em отрицательное соотношение} с собой. Он содержит в
себе противоположность и отталкивает себя внутрь себя как в существенный
мир, и внутрь себя же как в мир инобытия или мир явления. Таким образом, он
потому, что он есть целокупность, имеет бытие также и как лишь
{\em одна сторона} последней и образует в этом
определении разнящуюся от мира явлений самостоятельность. Являющийся мир
имеет в существенном мире свое отрицательное единство, в котором он идет ко
дну и в которое он возвращается, как в свое основание. Далее, существенный
мир есть также и полагающее основание являющегося мира; ибо его тождество с
собой, содержа в себе абсолютную форму в ее существенности, снимает себя,
превращает себя в положенность и как эта положенная непосредственность есть
являющийся мир.

Он, далее, есть не только вообще основание являющегося мира, а его
{\em определенное} основание. Уже как царство законов
он есть многообразное и притом существенное
{\em содержание} являющегося мира, а как содержательное
основание он есть {\em определенное} основание
{\em другого} мира, но лишь по этому содержанию; ибо
являющийся мир обладал еще более многообразным другим содержанием, чем то
царство законов, так как ему был еще своеобразно присущ отрицательный
момент. Но так как теперь царство законов содержит в себе также и этот
момент, то оно есть целокупность содержания являющегося мира и основание
всего его многообразия. Однако существенный мир есть вместе с тем отрицание
являющегося мира и, таким образом,
{\em противоположный} последнему мир. — А именно, в
тождестве обоих миров, поскольку один по форме определен, как существенный,
а другой, как тот же самый мир, но только положенный и несущественный,
восстановилось, правда, {\em соотношение основания}, но
оно восстановилось вместе с тем, как {\em соотношение
основания явления}, а именно, но как соотношение некоторого тождественного
содержания и не как соотношение некоторого просто разного, каков закон, а
как тотальное соотношение, или, иными словами, как отрицательное тождество
и {\em существенное соотношение содержания как
противоположного}. — Царство законов состоит не только в том, что
положенность одного содержания есть положенность некоторого другого, а еще
и в том, что это тождество есть по существу, как это выяснилось, также и
отрицательное единство; в отрицательном единстве каждая из двух сторон
закона есть {\em в ней же самой ее другое} содержание;
другое поэтому не есть неопределенно некоторое другое вообще, а оно есть
{\em ее} другое, или, иначе говоря, оно содержит в себе
также и определение содержания первой; таким образом, эти две стороны суть
противоположные. Так как царство законов имеет теперь в себе этот
отрицательный момент и противоположность и тем самым отталкивает себя, как
целокупность, от самого себя, раскалывая себя на в-себе-и-для-себя-сущий и
являющийся миры, то тождество их обоих есть
{\em существенное соотношение противоположения}. —
Соотношение основания как таковое есть противоположность, пошедшая ко дну в
своем противоречии, а существование есть сливающееся
{\em с самим собой} основание. Но существование
становится явлением; основание снято в существовании; оно снова
восстановляет себя, как возвращение явления в себя; однако здесь оно вместо
с тем выступает как снятое основание, а именно, как соотношение основания
противоположных определений; но тождество таковых определений есть по
существу становление и переход, а уже не соотношение основания как таковое.

В-себе-и-для-себя-сущий мир сам есть, следовательно, такой мир, который
внутри себя диференцирован в целокупность многообразного содержания; он
тождественен с являющимся или положенным миром и постольку есть его
основание; но тождественная связь обоих миров определена вместе с тем и как
противоположение, потому что форма являющегося мира есть рефлексия в его
инобытие, и он, следовательно, в сущем в себе и для себя мире поистине
настолько возвратился в себя самого, насколько этот мир ему противоположен.
Соотношение поэтому определенно следующее: в-себе-и-для-себя-сущий мир есть
{\em изнанка} являющегося.

\paragraph[С. \ Разложение явления]{С. \ Разложение явления}
\hypertarget{Toc478978748}{}В-себе-и-для-себя-сущий мир есть
{\em определенное} основание являющегося мира и служит
им лишь постольку, поскольку он есть в нем самом отрицательный момент и тем
самым та целокупность определений содержания и их изменений, которая
соответствует являющемуся миру, но вместе с тем составляет его полностью
противоположную сторону. Эти два мира относятся, следовательно, друг к
другу так, что то, что в являющемся мире положительно, в сущем в себе и для
себя мире отрицательно, и, наоборот, то, что в том мире отрицательно, в
этом мире положительно. Северный полюс в являющемся мире есть
{\em в себе и для себя} южный полюс и наоборот;
положительное электричество есть {\em в себе}
отрицательное и~т.~д. То, что в являющемся наличном бытии есть моральное
зло, несчастье и~т.~д., есть {\em в себе и для себя}
добро и счастье\footnote{ Ср. «Phänomenologie des Geistes», стр. 121 и сл.
\label{bkm:Ref474667017}.}.

На самом же деле как раз в этой противоположности обоих миров
{\em исчезло их различие}, и то, что должно было быть
в-себе-и-для-себя-сущим миром, само есть являющийся мир, а последний,
наоборот, есть в нем самом существенный мир. —
{\em Являющийся мир} ближайшим образом определен как
рефлексия в инобытие, так что его определения и существования имеют свое
основание и устойчивое наличие в некотором другом; но так как это другое
равным образом есть такое же {\em рефлектированное в
некоторое другое}, то они соотносятся в нем лишь с некоторым снимающим себя
другим и, стало быть, {\em с собой самим}; являющийся
мир тем самым есть {\em в нем самом} равный самому себе
закон. — Наоборот, сущий в себе и для себя мир есть ближайшим образом
тождественное с собой содержание, изъятое из-под власти инобытия и
изменения; но так как это содержание есть полная рефлексия являющегося мира
в себя самого, или, иначе говоря, так как его разность есть
рефлектированное в себя и абсолютное различие, то оно [это свободное от
инобытия содержание] заключает в себе отрицательный момент и соотношение с
собой как с инобытием; оно становится вследствие этого противоположным себе
самому, переворачивающим себя наизнанку, лишенным существенности
содержанием. Далее, это содержание в-себе-и-для-себя-сущего мира тем самым
получило также и форму {\em непосредственного
существования}. Ибо в-себе-и-для-себя-сущий мир есть ближайшим образом
основание являющегося мира; но так как он имеет противоположение в нем же
самом, то он есть вместе с тем снятое основание и непосредственное
существование.

Являющийся и существенный миры суть поэтому каждый в нем самом целокупность
тождественной с собой рефлексии и рефлексии в другое, или, иначе говоря,
в-себе-и-для-себя-бытия и явления. Они оба суть самостоятельные целые
существования; один должен был быть лишь рефлектированным, а другой
—~непосредственным существованием; но каждый
{\em непрерывно продолжается} в его другом и есть
поэтому в себе самом тождество обоих этих моментов. Имеется, стало быть,
эта целокупность, отталкивающая себя от себя самой и становящаяся двумя
целокупностями, из которых одна есть
{\em рефлектированная} целокупность, а другая
—~{\em непосредственная}. Оба мира суть прежде всего
самостоятельные, но они таковы лишь как целокупности, и таковы постольку,
поскольку каждый существенным образом содержит в себе момент другого.
Различенная самостоятельность каждого из этих миров, того, который
определен как {\em непосредственный}, и того, который
определен как {\em рефлектированный}, положена поэтому
теперь так, что каждый есть лишь существенное соотношение с другим и имеет
свою самостоятельность в {\em этом единстве обоих}.

Исходным пунктом послужил {\em закон явления}; он есть
тождество некоторого разного содержания с некоторым другим содержанием, так
что положенность одного есть положенность другого. Закону присуще пока что
то отличительное свойство, что тождество его сторон есть еще лишь некоторое
внутреннее тождество, и эти стороны еще не имеют тождества в них самих. Тем
самым это тождество, с одной стороны, не реализовано, содержание закона
имеет бытие не как тождественное, а как безразличное, разное содержание; с
другой стороны, оно тем самым лишь {\em в себе}
определено таким образом, что положенность одного есть положенность
другого; этого в нем еще нет налицо. Теперь же
{\em закон реализован}; его внутреннее тождество есть
вместе с тем налично-сущее тождество, и, наоборот, содержание закона
возведено в идеальность; ибо оно есть в нем самом снятое, рефлектированное
в себя содержание, так как каждая сторона содержит в себе свою другую
сторону и тем самым поистине тождественна и с ней, и с собой.

Таким образом, закон есть {\em существенное отношение}.
Истина несущественного мира есть ближайшим образом некоторый для него
{\em другой}, сущий в себе и для себя мир; но последний
есть целостность, так как он есть и он сам, и тот первый мир; таким
образом, оба суть непосредственные существования и тем самым рефлексии в
свое инобытие, равно как именно потому также и истинно рефлектированные в
себя. Слово «мир» означает вообще бесформенную целокупность многообразного;
этот мир и как существенный и как являющийся пошел ко дну, так как
многообразие перестало быть просто разным; таким образом, он, хотя и
остался еще целокупностью или вселенной, но как
{\em существенное отношение}. В явлении возникли две
целокупности содержания; они ближайшим образом определены как безразличные
друг к другу самостоятельности, и каждая из них в ней же самой имеет форму,
но не по отношению друг к другу; однако эта форма оказалась также и их
соотношением, и существенное отношение есть завершение единства их формы.

