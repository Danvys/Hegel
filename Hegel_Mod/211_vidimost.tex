\clearpage\section{Вторая Книга. Учение о сущности}
\clearpage\clearpage
{\em Истина бытия} есть
{\em сущность}.

Бытие есть непосредственное. Так как знание хочет познать истину, познать,
что такое бытие {\em в себе и для себя}, то оно не
останавливается на непосредственном и его определениях, а пробирается
сквозь него дальше, исходя из предположения, что
{\em за} этим бытием есть еще нечто другое, чем само
бытие, и что этот задний план составляет истину бытия. Это познание есть
опосредствованное знание, ибо оно не находится непосредственно при и в
сущности, а начинает с некоторого другого, с бытия, и должно пройти
предварительный путь, путь выхождения за пределы бытия или, вернее,
вхождения внутрь его. Только тогда, когда знание, выходя из
непосредственного бытия, углубляется вовнутрь, оно через это
опосредствование находит сущность. — Язык (немецкий.— {\em Перев}.)
сохранил в глаголе «быть» (sein) сущность (das Wesen) в прошедшем времени
(gewesen —~был); ибо сущность есть прошедшее, но вневременно прошедшее
бытие.

Когда мы представляем это шествие, как путь, который проходится знанием, то
это начинание с бытия и дальнейшее движение, снимающее это последнее и
приходящее к сущности как к некоторому опосредствованному, кажутся нам
деятельностью познания, внешней бытию и не имеющей никакого касательства к
его собственной природе.

Но это шествие есть движение самого бытия. В самом бытии обнаружилось, что
оно благодаря своей природе углубляется вовнутрь и через это ухождение в
себя становится сущностью.

Стало быть, если абсолютное было сначала определено как
{\em бытие}, то теперь оно определено как
{\em сущность}. Познание не может вообще остановиться
на многообразном {\em наличном бытии}, но оно не может
также остановиться и на {\em бытии}, на
{\em чистом бытии}; здесь нам непосредственно
напрашивается соображение, что это {\em чистое бытие},
{\em отрицание} всякого конечного, предполагает
{\em углубление вовнутрь} и движение, очистившее
непосредственное наличное бытие путем превращения его в чистое бытие. Бытие
согласно этому определяется как сущность, как такое бытие, в котором
подвергнуто отрицанию все определенное и конечное. Таким образом, оно есть
{\em не имеющее определений}, простое единство, из
которого {\em внешним образом} удалили определенное.
Этому единству само определенное было чем-то внешним, и указанное
определенное еще продолжает стоять наряду с единством также и после этого
удаления; ибо оно было снято не в себе, а лишь относительно, лишь по
отношению к этому единству. —
Выше~\label{bkm:Ref474666794}
уже было указано, что если определяют чистую сущность, как
{\em совокупность всех реальностей}, то эти реальности
равным образом покоряются природе определенности и абстрагирующей рефлексии
и эта совокупность сводится к пустой простоте. Сущность, таким образом,
есть лишь продукт, нечто сделанное. {\em Внешнее
отрицание}, которое есть абстракция, лишь
{\em устраняет} определенности бытия из того, что
остается как сущность;
оно~\label{bkm:Ref474666798}
всегда как бы ставит их лишь в другое место, и как до, так и после этого
устранения оставляет их как сущие. Но взятая таким образом сущность не есть
ни {\em в себе} ни {\em для себя
самой}; она есть {\em через некоторое другое}, через
внешнюю, абстрагирующую рефлексию, и есть {\em для
некоторого другого}, а именно, для абстракции и вообще для продолжающего
противостоять ей сущего. Она поэтому в своем определении есть мертвенное
внутри себя, пустое отсутствие определений.

Но сущность, каковой она стала здесь, есть то, что она есть, не через чуждую
ей отрицательность, а через свое собственное, бесконечное движение бытия.
Она есть {\em в-себе-и-для-себя-бытие} —~абсолютное
{\em в-себе-бытие}, так как она безразлична ко всякой
определенности бытия и так как инобытие и соотношение с другим
безоговорочно были сняты. Но она есть не только это в-себе-бытие: как голое
{\em в-себе-бытие}, она была бы лишь абстракцией чистой
сущности. Она столь же существенно есть и
{\em для-себя-бытие}; она сама есть эта
отрицательность, самоснятие инобытия и определенности.

Сущность как полное возвращение бытия внутрь себя есть, таким образом,
прежде всего неопределенная сущность; определенности бытия в ней сняты: она
содержит их {\em в себе}, но не так, как они
{\em в ней} положены. Абсолютная сущность в этом
простом единстве с собой {\em не обладает наличным
бытием}. Но она должна перейти к наличному бытию; ибо она есть
{\em в-себе-и-для-себя-бытие}, т.~е. она
{\em различает} определения, которые содержатся в ней
{\em в себе}; так как она есть отталкивание себя от
себя самой или, иначе говоря, безразличие в себе,
{\em отрицательное} соотношение с собою, то она тем
самым противополагает себя себе самой и есть лишь постольку бесконечное
для-себя-бытие, поскольку она есть единство с собой в этом своем отличии от
себя. — Этот процесс определения имеет, значит, другую природу, чем процесс
определения в сфере бытия, и определения сущности имеют другой характер,
чем определенности бытия. Сущность есть абсолютное единство в-себе-бытия и
для-себя-бытия; ее процесс определения остается поэтому внутри этого
единства и не есть ни становление, ни переход, равно как самые определения
не суть ни некоторое {\em другое} как другое, ни
соотношения {\em с другим}. Они суть самостоятельные,
но вместе с тем лишь такие самостоятельные, которые находятся в единстве
друг с другом. — Так как сущность есть сначала
{\em простая} отрицательность, то ей приходится теперь
положить {\em в своей} сфере ту определенность, которую
она содержит лишь {\em в себе}, чтобы сообщать себе
наличное бытие, а затем —~свое для-себя-бытие.

Сущность есть в {\em целом} то, чем было
{\em количество} в сфере бытия: абсолютное безразличие
в границе. Но количество есть это безразличие в
{\em непосредственном} определении, и граница в нем
есть непосредственно внешняя определенность, оно
{\em переходит} в определенное количество; внешняя
граница для него необходима и имеет в нем {\em бытие}.
Напротив, в сущности определенность не имеет
{\em бытия}: она только
{\em положена} самой сущностью,
{\em положена} ею не как свободная, а лишь в
{\em соотношении} с ее единством. — Отрицательность
сущности есть {\em рефлексия}, и определения суть
{\em рефлектированные}, положенные самой сущностью и
остающиеся в ней как снятые.

Сущность занимает место между {\em бытием} и
{\em понятием} и составляет их середину, а ее движение
—~{\em переход} от бытия в понятие. Сущность есть
{\em в-себе-и-для-себя-бытие}, но составляет таковое в
определении в-себе-бытия; ибо ее всеобщее определение заключается в том,
что она происходит из бытия или, иначе говоря, есть
{\em первое отрицание бытия}. Ее движение состоит в
том, что она полагает в ней [в самой себе] отрицание или определение,
сообщает себе этим {\em наличное бытие} и становится
как бесконечное для-себя-бытие тем, что она есть в себе. Таким образом, она
сообщает себе свое {\em наличное бытие},
{\em равное} ее в-себе-бытию, и становится
{\em понятием}. Ибо понятие есть абсолютное, как оно
абсолютно в своем наличном бытии, или, иначе говоря, как оно есть в себе и
для себя. Но то наличное бытие, которое сущность сообщает себе, еще не есть
наличное бытие, как оно есть в себе и для себя, а наличное бытие, как его
{\em сообщает} себе сущность, или, иначе говоря, как
его {\em полагают}, и оно поэтому еще отлично от
наличного бытия понятия.

Сущность, во-первых, сначала {\em светится внутри себя
самой}, или, иначе говоря, есть {\em рефлексия};
во-вторых, она {\em является}; в-третьих, она
{\em открывается}. Она полагает себя в своем движении в
следующих определениях:

I. как {\em простую}, в-себе-сущую сущность в ее
определениях внутри себя;

II. как выступающую в наличное бытие, или, иначе говоря, по ее существованию
и {\em явлению};

III. как сущность, которая едина со своим явлением, как
{\em действительность}.

\clearpage\subsection{Первый отдел. Сущность Как Рефлексия В Себе Самой}
Сущность происходит из бытия; она постольку не
есть непосредственно в себе и для себя, а есть некоторый
{\em результат} указанного движения. Или, если возьмем
предварительно сущность как некое непосредственное, то она есть
определенное наличное бытие, которому противостоит другое наличное бытие:
она есть лишь {\em существенное} наличное бытие,
противостоящее {\em несущественному}. Но сущность есть
в себе и для себя снятое бытие; то, что ей противостоит, есть только
{\em видимость}. Но видимость есть собственное
полагание сущности.

Сущность есть, {\em во-первых},
{\em рефлексия}. Рефлексия определяет себя; ее
определения суть некая положенность, которая вместе с тем есть рефлексия в
себя;

{\em во-вторых}, надлежит рассмотреть эти
{\em рефлективные определения} или
{\em определенные сущности} (die Wesenheiten);

{\em в-третьих}, сущность как рефлексия процесса
определения в себя самого делает себя {\em основанием}
и переходит в {\em существование} и
{\em явление}.

\subsubsection{Первая главаВидимость}
Сущность, происходя из бытия, кажется
противостоящей последнему; это непосредственное бытие есть
{\em ближайшим образом несущественное}.

Однако оно, {\em во-вторых}, есть нечто большее, чем
только несущественное, оно есть бытие, лишенное сущности, —
{\em видимость}.

{\em В-третьих}, эта видимость не есть некоторое
внешнее, другое по отношению к сущности, а она есть собственная видимость
сущности. Свечение сущности видимостью (das Scheinen des Wesens) внутри ее
самой есть {\em рефлексия}.

\paragraph[А. \ Существенное и несущественное]{А. \ Существенное и
несущественное}
Сущность есть {\em снятое
бытие}. Она есть простое равенство с самой собой, но постольку, поскольку
она есть отрицание сферы бытия вообще. Таким образом, сущности противостоит
непосредственность, как нечто такое, из чего она возникла и что сохранилось
и удержалось в этом снятии. Сама сущность есть в этом определении
{\em сущая}, непосредственная сущность, и бытие есть
некоторое отрицательное лишь {\em по отношению} к
сущности, а не само по себе; сущность есть, следовательно, некоторое
{\em определенное} отрицание. Бытие и сущность, таким
образом, снова относятся между собою, как {\em другие}
вообще, ибо {\em и то и другая обладают некоторым
бытием}, {\em некоторой непосредственностью},
безразличными друг к другу, и имеют одинаковую ценность cо стороны этого
бытия.

Но вместе с тем бытие противоположно сущности, есть
{\em несущественное}; по отношению, к ней оно имеет
определение снятого. Однако поскольку оно относится к сущности лишь вообще
как некоторое другое, то сущность есть, собственно говоря, не сущность, а
лишь некоторое другим образом определенное наличное бытие,
{\em существенное}.

Различие между существенным и несущественным заставило сущность впасть снова
в сферу {\em наличного бытия}, ибо сущность, какова она
ближайшим образом определена относительно бытия как непосредственное сущее,
и тем самым лишь как {\em другое} по отношению к бытию.
Сфера наличного бытия тем самым положена в основание, и то обстоятельство,
что то, что есть бытие в этом наличном бытии, есть в-себе-и-для-себя-бытие,
— это обстоятельство представляет собою дальнейшее, самому наличному бытию
внешнее определение; равно как и наоборот, сущность есть, правда,
в-себе-и-для-себя-бытие, но лишь по отношению к другому, в
{\em определенном} смысле. — Поэтому, поскольку мы
проводим в некотором наличном бытии различие между
{\em существенным} и
{\em несущественным}, это различие есть внешнее
полагание, не затрагивающее самого наличного бытия отделение одной его
части от другой, отделение, имеющее место в некотором
{\em третьем}. При этом оказывается неопределенным, что
принадлежит к существенному и что к несущественному. Это различие создается
каким-либо внешним соображением и рассуждением, и потому одно и то же
содержание может быть рассматриваемо то как существенное, то как
несущественное.

При более точном рассмотрении оказывается, что сущность становится некоторым
исключительно только существенным, противостоящим некоторому
несущественному, благодаря тому, что сущность берется лить как снятое бытие
или наличное бытие. Сущность есть, таким образом, лишь
{\em первое} отрицание, или, иначе говоря, отрицание,
представляющее собой {\em определенность}, через
которую бытие становится лишь наличным бытием, или наличное бытие —~лишь
некоторым {\em другим}. Но сущность есть абсолютная
отрицательность бытия; она есть само бытие, но не только определенное как
некоторое {\em другое}, а бытие, которое сняло себя и
как непосредственное бытие и как непосредственное отрицание (как отрицание,
обремененное некоторым инобытием). Бытие или наличное бытие тем самым
сохранилось не как другое, чем сущность, и то непосредственное, которое еще
отличается от сущности, есть не просто некоторое несущественное наличное
бытие, но и {\em само по себе} ничтожное
непосредственное; оно есть лишь некоторая
{\em не-сущность} (Unwesen),
{\em видимость}. 

\paragraph[B. \ Видимость]{B. \ Видимость}
1. {\em Бытие есть
видимость}. Бытие видимости состоит единственно только в снятости бытия, в
его ничтожности; эту свою ничтожность оно имеет в сущности и вне своей
ничтожности, вне сущности ее нет. Она есть отрицательное, положенное как
отрицательное.

Видимость есть весь остаток, еще сохранившийся от сферы бытия. Но она
кажется еще обладающей независимой от сущности непосредственной стороной и
представляющей собой вообще некоторое ее (сущности)
{\em другое}. В {\em другом}
содержатся вообще два момента, момент наличного бытия и момент неимения
наличного бытия. Так как несущественное уже больше не обладает бытием, то
ему остается от инобытия лишь {\em чистый момент
неимения наличного бытия}; видимость есть это
{\em непосредственное} неимение наличного бытия,
находящееся в определенности бытия таким образом, что оно обладает наличным
бытием лишь в соотношении с другим, лишь в своем неимении наличного бытия;
она есть несамостоятельное, имеющее бытие лишь в своем отрицании. На долю
несущественного остается, следовательно, только чистая определенность
{\em непосредственности}; оно дано (ist) как
{\em рефлектированная} непосредственность, т.~е. такая
непосредственность, которая имеет бытие лишь {\em через
посредство} своего отрицания и которая по отношению к своему
{\em опосредствованию} есть не что иное, как пустое
определение непосредственности неимения наличного бытия.

Таким образом, {\em видимость} есть феномен
{\em скептицизма} или также явление идеализма, — такая
{\em непосредственность}, которая не есть некое нечто
или некая вещь и вообще не есть такое безразличное бытие, которое
существовало бы вне своей определенности и соотношения с субъектом.
«{\em Есть}» —~этого скептицизм не позволял себе
сказать; новейший же идеализм не позволял себе рассматривать познание, как
некоторое знание о вещи-в-себе; эта видимость, по их воззрению, не имеет
вообще основой некоторое бытие, в эти познания не вступает вещь-в-себе. Но
вместе с тем скептицизм допускал многообразные определения своей видимости,
или, вернее, его видимость имела своим содержанием все многообразное
богатство мира. И точно так же явление идеализма охватывает собою весь
объем этих многообразных определенностей. Видимость скептиков и явление
новейших идеалистов {\em непосредственно} определены
так многообразно. Пусть, стало быть, не лежит в основании этого содержания
никакое бытие, никакая вещь или вещь-в-себе; это содержание само по себе
остается таким, каково оно есть; оно лишь перемещено из бытия в видимость,
так что видимость обладает внутри самой себя теми многообразными
определенностями, которые суть непосредственные, сущие, взаимно другие.
Видимость, следовательно, сама есть некое
{\em непосредственно} определенное. Она может иметь то
или другое содержание; но какое бы содержание она ни имела, оно все равно
положено не ею самой, а она имеет его непосредственно. Лейбницевский или
кантовский, фихтевский идеализм, равно как и другие формы последнего, так
же мало, как и скептицизм, вышли за пределы бытия как определенности, за
пределы этой непосредственности. Для {\em скептицизма}
содержание его видимости ему {\em дано}; для него
является чем-то {\em непосредственным} характер того
содержания, которым обладает его видимость.
{\em Лейбницевская монада} развивает из самой себя свои
представления; но она не есть порождающая и соединяющая сила, а они
возникают в ней, как мыльные пузыри; они безразличны, непосредственны по
отношению друг к другу, а также и по отношению к самой монаде. Точно так же
и {\em кантовское} явление есть некоторое
{\em данное} содержание восприятия; оно (содержание)
предполагает воздействия, определения субъекта, которые по отношению к
самим себе и по отношению к последнему суть непосредственные. Бесконечный
толчок~\label{bkm:Ref474666813}
{\em фихтевского} идеализма не имел, правда, в своем
основании никакой вещи-в-себе, так что он становится исключительно
некоторой определенностью в «я». Но эта определенность есть вместе с тем
для «я», делающего ее своей и снимающего ее внешний характер,
{\em непосредственная} определенность,
{\em предел} «я», за который «я» может выйти, но
который, однако, имеет в себе сторону безразличия, согласно которой он,
хотя сам и есть в «я», все же содержит в себе
{\em непосредственное} небытие последнего. —

2. Видимость, следовательно, содержит в себе некоторую непосредственную
предпосылку, сторону, независимую по отношению к сущности. Но поскольку
видимость отлична от последней, нельзя показать относительно ее, что она
снимает себя и возвращается опять в сущность; ибо бытие целиком
возвратилось в сущность; видимость есть ничтожное в себе; следует только
показать, что определения, отличающие ее от сущности, суть определения
самой сущности, и далее, что та {\em определенность
сущности}, которую представляет собой видимость, снята в самой сущности.

Непосредственность {\em небытия} есть как раз то, что
составляет видимость; но это небытие есть не что иное, как отрицательность
сущности в ней самой. Бытие есть небытие в сущности. Его
{\em ничтожность} в себе есть
{\em отрицательная природа самой сущности}.
Непосредственность же или безразличие, которое содержится в этом небытии,
есть собственное абсолютное в-себе-бытие сущности. Отрицательность сущности
есть равенство последней с самой собой, или, иначе говоря, ее простая
непосредственность и безразличие. Бытие сохранилось в сущности, поскольку
последняя в лице своей бесконечной отрицательности обладает этим равенством
с самой собой; благодаря этому, сущность сама есть бытие.
Непосредственность, которой определенность обладает в видимости по
отношению к сущности, есть поэтому не что иное, как собственная
непосредственность сущности, но не сущая непосредственность, а
безоговорочно опосредствованная, или, иначе говоря, рефлектированная
непосредственность, которую представляет собой видимость, — бытие не как
бытие, а лишь как определенность бытия в противоположность к
опосредствованию:\ \ бытие, как момент.

Эти два момента, ничтожность, но как устойчивое наличие, и бытие, но как
момент, или, иначе говоря, в себе сущая отрицательность и рефлектированная
непосредственность, составляющие {\em моменты
видимости}, суть, стало быть, {\em моменты самой
сущности}; нет видимости бытия в сущности или видимости сущности в бытии;
видимость внутри сущности не есть видимость некоторого другого, а она есть
{\em видимость в себе},
{\em видимость самой сущности}.

Видимость есть сама сущность в определенности бытия. То, вследствие чего
сущность имеет некоторую видимость, состоит в том, что сущность
{\em определена} внутри себя и вследствие этого
отличается от своего абсолютного единства. Но эта определенность вместе с
тем безоговорочно снята в ней самой. Ибо сущность есть самостоятельное,
т.~е. опосредствующее себя с собою через свое отрицание, которое есть она
же сама; она есть, следовательно, тождественное единство абсолютной
отрицательности и непосредственности. — Отрицательность есть
отрицательность в себе; она есть свое соотношение с собой; таким образом,
она есть в себе непосредственность. Но она есть отрицательное соотношение с
собой, отталкивающее отрицание себя самой; таким образом, в-себе-сущая
непосредственность есть отрицательное или
{\em определенное} по отношению к ней. Но эта
определенность сама есть абсолютная отрицательность и тот процесс
определения, который непосредственно как процесс определения, есть снятие
самого себя, возвращение в себя.

Видимость есть отрицательное, обладающее бытием, но в некотором другом, в
своем отрицании; она есть та несамостоятельность, которая в самой себе
снята и ничтожна. Таким образом, она есть возвращающееся в себя
отрицательное, несамостоятельное, как в себе самом несамостоятельное. Это
{\em соотношение} отрицательного или
несамостоятельности {\em с собой} есть его
{\em непосредственность}; это соотношение есть некое
{\em другое}, чем само это отрицательное; оно есть
определенность последнего по отношению к себе; или, иначе говоря, оно есть
отрицание по отношению к отрицательному. Но отрицание по отношению к
отрицательному есть соотносящаяся лишь с собой отрицательность, абсолютное
снятие самой определенности.

Стало быть, та {\em определенность}, которую
представляет собою видимость внутри сущности, есть бесконечная
определенность; она есть лишь сливающееся {\em с собой}
отрицательное; она есть, таким образом, та определенность, которая, как
таковая, есть самостоятельность и не определена. — Обратно,
самостоятельность, как соотносящаяся с собой{\em 
непосредственность}, есть столь же безоговорочно определенность и момент и
дана (ist) лишь как соотносящаяся с собой отрицательность. — Эта
отрицательность, которая тождественна с непосредственностью и, таким
образом, и непосредственность, тождественная с отрицательностью, есть
{\em сущность}. Видимость есть, стало быть, сама
сущность, но сущность в некоторой определенности, однако таким образом, что
эта определенность есть лишь ее момент, и сущность есть излучение своей
видимости внутри самой себя.

В сфере бытия {\em возникает} в противоположность бытию,
как {\em непосредственному}, небытие равным образом,
как {\em непосредственное}, и их истиной служит
становление. В сфере сущности оказываются сначала противопоставленными
сущность и несущественное, а затем —~сущность и видимость, — несущественное
и видимость, как остатки бытия. Но и то, и другая, равно как и отличие
сущности от них, состоят не в чем дальше, как в том, что сущность сначала
берется как некоторое {\em непосредственное}, не так,
как она есть в себе, а именно, не как такая непосредственность, которая
есть непосредственность как чистое опосредствование или, иначе говоря, как
абсолютная отрицательность. Та первая непосредственность есть,
следовательно, лишь {\em определенность}
непосредственности. Снятие этой определенности сущности состоит поэтому не
в чем дальше, как в вскрытии того обстоятельства, что несущественное есть
лишь видимость и что сущность содержит в самой себе видимость как
бесконечное движение внутри себя, которое определяет ее непосредственность
как отрицательность, а ее отрицательность как непосредственность, и, таким
образом, есть излучение своей видимости внутри самой себя. Сущность в этом
своем самодвижении есть {\em рефлексия}.

\paragraph[С. \ Рефлексия]{С. \ Рефлексия}
Видимость есть то же самое, что
{\em рефлексия}; но она есть рефлексия как
{\em непосредственная}; для видимости, ушедшей в себя
и, следовательно, отчудившейся от своей непосредственности, мы имеем
иностранное слово «рефлексия».

Сущность есть рефлексия, движение становления и перехода, остающегося внутри
себя самого, движение, в котором различенное определено безоговорочно лишь
как в себе отрицательное, как видимость.

В становлении бытия лежит в основании определенности бытие, и она есть
соотношение с {\em другим}. Напротив, рефлектирующее
движение есть другое как {\em отрицание в себе},
обладающее бытием лишь {\em как} соотносящееся с собою
отрицание. Или, иначе говоря, так как это соотношение с собой есть как раз
указанное подвергание отрицания отрицанию, то перед нами
{\em отрицание как отрицание}, как нечто такое, что
имеет свое бытие в своей отрицаемости, имеет свое бытие как видимость.
Другое есть здесь, следовательно, не {\em бытие с
отрицанием} или границей, а {\em отрицание с
отрицанием}. Но {\em первое} по отношению к этому
другому, непосредственное или бытие, есть лишь само это равенство отрицания
с собою, подвергшееся отрицанию отрицание, абсолютная отрицательность. Это
равенство с собою или {\em непосредственность} не есть
поэтому некоторое {\em первое}, с которого начинают и
которое затем перешло бы в свое отрицание; равным образом оно не есть и
некоторый сущий субстрат, который двигался бы через рефлексию, проходя
сквозь нее, а непосредственность есть лишь само это движение.

Становление в сущности, ее рефлектирующее движение, есть поэтому
{\em движение от ничто к ничто и вследствие этого
движение назад к самой себе}. Переход или становление снимает себя в своем
переходе; другое, становящееся в этом переходе, не есть небытие некоторого
бытия, а ничто некоторого ничто, и это, т.~е. то обстоятельство, что оно
есть отрицание некоторого ничто, и составляет здесь бытие. — Бытие дано
здесь лишь как движение ничто (des Nichts) к ничто; таким образом, оно есть
сущность; и последняя не {\em имеет} этого движения
{\em внутри себя}, а есть это движение как сама
абсолютная видимость, чистая отрицательность, не имеющая вне себя ничего
такого, что она отрицала бы, а лишь отрицающая само свое отрицательное,
которое имеет бытие только в этом отрицании.

Эта чистая абсолютная рефлексия, которая есть движение от ничто к ничто,
сама определяет себя далее.

Она есть, {\em во-первых},
{\em полагающая рефлексия};

она, {\em во-вторых}, {\em начинает
с предположенного непосредственного} и есть, таким образом,
{\em внешняя} рефлексия;

но, {\em в-третьих}, она снимает это предположение, и
так как она в этом снятии предположения {\em вместе с
тем} сама оказывается предполагающей, то она есть
{\em определяющая} рефлексия.

\subparagraph[1. \ Полагающая рефлексия]{1. \ Полагающая рефлексия}
Видимость есть ничтожное или лишенное сущности;
но ничтожное или лишенное сущности не имеет своего бытия в некотором
{\em другом}, в котором оно светится видимостью, а его
бытие есть его собственное равенство с собой; это чередование
отрицательного с самим собой определилось как абсолютная рефлексия
сущности.

Эта соотносящаяся с собой отрицательность есть, следовательно, подвергание
отрицанию себя самой. Она тем самым есть вообще
{\em настолько же снятая} отрицательность, насколько
она есть отрицательность. Или, иначе говоря, она сама есть и отрицательное
и простое равенство с собой или непосредственность. Она, следовательно,
состоит в том, что она есть {\em она сама и не она
сама} и притом в едином единстве.

Рефлексия есть ближайшим образом движение ничто к ничто и тем самым
сливающееся с собой самим отрицание. Это слияние с собой есть вообще
простое равенство с собой, непосредственность. Но это совпадение не есть
переход отрицания в равенство с собой как переход в свое
{\em инобытие}, а рефлексия есть переход как снятие
перехода; ибо она есть непосредственное совпадение отрицательного
{\em с самим собой}. Таким образом, это слияние есть,
{\em во-первых}, равенство с собой или
непосредственность; но, {\em во-вторых}, эта
непосредственность есть равенство {\em отрицательного}
с собой и тем самым отрицающее само себя равенство; это —~такая
непосредственность, которая есть в себе отрицательное, отрицательное самой
себя, состоящее в том, что она есть то, что она не есть.

Соотношение отрицательного с самим собой есть, следовательно, его
возвращение в себя; оно есть непосредственность как снятие отрицательного,
но представляет собою непосредственность безоговорочно лишь как это
соотношение или как {\em возвращение из чего-либо} (aus
Einem) и тем самым как снимающую саму себя непосредственность. — Это есть
{\em положенность}, непосредственность исключительно
лишь как {\em определенность} или как рефлектирующая
себя. Эта непосредственность, имеющая бытие лишь как
{\em возвращение} отрицательного в себя, есть та
непосредственность, которая составляет определенность видимости и с
которой, как казалось раньше, рефлектирующее движение начинается. Но на
самом деле невозможно начинать с этой непосредственности, и она, наоборот,
имеет бытие лишь как возвращение или как сама рефлексия. Рефлексия есть,
стало быть, такое движение, которое, будучи возвращением, впервые лишь в
этом возвращении есть то, что начинается, или то, что возвращается.

Она есть {\em полагание}, поскольку она есть
непосредственность как возвращение; а именно, здесь нет никакого другого,
ни такого, из которого она возвращалась бы, ни такого, в которое она
возвращалась бы; она, следовательно, имеет бытие лишь как возвращение или
как отрицательное самой себя. Но, далее, эта непосредственность есть снятое
отрицание и снятое возвращение в себя. Как снятие отрицательного рефлексия
есть снятие {\em своего другого}, непосредственности.
Следовательно, так как она есть непосредственность как возвращение, слияние
отрицательного с самим собой, то она есть также и отрицание отрицательного
как отрицательного. Таким образом, она есть
{\em предполагание}. — Или, скажем иначе,
непосредственность как возвращение есть лишь отрицательное ее самой,
состоит лишь в том, что она не есть непосредственность; но рефлексия есть
снятие отрицания самого себя, она есть слияние с собой; она, следовательно,
снимает свое полагание, и, будучи в своем полагании снятием полагания, она
есть предполагание. В предполагании рефлексия определяет возвращение в себя
как отрицательное ее самой, как то, снятие чего есть сущность. Оно есть
свое отношение к самому себе, но к себе как к отрицательному себя; лишь
таким образом оно есть остающаяся внутри себя, соотносящаяся с собой
отрицательность. Непосредственность выступает вообще лишь как возвращение и
есть то отрицательное, которое есть видимость того начала, которое
подвергается отрицанию через возвращение. Возвращение сущности есть, стало
быть, ее отталкивание себя от самой себя. Или, иначе говоря, рефлексия в
себя есть по существу предполагание того, возвращение из чего она
представляет собой.

Именно через снятие своего равенства с собой сущность впервые есть равенство
с собой. Она предполагает самоё себя, и снятие этого предположения есть
сама же она, и, наоборот, это снятие ее предположения есть само же
предположение. — Рефлексия, стало быть,
{\em преднаходит} некоторое непосредственное, за
которое она переходит и возвращение из чего она представляет собой. Но это
возвращение впервые и есть предполагание преднайденного. Это преднайденное
{\em становится} лишь в силу того, что оно
{\em покидается}; его непосредственность есть снятая
непосредственность. — Снятая непосредственность есть, обратно, возвращение
в себя, {\em прибытие} сущности к себе, простое, равное
самому себе бытие. Тем самым это прибытие к себе, есть снятие себя и
отталкивающая от самой себя, предполагающая рефлексия, а ее отталкивание от
себя есть прибытие к самой себе.

Рефлектирующее движение, стало быть, следует понимать как
{\em абсолютный обратный толчок} внутрь самого себя.
Ибо предполагание возвращения в себя —~то, из чего
{\em происходит} сущность, представляющая собою лишь
это возвращение в себя, — имеет бытие лишь в самом возвращении. Выхождение
же за непосредственное, которым начинает рефлексия, имеет бытие впервые
лишь через это выхождение; и выхождение за непосредственное есть прибытие к
нему. Движение как поступательное шествие оборачивается непосредственно в
самом себе, и лишь таким образом оно есть самодвижение, движение, исходящее
из себя, поскольку {\em полагающая} рефлексия есть
{\em предполагающая}, но как
{\em предполагающая} рефлексия есть безоговорочно
{\em полагающая}.

Таким образом, рефлексия есть и она сама и ее небытие, и она есть она сама,
лишь будучи отрицательным себя, ибо только таким образом снятие
отрицательного есть вместе с тем слияние с собою.

То непосредственное, которое она, как снятие, предполагает себе (sich
voraussetzt), имеет бытие безоговорочно лишь как
{\em положенность}, как снятое
{\em в себе}, которое не разнится от возвращения в себя
и само есть только это возвращение. Но оно вместе с тем определено как
{\em отрицательное}, как непосредственно
{\em противостоящее} чему-либо (gegen eines) и,
следовательно, как противостоящее некоторому другому. Таким образом,
рефлексия {\em определена}; поскольку она по этой
определенности {\em имеет} некоторое предположение и
начинает с непосредственного, как с ее другого, постольку она есть
{\em внешняя рефлексия}.

\subparagraph[2. \ Внешняя рефлексия]{2. \ Внешняя рефлексия}
Рефлексия как абсолютная рефлексия есть
сущность, светящаяся видимостью в себе самой, и предполагает себе только
видимость, положенность; она, как пред-полагающая, есть непосредственно
лишь полагающая рефлексия. Но внешняя или реальная рефлексия предполагает
себя как снятую, как отрицательное себя. Она в этом определении двояка:
во-первых, как предполагаемое или, иначе говоря, как рефлексия в себя,
которая есть непосредственное; во-вторых, как отрицательно соотносящаяся с
собой рефлексия; она соотносится с собой, как с тем своим небытием.

Внешняя рефлексия, следовательно, {\em предполагает}
некоторое бытие, {\em во-первых}, не в том смысле, что
его непосредственность есть только положенность или момент, а вернее в том
смысле, что эта непосредственность есть соотношение с собой, и
определенность имеет бытие лишь как момент. Она соотносится со своим
пред-положением таким образом, что последнее есть отрицательное рефлексии,
но таким образом, что это отрицательное, {\em как}
отрицательное, снято. — Рефлексия в своем полагании непосредственно снимает
свое полагание; таким образом, она имеет некоторое
{\em непосредственное предположение}. Она,
следовательно, {\em преднаходит} это полагание как
нечто такое, с чего она начинает, и, лишь исходя из него, она есть
возвращение в себя, отрицание этого своего отрицательного. Но то
обстоятельство, что это предположенное есть отрицательное или полагаемое,
не касается предположенного; эта определенность принадлежит только
полагающей рефлексии, но в предполагании положенность имеет бытие лишь как
снятая. Постольку то, что внешняя рефлексия определяет и полагает в
непосредственном, представляет собой внешние последнему определения. — Она
была в сфере бытия бесконечным, — конечное признается первым, реальным, с
чего, как с лежащего и продолжающего лежать в основании, начинают, а
бесконечное есть противостоящая рефлексия в себя.

Эта внешняя рефлексия есть силлогизм, в котором двумя крайними терминами
служат непосредственное и рефлексия в себя; его средним термином служит
соотношение этих двух крайних терминов, определенное непосредственное, так
что одна часть этого определенного непосредственного, непосредственность,
присуща лишь одному крайнему термину, а другая, определенность или
отрицание, лишь другому.

Но при ближайшем рассмотрении того, что делает внешняя рефлексия,
оказывается, что она есть, {\em во-вторых}, полагание
непосредственного, которое (непосредственное) постольку становится
отрицательным или определенным; но она есть непосредственно также и снятие
этого своего полагания; ибо она {\em предполагает}
непосредственное; она есть в подвергании отрицанию подвергание отрицанию
этого своего подвергания отрицанию. Но тем самым она есть непосредственно
также и {\em полагание}, снятие отрицательного по
отношению к ней непосредственного, и это последнее, с которого, казалось,
она начинала, как с некоторого чуждого, имеет бытие впервые только в этом
ее начинании. Непосредственное есть, таким образом, то же самое, что
рефлексия, не только в себе (что означало бы для нас или во внешней
рефлексии), но и {\em положено}, что оно есть то же
самое. А именно, оно определено рефлексией как ее отрицательное, или, иначе
говоря, как ее другое, но она же сама и подвергает отрицанию этот процесс
определения. — Тем самым внешность рефлексии по отношению к
непосредственному снята, ее отрицающее само себя полагание есть слияние ее
с ее отрицательным, с непосредственным, и это слияние есть сама же
существенная непосредственность. — Выходит, следовательно, что внешняя
рефлексия есть не внешняя, а в такой же степени и имманентная рефлексия
самой непосредственности, или, что то, что имеет бытие через полагающую
рефлексию, есть в-себе-и-для-себя-сущая сущность. Таким образом, она есть
{\em определяющая рефлексия}.

{\centering
Примечание
\par}

Рефлексия обычно понимается в субъективном смысле, а именно, как движение
силы суждения, выходящей за пределы некоторого данного определенного
представления и ищущей для него или сравнивающей с ним всеобщие
определения. Кант противополагает {\em рефлектирующую
силу суждения определяющей} («Kritik der Urteilskraft», Введение, стр.
XXIII и
сл.)~\label{bkm:Ref474666829}.
Он определяет силу суждения вообще, как способность
{\em мыслить особенное как содержащееся под всеобщим}.
Если всеобщее (правило, принцип, закон) дано, то сила суждения, подводящая
под него особенное, есть {\em определяющая}. Если же
дано лишь особенное, {\em к которому она должна
подыскивать всеобщее}, то сила суждения есть лишь
{\em рефлектирующая}~\label{bkm:Ref474666833}.
— Рефлексия, стало быть, есть здесь равным образом выхождение за пределы
некоторого непосредственного к всеобщему. Непосредственное определяется,
как особенное, отчасти только благодаря этому своему соотношению со своим
всеобщим; само же по себе оно есть лишь некоторое единичное или некоторое
непосредственное сущее. Отчасти же то, с чем его соотносят, есть
{\em его} всеобщее, {\em его}
правило, принцип, закон —~вообще рефлектированное в себя, соотносящееся с
самим собой, сущность или существенное.

Но здесь не идет речь ни о рефлексии сознания, ни о более определенной
рефлексии рассудка, имеющей своими определениями особенное и всеобщее, а
говорится о рефлексии вообще. Та рефлексия, которой Кант приписывает
подыскивание всеобщего для данного особенного, есть, как явствует, равным
образом только {\em внешняя} рефлексия, соотносящаяся с
непосредственным, как с некоторым данным. — Но в ней заключается также и
понятие абсолютной рефлексии; ибо всеобщее, принцип или правило и закон, к
которому она далее переходит в своем процессе определения, признается
сущностью того непосредственного, с которого начинают, а тем самым
последнее признается ничтожным, и только возвращение из него, процесс
определения, совершаемый рефлексией, признается полаганием
непосредственного по его истинному бытию; следовательно, то, что рефлексия
делает с ним, и определения, которые исходят от нее, признаются не чем-то
внешним этому непосредственному, а его настоящим бытием.

Внешняя рефлексия имелась также в виду, когда рефлексии, как это некоторое
время было хорошим тоном в новейшей философии, приписывалось вообще все
дурное, и она с ее приемом определения (ihrem Bestimmen) рассматривалась,
как антипод и вечный враг абсолютного способа
рассмотрения~\label{bkm:Ref474666843}.
И в самом деле, и мыслительная рефлексия, поскольку она ведет себя как
внешняя, равным образом безоговорочно исходит из некоторого данного,
чуждого ей непосредственного и рассматривает себя как лишь формальное
действие, которое получает содержание и материю извне, а само по себе есть
лишь обусловленное последней движение. — Далее, как мы в этом тотчас
убедимся ближе при рассмотрении определяющей рефлексии,
{\em рефлектированные определения} суть другого рода,
чем чисто непосредственные определения бытия. Относительно последних легче
соглашаются рассматривать их как преходящие, только относительные,
находящиеся в соотношении с другим; рефлектированные же определения имеют
форму в-себе-и-для-себя-бытия; они поэтому притязают на то, что они суть
{\em существенное}, и вместо того чтобы быть
переходящими в свою противоположность, они представляются, наоборот,
абсолютными, свободными и безразличными друг к другу. Они поэтому упорно
сопротивляются своему движению; их {\em бытие} есть их
тождество с собой в их определенности, по которой они хотя и взаимно
предполагают друг друга, сохраняются в этом соотношении безоговорочно
раздельными.

\subparagraph[3. \ Определяющая рефлексия]{3. \ Определяющая рефлексия}
Определяющая рефлексия есть вообще единство
{\em полагающей} и {\em внешней}
рефлексии. Мы должны это рассмотреть ближе.

1. Внешняя рефлексия начинает с непосредственного бытия,
{\em полагающая} же —~с ничто. Внешняя рефлексия,
становящаяся определяющей, полагает некоторое другое, но это другое есть
сущность, которую рефлексия полагает на место снятого бытия; полагание
полагает свое определение не вместо другого определения, оно не имеет
никакого предположения. Но это не означает, что оно есть завершенная,
определяющая рефлексия; определение, которое оно полагает, есть поэтому
{\em только} некоторое положенное; оно есть
непосредственное, но не как равное самому себе, а как отрицающее себя; оно
находится в абсолютном соотношении с возвращением в себя; оно имеет бытие
только в рефлексии внутрь себя, но оно не есть сама эта рефлексия.

{\em Положенное} есть поэтому некоторое
{\em другое}, но таким образом, что равенство рефлексии
с собой безоговорочно сохраняется; ибо положенное имеет бытие лишь как
снятое, как соотношение с возвращением в себя само. —
{\em В сфере бытия наличное бытие }было бытием, имевшим
отрицание в себе самом, и бытие было непосредственной почвой и стихией
этого отрицания, которое поэтому само было непосредственным.
{\em В сфере сущности} наличному бытию соответствует
{\em положенность}. Она также есть некоторое наличное
бытие, но ее почвой служит бытие как сущность, как чистая отрицательность;
она есть некоторая определенность или отрицание не как сущая, а
непосредственно как снятая. {\em Наличное бытие есть
лишь положенность}, таково суждение (der Satz) сущности о наличном бытии.
Положенность противостоит, с одной стороны, наличному бытию, а, с другой
—~сущности и должна быть рассматриваема как средний термин, смыкающий
наличное бытие с сущностью и обратно —~сущность с наличным бытием. —
Поэтому, когда говорят, что то или другое определение есть
{\em только} некоторая положенность, то это может иметь
двоякий смысл; оно таково или в противоположность наличному бытию или в
противоположность сущности. В первом смысле наличное бытие принимается за
нечто более высокое, чем положенность, и последняя приписывается внешней
рефлексии, субъективному. На самом же деле положенность выше, ибо как
положенность наличное бытие (как то, что оно есть в себе, как
отрицательное) есть нечто всецело соотнесенное только с возвращением в
себя. Поэтому положенность является положенностью
{\em только} по отношению в сущности, как отрицание
возвращенности в себя самое.

2. Положенность еще не есть определение рефлексии; она есть лишь
определенность как отрицание вообще. Но полагание достигло теперь единства
с внешней рефлексией; последняя есть в этом единстве абсолютное
предполагание, т.~е. отталкивание рефлексии от себя самой или, иначе
говоря, полагание определенности как {\em ее самой}.
Положенность есть поэтому как таковая отрицание; но как предположенное оно
(отрицание) есть рефлектированное в себя. Таким образом, положенность есть
{\em определение рефлексии}.

Определение рефлексии отлично от определенности бытия, от качества;
последнее есть непосредственное соотношение с другим вообще; равным
образом, и положенность есть соотношение с другим, но с рефлектированностью
в себя. Отрицание как качество есть отрицание как
{\em сущее}; бытие составляет его основание и стихию.
Напротив, определение рефлексии имеет этим основанием рефлектированность в
себя само. Положенность фиксируется в определение именно потому, что
рефлексия есть равенство с самой собой в своей отрицаемости; ее
отрицаемость поэтому сама есть рефлексия в себя. Определение устойчиво
налично здесь не через бытие, а через свое равенство с собой. Так как
бытие, носящее качество, есть неравное отрицанию, то качество неравно в
самом себе и поэтому представляет собою преходящий, исчезающий в другом
момент. Напротив, определение рефлексии есть положенность
{\em как} отрицание, отрицание, имеющее своим
основанием подвергнутость отрицанию и, следовательно, не неравное себе в
самом себе, а тем самым оно есть {\em существенная}, а
не преходящая определенность. {\em Саморавенство
рефлексии}, в которой отрицательное содержится только как отрицательное,
как снятое или положенное, именно и есть то, что сообщает этому,
отрицательному устойчивое наличие.

{\em В силу этой рефлексии в себя} определения рефлексии
представляются свободными, витающими в пустоте, не притягивающими и не
отталкивающими друг друга {\em определенными
сущностями} (Wesenheiten). В них определенность укрепилась и бесконечно
фиксировалась через соотношение с собой. Это —~определенное, подчинившее
себе свой переход и свою голую положенность, или, иначе говоря, обернувшее
свою рефлексию в другое так, что она стала рефлексией в себя. Указанные
определения образуют вследствие этого определенную видимость, как она имеет
бытие в сущности, — существенную видимость. Поэтому
{\em определяющая рефлексия} есть рефлексия, перешедшая
вовне себя; равенство сущности с самой собой исчезло в отрицание, которое
теперь господствует.

В определении рефлексии имеются, следовательно, две стороны, которые вначале
различаются между собой. {\em Во-первых}, оно есть
положенность, отрицание как таковое; {\em во-вторых},
оно есть рефлексия в себя. Со стороны положенности оно есть отрицание как
отрицание; это, стало быть, уже есть его единство с собой самим. Но оно
таково пока что лишь {\em в себе}; или, иначе говоря,
оно есть непосредственное, как снимающее себя в себе самом, как другое себя
самого. — Постольку рефлексия есть остающийся внутри себя процесс
определения. В ней сущность не выходит вне себя; различия безоговорочно
{\em положены}, вобраны обратно в сущность. Но, с
другой стороны, они суть не положенные, а рефлектированные в себя самих;
отрицание {\em как} отрицание рефлектировало в свое
равенство с самим собой, а не в свое другое, не в свое небытие.

3. Если определение рефлексии есть как соотношение рефлектированное в себя
само, так и положенность, то отсюда непосредственно уясняется ближе его
природа. А именно, как положенность оно есть отрицание, как таковое,
некоторое небытие, противостоящее некоторому другому, а именно
противостоящее абсолютной рефлексии в себя или сущности. Но как соотношение
с собой оно рефлектировано в себя.— Эта его рефлексия и та его положенность
разны; его положенность есть, наоборот, его снятость; его же
рефлектированность в себя есть его устойчивое наличие. Следовательно,
поскольку именно положенность и есть вместе с тем рефлексия в себя самое,
постольку определенность рефлексии есть
{\em соотношение со своим инобытием в себе самой}. —
Она имеет бытие не как некоторая сущая, покоящаяся определенность, которую
соотносили бы с некоторым другим, так что соотнесенное и его соотношение
оказались бы отличными друг от друга, причем первое было бы некоторым
внутри-себя-сущим, некоторым нечто, исключающим из себя свое другое и свое
соотношение с этим другим. Дело обстоит не так, а так, что определение
рефлексии есть в себе самом {\em определенная сторона}
и {\em соотношение} этой определенной стороны, как
определенной, т.~е. соотношение с ее отрицанием. — Вследствие своего
соотношения качество переходит в другое; в его соотношении начинается его
изменение. Напротив, определение рефлексии вобрало в себя обратно свое
инобытие. Оно есть {\em положенность}, отрицание,
которое, однако, поворачивает соотношение с другим обратно внутрь себя,
такое отрицание, которое равно самому себе, представляет собою единство
самого себя и своего другого, и только через это есть
{\em определенная сущность} (Wesenheit). Оно есть,
следовательно, положенность, отрицание, но как рефлексия в себя оно есть
вместе с тем снятость этой положенности, бесконечное соотношение с собой.

