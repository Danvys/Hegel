
Мера определилась в некоторое соотношение мер,
составляющих качество различенных, самостоятельных нечто, выражаясь обычнее
—~{\em вещей}. Только что рассмотренные отношения меры
принадлежат абстрактным качествам, как например, пространству и времени;
примерами тех отношений мер, которые нам теперь предстоит рассмотреть,
служат удельный вес и, далее, химические свойства, представляющие собою
определения {\em материальных} существований.
Пространство и время суть тоже моменты таких мер, однако теперь они
подчинены дальнейшим определениям и уже более не относятся друг к другу
лишь по их собственным понятийным определениям. В звуке, например,
{\em время}, в продолжение которого происходит
известное число колебаний, и {\em пространственные}
определения длины, толщины колеблющегося тела принадлежат к определяющим
моментам; но величины этих идеализованных моментов определены извне; они
теперь уже находятся не в степенном, а в обычном прямом отношении друг к
другу, и гармония сводится к совершенно внешней простоте чисел, отношения
которых всего легче воспринимаются и тем самым доставляют удовлетворение,
всецело принадлежащее области ощущения, так как для духа здесь не
оказывается никакого наполняющего его представления, образа фантазии, мысли
и тому подобного. Так как стороны, образующие теперь отношение меры, сами
суть меры, но вместе с тем также и реальные нечто, то их меры суть
ближайшим образом непосредственные меры и, взятые как отношения в них, —
прямые отношения. Теперь мы должны рассмотреть взаимное отношение таких
отношений в его поступательном определении.

Мера, как оказавшаяся теперь реальной, есть
{\em во-первых}, самостоятельная мера некоторого тела
(einer Körperlichkeit), относящаяся к {\em другим} и в
этом отношении специфицирующая как их, так и тем самым самостоятельную
материальность. Эта спецификация, как внешнее соотнесение со многими
другими вообще, есть процесс порождения других отношений и тем самым других
мер, и специфическая самостоятельность не застревает в
{\em одном} прямом отношении, а переходит в
{\em специфическую определенность}, представляющую
собою {\em ряд мер}.

{\em Во-вторых}, возникающие благодаря этому прямые
отношения суть в себе определенные и исключающие меры (избирательное
сродство); но так как их отличие друг от друга вместе с тем только
количественно, то получается дальнейшее движение отношений, которое отчасти
лишь внешне- количественно, но также и прерывается качественными
отношениями и образует {\em узловую линию специфических
самостоятельных} нечто.

Но, {\em в-третьих}, в этом поступательном движении
появляется в качестве меры {\em безмерность} вообще и,
определеннее, {\em бесконечность} меры, в которой
исключающие друг друга самостоятельности едины между собою и
самостоятельное вступает в отрицательное соотношение с самим собою.


\section[А. Отношение самостоятельных мер]{А. Отношение самостоятельных мер}

Теперь меры признаются уже не просто
непосредственными, а самостоятельными, поскольку они в них самих становятся
отношениями таких мер, которые специфицированы, и, таким образом, в этом
для-себя-бытии суть нечто, — физические, ближайшим образом материальные
вещи. Но целое, представляющее собою отношение таковых мер,

a) само ближайшим образом {\em непосредственно}; таким
образом, обе стороны, которые определены, как таковые самостоятельные меры,
существуют вне друг друга в особых вещах и приводятся в
{\em соединение извне};

b) но самостоятельные материальности суть то, что они представляют собою
качественно, лишь благодаря тому количественному определению, которым они
обладают как меры, стало быть, благодаря тому, что само есть количественное
соотношение с другими, причем они определены как относящиеся к этим другим
{\em по-разному} (так называемое
{\em сродство}), и притом как
{\em члены некоторого ряда} такого количественного
отношения;

c) это безразличное многообразное отношение вместе с тем замыкается,
становится {\em исключающим} для-себя-бытием —~так
называемое {\em избирательное сродство}.

\subsection[а) Соединение двух мер]{а) Соединение двух мер}
Нечто определено внутри себя как отношение меры
определенных количеств, которым, далее, присущи качества, и нечто есть
соотношение этих качеств. Одно качество есть его
{\em внутри-себя-бытие}, по которому оно есть некоторое
для-себя-сущее —~некоторое материальное (например, взятое как интенсивное,
вес, а как экстенсивное, — {\em множество} материальных
частей); другое есть {\em внешность} этого
внутри-себя-бытия (абстрактное, идеализованное, — пространство). Эти
качества определены количественно, и их взаимное отношение составляет
качественную природу материального нечто —~отношение веса к объему,
определенная удельная тяжесть. Объем, идеализованное, должен быть принимаем
за единицу, интенсивное же, выступающее в количественной определенности и в
сравнении с первым как экстенсивная величина, как множество для-себя-сущих
одних, должно быть принимаемо за численность. — Чисто качественное
отношение этих двух количественных определенностей, степенное отношение,
тут исчезло потому, что в самостоятельности для-себя-бытия (материального
бытия) возвратилась та непосредственность, в которой определенность
величины есть некоторое определенное количество как таковое, а отношение
такого определенного количества к другой стороне равным образом определено
обычным показателем прямого отношения.

Этот показатель есть специфическое определенное количество данного нечто, но
он есть непосредственное определенное количество, и последнее (а, стало
быть, и специфическая природа такого нечто) определено лишь при
{\em сравнении} его с другими показателями таких
отношений. Он составляет {\em специфическую в-себе}-определенность —~внутреннюю своеобразную меру нашего нечто; но
так как эта его мера покоится на определенном количестве, то она также есть
лишь внешняя, безразличная определенность, и вследствие этого такое нечто,
несмотря на внутреннее определение меры, изменчиво. Другим, к которому оно
может относиться как изменчивое, служит не некоторое множество материй, не
определенное количество вообще —~против этого его специфическая
в-себе-определенность может устоять, — а такое определенное количество,
которое вместе с тем есть также показатель такого специфического отношения.
В соотношении находятся и вступают в соединение две вещи различной
внутренней меры —~например, два металла различного удельного веса; здесь
нам нет надобности разбирать вопрос о том, какая еще однородность их
природы требуется помимо этого для того, чтобы такое соединение было
возможно, — скажем например, однородность, заключающаяся в том, что то, о
соединении чего с водою могла бы идти речь, не должно быть металлом. — И
вот, с одной стороны, каждая из обеих мер сохраняется в изменении, которое
должно было постигнуть ее вследствие внешности определенного количества,
сохраняется, потому что она есть мера, но с другой стороны, это сохранение
себя само есть отрицательное отношение к этому определенному количеству,
его спецификация, а так как это определенное количество есть показатель
отношения меры, то это самосохранение есть изменение самой меры, а именно,
взаимная спецификация.

Взятое со стороны чисто количественного определения, соединение было бы
голым суммированием двух величин одного качества и двух величин другого
качества, например, суммой двух весов и двух объемов при соединении двух
материй различной удельной тяжести, так что не только вес смеси оставался
бы равным сумме, но и пространство, занимаемое этой смесью, было бы равно
сумме тех двух пространств. Однако лишь вес оказывается суммой весов,
имевшихся до соединения; суммируется лишь та сторона, которая, как
для-себя-сущая, стала прочным существованием (Dasein) и потому обладает
пребывающим непосредственным определенным количеством, — вес материи или
то, что с точки зрения количественной определенности признается
тождественным весу, — множество материальных частей. В показателях же
происходит изменение потому, что они, как отношения меры, суть выражение
качественной определенности, для-себя-бытия, которое, в то самое время,
когда определенное количество как таковое подвергается случайному, внешнему
изменению через суммируемый прибавок, являет себя вместе с тем отрицающим
по отношению к этой внешности. Так как этот имманентный процесс определения
количественно, как было показано, не может проявиться в весе, то он являет
себя в другом качестве, составляющем идеализованную сторону отношения. Для
чувственного восприятия может казаться странным, что после смешения двух
специфически различных материй обнаруживается изменение —~обыкновенно
уменьшение —~суммированного объема; само пространство составляет
устойчивость внеположной материи. Но эта устойчивость против
отрицательности, содержащейся в для-себя-бытии, есть не сущее в себе, а
изменчивое; пространство, таким образом, положено, как то, что оно есть
поистине, — как идеализованное.

Но тем самым не только положена как изменчивая одна из качественных сторон
отношения, а сама мера (и, стало быть, основанная на ней качественная
определенность данного нечто) оказывается устойчивой не в самой себе, а
имеющей, как и определенное количество вообще, свою определенность в других
отношениях меры.

\subsection[b) Мера как ряд отношений мер]{b) Мера как ряд отношений мер}
1. Если бы нечто, соединяемое с другим, а также
и это другое, было бы тем, что оно есть, лишь через определение простым
качеством, то они в этом соединении лишь снимали бы себя. Но то нечто,
которое есть внутри себя отношение меры, самостоятельно, а тем самым оно
вместе с тем соединимо с таким же именно самостоятельным; снимаясь в этом
единстве, оно сохраняется через свое безразличное количественное
существование и ведет себя вместе с тем как специфицирующий момент
некоторого нового отношения меры. Его качество закутано в количественное;
тем самым оно также и безразлично к другой мере, продолжается в ней и в
новообразованной мере; показатель новой меры сам есть лишь какое-либо
определенное количество, внешняя определенность; его безразличие
сказывается в том, что специфически-определенное нечто вступает с другими
такими же мерами в точно такие же отношения нейтрализации, обусловливаемые
взаимоотношением обеих сторон как мер: специфическое своеобразие этого
нечто не может быть выражено в каком-нибудь только
{\em одном} отношении, сторонами которого служит оно
само и какое-нибудь другое нечто.

2. Это соединение с несколькими, которые также суть в них самих меры, дает в
результате разные отношения, имеющие, следовательно, разные показатели.
Самостоятельное нечто имеет показатель своей в-себе-определенности лишь в
сравнении с другими; но нейтрализация с другими и составляет его реальное
сравнение с ними; это —~его сравнение с ними через себя самого. Но
показатели этих отношений —~разные, и, стало быть, оно [то самостоятельное
нечто, о котором идет речь] представляет свой качественный показатель как
{\em ряд} этих {\em разных
численностей}, относительно которых оно есть единица, — как
{\em ряд специфического отношения к другим}.
Качественный показатель, как {\em одно}
непосредственное определенное количество, выражает собою некоторое
отдельное отношение. Поистине же, самостоятельное нечто отличается
{\em своеобразным рядом} показателей, который оно,
принятое за единицу, образует с другими такими самостоятельностями, тогда
как какое-нибудь другое самостоятельное нечто, таким же образом приведенное
в соотношение с теми же самыми самостоятельностями и принятое за единицу,
образует некоторый другой ряд. — Отношение такого ряда внутри его и
составляет качественное в самостоятельном.

Поскольку такое самостоятельное образует с некоторым рядом самостоятельных
ряд показателей, то сначала кажется, что оно отличается от некоторого
другого самостоятельного, которое находится вне данного ряда и с которым
оно {\em сравнивается}, тем, что последнее составляет с
теми же самыми противостоящими некоторый другой ряд показателей. Но таким
путем эти два самостоятельных {\em не} были бы
{\em сравнимы}, поскольку каждое из них
рассматривается, таким образом, как {\em единица}
относительно своих показателей, и те два ряда, которые возникают из этого
соотношения, суть {\em неопределенно другие}. Оба эти
подлежащие сравнению как самостоятельные пока что отличаются друг от друга
лишь как определенные количества; для определения их отношения между собою
требуется в свою очередь некоторая общая им самостоятельная единица. Эту
определенную единицу следует искать только в том, в чем подлежащие
сравнению, как было показано, имеют специфическое наличное бытие своей
меры, следовательно, в том отношении, в котором находятся друг к другу
показатели отношений данного ряда. Но само это отношение показателей есть
лишь постольку самостоятельная, в самом деле определенная единица,
поскольку члены ряда имеют это отношение (как некоторое
{\em константное} отношение между собой) к обеим
сравниваемым самостоятельным нечто; только таким образом оно может быть
{\em их общей единицей}. Следовательно, исключительно
только в ней заключается сравнимость этих двух самостоятельных нечто,
которые принимались не за взаимно нейтрализующиеся друг с другом, а за
безразличные одно к другому. Каждое из них, взятое изолированно, вне
сравнения, есть единица отношений с противостоящими членами, которые суть
численности относительно этой единицы и, стало быть, представляют собой ряд
показателей. Напротив, этот ряд есть, наоборот, единица для тех двух,
которые, будучи сравниваемы между собою, суть определенные количества одно
относительно другого; как таковые, они сами суть разные численности их
только что указанной единицы.

Но, далее, те нечто, которые, с противостоящими сравниваемыми между собою
двумя или, вернее, {\em многими} вообще, дают ряд
показателей их поведения (Verhaltens), сами по себе суть равным образом
самостоятельные; каждое из них есть некоторое специфическое нечто,
обладающее некоторым свойственным ему в себе отношением меры. Постольку их
равным образом следует брать каждое как единицу, так что они в лице только
что названных просто лишь сравниваемых между собою двух (или, лучше
сказать, неопределенно многих) имеют ряд показателей, каковые показатели
суть числа, выражающие собою результат сравнения этих двух или многих друг
с другом; равно как и, наоборот, числа, выражающие собою результат
сравнения между собою тех нечто, которые теперь также и в отдельности
берутся как самостоятельные, равным образом представляют собою ряд
показателей для членов первого ряда. Обе стороны суть, таким образом, ряды,
в которых каждое число есть {\em во-первых}, единица
вообще относительно противостоящего ему ряда, в котором оно имеет свою
особую определенность (Fürsichbestimmtsein) как некоторый ряд показателей;
{\em во-вторых}, оно само есть один из показателей для
каждого члена противостоящего ряда; и, {\em в-третьих},
оно есть сравнительное число (Vergleichungszahl) для прочих чисел своего
ряда и, как такого рода численность, свойственная ему так же и как
показателю, имеет свою особо определенную (fürsichbestimmte) единицу в
противостоящем ряде.

3. В этом отношении (in diesem Verhalten) получается возврат к тому виду и
способу, каким определенное количество положено как для-себя-сущее, а
именно, как градус, положено быть простым, но иметь определенность величины
в некотором вне его сущем определенном количестве, которое представляет
собою некоторый круг определенных количеств. Но в мере это внешнее есть не
просто лишь определенное количество и круг определенных количеств, а ряд
сравнительных чисел, и именно в их совокупности и заключается особая
определенность (Fürsichbestimmtsein) меры. Подобно тому, как это имеет
место касательно для-себя-бытия определенного количества как градуса,
природа самостоятельной меры превратилась в эту внешность себя самой. Ее
соотношение с собою есть ближайшим образом
{\em непосредственное} отношение, и тем самым ее
безразличие к другому сразу же оказывается состоящим лишь в определенном
количестве. В этой внешности заключается поэтому ее качественная сторона, и
{\em ее отношение к другому} становится тем, что
составляет специфическое определение этого самостоятельного. Это
определение таким образом безоговорочно состоит в количественном роде и
способе этого отношения, и этот род и способ определен столь же другим,
сколь и им самим [данным самостоятельным нечто], причем это другое есть
некоторый ряд определенных количеств, и оно само со своей стороны есть
также некоторое определенное количество. Но это соотношение, в котором два
специфических специфицируются в нечто, в некоторое третье, в показатель,
содержит в себе, далее, то, что одно не перешло в нем в другое, и,
следовательно, не только {\em одно} отрицание вообще,
но и {\em оба} в нем положены отрицательно, а так как
каждое при этом сохраняет себя безразличным, то
{\em его отрицание} также снова
{\em подвергается отрицанию}. Это их качественное
единство есть тем самым для-себя-сущее,
{\em исключающее} единство. Показатели, которые
ближайшим образом суть сравнительные между собою числа, обладают в них друг
относительно друга истинно специфической определенностью лишь и впервые в
моменте исключения, и их различие таким образом получает вместе с тем
качественную природу. Различие это, однако, основывается на количественном.
Самостоятельное, {\em во-первых}, относится к некоторой
{\em множественности} своей качественно другой стороны
лишь потому, что оно в этом своем отношении вместе с тем безразлично;
{\em во-вторых}, нейтральное соотношение благодаря
содержащейся в нем количественности есть теперь не только изменение, а
положено как отрицание отрицания и представляет собою исключающее единство.
Благодаря этому {\em сродство} некоторого
самостоятельного с тем множеством, которое составляет другую сторону
отношения, уже более не есть безразличное соотношение, а представляет собою
некоторое {\em избирательное сродство}.

\subsection[с) Избирательное сродство]{с) Избирательное сродство}
Здесь мы употребляем выражение
«{\em избирательное сродство}» точно так же, как раньше
употребляли выражения «нейтральность», «сродство», — выражения, касающиеся
{\em химического} отношения. Ибо в химической области
материальное имеет по существу свою специфическую определенность в
соотношении со своим другим; оно существует лишь как это различие
(Differenz). Это специфическое соотношение связано далее с количеством и
есть вместе с тем не только соотношение с одним отдельным другим, а с целым
рядом таких противостоящих ему различных; соединения с этим рядом покоятся
на так называемом {\em сродстве с каждым} из его
членов, но при таком безразличии каждое соединение есть вместе с тем
исключающее относительно других; каковое соотношение противоположных
определений нам еще предстоит рассмотреть.

Но не только в химической области специфическое выявляет себя (stellt sich
dar) в некотором круге соединений; отдельный тон также получает свой смысл
лишь в соотношении и соединении с некоторым другим и рядом других; гармония
или дисгармония в таком круге соединений составляет его качественную
природу, которая вместе с тем покоится на количественных отношениях,
образующих некоторый ряд показателей и представляющих собою отношения тех
двух специфических отношений, которые каждый из соединенных тонов есть в
нем самом. Отдельный тон есть основной тон некоторой системы, но в свою
очередь также и один из членов в системе каждого другого основного тона.
Гармонии суть исключающие избирательные сродства, качественное своеобразие
которых, однако, вместе с тем снова разрешается в носящее внешний характер,
лишь количественное поступательное движение.

Но в чем заключается принцип меры по отношению к тем средствам, которые
(будь они химические или музыкальные или какие-либо другие) суть
избирательные сродства среди других и в противоположность к другим? Об этом
в дальнейшем будет еще сказано в примечании о химическом сродстве; но этот
высший вопрос теснейшим образом связан со специфической чертой собственно
качественного и должен рассматриваться в особенной части конкретного
естествознания.

Поскольку член некоторого ряда имеет свое качественное единство в своем
отношении к целому некоторого противостоящего ряда, но члены последнего
разнятся друг от друга лишь тем определенным количеством, по которому они
нейтрализуются с этим членом, — постольку более специальная определенность
этого многообразного сродства есть равным образом лишь количественная
определенность~\pagenote{Здесь, как и в
предыдущем разделе «Мера как ряд отношений мер», Гегель имеет в виду учение
шведского химика Торберна Бергмана (1735–1784) о количественном выражении
сродства между основаниями и кислотами. Бергман предполагал, что одно и то
же количество какого-нибудь химического основания тем больше требует
кислоты для своего насыщения или нейтрализации, чем больше у них сродства
друг с другом. Он нашел, что для насыщения, например, 100 весовых частей
едкого кали требуется 78,5 весовых частей серной кислоты, или 64 весовых
частей азотной кислоты, или 51,5 весовых частей соляной кислоты и~т.~д.;
для насыщения же 100 весовых частей едкого натра нужно 177 весовых частей
верной кислоты, либо 135,5 весовых частей азотной кислоты, либо 125 весовых
частей соляной кислоты и~т.~д. В том и другом случае {\em порядок}
кислот остается {\em один и тот же}. Получается некоторой ряд пропорций
или мер насыщения (нейтрализации), который, по Гегелю, и характеризует
собой специфическую природу исследуемого вещества, выступающего в качестве
{\em противочлена} этого ряда. Учение Бергмана о химическом сродстве и
его количественном выражении было господствующей теорией в последней
четверти XVIII в. В начале XIX в. появилась новая теория химического
сродства, связанная с именем французского химика Клода Бертоллэ
(1748–1822), в значительной мере направленная против теории Бергмана.
Бертоллэ считал, что, наоборот, чем меньшее количество вещества $А$
требуется для нейтрализации вещества $В$, тем больше сродство между
ними. Кроме того, также и числовые значения мер насыщения, найденные
Бергманом, оказались при более тщательных экспериментах весьма неточными.
Сочинения Бергмана были изданы в немецком переводе в 1782–1799 гг. и были
широко известны в Германии. Между прочим, от Бергмана идет термин
«attractio electiva» (избирательное притяжение), который в немецком
переводе был передан термином «Wahlverwandtschaft» (избирательное
сродство), употребляемым здесь у Гегеля для обозначения одной из категорий
меры. Подробнее о теориях Бергмана и Бертоллэ см. у {\em Hermann Корр},
Geschichte der Chemie, Neudruck der Originalausgabe, Leipzig 1931, Bd. II,
S. 297–324, откуда и взяты вышеприведенные сведения. — Что касается
«избирательного сродства», как химической категории, то принцип этого
сродства был сформулирован еще задолго до Бергмана французским химиком
Этьеном Жоффруа (1672–1731), который в 1718 г. выставил следующее
положение: «Всякий раз, когда мы имеем соединение двух веществ, обладающих
склонностью соединяться друг с другом, если к этому соединению
примешивается третье вещество, имеющее более сильное сродство с одним из
первых двух, то это третье вещество соединяется с ним, отбивая его от
другого» (цитировано у Коппа, стр.~296 второго тома),~т.~е. указанное
третье вещество разлагает первоначально данное соединение, соединяясь с
одним из компонентов и вытесняя из соединения другой компонент.}.
В избирательном сродстве, как исключающем, качественном соотношении,
отношение изымает себя из этого количественного различия. Ближайшее
представляющееся здесь определение таково: с различием множества,
следовательно, {\em экстенсивной} величины, имеющим
место между членами одной стороны для нейтрализации того или другого члена
другой стороны, сообразуется также и избирательное сродство этого члена с
членами другого ряда, с каждым из которых он находится в сродстве. То
исключение, которое было бы этим обосновано и которое представляло бы собою
{\em более прочное} сцепление, противостоящее другим
возможностям соединения, превращается, казалось бы, в тем большую
{\em интенсивность}, согласно указанному ранее
тождеству форм экстенсивной и интенсивной величины, в каковых обеих формах
определенность величины одна и та же. Но этот переход односторонней формы
экстенсивной величины также и в ее другую форму, в интенсивную величину,
ничего не меняет в природе основного определения, которое есть одно и то же
определенное количество; так что, стало быть, на самом деле не полагалось
бы никакого исключения, а могло бы безразлично иметь место или только
{\em одно} соединение или также и комбинации с
неопределенным числом членов, если бы только порции этих членов, входящие в
соединения, соответствовали требуемому определенному количеству сообразно
их взаимному отношению между собою.

Однако то соединение, которое мы назвали также и нейтрализацией, есть не
только форма интенсивности; показатель есть по существу определение меры и
тем самым исключающий; в этой стороне исключающего отношения числа потеряли
свою непрерывность и способность сливаться друг с другом: определения
«более» и «менее» получают отрицательный характер, и то
{\em преимущество}, которое один показатель имеет перед
другими, не останавливается на количественной определенности. Но в точно
такой же степени имеется также и та другая сторона, по которой тому или
другому моменту опять-таки безразлично получать от нескольких
противостоящих ему моментов нейтрализующее определенное количество, от
каждого сообразно его специфической определенности относительно других;
исключающее, отрицательное отношение испытывает вместе с тем этот ущерб,
наносимый количественной стороной. — Этим положено превращение
безразличного, чисто количественного отношения в качественное и, наоборот,
переход специфической определенности в чисто внешнее отношение, — положен
ряд отношений, которые суть то обладающие чисто количественной природой, то
специфические отношения и меры.


\subsubsection[Примечание Бертоллэ о химическом избирательном сродстве и теория Берцелиуса по этому предмету]
{Примечание Бертоллэ о химическом избирательном сродстве и теория Берцелиуса по этому предмету}

{\em Химические вещества} суть самые характерные
(eigentümlichste) примеры таких мер, которые суть моменты меры, обладающие
тем, что составляет их определение, исключительно в своем отношении к
другим. Кислоты и щелочи или вообще основания представляются
непосредственно в себе определенными вещами; но на самом деле они суть
незавершенные элементы тел, составные части, которые, собственно говоря, не
существуют сами по себе, а имеют лишь то существование, что снимают свою
изолированность и соединяются с другими. И далее, то различие, в силу
которого они существуют как {\em самостоятельные},
состоит не в этом непосредственном качестве, а в количественном роде и
способе отношения. А именно, различие это не ограничивается химической
противоположностью кислоты и щелочи или основания вообще, но
специфицируется в некоторую {\em меру насыщения} и
состоит в специфической определенности количества нейтрализующихся веществ.
Это количественное определение касательно насыщения составляет качественную
природу того или другого вещества; оно делает последнее тем, что оно есть
само по себе, и то число, которое это выражает, есть по существу один из
нескольких показателей для некоторой противостоящей единицы. — Такое
вещество находится с некоторым другим веществом в так называемом сродстве;
поскольку это соотношение сохраняло бы чисто качественную природу,
постольку одна определенность, например в соотношении магнитных полюсов или
электричеств, была бы лишь отрицанием другой и эти две стороны не
оказывались бы вместе с тем также и безразличными друг к другу. Но так как
соотношение имеет также и количественную природу, то каждое из этих веществ
способно нейтрализоваться с {\em несколькими} и не
ограничиваться одним противостоящим ему. Относятся между собою не только
кислота и щелочь или основание, но кислоты и щелочи или основания. Они
характеризуются друг относительно друга ближайшим образом тем, что,
например, одна кислота требует для своего насыщения больше щелочи, чем
другая. Но для-себя-сущая самостоятельность обнаруживается в том, что
сродства относятся между собою исключающим образом и что одно сродство
имеет преимущество перед другим, между тем как, сама по себе взятая,
какая-нибудь кислота может вступать в соединение со всеми щелочами и
обратно. Таким образом, главное различие между одной кислотой и другой
состоит в том, что какая-нибудь одна имеет более близкое сродство с данным
основанием, чем другая, т. е, в так называемом избирательном сродстве.

Относительно химических сродств между кислотами и щелочами найден закон,
согласно которому при смешении двух нейтральных растворов, вследствие чего
получается разложение и образуются два новых соединения, эти продукты
равным образом нейтральны. Отсюда следует, что количества двух щелочных
оснований, потребные для насыщения какой-нибудь кислоты,
{\em в такой же пропорции} необходимы и для насыщения
другой кислоты; вообще, если для одной какой-нибудь щелочи, принятой за
единицу, установлен {\em ряд сравнительных чисел}, в
которых ее насыщают разные кислоты, то для каждой другой щелочи этот ряд
остается одним и тем же, и только разные щелочи должны быть взяты по
отношению друг к другу в разных численностях —~в численностях, которые
опять-таки со своей стороны образуют точно такой же постоянный ряд
показателей для каждой из противостоящих кислот, так как они тоже к каждой
отдельной кислоте относятся в той же пропорции, как и к каждой другой. —
{\em Фишер} первый вывел в их простоте эти
{\em ряды} на основании работ Рихтера (см. его
примечания к переводу статьи {\em Бертоллэ} о законах
сродства в химии, стр. 232 и Berthollet, Statique chimique, I часть, стр.
134 и сл.). — Желание принять во внимание знание сравнительных чисел для
соединений химических элементов, столь всесторонне разработанное с тех пор,
как это впервые было написано, было бы здесь уклонением в сторону также и
потому, что это эмпирическое и к тому же отчасти лишь гипотетическое
расширение сведений продолжает вращаться в тех же определениях понятия. Но
о применяемых при этом категориях и, далее, о воззрениях на само химическое
избирательное сродство и его связь с количественным, равно как и о попытке
обосновать это сродство на определенных физических качествах, мы прибавим
еще несколько замечаний.

Как известно, {\em Бертоллэ} видоизменил общее
представление об избирательном сродстве тем, что внес понятие о действии
{\em некоторой химической массы}. Это видоизменение,
следует заметить, не оказывает никакого влияния на количественные отношения
самих законов химического насыщения, но качественный момент исключающего
избирательного сродства как такового им не только ослабляется, но даже
устраняется. Если две кислоты действуют на одну и ту же щелочь и та из них,
о которой говорят, что она имеет к данной щелочи большее избирательное
сродство, имеется вместе с тем в таком определенном количестве, которое
способно насытить данное определенное количество основания, то согласно
представлению об избирательном
сродстве~\pagenote{Имеется в виду учение Бергмана (см. предыдущее примечание).}
получается только это насыщение; другая кислота остается совершенно
недейственной и исключенной из нейтрального соединения. Согласно же
введенному Клодом Бертоллэ понятию о действии некоторой
{\em химической массы}, каждая из этих двух кислот
оказывает действие в отношении, сторонами которого являются их наличные
количества и их способность насыщения или так называемое сродство.
Исследования Бертоллэ указали те ближайшие обстоятельства, при которых это
действие химической массы устраняется, и одна (более сродная) кислота
вытесняет, повидимому, другую (менее сродную) и
{\em исключает} ее действие,~т.~е. действует в смысле
избирательного сродства. Он показал, что это исключение происходит при
таких {\em обстоятельствах}, как, например, сила
сцепления, нерастворимость образующейся соли в воде, а не благодаря
качественной {\em природе} действующих агентов как
таковой. Действие этих обстоятельств может в свою очередь быть уничтожено
другими обстоятельствами, например, температурой. С устранением этих
препятствий химическая масса начинает действовать, ничем не стесняемая,
 и то, что
казалось чисто качественным исключением, избирательным сродством,
оказывается состоящим лишь во внешних видоизменениях.

Следовало бы, далее, выслушать главным образом то, что сказал об этом
предмете {\em Берцелиус}. Но он в своем «Учебнике
химии»~\pagenote{«Учебник химии» Берцелиуса вышел в трех томах в 1808–1828 гг.}
не дает по этому вопросу ничего своеобразного и более определенного. Он
принимает взгляды Бертоллэ и повторяет их дословно, уснащая их только
своеобразной метафизикой некритической рефлексии, категории которой, стало
быть, только и подлежат ближайшему, рассмотрению. Его теория выходит за
пределы опыта и отчасти придумывает чувственные представления, которые не
даны даже в опыте, отчасти же применяет определения мысли и делает себя с
обеих сторон предметом логической критики. Мы поэтому разберем здесь
сказанное об этой теории в самом вышеуказанном учебнике, III том, 1-й отдел
(перев. Велера, стр. 82 и сл.). Там мы читаем: «Мы
{\em необходимо должны представить} себе, что в
равномерно смешанной жидкости каждый {\em атом}
растворенного тела {\em окружен одинаковым числом
атомов} растворяющего вещества; а если растворены несколько субстанций
вместе, то они должны {\em поделить между собою
промежутки} между атомами растворяющего вещества, так что при равномерном
смешении жидкости возникает такая {\em симметрия в
расположении} атомов, что {\em все атомы} отдельных тел
{\em одинаково расположены по отношению к атомам}
других тел; можно поэтому сказать, что раствор характеризуется
{\em симметрией} в
{\em расположении} атомов, точно так же как
{\em соединение} характеризуется
{\em определенными пропорциями}». — Сказанное
поясняется затем примером соединений, получающихся в растворе хлористой
меди, к которому прибавлена серная кислота; но на этом примере Берцелиус,
разумеется, не доказывает ни того, что {\em атомы}
существуют, ни того, что атомы жидкости {\em окружены}
некоторым числом атомов растворенных тел и что свободные атомы обеих кислот
{\em располагаются} вокруг остающихся связанными
(окисью меди), ни того, что существует {\em симметрия}
в {\em расположении и положении} атомов, ни того, что
имеются промежутки между атомами, — и уже меньше всего доказывается, что
растворенные субстанции {\em поделяют между собою
промежутки} между атомами растворяющего вещества. Это означало бы, что
растворенные вещества занимают место там, где {\em нет}
растворяющего вещества, ибо промежутками последнего являются пространства,
которых оно {\em не наполняет}, и что, стало быть,
растворенные субстанции {\em не} находятся в
растворяющем веществе, а хотя и обволакивают и окружают последнее или
обволакиваются и окружаются им, все же находятся
{\em вне его}, следовательно, несомненно также и не
растворены им. Стало быть, не усматривается, почему нужно составить себе
такие {\em представления}, которые не основаны на
опыте, в основном сразу же оказываются противоречивыми и не подтверждены
каким-нибудь другим образом. Такое подтверждение могло бы получиться только
посредством рассмотрения самих этих представлений,~т.~е. посредством такой
метафизики, которая есть логика; но последняя так же мало подтверждает их,
как и опыт, — как раз наоборот! —~Впрочем, Берцелиус признает, как это тоже
сказано выше, что положения Бертоллэ не противоречат теории определенных
пропорций; он, правда, прибавляет, что они не противоречат также и взглядам
корпускулярной философии,~т.~е. вышеуказанным представлениям об атомах, о
наполнении {\em промежутков} растворяющей жидкости
атомами твердых тел и~т.~д., но эта лишенная всякого обоснования метафизика
не имеет по существу ничего общего с самими пропорциями насыщения.

То специфическое, что выражается в законах насыщения, касается, стало быть,
лишь {\em множества} самих количественных единиц (не
атомов) некоторого тела, с каковым множеством нейтрализуется количественная
{\em единица} (также не атом) другого тела, химически
отличного от первого; разница между ними состоит единственно только в этих
разных пропорциях. Если Берцелиус, несмотря на то, что его учение о
пропорциях всецело представляет собою лишь определение
{\em множеств}, все же говорит затем, например, на стр.
86, также и о {\em степенях} сродства, объясняя
{\em химическую массу} Бертоллэ как сумму
{\em степени сродства} из наличного количества
действующего тела, вместо чего Бертоллэ более последовательно употребляет
выражение {\em capacité }de saturation (способность,
емкость насыщения), то тем самым он сам впадает в форму
{\em интенсивной} величины. Но это —~форма,
составляющая своеобразие так называемой
{\em динамической} философии, которую он раньше, на
стр. 29, называет «спекулятивной философией известных немецких школ» и
определенно отвергает в пользу превосходной «корпускулярной философии». Об
этой динамической философии он там сообщает, что она принимает, что
элементы при своем химическом соединении взаимно
{\em проникают} друг в друга и что нейтрализация
состоит в таком {\em взаимном проникновении}; но это
означает не что иное, как то, что химически различные частицы, выступающие
по отношению друг друга как {\em множество}, сжимаются
в простоту некоторой {\em интенсивной} величины, что
проявляется также и как уменьшение объема. Напротив, согласно
корпускулярной теории атомы и при {\em химическом
соединении} сохраняются в вышеуказанных промежутках,~т.~е. остаются
{\em вне друг друга} (рядоположность); при таком
поведении химических тел исключительно как лишь экстенсивных величин, как
увековеченного {\em множества},
«{\em степень} сродства» не имеет никакого смысла. Если
там же указывается, что явления определенных пропорций оказались совершенно
неожиданными для динамической теории, то это —~лишь внешнее историческое
обстоятельство, не говоря уже о том, что стехиометрические ряды
{\em Рихтера} в изложении Фишера уже были известны
Бертоллэ и приведены в первом издании настоящей «Логики», где показывается 
ничтожность категорий, на которых покоится как старая, так и претендующая на
новизну корпускулярная теория. Но Берцелиус ошибается, утверждая, будто при
господстве «динамического воззрения» явления определенных пропорций
остались бы неизвестными «навсегда», — в том смысле, что указанное
воззрение якобы несовместимо с определенностью пропорций. Последняя есть во
всяком случае лишь определенность величины, причем безразлично, имеет ли
она форму экстенсивной или интенсивной величины, так что даже сам
Берцелиус, хотя он и рьяный сторонник первой формы —~множества, все же
пользуется представлением о степенях сродства.

Поскольку, таким образом, сродство сведено к количественному различию,
постольку от него отказались как от избирательного сродства; имеющее же
место при последнем {\em исключающее} сведено к
обстоятельствам,~т.~е. к таким определениям, которые являются чем-то
внешним сродству, — к сцеплению, нерастворимости получившихся соединений
и~т.~д. С этим представлением можно отчасти сравнить способ рассуждения,
применяемый при рассмотрении действия тяжести, когда то, что
{\em в себе} присуще самой тяжести, а именно, тот факт,
что движущийся маятник благодаря ей необходимо переходит в состояние покоя,
принимается лишь за одновременно существующее обстоятельство внешнего
сопротивления воздуха, нити и~т.~д. и, вместо того чтобы остановку маятника
приписывать тяжести, ее приписывают исключительно только
{\em трению}~\pagenote{Гегель неправ:
причиной остановки маятника является не сила тяжести, а трение в том месте,
где маятник прикреплен или привязан, а также и сопротивление воздуха (или
какой-нибудь другой среды, в которой качается маятник).}.
— Здесь, по отношению к природе того
{\em качественного}, которое присуще избирательному
сродству, не получается разницы от того, что оно выступает и понимается в
форме этих обстоятельств как его условий. Вместе с качественным как таковым
начинается некоторый новый порядок, спецификация которого уже не есть
только количественное различие.

Если, таким образом, различие химического сродства точно устанавливается в
некотором ряде количественных отношений в противоположность избирательному
сродству, как появляющейся качественной
определенности~\pagenote{Немецкий текст
здесь испорчен. Перевод сделан на основе конъектуры Б. Г. Столпнера,
предлагающего вместо «eintretender qualitativer Bestimmtheit» читать
«eintretende qualitative Bestimmtheit». Лассон предлагает другую
конъектуру, вставляя перед указанными словами слова «ein Unterschied». В
этом случае надо было бы перевести эту фразу следующим образом: «Если,
таким образом, различие химического сродства, в противоположность
избирательному сродству, точно устанавливается в некотором ряде
количественных отношений, как различие появляющейся качественной
определенности...» и~т.~д.},
поведение которой отнюдь не совпадает с первым порядком, то это различие
снова полностью запутывается тем способом, каким в новейшее
время~\pagenote{Имеются в виду
электрохимические теории английского химика Дэви (1778—1829) и шведского
химика Берцелиуса (1779–1848), явившиеся крупным шагом вперед в развитии
химии. Гегель недооценивал их значение, так же, как он недооценивал
прогрессивное значение химических теорий Бертоллэ.}
приводятся в связь {\em электрическое} отношение с
химическим, причем устанавливающие эту связь всецело обманываются в своей
надежде на то, что, исходя из этого якобы более глубокого принципа, им
удастся выяснить себе самое главное,~т.~е. отношение меры. Здесь не
приходится рассматривать ближе эту теорию, в которой совершенно
{\em отождествляются} электрические явления с явлениями
химизма, поскольку она касается физической стороны, а не только отношений
меры, и нужно коснуться ее лишь постольку, поскольку ею запутываются
отличительные черты определений меры. Взятая сама по себе, она должна быть
названа поверхностной, ибо поверхностность состоит в том, что, упуская из
виду разницу, принимают разное за тождественное. Что касается сродства, то,
поскольку, таким образом, эта теория отождествляет химические процессы с
электрическими, равно как и с явлениями пламени и света, его сводили «к
нейтрализации противоположных электричеств». Мы можем найти такое почти
комическое изображение тождества электричества и химизма (там же, стр. 63):
«Электрические явления служат, правда, объяснением действия тел на большем
или меньшем расстоянии, их {\em притяжения} до
соединения (т.~е. их {\em еще не} химического
поведения) и возникающего через это соединение
{\em пламени} (?), но {\em не дают
нам никакого ключа} к пониманию причины того, почему
{\em продолжается} с такой великой силой
{\em соединение} тел {\em после
того}, как {\em уничтожается} противоположное
{\em электрическое состояние}»;~т.~е. эта теория
разъясняет нам, что электричество есть причина химического поведения тел,
но электричество не дает никакого ключа к пониманию того, что в химическом
процессе есть химического.

Тем, что химическое различие сводится вообще к противоположности
положительного и отрицательного электричества, различие в сродстве между
агентами, стоящими на той и на другой стороне, определяется как порядок
двух рядов электроположительных и электроотрицательных тел. При
отождествлении электричества и химизма по их всеобщему определению
упускается из виду уже то обстоятельство, что первое вообще и его
нейтрализация {\em не постоянны} и остаются
{\em внешними} качеству тел, химизм же в своем действии
и в особенности в акте нейтрализации {\em захватывает и
меняет всю} качественную природу тел. Точно так же в пределах самого
электричества непостоянна его противоположность как положительного и
отрицательного; она столь непостоянна, что находится в зависимости от
ничтожнейших внешних обстоятельств и не может идти ни в какое сравнение с
определенностью и прочностью, например, противоположности кислот и металлов
и~т.~д. Изменчивость, которая может получить место в химическом поведении
благодаря в высшей степени насильственным воздействиям, например повышенной
температуре и~т.~д., не идет ни в какое сравнение с поверхностным
характером электрической противоположности. Дальнейшее же различие
{\em внутри ряда} каждой из сторон между более или
менее электроположительным или более или менее электроотрицательным
характером есть уже нечто окончательно недостоверное и неконстатированное.
Но из этих рядов тел (Берцелиус в указанном месте, стр. 84 и сл.) «должна
возникнуть в соответствии с их электрическими предрасположениями такая
электрохимическая система, которая лучше всех других способна дать нам
{\em идею о химии}». Он затем указывает эти ряды; но
каковы они на самом деле, об этом он прибавляет на стр. 67 следующее:
«Таков {\em приблизительно} порядок этих тел; но этот
предмет так мало исследован, что нельзя еще утверждать
{\em ничего вполне достоверного} об этом относительном
порядке». — Как сравнительные числа этих (впервые установленных Рихтером)
рядов сродства, так и сделанное Берцелиусом в высшей степени интересное
приведение соединений двух тел к немногим простым количественным отношениям
совершенно независимы от этой, якобы электрохимической, бурды. Если в
установлении этих пропорций и в достигнутом со времени» Рихтера
всестороннем их расширении правильной путеводной звездой служил путь опыта,
то тем более разительный контраст с этими успехами представляет смешение
этих великих открытий с лежащей вне опыта скудостью (Oede) так называемой
корпускулярной теории. Лишь эта попытка покинуть принцип опыта могла
послужить мотивом к тому, чтобы снова подхватить шедшую ранее главным
образом от {\em Риттера} шальную мысль установить
прочные порядки электроположительных и электроотрицательных тел,
долженствующие иметь вместе с тем и химическое значение.

Неверность того предположения, что в основе химического сродства лежит
противоположность электроположительных и электроотрицательных тел, даже в
том случае, если бы взятая сама по себе эта противоположность была
фактически правильнее, чем она есть на самом деле, — неверность этого
предположения вскоре обнаруживается даже экспериментальным путем, причем
это обстоятельство, однако, в свою очередь приводит к дальнейшей
непоследовательности. Берцелиус на стр. 73 (в вышеуказанном сочинении)
признает, что такие два так называемых электроотрицательных тела, как сера
и кислород, образуют между собой гораздо более тесное соединение, чем,
например, кислород и медь, хотя последняя электроположительна. Стало быть,
основа сродства, базирующаяся на всеобщей противоположности отрицательного
и положительного электричеств, должна быть здесь отодвинута на задний план
перед голым «более» или «менее» в пределах одного и того же ряда
электрической определенности. Из этого делают тот вывод, что
{\em степень сродства} тел зависит не только от их
специфической {\em однополярности} (с какой гипотезой
находится в связи последнее определение, этот вопрос не имеет здесь никакой
важности; оно здесь имеет лишь смысл указания:
«{\em либо} положительное,
{\em либо} отрицательное»); степень сродства мы должны
выводить главным образом из {\em интенсивности} их
{\em полярности} вообще. Здесь, стало быть,
рассмотрение сродства переходит более определенно к отношению
{\em избирательного сродства}, которое нас главным
образом интересует; посмотрим же, какие выводы получаются относительно
последнего. Так как сразу же (там же, стр. 73) признается, что
{\em степень} сказанной полярности, если она существует
не только в нашем представлении, {\em не} есть,
по-видимому, {\em постоянная} величина, а весьма зависит
от температуры, то нам согласно всему этому сообщают как вывод, что не
только всякое химическое действие есть {\em в своей
основе} некоторый {\em электрический} феномен, но что
также и то, что кажется действием так называемого
{\em избирательного сродства}, производится
{\em исключительно только более сильной} в одних телах,
чем в других, электрической полярностью. Следовательно, в конечном
результате всего этого кружения в гипотетических представлениях мы остаемся
при категории {\em более сильной интенсивности},
которая есть такая же формальная категория, как избирательное сродство
вообще, и тем, что последнее сводят к большей интенсивности электрической
полярности, мы нисколько не приближаемся более прежнего к какому-нибудь
физическому основанию. Но и то, что здесь, как утверждает Берцелиус, должно
быть определено, как большая специфическая интенсивность, в дальнейшем
сводится им лишь к указанным выше, установленным Бертоллэ модификациям.

Заслуга и слава, которую приобрел {\em Берцелиус}
распространением учения о пропорциях на все химические отношения, не должны
служить основанием для того, чтобы удержать нас от разъяснения слабой
стороны этой теории; но более определенным основанием для того, чтобы
сделать это, должно служить то обстоятельство, что такая заслуга в одной
стороне науки обыкновенно, как это показывает пример
{\em Ньютона}, сообщает
{\em авторитетность} поставленной в связь с нею
необоснованной конструкции из плохих категорий и что именно такая
метафизика провозглашается, а затем и повторяется другими с величайшей
притязательностью.

Кроме тех форм отношений меры, которые связаны с химическим сродством и
избирательным сродством, могут быть рассмотрены также и другие, касающиеся
количеств, окачествующихся в некоторую систему. Химические тела образуют,
что касается насыщения, систему отношений; самое насыщение покоится на
определенной пропорции, в которой соединяются стоящие на одной и другой
стороне количества, имеющие друг относительно друга самостоятельное
материальное существование. Но имеются также и такие отношения меры,
моменты которых нераздельны и не могут быть представлены в собственном,
отличном друг от друга существовании. Эти отношения суть то, что мы выше
назвали {\em непосредственными самостоятельными}
мерами, и их представителями служат {\em удельные
тяжести} тел. — Эти удельные тяжести суть в телах отношения веса к объему;
показатель отношения, выражающий определенность одной удельной тяжести в
отличие от других, есть определенное количество, получающееся лишь
{\em из сравнения}, внешнее им отношение, имеющее место
во внешней рефлексии, не основывающееся на собственном качественном
отношении к противостоящему существованию. Здесь следовало бы поставить
себе задачу познать показатели отношений {\em ряда
удельных тяжестей} как некоторую {\em систему},
вытекающую из {\em правила}, которое специфицировало бы
чисто арифметическую множественность в ряд гармонических узлов. — Такое же
требование должно было бы быть поставлено и познанию указанных выше рядов
химических средств. Но наука еще далека от того, чтобы достигнуть этого,
равно как и от того, чтобы постигнуть в системе мер числа, указывающие
расстояния планет солнечной системы.

Хотя сначала кажется, что удельные тяжести не имеют никакого качественного
отношения друг к другу, они однако вступают также и в качественное
соотношение. Когда тела химически соединяются или даже только
{\em амальгамируются} или смешиваются, то появляется
также и {\em нейтрализация} удельных тяжестей. Выше мы
указали на то явление, что объем даже и смеси остающихся, собственно
говоря, химически безразличными друг к другу материй не равен сумме их
объемов до смешения. Они в этой смеси взаимно изменяют то определенное
количество этой определенности, с которым они вступают в это свое
соотношение, и, таким образом, являют себя качественно сохраняющимися друг
относительно друга. Здесь определенное количество удельной тяжести
проявляется не только как постоянное {\em сравнительное
число}, но и как некоторое {\em относительное число}
(Verhältniszahl), которое может изменяться; и показатели смесей дают ряды
мер, поступательное движение которых определяется принципом, отличным от
относительных чисел соединяемых друг с другом удельных тяжестей. Показатели
этих отношений не суть исключающие определения мер; их поступательное
движение есть нечто непрерывное, но содержит в себе специфицирующий закон,
отличный от тех формально движущихся вперед отношений, в которых
соединяются множества, и делающий первое поступательное движение
несоизмеримым со вторым.


\section[В. Узловая линия отношений меры]{В. Узловая линия отношений меры}

Последним определением отношения меры было то,
что оно как специфическое есть {\em исключающее};
исключение присуще нейтральности, как
{\em отрицательному} единству различенных моментов. Для
этого {\em для-себя-сущего} единства, для
избирательного сродства, что касается его соотношения с другими
нейтральностями, не оказалось никакого дальнейшего принципа спецификации, —
последняя остается лишь при количественном определении сродства вообще,
согласно которому нейтрализуются-то именно определенные множества, которые
тем самым противостоят другим относительным избирательным средствам своих
моментов. Но, далее, в силу основного количественного определения
{\em исключающее} избирательное сродство
{\em продолжается} также и в другие для него
нейтральности, и эта продолжаемость есть не только внешнее соотнесение
разных отношений нейтрализации, как некоторое сравнение, а нейтральность
как таковая имеет в себе некоторую {\em разделимость},
так как те, от объединения которых она произошла, вступают в соотношение
как самостоятельные нечто, каждое как безразлично готовое соединиться, хотя
и в разных, специфически определенных количествах, с тем или другим членом
противостоящего ряда. Вследствие этого указанная мера, покоящаяся на такого
рода отношении в ней самой, обременена собственной безразличностью; она
есть некоторое в самом себе внешнее и в своем соотношении с собою
изменчивое.

Это {\em соотношение} покоящейся на отношении меры (des
Verhältnismasses) {\em с собою} отлично от ее внешности
и изменчивости как ее количественной стороны; мера эта как соотношение с
собою, в противоположность ее количественной стороне, есть некоторая сущая,
качественная основа, — пребывающий, материальный субстрат, который вместе с
тем, как продолжаемость меры в своей внешности, ее непрерывность
{\em с самой собой}, должен содержать в своем качестве
сказанный принцип спецификации этой внешности.

Исключающая мера по этому своему более детальному (nähern) определению,
будучи внешней себе в своем для-себя-бытии, отталкивает себя от себя самой,
полагает себя и как некоторое другое, лишь количественное отношение, и как
такое другое отношение, которое вместе с тем есть другая мера; она
определена как в самом себе специфицирующее единство, которое в самом себе
продуцирует отношения меры. Эти отношения отличны от вышеуказанного вида
сродств, в котором одно самостоятельное относится к самостоятельному
другого качества и к некоторому ряду таковых. Они имеют место
{\em в одном и том же} субстрате, в пределах одних и
тех же моментов нейтральности; мера, отталкиваясь от себя, определяет себя
к другим, лишь количественно разным отношениям, которые тоже образуют
{\em сродства} и {\em меры},
{\em перемежаясь} с такими, которые остаются лишь
{\em количественными разностями}. Они, таким образом,
образуют некоторую {\em узловую линию} мер по шкале
большего и меньшего.

Дано отношение меры, некоторая самостоятельная реальность, качественно
отличная от других. Такое для-себя-бытие, ввиду того, что оно вместе с тем
существенным образом есть некоторое отношение определенных количеств,
открыто для внешности и для количественного изменения; оно имеет известную
ширь, в пределах которой оно остается безразличным к этому изменению и не
изменяет своего качества. Но наступает некоторая точка этого изменения
количественного момента, в которой изменяется качество, определенное
количество оказывается специфицирующим, так что измененное количественное
отношение перешло в некоторую меру и тем самым в некоторое новое качество,
в новое нечто. Отношение, заместившее первое, определено им отчасти с той
стороны, что моменты, находящиеся в сродстве, качественно те же, отчасти же
с той стороны, что здесь имеется количественная непрерывность. Но так как
различие имеет место именно в этой количественной стороне, то новое нечто
относится безразлично к предыдущему; различие между ними есть внешнее
различие определенного количества. Оно появилось поэтому не из предыдущего,
а непосредственно из себя,~т.~е. из внутреннего, еще не вступившего в
наличное бытие специфицирующего единства. — Новое качество или новое нечто
подвергается такому же дальнейшему процессу своего изменения и так далее до
бесконечности.

Поскольку движение от одного качества к другому совершается в постоянной
непрерывности количества, постольку отношения, приближающиеся к некоторой
окачествующей точке, рассматриваемые количественно, различаются лишь как
большее и меньшее. Изменение с этой стороны
{\em постепенно}. Но постепенность касается только
внешней стороны изменения, а не качественной его стороны; предшествующее
количественное отношение, бесконечно близкое к последующему, все еще есть
другое качественное существование. Поэтому с качественной стороны
исключительно количественное постепенное движение вперед, не имеющее
границы в самом себе, абсолютно прерывается; так как появляющееся новое
качество по своему чисто количественному соотношению есть относительно
исчезающего неопределенно другое, безразличное качество, то переход есть
{\em скачок}; оба качества положены как совершенно
внешние друг другу.

Обыкновенно стремятся сделать изменение {\em понятным}
посредством представления о постепенности перехода; но постепенность есть
скорее, наоборот, как раз исключительно только безразличное изменение,
противоположность качественному изменению. В постепенности, скорее,
наоборот, упраздняется связь обеих реальностей, все равно, принимают ли их
за состояния или за самостоятельные вещи; положено, что ни одна из них не
есть граница другой и что они безоговорочно внешни друг другу; тем самым
устраняется как раз то, что требуется для
{\em понимания}, как бы малы ни были требования в этом
отношении.


\subsubsection[Примечание Примеры таких узловых линий; о том, что якобы в природе нет скачков]
{Примечание Примеры таких узловых линий; о том, что якобы в природе нет скачков}

Уже натуральный ряд чисел обнаруживает такую
{\em узловую линию} качественных моментов,
проявляющихся в чисто внешнем поступательном движении. Этот ряд есть, с
одной стороны, чисто количественное движение вперед и назад, постоянное
прибавление или убавление, так что каждое число находится в том же
{\em арифметическом} отношении к своему предшествующему
и последующему, в каком эти последние находятся к своим предшествующим и
последующим, и~т.~д. Но возникающие благодаря этому числа имеют к другим
предыдущим и последующим еще и некоторое
{\em специфическое} отношение: например, одно число
есть кратное другого, будучи больше его в некоторое целое число раз, или
одно число есть корень или степень другого. — В
{\em музыкальных} отношениях в шкале количественного
движения вперед появляется благодаря некоторому определенному количеству
некоторое гармоническое отношение, причем нет того, чтобы это определенное
количество, взятое особо, имело к своему предыдущему и последующему иное
отношение, чем они в свою очередь к своим предыдущим и последующим. Между
тем как последующие тоны кажутся все более и более удаляющимися от
исходного тона или, беря другой пример, числа вследствие арифметического
движения вперед кажутся становящимися лишь еще более иными, вдруг
наступает, наоборот, некоторый {\em возврат},
поражающее соответствие, которое не было качественно подготовлено
непосредственно предыдущим, но является некоторым actio in distans
(действием на расстоянии), отношением к некоторому далеко отстоящему члену;
движение вперед вдоль чисто безразличных отношений, которые не изменяют
предшествующей специфической реальности или даже вообще не образуют
таковой, внезапно прерывается, и между тем как с количественной стороны оно
продолжается по-прежнему, вместе с тем путем скачка возникает некоторое
специфическое отношение.

В {\em химических соединениях} встречаются при
прогрессирующем изменении пропорций смешивания такие качественные узлы и
скачки, что два вещества на определенных точках шкалы смешения образуют
продукты, обнаруживающие особенные качества. Эти продукты отличаются друг
от друга не только количественными моментами, и равным образом они еще не
появляются, хотя бы, скажем, в более слабой степени, вместе с теми
отношениями, которые находятся близко к этим узловым отношениям, а связаны
именно с такого рода узловыми точками. Например, соединения кислорода и
азота дают ряд окислов азота, появляющихся лишь при определенных
количественных отношениях смешения и обладающих существенно различными
качествами, так что на промежуточных точках шкалы смешения не получается
никаких специфических соединений. — {\em Окислы
металлов}, например свинца, образуются на известных количественных точках
шкалы окисления и различаются цветом и другими качествами. Они не переходят
постепенно один в другой; отношения, лежащие между указанными узлами, не
дают никакого нейтрального, никакого специфического существования. Без
того, чтобы переходить сперва через промежуточные ступени, появляется
специфическое соединение, покоящееся на некотором отношении меры и
обладающее собственными качествами. — Или, например,
{\em вода}, при изменении ее температуры не только
становится от этого менее теплой, но и проходит через состояния твердости
капельной и упругой жидкости; эти различные состояния наступают не
постепенно, а постепенный ход изменения температуры вдруг прерывается и
задерживается этими точками, и наступление другого состояния является
скачком. — Всякое {\em рождение} и всякая
{\em смерть} вместо того, чтобы быть продолжающейся
постепенностью, есть, наоборот, перерыв такой постепенности и скачок из
количественного изменения в качественное.

Говорят: {\em в природе не бывает скачков}, и обычное
представление, когда оно хочет постичь некоторое
{\em возникновение} или
{\em прехождение}, полагает, как мы уже сказали выше,
что постигнет их, представляя их себе как
{\em постепенное} происхождение или исчезновение. Но мы
показали, что вообще изменения бытия суть не только переход одной величины
в другую, но и переход качественного в количественное и наоборот,
становление иным, представляющее собою перерыв постепенности и качественно
другое по сравнению с предшествующим существованием. Вода через охлаждение
не становится твердой постепенно, не делается сначала кашеобразной, чтобы
затем, делаясь постепенно все тверже и тверже, достигнуть консистенции
льда, а сразу затвердевает; уже достигнув температуры точки замерзания, она
все еще может полностью сохранить свое жидкое состояние, если она останется
в покое, и ничтожное сотрясение приводит ее в состояние твердости.

В основании предположения о постепенности возникновения лежит представление
о том, что {\em возникающее} уже до своего
возникновения {\em наличествует} чувственно или вообще
{\em в действительности} и лишь вследствие своей
малости {\em еще не может быть воспринимаемо}, равно
как в основании предположения о постепенности исчезновения лежит
представление о том, что {\em небытие} или то
{\em другое}, которое занимает место исчезнувшего,
также уже {\em наличествует}, но
{\em еще не может быть замечено}, и притом наличествует
не в том смысле, что в наличном другом это другое содержится
{\em в себе}, а в том смысле, что оно
{\em наличествует как существование}, но лишь
незаметно. Этим упраздняются вообще возникновение и прохождение, или,
скажем иначе, «{\em в себе}»,~т.~е. то внутреннее, в
котором нечто есть до своего существования (Dasein), превращают в
{\em малость внешнего существования}, и существенное
или понятийное различие —~во внешнее различие, в различие исключительно по
величине. — Способ делать понятным возникновение или прехождение
предположением о постепенности изменения имеет в себе скуку, свойственную
тавтологии; он уже заранее имеет совершенно готовым возникающее или
преходящее и делает изменение изменением исключительно внешнего различия,
так что вследствие этого оно на самом деле есть только тавтология.
Трудность для такого стремящегося к пониманию рассудка заключается в
качественном переходе какого-нибудь нечто в свое другое вообще, и в свою
противоположность; чтобы избегнуть этой трудности, он обманывает себя
представлением о {\em тождестве} и об
{\em изменении}, как о безразличном, внешнем изменении
{\em количественного}.

В области {\em моральной}, поскольку моральное
рассматривается в сфере бытия, имеет место такой же переход количественного
в качественное, и разность качеств оказывается основанной на разности
величин. Количественные изменения являются тем, посредством чего мера
легкомыслия оказывается превзойденной и вместо легкомыслия появляется нечто
совершенно другое, а именно преступление, — являются тем, посредством чего
право переходит в несправедливость, добродетель в порок. — Точно так же
государства при прочих равных условиях получают разный качественный
характер благодаря различию в их величине. Законы и государственный строй
превращаются в нечто другое, когда увеличивается объем государства и
возрастает число граждан. Государство имеет меру своей величины, превзойдя
которую оно неудержимо распадается при том же государственном устройстве,
которое при другом размере составляло его счастье и силу.

\section[С. Безмерное]{С. Безмерное}
Исключающая мера остается даже в своем
реализованном для-себя-бытии обремененной моментом количественного
наличного бытия, а потому способной к восхождению и нисхождению по той
шкале определенного количества, по которой изменяются отношения. Нечто или
некоторое качество, покоящееся на таком отношении, выгоняется за свои
пределы в {\em безмерное} и гибнет только вследствие
изменения своей величины. Величина есть тот характер некоторого
существования, за который оно может быть ухвачено как будто безобидно и
который может привести его к разрушению.

Абстрактно безмерное есть определенное количество вообще как лишенное
определений внутри себя и как лишь безразличная определенность, которою
мера не изменяется. В узловой линии мер эта определенность вместе с тем
положена как специфицирующая. Это абстрактно-безмерное снимает себя,
переходя в качественную определенность; новое отношение меры, в которое
переходит то, которое имелось сначала, есть безмерное по отношению к
последнему, в самом же себе оно тоже есть для-себя-сущее качество; таким
образом положено чередование специфических существований между собою и их
же чередование с отношениями, остающимися чисто количественными, — и~т.~д.
до {\em бесконечности}. Следовательно, что на самом
деле имеется в указанном переходе, — это как отрицание специфических
отношений, так и отрицание самого количественного поступательного движения,
— для-себя-сущее {\em бесконечное}. —
{\em Качественная} бесконечность, каковой мы ее видели
в наличном бытии, была внезапным появлением бесконечного в конечном как
{\em непосредственный переход} и исчезание
посюстороннего в своем потустороннем. Напротив,
{\em количественная} бесконечность есть по своей
определенности уже {\em непрерывность} определенного
количества, {\em продолжаемость} последнего за свои
пределы. Качественно конечное {\em становится}
бесконечным, количественно конечное есть его потустороннее в нем же самом и
{\em указует за свои пределы}. Но эта бесконечность
спецификации меры {\em полагает} и качественное и
количественное как {\em снимающие} себя друг в друге и
тем самым полагает первое, непосредственное их
{\em единство}, которое есть мера вообще, как
возвратившееся в себя и тем самым как то, что само
{\em положено}. Качественное, некоторое специфическое
существование, переходит в другое существование таким образом, что
происходит лишь изменение количественной определенности некоторого
отношения. Изменение самого качественного в [другое] качественное положено,
стало быть, как внешнее и безразличное изменение и как
{\em слияние с самим собою}; количественное же и помимо
этого снимает себя как превращающееся в качественное, во
в-себе-и-для-себя-определенность. Это, таким образом, в смене своих мер
само себя внутри себя продолжающее единство есть то, что поистине остается,
есть сохраняющаяся, самостоятельная {\em материя},
{\em суть} (die Sache).

Следовательно, что здесь имеется, это ({\em $\alpha $})
одна и та же суть, которая положена как основа в своих различениях и как
вековечное (perennierend). Уже в определенном количестве вообще начинается
это отделение бытия от его определенности; нечто обладает
{\em величиной} (ist gross) безразлично к своей сущей
определенности. В мере сама вещь уже есть в себе единство качественного и
количественного, тех двух моментов, которые составляют различие внутри
всеобщей сферы бытия и каждый из которых есть потустороннее другого;
вековечный субстрат имеет, таким образом, прежде всего в самом себе
определение сущей бесконечности. ({\em $\beta $}) Эта
тождественность субстрата {\em положена} в том, что
качественные самостоятельности, в которые растолкнулось определяющее меру
единство, состоят лишь в количественных различиях, так что субстрат
продолжает себя в это свое дифференцирование.
({\em $\gamma $}) В бесконечном прогрессе узлового ряда
положена продолжаемость качественного в количественное движение вперед как
в некоторое безразличное изменение, но также положено содержащееся здесь
{\em отрицание} качественного и одновременно вместе с
тем чисто количественной внешности. Количественное указывание за свою
границу, указывание на некоторое другое как на другое количественное,
исчезает в выступлении некоторой покоящейся на отношении меры, некоторого
качества, а качественный переход упраздняется как раз в том, что само новое
качество есть лишь некоторое количественное отношение. Этот переход
качественного и количественного друг в друга происходит на почве их
единства, и смыслом этого процесса служит лишь
{\em наличное бытие},
{\em указание} или {\em положение},
что в основании этого процесса лежит такого рода субстрат, который есть их
единство.

В рядах самостоятельных отношений мер стоящие на одной стороне члены ряда
суть непосредственные качественные нечто (например, удельные тяжести или
химические вещества, основания или щелочи, кислоты), а затем и их
нейтрализации (под которыми здесь должно разуметь также и соединения
веществ разной удельной тяжести) суть также самостоятельные и даже
исключающие отношения меры, безразличные друг к другу целостности
для-себя-сущего наличного бытия. Теперь такие отношения определены лишь как
узлы одного и того же субстрата. Тем самым меры и положенные с ними
самостоятельности низводятся до {\em состояний}.
Изменение есть лишь изменение некоторого
{\em состояния}, и
{\em переходящее} положено как остающееся в этом
изменении {\em тем же самым}.

Для обозрения пути все дальнейших и дальнейших определений, который пройден
мерой, скажем, что моменты этого пути суть, коротко говоря, следующие: мера
есть прежде всего {\em непосредственное} единство
качества и количества, как некоторое обыкновенное определенное количество,
которое, однако, специфично. Тем самым она, как соотносящаяся не с другим,
а с собою определенность количества, есть по существу
{\em отношение}. Поэтому она, далее, содержит в себе
свои моменты, как снятые и нераздельные; как это всегда бывает в понятии,
различие в ней таково, что каждый из ее моментов сам есть единство
качественного и количественного. Это, стало быть,
{\em реальное} различие дает множество отношений меры,
каждое из которых как формальная целостность, самостоятельно внутри себя.
Ряды, образуемые сторонами этих отношений, представляют собою для каждого
отдельного члена (относящегося, как принадлежащий одной стороне, ко всему
противостоящему ряду) один и тот же постоянный порядок. Последний, как
представляющий собой исключительно только
{\em порядок}, еще совершенно внешнее единство,
оказывается, правда, как имманентное специфицирующее единство некоторой
для-себя-сущей меры, отличным от своих спецификаций; однако специфицирующий
принцип еще не есть свободное понятие, единственно только которое и дает
своим различиям имманентное определение, а принцип есть ближайшим образом
лишь субстрат, некоторая материя, различиям которой для того, чтобы каждое
из них было целостностью,~т.~е. содержало внутри себя природу остающегося
самому себе равным субстрата, выпадает на долю лишь внешнее количественное
определение, которое вместе с тем оказывается разностью качеств. В этом
единстве субстрата с самим собою определение меры есть снятое, ее качество
есть определенное количеством, внешнее состояние. — Этот процесс есть столь
же реализующее дальнейшее определение меры, сколь и понижение последней до
уровня момента.

