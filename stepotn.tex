\paragraph[С. \ Степенное отношение]{С. \ Степенное отношение}
\hypertarget{Toc478978668}{}1. Определенное количество, полагающее себя в
своем инобытии тождественным с собою, определяющее само свое выхождение за
себя, достигло для-себя-бытия. Таким образом оно представляет собой
некоторую качественную целокупность, которая, поскольку она полагает себя
как развернутую, имеет своими моментами определения понятия числа —~единицу
и численность; в обратном отношении численность есть некоторое такое
множество, которое еще не определено самой единицей как таковою, а
определено откуда-то извне, некоторым третьим; теперь же численность
положена как определяемая лишь ею же. Это происходит в степенном отношении,
где единица, которая сама по себе есть численность, есть вместе с тем
численность в отношении себя как единицы. Инобытие, численность единиц,
есть самая же единица. Степень есть некоторое множество единиц, каждая из
которых есть самое это множество. Определенное количество как безразличная
определенность изменяется; но поскольку это изменение есть возвышение в
степень, это его инобытие ограничено исключительно самим собою. — Таким
образом, определенное количество положено в степени, как возвратившееся в
себя само; оно непосредственно есть оно само и также и свое инобытие.

{\em Показатель} этого отношения уже более не есть
некоторое непосредственное определенное количество, как в прямом, а также и
в обратном отношении. Он имеет в степенном отношении совершенно
{\em качественную} природу, есть та
{\em простая} определенность, что численность есть
самая же единица и что определенное количество
{\em тождественно} в своем инобытии с самим собою. В
этом обстоятельстве заключается вместе с тем та сторона его
{\em количественной} природы, что граница или отрицание
не положена как непосредственно сущее, а наличное бытие положено, как
продолженное в свое инобытие; ибо истина качества заключается именно в том,
что оно есть количество, непосредственная определенность как снятая.

2. Степенное отношение представляется сначала некоторым внешним изменением,
которому подвергают какое-нибудь определенное количество; но оно имеет ту
более тесную связь с {\em понятием} определенного
количества, что последнее в том наличном бытии, до которого оно развилось в
указанном отношении, достигло этого понятия, полностью реализовало его; это
отношение есть изображение того, что определенное количество есть
{\em в себе}, и выражает ту его определенность или то
{\em качество}, которым оно отличается от другого.
Определенное количество есть {\em безразличная},
{\em положенная} как {\em снятая},
определенность,~т.~е. определенность как граница, которая также и не есть
граница, продолжается в свое инобытие, остается, следовательно, в нем
тождественной с самой собой; таким оно {\em положено} в
степенном отношении; его инобытие, выхождение за само себя в некоторое
другое определенное количество, определено им же самим.

Сравнивая между собой этапы этой реализации в рассмотренных доселе
отношениях, мы видим, что качество определенного количества, заключающееся
в том, что оно положено как свое собственное отличие от самого себя,
состоит вообще в том, чтобы быть отношением. Как прямое отношение, оно есть
таковое положенное различие пока что лишь вообще или непосредственно, так
что его соотношение с самим собою, которое оно, как показатель, имеет
относительно своих различий, признается лишь неизменностью некоторой
численности единиц. В обратном отношении определенное количество есть в
отрицательном определении некоторое свое отношение к себе самому, к себе,
как к своему отрицанию, в котором оно, однако, имеет свое численное
значение; как утвердительное соотношение с собою, оно есть такой
показатель, который как определенное количество есть определяющий свои
моменты лишь {\em в себе}. В степенном же отношении оно
наличествует в различии {\em как различии себя от
самого себя}. {\em Внешность} определенности есть
качество определенного количества: теперь эта внешность положена, таким
образом, соответственно его понятию, как его собственный процесс
определения, {\em как} его соотношение с самим собою,
его {\em качество}.

3. Но тем, что определенное количество {\em положено}
так, как оно соответствует своему понятию, оно перешло в другое определение
или, как это можно также выразить, его
{\em определение} теперь дано (ist) также и как
{\em определенность}, его
{\em в-себе-бытие} дано (ist) также и как
{\em наличное бытие}. Оно есть
{\em определенное количество}, поскольку внешность или
безразличие к тому, как оно определено (то обстоятельство, что оно есть то,
что, как говорится, может быть увеличено или уменьшено), значимо и положено
лишь {\em просто} или, иначе говоря,
{\em непосредственно}; оно стало своим другим,~т.~е.
качеством, поскольку указанная внешность теперь положена, как
опосредствованная через него самого, положена как момент так, что оно
именно {\em в ней же соотносится с самим собой}, есть
бытие как качество.

Итак, первоначально количество как таковое выступает как нечто
противостоящее качеству. Но само количество есть
{\em некоторое} качество, соотносящаяся с собою
определенность вообще, отличенная от другой для нее определенности, от
качества как такового. Однако, оно не только есть
{\em некоторое} качество, а истина самого качества есть
количество; качество явило себя переходящим в количество. И обратно:
количество в своей истине есть возвратившаяся в себя самое, небезразличная
внешность. Таким образом, оно есть само качество, так что качество как
таковое не есть еще что-то помимо этого определения. — Для того чтобы была
{\em положена} целокупность как таковая, требуется
{\em двойной} переход, не только переход одной
определенности в ее другую, но также и переход этой другой, возвращение ее
в первую. Через первый переход тождество этих двух определенностей имеется
пока что лишь {\em в себе}; —~качество содержится в
количестве, которое однако вместе с тем есть пока что еще односторонняя
определенность. Что последняя, наоборот, также содержится в первой, что она
также есть лишь снятая, это получается во втором переходе, — в ее
возвращении в первую. Это замечание о необходимости
{\em двойного} перехода имеет большую важность для
всего научного метода.

И вот определенное количество как безразличное или внешнее определение, так
что оно вместе с тем снято как такое
определение~\textstyleEndnodeLink{(\ref{bkm:Ref474666560}}\label{bkm:bm58}\textstyleEndnodeLink{)}
и есть качество и то, через что нечто есть то, что оно есть, — такое
определенное количество есть истина определенного количества,
{\em мера}.

{\centering
\hypertarget{Toc478978669}{}\textstyleHeadviiFirstLine{Примечание}
\par}

Выше, в примечаниях о количественно бесконечном, было разъяснено, что
последнее, равно как и трудности, возникающие относительно него, имеют свое
происхождение в {\em качественном} моменте,
обнаруживающемся в количественном, и что в особенности качественное в
степенном отношении обнаруживает многообразное развитие и усложнение; как
на основной недостаток, служащий помехой усвоению понятия, было указано на
то, что при рассмотрении бесконечного останавливаются только на
отрицательном его определении, на том, что оно есть отрицание определенного
количества, и не идут дальше, не устанавливают того простого,
утвердительного определения, что оно есть качественное. — Здесь нам
остается сделать еще одно замечание о происходившем в философии
примешивании форм количественного к чистым качественным формам мышления. С
особенным усердием применяли в новейшее время к
{\em определениям понятия степенные
отношения}~\textstyleEndnodeLink{(\ref{bkm:Ref474666566}}\label{bkm:bm59}\textstyleEndnodeLink{)}.
Понятие в своей непосредственности было названо
{\em первой} степенью, понятие в своем инобытии или
различии, в существовании его моментов —~{\em второй},
а понятие в своем возвращении в себя или, иначе говоря, понятие как
целостность —~{\em третьей} степенью. — Как возражение
против этого сразу приходит в голову, что категория «степень»,
употребляемая таким образом, есть категория, существенно принадлежащая
области определенного количества; говорившие об этих Potenzen не имели в
виду potentia, $\delta \text{\textgreek{'u}}\nu \alpha \mu \iota \varsigma
$ Аристотеля (Potenz по-немецки означает и степень и возможность, потенцию.
— {\em Перев.}). Таким образом, степенное отношение выражает
определенность как различие, взятое так, как оно есть в
{\em особенном понятии} определенного количества,
выражает, как это различие достигает своей истины, но не выражает его,
взятого так, как оно есть в понятии как таковом. Определенное количество
содержит в себе отрицательность, принадлежащую к природе понятия, еще вовсе
не как положенную в своеобразном определении последнего; различия, присущие
определенному количеству, суть поверхностные определения для самого
понятия; они еще весьма далеки от того, чтобы быть определенными так, как
они определены в понятии. Как раз в детском периоде философствования числа
—~а первая, вторая и~т.~д. степень не имеют в этом отношении никакого
преимущества перед числами —~употреблялись, например,
{\em Пифагором} для обозначения всеобщих существенных
различий. Это было подготовительной ступенью к чистому, мыслительному
пониманию; лишь после Пифагора были изобретены,~т.~е. были осознаны
{\em особо} сами определения мысли. Но возвращаться от
последних назад к числовым определениям —~это свойственно чувствующему себя
бессильным мышлению, которое в противоположность существующей философской
культуре, привыкшей к определениям мысли, прибавляет к своему бессилию
прямо-таки смешное желание выдавать эту слабость за нечто новое,
возвышенное и за прогресс.

Поскольку выражение понятий через степени применяется лишь как
{\em символ}, против этого приходится столь же мало
возражать, как против употребления чисел или другого рода символов для
выражения понятия; но вместе с тем против этого приходится возражать столь
же много, как против всякой символики вообще, при помощи которой нам
предлагают изображать чистые понятийные или, иначе говоря, философские
определения. Философия не нуждается в такой помощи, не нуждается ни в
помощи, приходящей из области чувственного мира, ни в помощи, приходящей от
представляющей силы воображения, ни даже в помощи из находящихся на ее
собственной почве подчиненных сфер, определения которых вследствие этой
подчиненности не подходят для более высоких ее кругов и для целого.
Последнее происходит вообще в тех случаях, когда применяют категории
конечного к бесконечному; привычные определения силы или субстанциальности,
причины и действия и~т.~п. равным образом суть лишь символы для выражения,
например, жизненных или духовных отношений,~т.~е. суть неистинные
определения применительно к последним, а тем паче это справедливо о
применении степеней определенного количества и числовых степеней к таким и
вообще к спекулятивным отношениям. — Если хотят употреблять числа, степени,
математически-бесконечное и тому подобное не как символы, а как формы для
философских определений и, стало-быть, как философские формы, то следовало
бы прежде всего вскрыть их философское значение,~т.~е. их понятийную
определенность. А если это сделают, то они сами окажутся излишними
обозначениями; определенность понятия сама себя обозначает, и ее
обозначение является единственно правильным и подходящим. Употребление
указанных форм представляет собою поэтому не что иное, как только удобное
средство избавить себя от труда понимания, указания и оправдания
определений понятия.

\clearpage\setcounter{page}{1}
\bigskip
