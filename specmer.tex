\paragraph[В. \ Специфицирующая мера]{В. \ Специфицирующая мера}
\hypertarget{Toc478978673}{}Эта последняя есть

{\em во-первых}, некоторое правило, некоторая мера,
внешняя по отношению к голому определенному количеству;

{\em во-вторых}, специфическое количество, определяющее
собою внешнее определенное количество;

{\em в-третьих}, {\em обе стороны},
как {\em качества}, которым присуща специфическая
количественная определенность, относятся друг к другу, как
{\em единая} мера.

\subparagraph[а) \ Правило (61)]{а) \ Правило
\textstyleEndnodeLink{(\ref{bkm:Ref474666643}}\textstyleEndnodeLink{)}}
\hypertarget{Toc478978674}{}\label{bkm:bm61}Правило, или масштаб, о котором
мы уже говорили, есть прежде всего некоторая в себе определенная величина,
служащая единицей по отношению к некоторому определенному количеству,
которое есть особое существование, существует в некотором другом нечто, чем
нечто, служащее масштабом, и измеряется последним,~т.~е. определяется как
численность указанной единицы. Это сравнение есть некоторое внешнее
действие; сама та единица есть произвольная величина, которую в свою
очередь можно положить как численность (фут, например, как определенное
число дюймов). Но мера есть не только внешнее правило, но, как
специфическая, она состоит в том, чтобы в себе самой относиться к своему
другому, которое есть некоторое определенное количество.

\subparagraph[b) \ Специфицирующая мера]{b) \ Специфицирующая мера}
\hypertarget{Toc478978675}{}Мера есть специфическое определение
{\em внешней},~т.~е. безразличной величины, полагаемой
теперь некоторым другим существованием в то нечто, которое служит мерой и
которое, хотя само оно есть определенное количество, все же в отличие от
такового есть нечто качественное, нечто такое, что определяет исключительно
безразличное, внешнее определенное количество. Нечто имеет в нем ту сторону
бытия-для-другого, которой присуще безразличное увеличение и уменьшение.
Указанное имманентное измеряющее есть такое присущее данному нечто
качество, которому противостоит то же самое качество в другом нечто, но в
последнем это качество имеется ближайшим образом с относительно безмерным
определенным количеством вообще, в противоположность первому качеству,
которое определено как измеряющее.

В нечто, поскольку оно есть мера внутри себя, приходит извне некоторое
изменение величины его качества; оно не принимает оттуда
{\em арифметического} множества. Его мера этому
противодействует, держится по отношению к множеству как некоторое
интенсивное и вбирает его своеобразным способом; она изменяет положенное
извне изменение, делает из этого определенного количества некоторое другое
и {\em являет} себя через эту спецификацию
для-себя-бытием в этой внешности. — Это
{\em специфически-вобранное} множество само есть
некоторое определенное количество, также зависимое от другого множества
или, иначе говоря, от последнего как лишь {\em внешнего
множества}. Специфицированное множество поэтому также изменчиво, но не есть
вследствие этого определенное количество как таковое, а есть внешнее
определенное количество, специфицированное константным образом. Таким
образом, мера имеет свое наличное бытие как
{\em отношение} и специфическое в ней есть вообще
{\em показатель} этого отношения.

В {\em интенсивном} и
{\em экстенсивном} определенном количестве имеется, как
оказалось при рассмотрении этих определений, {\em одно
и то же} определенное количество, которое в одном случае дано в форме
интенсивности, а в другом —~в форме экстенсивности. Лежащее в основании
определенное количество не подвергается в этом различии никакому изменению,
это различие есть лишь некоторая внешняя форма. Напротив, в специфицирующей
мере определенное количество то берется в его непосредственной величине, то
оно, проходя через показатель отношения, берется в некоторой другой
численности.

Показатель, составляющий специфическое, может на первый взгляд показаться
постоянным определенным количеством, как частное отношения между внешним и
качественно-определенным количеством. Но в таком случае он был бы не чем
иным, как некоторым внешним определенным количеством; под словом
«показатель» здесь следует понимать не что иное, как момент самого
качественного, специфицирующий определенное количество, как таковое.
Собственным имманентным качественным моментом определенного количества
служит, как это оказалось выше, лишь {\em степенное
определение}. Степенное определение и должно быть тем, что конституирует
рассматриваемое отношение и что, в качестве в-себе-сущего определения,
противостоит здесь определенному количеству как внешнему характеру.
Последнее имеет своим принципом нумерическую единицу, составляющую его
в-себе-определенность; соотношение нумерической единицы есть внешнее
соотношение, и изменение, определяемое лишь природой непосредственного
определенного количества как такового, и состоит само по себе в
присовокуплении такой нумерической единицы и снова такой же единицы и~т.~д.
Таким образом, если внешнее определенное количество изменяется в
арифметической прогрессии, то специфицирующая реакция качественной природы
меры порождает другой ряд, который соотносится с первым, возрастает и
убывает вместе с ним, но не в отношении, определяемом некоторым численным
показателем, а в отношении, несоизмеримом с каким бы то ни было числом,
соотносится согласно некоторому степенному определению.

{\centering
\hypertarget{Toc478978676}{}\textstyleHeadviiFirstLine{Примечание}
\par}

Чтобы привести пример, укажем на {\em температуру}; она
представляет собою такое {\em качество}, в котором
различаются эти обе стороны, — то обстоятельство, что она есть внешнее
определенное количество, и то обстоятельство, что она есть
специфицированное определенное количество. Как определенное количество она
есть внешняя температура (и притом тоже некоторого тела как некоторой общей
среды), относительно каковой температуры принимается, что ее изменение
происходит по шкале арифметической прогрессии и что она равномерно
возрастает или убывает; напротив, разными находящимися в ней отдельными
телами температура эта воспринимается по-разному, так как они определяют
воспринятую извне температуру своей имманентной мерой, и их температурные
изменения не находятся в прямом отношении с изменением температуры среды
или между собою. Сравнение разных тел, подвергающихся действию одной и той
же температуры, дает числовые отношения их специфических теплот, их
теплоемкости. Но эти теплоемкости тел неодинаковы в разных температурах, и
с этим связано изменение специфического состояния. В увеличении или
уменьшении температуры сказывается, следовательно, некоторая особая
спецификация. Отношение температуры, представляемой как внешняя, к
температуре некоторого определенного тела, находящейся вместе с тем в
зависимости от первой температуры, не имеет неизменного показателя
отношения; увеличение или уменьшение этой последней теплоты не идет
равномерно с возрастанием и убыванием внешней температуры. — При этом
некоторая температура принимается вообще за внешнюю, изменение которой лишь
внешне или чисто количественно. Она, однако, сама есть температура воздуха
или какая-нибудь другая специфическая температура. Поэтому при более
близком рассмотрении следовало бы, собственно говоря, брать это отношение
не как отношение просто количественного к некоторому
качественно-определенному, а как отношение двух специфических определенных
количеств. И в самом деле, в непосредственно последующем изложении
специфицирующего отношения тотчас же выяснится, что моменты меры состоят не
только в некоторой количественной и некоторой качественно-количественной
сторонах одного и того же качества, а в отношении двух качеств, которые в
самих себе суть меры.

\subparagraph[с) \ Отношение обеих сторон как качеств]{с) \ Отношение обеих
сторон как качеств}
\hypertarget{Toc478978677}{}\ 1. Качественная, в себе определенная сторона
определенного количества дана лишь как соотношение с внешне количественным;
как специфицирование последнего она есть снятие его внешности, через
которую определенное количество имеет бытие как таковое; она, таким
образом, имеет последнее своей предпосылкой и начинает с него. Но
количество само отлично от качества также и качественно. Это их различие
должно быть положено в той {\em непосредственности}
бытия вообще, в стадии которой мера еще находится; взятые таким образом обе
стороны качественны по отношению друг друга, и каждая есть сама по себе
такого рода наличное бытие; и то одно определенное количество, которое
ближайшим образом представляет собою лишь формальное, неопределенное в себе
определенное количество, есть определенное количество некоторого нечто и
его качества, а так как теперь взаимоотношение обеих сторон определилось в
меру вообще, то оно есть также и специфическая величина этих качеств.
Согласно определению меры эти качества находятся во взаимном отношении друг
к другу; определение меры есть их показатель; но они в себе соотнесены друг
с другом уже в {\em для-себя-бытии} меры; определенное
количество имеет двойное бытие, есть внешнее и специфическое, так что
каждое из различенных количеств заключает в себе это двойственное
определение и вместе с тем безоговорочно переплетено с другим; именно
только в этом и заключается определенность обоих качеств. Они, таким
образом, не суть лишь вообще сущее друг для друга наличное бытие, а
положены нераздельными, и связанная с ними определенность величины есть
некоторая качественная единица, — {\em одно}
определение меры, в котором они согласно своему понятию, в себе, связаны
друг с другом. Мера есть, таким образом,
{\em имманентное} количественное отношение друг к другу
{\em двух} качеств.

2. В мере появляется существенное определение
{\em переменной величины}, ибо она есть определенное
количество как снятое и, стало быть, уже более не есть то, чем оно должно
быть чтобы быть определенным количеством, а есть определенное количество и
вместе с тем нечто другое; этим другим служит качественное и, как было
определено, это —~не что иное, как его степенное отношение. В
непосредственной мере это изменение еще не положено; качество связано лишь
с каким-либо и притом отдельным определенным количеством вообще. В
специфицировании меры,~т.~е. в предыдущем определении, где исключительно
внешнее определенное количество подвергалось изменению со стороны
качественного момента, положена различенность этих двух определений
величины и тем самым вообще множественность мер в некотором общем им
внешнем определенном количестве. Определенное количество впервые являет
себя налично сущей мерой лишь в такой своей различенности от самого себя,
когда оно, будучи одним и тем же (например, той же самой температурой
среды), вместе с тем выступает как разное и притом количественное наличное
бытие (в разных температурах находящихся в этой среде тел). Эта различность
определенного количества в разных качествах —~в разных телах —~дает
дальнейшую форму, ту форму меры, в которой обе стороны относятся друг к
другу, как качественно определенные количества, что можно назвать
{\em реализованной мерой}.

Величина как некоторая величина вообще переменна, ибо ее определенность
имеет бытие как некоторая граница, которая вместе с тем не есть граница;
постольку изменение затрагивает лишь некоторое особое определенное
количество, на место которого ставится некоторое другое определенное
количество; но истинным изменением является лишь изменение определенного
количества как такового; отсюда получается понимаемое таким образом
интересное определение переменной величины в высшей математике; причем не
приходится ни останавливаться на формальной стороне, на
{\em переменности} вообще, ни привлекать другие
определения, кроме того простого определения понятия, по которому
{\em другим определенного количества} служит лишь
{\em качественное}. Стало быть, истинное определение
реальной переменной величины заключается в том, что она есть величина,
определяемая качественно и, следовательно, как мы это достаточно показали,
определяемая степенным отношением. В этой переменной величине
{\em положено}, что определенное количество значимо не
как таковое, а по своему другому для него определению, по качественному
определению.

Стороны этого отношения имеют по своей абстрактной стороне, как качества
вообще, какое-нибудь особенное значение, например пространства и времени.
Взятые ближайшим образом вообще в отношении их мер, как определенности
величины, одна из них есть численность, увеличивающаяся и уменьшающаяся во
внешней, арифметической прогрессии, а другая есть численность, специфически
определяемая первой, которая служит для нее единицей. Если бы каждая из них
была лишь некоторым особенным качеством вообще, то между ними не было бы
различия, по которому можно было бы сказать, какая из этих двух должна быть
принимаема в отношении ее количественного определения за чисто внешне
количественную и какая —~за изменяющуюся при количественной спецификации.
Если они, например, относятся между собою, как квадрат и корень, то
безразлично, в какой из них мы рассматриваем увеличение и уменьшение как
чисто внешнее, нарастающее в арифметической прогрессии, и какая из них
рассматривается, напротив, как специфически определяющая себя в этом
определенном количестве.

Но качества не суть неопределенно разные в отношении друг друга, ибо в них
как моментах меры должно заключаться окачествование последней. Ближайшая
определенность самих качеств заключается в том, что одно есть
{\em экстенсивное}, внешность в самой себе, а другое
—~{\em интенсивное}, внутри-себя-сущее, или, иначе
сказать, отрицательное по отношению к первому. Из количественных моментов
на долю первого приходится согласно этому численность, а на долю второго
—~единица; в простом прямом отношении первое должно быть принимаемо за
делимое, а второе —~за делитель, а в специфицирующем отношении —~первое за
степень или за становление другим и второе —~за корень. Поскольку здесь еще
занимаются счетом,~т.~е. обращают внимание на внешнее определенное
количество (которое, таким образом, есть совершенно случайная, эмпирически
называемая определенность величины) и, стало быть, изменение также
принимается за нарастающее во внешней, арифметической прогрессии,
постольку, это изменение падает на ту сторону, которая служит единицей, на
интенсивное качество; внешнюю же, экстенсивную, сторону мы, напротив,
должны представлять изменяющейся в специфицированном ряду. Но прямое
отношение (как, например, скорость вообще,  $\frac s t$) снижено здесь до
формального, не существующего, принадлежащего лишь абстрагирующей рефлексии
определения; и если в отношении корня и квадрата (как например, в 
$s=at^2$) мы все еще должны принимать корень за эмпирическое определенное
количество, возрастающее в арифметической прогрессии, а другую сторону
отношения за специфицированную, то высшая, более соответствующая понятию
реализация окачествования количественного состоит в том, что обе стороны
относятся между собою в высших степенных определениях (как это, например,
имеет место в  $s^3=at^2$).

{\centering
\hypertarget{Toc478978678}{}\textstyleHeadviiFirstLine{Примечание}
\par}

Данное нами здесь разъяснение касательно связи качественной природы
некоторого существования (eines Daseins) и его количественного определения
в мере находит свое применение в уже указанном вкратце примере движения;
это применение заключается прежде всего в том, что в
{\em скорости}, как прямом отношении пройденного
пространства и протекшего времени, величина времени принимается за
знаменатель, а величина пространства, напротив, — за числитель. Если
скорость есть вообще лишь отношение между пространством и временем
некоторого движения, то безразлично, какой из этих двух моментов
рассматривается как численность и какой как единица. Но на самом деле
пространство так же, как в удельной тяжести вес, есть внешнее реальное
целое вообще и, стало быть, численность; время же, точно так же как объем,
есть, напротив, идеализованное, отрицательное, сторона, служащая единицей.
— Но существенным применением служит здесь то более важное отношение, что в
{\em свободном движении} —~прежде всего в еще
обусловленном движении {\em падения} тел —~количество
времени и количество пространства определены друг относительно друга первое
как корень, а второе как квадрат,— или в абсолютно свободном движении
небесных тел время обращения и расстояние —~первое на одну степень ниже,
чем второе, — определены друг относительно друга первое как квадрат, второе
как куб. Подобные основные отношения покоятся на природе находящихся в
отношении качеств пространства и времени и на роде соотношения, в котором
они находятся, зависят от того, является ли это отношение механическим
движением,~т.~е. несвободным, не определяемым понятием моментов, или
падением,~т.~е. условно свободным движением, или, наконец, абсолютно
свободным небесным движением, каковые роды движения, точно так же как и их
законы, покоятся на развитии понятия их моментов, пространства и времени,
так как эти качества как таковые оказываются {\em в
себе},~т.~е. в понятии, {\em нераздельными}, и их
количественное отношение есть {\em для-себя-бытие}
меры, есть лишь {\em одно} определение меры.

По поводу абсолютных отношений меры следует сказать, что
{\em математика природы}, если она хочет быть достойной
этого имени, по существу должна быть наукой о мерах, наукой, для которой
эмпирически, несомненно, сделано очень много, но собственно научно,~т.~е.
философски, сделано еще весьма мало.
{\em Математические начала философии природы}, как
Ньютон назвал свое сочинение, если они должны выполнять это назначение в
более глубоком смысле, чем тот, в котором это делали он и все пошедшее от
Бэкона поколение философов и ученых, должны были бы содержать в себе нечто
совсем иное, чтобы внести свет в эти еще темные, но в высшей степени
достойные рассмотрения области. Велика заслуга познакомиться с
эмпирическими числами природы, например, с расстояниями планет друг от
друга; но бесконечно бóльшая заслуга состоит в том, чтобы заставить
исчезнуть эмпирические определенные количества и возвести их во
{\em всеобщую форму} количественных определений так,
чтобы они стали моментами некоторого {\em закона} или
некоторой меры, — бессмертные заслуги, которые приобрели себе, например,
{\em Галилей} относительно падения тел и
{\em Кеплер} относительно движения небесных тел. Они
так {\em доказали} найденные ими законы, что показали,
что им соответствует весь объем подробностей, доставляемых восприятием. Но
следует требовать еще высшего {\em доказывания} этих
законов, а именно не чего иного, как того, чтобы их количественные
определения были познаны из качеств или, иначе говоря, из соотнесенных друг
с другом определенных понятий (как, например, пространство и время). Этого
рода доказательств еще нет и следа в указанных математических началах
{\em философии природы}, равно как и в дальнейших
подобного рода работах. Выше, по поводу видимости математических
доказательств встречающихся в природе отношений, — видимости, основанной на
злоупотреблении бесконечно малым, — мы заметили, что попытка вести такие
доказательства собственно {\em математически},~т.~е. не
черпая их ни из опыта, ни из понятия, есть бессмысленное предприятие. Эти
доказательства {\em предполагают наперед} свои
теоремы,~т.~е. как раз сказанные законы, исходя из опыта; и они лишь
приводят эти законы к абстрактным выражениям и удобным формулам. Всю
приписываемую {\em Ньютону} реальную заслугу, в которой
видят его преимущество перед {\em Кеплером} по
отношению к одним и тем же предметам, если отвлечься от мнимого здания
доказательств, несомненно придется в конце концов (когда наступит более
очищенное соображение относительно того, что сделала математика и что она в
состоянии сделать) ограничить с ясным пониманием сути дела тем, что он дал
известное {\em преобразование
выражения\footnotemark{}}\footnotetext{ См. «Энциклопедию философских
наук», примечание к §~270 о преобразовании кеплеровой формулы 
$\frac{S^3}{T^2}$ в ньютоновскую  \ причем  было названо силой тяготения.}
 и ввел согласно своим {\em началам} аналитическую
трактовку.
