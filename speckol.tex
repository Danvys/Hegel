\subsubsection[Первая главаСпецифическое
Количество]{\textstyleHeadiiiFirstLine{Первая глава}\newline
Специфическое Количество}
\hypertarget{Toc478978671}{}Качественное количество есть
{\em ближайшим образом} непосредственное
{\em специфическое определенное количество}, которое

{\em во-вторых}, как относящееся к другому, становится
некоторым количественным специфицированием, снятием безразличного
определенного количества. Постольку эта мера есть
{\em правило} и содержит в себе
{\em различенными оба момента} меры, а именно,
в-себе-сущую количественную определенность и внешнее определенное
количество. Но в этом различии эти две стороны становятся качествами, а
правило —~некоторым отношением этих качеств; мера поэтому представляется

{\em в-третьих}, {\em отношением
качеств}, имеющих ближайшим образом одну меру, которая, однако, затем
специфицируется внутри себя в некоторое различие мер.

\paragraph[А. \ Специфическое определенное количество]{А. \ Специфическое
определенное количество}
\hypertarget{Toc478978672}{}1. Мера есть простое соотношение определенного
количества с собою, его собственная определенность в самом себе; таким
образом, определенное количество качественно. Ближайшим образом мера, как
непосредственная мера, есть некоторое непосредственное и поэтому некоторое
определенным образом определенное количество; столь же непосредственным
является и сопряженное с ним качество, оно есть какое-нибудь определенное
качество. — Определенное количество, как эта уже более не безразличная
граница, как соотносящаяся с собою внешность, само, таким образом, есть
качество и, будучи отличным от последнего, оно не простирается дальше его,
равно как и это качество не идет дальше этого определенного количества. Оно
есть, таким образом, возвратившаяся в простое равенство с собою
определенность; оно едино с определенным наличным бытием, точно так же, как
это последнее едино со своим определенным количеством.

Если из полученного теперь определения хотят образовать предложение, то
можно выразиться так: {\em все налично сущее имеет
некоторую меру}. Всякое наличное бытие обладает некоторой величиной, и эта
величина принадлежит к самой природе нечто; она составляет его определенную
природу и его внутри-себя-бытие. Нечто не безразлично к этой величине, не
остается тем, что оно есть, если изменяется эта величина, а изменение
последней изменяет его качество. Определенное количество как мера перестало
быть такой границей, которая не есть граница; оно есть отныне определение
вещи, так что если увеличить или уменьшить эту вещь за пределы этого
определенного количества, она погибнет.

Мера как масштаб в обычном смысле есть некоторое определенное количество,
которое произвольно принимается за {\em в себе
определенную} единицу по отношению к внешней численности. Такого рода
единица может, правда, и в самом деле быть определенной в себе единицей,
как например,
{фут}~\textstyleEndnodeLink{(\ref{bkm:Ref474666577}}\label{bkm:bm60}\textstyleEndnodeLink{)}
и тому подобные первоначальные меры; однако, поскольку она употребляется
вместе с тем и как масштаб для других вещей, она для них —~только внешняя,
а не первоначальная мера. — Так, например, диаметр земли или длина маятника
могут быть сами по себе рассматриваемы как специфические определенные
количества; но решение брать именно такую-то часть диаметра земли или длины
маятника и употреблять последнюю как масштаб именно на таком-то градусе
широты является произвольным. Но еще в большей степени такого рода масштаб
является чем-то внешним для других вещей. Последние специфицировали
всеобщее специфическое определенное количество опять-таки на особый лад и
этим сделались особыми вещами. Нелепо поэтому говорить о естественном
{\em масштабе} вещей. Да и помимо этого всеобщий
масштаб должен, как полагают, служить лишь для внешнего
{\em сравнения}; в этом наиболее поверхностном смысле,
в котором он принимается {\em за всеобщую меру},
совершенно безразлично, что для этого употребляется. Это —~не основная мера
в том смысле, что в ней представлены естественные меры особых вещей и из
нее последние познаются согласно некоторому правилу как спецификации единой
всеобщей меры, меры их всеобщего тела. Но без этого смысла абсолютный
масштаб имеет лишь интерес и значение некоторого
{\em общего} (eines Gemeinschaftlichen), а таковое есть
всеобщее не {\em в себе}, а только по соглашению.

Эта непосредственная мера есть некоторое простое определение величины, как
например величина органических существ, их членов и~т.~д. Но всякое
существующее, чтобы быть тем, что оно есть, и чтобы вообще обладать
существованием, имеет некоторую величину. — Как определенное количество,
она есть безразличная величина, открытая внешнему определению и способная
подниматься по лестнице большего и опускаться по лестнице меньшего. Но как
мера, она вместе с тем отлична от себя самой как определенного количества,
как такого безразличного определения, и есть ограничение этого
безразличного движения взад и вперед вдоль границы.

Поскольку, таким образом, количественная определенность оказывается в
наличном бытии двоякой —~с одной стороны, такой определенностью, с которой
связано качество, а с другой стороны, такой определенностью, вдоль которой,
не нанося ущерба качеству, можно двигаться взад и вперед, — постольку
гибель того нечто, которое имеет меру, может произойти в результате того,
что изменяется его определенное количество. Эта гибель представляется, с
одной стороны, {\em неожиданной}, поскольку можно ведь
вносить изменения в определенное количество, не изменяя меры и качества, а
с другой стороны, ее делают чем-то совершенно понятным, посредством понятия
{\em постепенности}. К, этой категории охотно
прибегают, чтобы сделать представимым или
«{\em объяснить}» прехождение какого-либо качества или
нечто, так как кажется, что таким образом можно почти видеть своими глазами
процесс исчезания, потому что определенное количество берется как внешняя,
по своей природе изменчивая граница, и, стало быть,
{\em изменение} как изменение исключительно только
определенного количества само собою понятно. Но на самом деле этим ничего
не объясняется; изменение есть вместе с тем по существу переход одного
качества в другое, или более абстрактный переход от наличного бытия в
отсутствие наличного бытия; в этом заключается иное определение, чем в
постепенности, которая есть лишь уменьшение или увеличение и одностороннее
цепляние за величину.

2. Но что изменение, выступающее как чисто количественное, переходит также и
в качественное, — на эту связь обратили внимание уже древние и представили
коллизии, возникающие на почве незнания этого обстоятельства, в популярных
примерах. Относящиеся сюда «{\em эленхи}»,~т.~е.
согласно объяснению Аристотеля способы, посредством которых вынуждаются
сказать противоположное тому, что утверждали до этого, известны под
названием «лысый», «куча». Задавался вопрос: получается ли лысина, если
выдернуть один волос из головы или из лошадиного хвоста, или: перестает ли
куча быть кучей, если возьмем из нее одно зернышко? Можно не задумываясь
согласиться с тем, что при этом не получается лысины и что куча не
перестает быть кучей, так как такое отнятие составляет только
количественную и притом даже весьма незначительную разницу; таким образом
отнимают один волос, одно зернышко и повторяют это так, что всякий раз,
следуя тому, с чем согласились, отнимается лишь один волос или одно
зернышко; под конец обнаруживается качественное различие: голова, хвост
становятся лысыми, куча исчезает. Когда соглашались, что отнимание одного
волоса не делает лысым и~т.~д., забывали не только о повторении, но и о
том, что сами по себе незначительные количества (например, сами по себе
незначительные траты состояния) {\em суммируются} и
сумма составляет качественное целое, так что под конец это целое
оказывается исчезнувшим, голова сделалась лысой, кошелек опустел.
Затруднение, противоречие, получающееся в результате, не есть нечто
софистическое в обычном смысле этого слова, не есть ложная уловка, введение
в обман. Ложным является то, что совершает предполагаемый другой
собеседник,~т.~е. наше обыденное сознание, принимающее количество лишь за
безразличную границу,~т.~е. берущее его именно в том определенном смысле, в
каком оно есть количество как таковое. Это предположение изобличается как
ложное той истиной, к которой оно приводится, истиной, заключающейся в том,
что количество есть момент меры и находится в связи с качеством; что здесь
опровергается, — это одностороннее цепляние за абстрактную определенность
определенного количества. — Вышеуказанные обороты рассуждения и не суть
поэтому пустая или педантическая шутка, а внутренне правильны и являются
порождениями сознания, интересующегося явлениями, встречающимися в области
мышления.

Определенное количество в том случае, когда его принимают за безразличную
границу, есть та сторона, с которой нечто существующее (ein Dasein)
подвергается неожиданному нападению и неожиданной гибели. В том-то и
заключается {\em хитрость} понятия, что она схватывает
существующее с той стороны, с которой, как ему кажется, его качество
нисколько не затрагивается и притом настолько не затрагивается, что
увеличение государства, состояния и~т.~д., ввергающее государство,
собственника в несчастье, сначала даже кажется его счастьем.

3. Мера есть в своей непосредственности некоторое обычное качество,
обладающее определенной, принадлежащей ему величиной. От той стороны, по
которой определенное количество есть безразличная граница, вдоль которой
можно, не изменяя качества, ходить взад и вперед, отлична его другая
сторона, по которой оно качественно, специфично. Обе стороны суть
определения величины одного и того же. Но согласно той непосредственности,
которая первоначально присуща мере, мы должны, далее, брать это различие
как непосредственное; обе стороны имеют согласно этому также и разное
существование. Тогда то существование меры, которое есть определенная
{\em в себе} величина, есть в своем отношении к
существованию изменчивой, внешней стороны снятие своего безразличия,
{\em специфицирование} меры.
