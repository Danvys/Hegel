\documentclass[b5paper, 11pt, twoside, onecolumn, openany]{memoir}

%%% PACKAGES 
%%%------------------------------------------------------------------------

\usepackage[utf8]{inputenc}
\usepackage[T1]{fontenc}
\usepackage[russian]{babel}
\usepackage[final]{microtype} % Less badboxes
\usepackage{bookmark}

% \usepackage{kpfonts} %Font

\usepackage{amsmath,amssymb,mathtools} % Math

% \usepackage{tikz} % Figures
\usepackage{graphicx} % Include figures

%%% PAGE LAYOUT 
%%%------------------------------------------------------------------------

\setlrmarginsandblock{0.15\paperwidth}{*}{1} % Left and right margin
\setulmarginsandblock{0.2\paperwidth}{*}{1}  % Upper and lower margin
\checkandfixthelayout

%%% SECTIONAL DIVISIONS
%%%------------------------------------------------------------------------

\maxsecnumdepth{section} % Subsections (and higher) are numbered
\setsecnumdepth{section}

\makeatletter %
\makechapterstyle{standard}{
  \setlength{\beforechapskip}{0\baselineskip}
  \setlength{\midchapskip}{1\baselineskip}
  \setlength{\afterchapskip}{2\baselineskip}
  \renewcommand{\chapterheadstart}{\vspace*{\beforechapskip}}
  \renewcommand{\chapnamefont}{\centering\normalfont\Large}
  \renewcommand{\printchaptername}{\chapnamefont \@chapapp}
  \renewcommand{\chapternamenum}{\space}
  \renewcommand{\chapnumfont}{\normalfont\Large}
  \renewcommand{\printchapternum}{\chapnumfont \thechapter}
  \renewcommand{\afterchapternum}{\par\nobreak\vskip \midchapskip}
  \renewcommand{\printchapternonum}{\vspace*{\midchapskip}\vspace*{5mm}}
  \renewcommand{\chaptitlefont}{\centering\bfseries\LARGE}
  \renewcommand{\printchaptertitle}[1]{\chaptitlefont ##1}
  \renewcommand{\afterchaptertitle}{\par\nobreak\vskip \afterchapskip}

  % Part titles
  \renewcommand{\beforepartskip}{}
  \renewcommand{\afterpartskip}{\bigskip}

  \renewcommand{\clearforchapter}{}
}
\makeatother

\chapterstyle{standard}

\setsecheadstyle{\normalfont\large\bfseries}
\setsubsecheadstyle{\normalfont\normalsize\bfseries}
\setparaheadstyle{\normalfont\normalsize\bfseries}
\setparaindent{0pt}\setafterparaskip{0pt}

%%% FLOATS AND CAPTIONS
%%%------------------------------------------------------------------------

\makeatletter % You do not need to write [htpb] all the time
\renewcommand\fps@figure{htbp}
\renewcommand\fps@table{htbp}
\makeatother

\captiondelim{\space } % A space between caption name and text
\captionnamefont{\small\bfseries} % Font of the caption name
\captiontitlefont{\small\normalfont} % Font of the caption text

\changecaptionwidth          % Change the width of the caption
\captionwidth{1\textwidth} %

%%% ABSTRACT
%%%------------------------------------------------------------------------

\renewcommand{\abstractnamefont}{\normalfont\small\bfseries} % Font of abstract title
\setlength{\absleftindent}{0.1\textwidth} % Width of abstract
\setlength{\absrightindent}{\absleftindent}

%%% HEADER AND FOOTER 
%%%------------------------------------------------------------------------

\makepagestyle{standard} % Make standard pagestyle

\makeatletter                 % Define standard pagestyle
\makeevenfoot{standard}{}{}{} %
\makeoddfoot{standard}{}{}{}  %
\makeevenhead{standard}{\bfseries\thepage\normalfont\qquad\small\leftmark}{}{}
\makeoddhead{standard}{}{}{\small\rightmark\qquad\bfseries\thepage}
% \makeheadrule{standard}{\textwidth}{\normalrulethickness}
\makeatother                  %

\makeatletter
\makepsmarks{standard}{
\createmark{chapter}{both}{nonumber}{\@chapapp\ }{ \quad }
\createmark{chapter}{right}{nonumber}{}{ \quad }
\createmark{section}{right}{nonumber}{}{ \quad }
\createmark{subsection}{right}{nonumber}{}{ \quad }
\createplainmark{toc}{both}{\contentsname}
\createplainmark{lof}{both}{\listfigurename}
\createplainmark{lot}{both}{\listtablename}
\createplainmark{bib}{both}{\bibname}
\createplainmark{index}{both}{\indexname}
\createplainmark{glossary}{both}{\glossaryname}
}
\makeatother                               %

\makepagestyle{chap} % Make new chapter pagestyle

\makeatletter
\makeevenfoot{chap}{}{}{} % Define new chapter pagestyle
\makeoddfoot{chap}{}{}{}  %
\makeevenhead{chap}{}{}{}   %
\makeoddhead{chap}{}{}{}    %
% \makeheadrule{chap}{\textwidth}{\normalrulethickness}
\makeatother

\nouppercaseheads
\pagestyle{standard}               % Choosing pagestyle and chapter pagestyle
\aliaspagestyle{chapter}{chap} %

%%% NEW COMMANDS
%%%------------------------------------------------------------------------

\maxtocdepth{subsection} % ToC depth
\settocdepth{subsection}

\AtEndDocument{\addtocontents{toc}{\par}} % Add a \par to the end of the TOC

\usepackage{hyperref}   % Internal hyperlinks
\hypersetup{
  pdfborder={0 0 0},      % No borders around internal hyperlinks
  pdfauthor={ФРА} % author
}
\usepackage{memhfixc}   %

\author{Г.~В.~Ф.~Гегель}
\title{Наука логики}
\date{}

\renewcommand{\partnumberline}[1]{} % Remove part number in ToC
\renewcommand{\cftpartdotsep}{\cftdotsep} % Part dots in ToC
\renewcommand{\chapternumberline}[1]{} % Remove chapter number in ToC
\renewcommand{\cftchapterdotsep}{\cftdotsep} % Chapter dots in ToC

% Remove section numbers in ToC
\let\oldcftsf\cftsectionfont% save definition of \cftsectionfont
\let\oldcftspn\cftsectionafterpnum% and of \cftsectionafterpnum
\renewcommand*{\cftsectionfont}{%
\let\oldnl\numberline% save definition of \numberline
\renewcommand*{\numberline}[1]{}% change it
\oldcftsf} % use original \cftsectionfont
\renewcommand*{\cftsectionafterpnum}{%
\let\numberline\oldnl% % restore orginal \numberline
\oldcftspn} % use original \cftsectionafterpnum

\begin{document}

\mainmatter

\begin{center}
  {\Large Г.~В.~Ф.~ГЕГЕЛЬ} \\
  \vspace{130pt}
  \textbf{\Huge НАУКА ЛОГИКИ} \\
  \vspace{12pt}
  {\Large Том~I. Объективная логика} \\
  \vspace{8pt}
  {\large Книга~I. Учение о бытии} \\
  \vspace{45pt}
  \textit{перевод} \\
  \textit{Б.~Г.~СТОЛПНЕРА} \\
  \vspace{10pt}
  \textit{под редакцией} \\
  \textit{М.~Б.~МИТИНА}
\end{center}

\clearpage

\part*{О значении воинствующего материализма}

\input{materializm.tex}

\part*{Предисловие к~первому изданию}

\input{predislovie1.tex}

\part*{Предисловие ко~второму изданию}

\input{predislovie2.tex}

\part*{Введение}

\input{vvedenie.tex}

\part*{Первая книга. Учение о бытии.}

\chapter*{С чего следует начинать логику}
\input{schego.tex}


\part*{Первый отдел. Определённость (качество).}

%         Первая глава. Бытие.
Бытие есть неопределенное непосредственное. Оно свободно от определенности
по отношению к сущности, равно как еще свободно от всякой определенности,
которую оно может получить внутри самого себя. Это не имеющее рефлексии
бытие есть бытие, как оно есть непосредственно лишь в самом себе.

Так как оно неопределённо, то оно есть бескачественное бытие.
Но~{\em в~себе} ему принадлежит характер неопределённости
лишь в~противоположность к~{\em определённому} или
качественному. Но~бытию~вообще противостоит
{\em определённое} бытие как~таковое, а~благодаря этому
сама его неопределённость составляет его качество. Тем~самым
обнаружится, что {\em первое}~бытие есть определенное~в~себе
и~что, следовательно:

{\em во-вторых}, оно переходит в
{\em наличное бытие}, есть
{\em наличное бытие}, но что последнее как конечное
бытие снимает себя и переходит в бесконечное соотношение бытия с самим
собою,

переходит, {\em в-третьих}, в
{\em для-себя-бытие}.

\chapter*{Первая глава. Бытие.}
\section*{А. Бытие}

{\em Бытие}, {\em чистое бытие} "--- без всякого дальнейшего определения.
В~своей неопределенной непосредственности оно равно лишь
самому себе, и~оно также и~не~неравно по отношению к~другому, не~имеет
никакой разности ни внутри себя, ни по~отношению к~внешнему. Если бы
в~нём было какое-либо определение или
содержание, отличное от~другого определения в~нём~же, или же такое
определение или содержание, которым оно отличается от~некоего другого
бытия, то~такое различие нарушило~бы его чистоту. Бытие есть чистая
неопределенность и~пустота. --- В~нём нечего созерцать, если здесь
может итти речь о~созерцании, или, иначе~говоря, оно есть~только само
это чистое, пустое созерцание.
В~нём столь же мало есть нечто такое, что можно было~бы
мыслить, или, иначе~говоря, оно равным образом есть~лишь это пустое
мышление. Бытие, неопределенное, непосредственное, есть~на~деле
{\em ничто} и~не~более и~не~менее, чем~ничто.

\section*{В. Ничто}
{\em Ничто}, {\em чистое ничто}; оно
есть простое равенство с~самим собою, совершенная пустота, отсутствие
определений и~содержания; неразличенность в~самом~себе. --- Поскольку
здесь можно говорить о~созерцании или мышлении, следует сказать,
что~считается небезразличным, созерцаем~ли мы, а~также мыслим~ли мы
нечто~или {\em ничто}. Выражение <<созерцать или мыслить ничто>>,
следовательно, что-то означает. Мы проводим различие между этими двумя
случаями; таким образом, ничто {\em есть} (существует)
в нашем созерцании или мышлении; или, вернее, оно~и~есть само~это пустое
мышление и~созерцание; и~оно~есть то~же~самое пустое созерцание или
мышление, что~и чистое бытие. --- Ничто есть, стало быть, то~же~самое
определение или, вернее, то~же~самое отсутствие определений и,~значит,
вообще то~же~самое, что~и чистое {\em бытие}.

\section*{С. Становление}
\subsection*{1. Единство бытия и ничто}
{\em Чистое бытие и~чистое ничто есть},
{\em следовательно}, {\em одно и~то~же}.
Истина состоит не~в~бытии и~не~в~ничто, а~в~том, что~бытие "---
не~переходит, а "--- перешло в~ничто, и~ничто "---~не~переходит,
а "---~перешло в~бытие. Но равным образом истина заключается
не~в~их неразличенности, а~в~том, что {\em они не одно~и~то~же}, что
{\em они абсолютно различны}, но~столь~же и~нераздельны
и~неотделимы и~что {\em каждое} из~них непосредственно
{\em исчезает в~своей противоположности}. Их истина
есть, следовательно, это {\em движение}
непосредственного исчезновения одного в другом:
{\em становление}; такое~движение, в~котором они оба
различны, но~таким~различием, которое столь~же непосредственно
растворилось.

\subsection*{Примечание 1. Противоположность бытия и ничто в представлении [ссылка!]}

{\em Ничто} обыкновенно противополагается [категории]
{\em нечто}; но нечто есть уже некое определенное,
сущее, отличающееся от другого {\em нечто}; таким
образом и ничто, противополагаемое [категории] нечто, есть ничто
какого-либо нечто, некое определенное ничто. Но здесь мы должны брать ничто
в его неопределенной простоте. "--- Если бы кто-нибудь считал более правильным
противополагать бытию {\em небытие} вместо ничто, то мы
не имели бы, что возразить против этого в рассуждении получающегося
результата, ибо в {\em небытии} содержится соотношение
с {\em бытием}; оно есть и то и другое, бытие и его
отрицание, высказанные одним духом, ничто, каково оно в становлении. Но
здесь ближайшим образом идет речь не о форме противоположения,~т.~е.
одновременно и о форме {\em соотношения}, а об
абстрактном, непосредственном отрицании, о ничто, взятом чисто само по
себе, о безотносительном отрицании, "--- что, если угодно, можно было бы
выразить также и простым <<{\em не}>>.

Простую мысль о {\em чистом бытии} как об абсолютном и
как единственную истину высказали впервые {\em элеаты},
преимущественно {\em Парменид}, и последний в
сохранившихся после него фрагментах высказал ее с чистым воодушевлением
мышления, в первый раз постигшего себя в своей абсолютной абстрактности:
{\em лишь бытие есть}, {\em а
небытия вовсе нет}. "--- В восточных системах, главным образом в буддизме,
{\em ничто}, пустота, является, как известно,
абсолютным принципом. "--- Глубокомысленный {\em Гераклит}
выдвинул в противоположность вышеуказанной простой и односторонней
абстракции более высокое, целостное понятие становления и сказал:
{\em бытие столь же мало есть},
{\em как и небытие}, или, выражая эту мысль также и
иначе, говорил: <<все{\em  течет}>>,~т.~е. все есть
{\em становление}. "--- Популярные, в особенности
восточные, изречения, гласящие, что все, что есть, носит зародыш своего
уничтожения в самом своем рождении, а смерть есть, наоборот, вступление в
новую жизнь, выражают в сущности то же самое единение бытия и ничто. Но эти
выражения предполагают субстрат, в котором совершается переход: бытие и
ничто удерживаются вне друг друга во времени, представляются как бы
чередующимися в нем, а не мыслятся в их абстрактности, и поэтому также и не
мыслятся так, чтобы они сами по себе были одним и тем же.

<<Ex nihilo nihil fit>> (ничто не происходит из ничего) "---~есть одно из тех
положений, которым некогда приписывалось в метафизике большое значение. В
этом положении можно либо усматривать лишь бессодержательную тавтологию:
ничто есть ничто; либо, если действительным смыслом этого положения
является высказывание о {\em становлении}, приходится
сказать, что так как из {\em ничего становится}
{\em ничто же}, то на самом деле здесь нет речи о
становлении, ибо ничто здесь так и остается ничем. <<Становление>> означает,
что ничто не остается ничем, а переходит в свое другое, в бытие. "--- Если
позднейшая метафизика, главным образом христианская, отвергла положение,
гласящее, что из ничего ничего не происходит, то она этим утверждала
переход ничто в бытие; как бы синтетически или, иначе сказать, в форме
просто представления она ни брала последнее положение, все же даже в самом
несовершенном соединении имеется точка, в которой бытие и ничто встречаются
и их различие исчезает. "--- Подлинную свою важность положение:
{\em из ничего ничего не происходит},
{\em ничто есть именно ничто}, получает благодаря его
антагонизму к {\em становлению} вообще и,
следовательно, также к сотворению мира из ничего. Те, которые утверждают
положение: ничто есть именно ничто, и даже горячо отстаивают его, не
сознают того, что они тем самым соглашаются с абстрактным
{\em пантеизмом} элеатов и по сути дела также и со
спинозовским пантеизмом. Философское воззрение, которое считает принципом
положение: <<бытие есть только бытие, ничто есть только ничто>>, заслуживает
названия системы тождества; это абстрактное тождество представляет собою
сущность пантеизма.

Если вывод, что бытие и ничто суть одно и то же, взятый сам по себе, кажется
удивительным или парадоксальным, то мы в дальнейшем не должны обращать на
это внимания; скорее приходится удивляться этому удивлению, которое
показывает себя таким новичком в философии и забывает, что в этой науке
встречаются совсем иные определения, чем те, которые имеют место в
обыденном сознании и в так называемом здравом человеческом рассудке,
который как раз не всегда есть здравый, а есть также рассудок, специально
культивированный для абстракций и для веры в них или, вернее, для
суеверного отношения к абстракциям. Было бы нетрудно показать наличие этого
единства бытия и ничто на всяком примере во всякой действительной вещи или
мысли. Относительно {\em бытия} и
{\em ничто} следует сказать то же самое, что мы сказали
выше относительно непосредственности и опосредствования (каковое последнее
заключает в себе некое соотношение {\em друг с другом}
и, значит, {\em отрицание}), а именно, что
{\em нет ничего ни на небе, ни на земле, что не
содержало бы в себе и того и другого, и бытия и ничто}. Так как при этом
начинают говорить о {\em каком-нибудь определенном
нечто и действительном}, то, разумеется, в этом нечто указанные определения
уже больше не наличествуют в той совершенной неистинности, в каковой они
выступают как бытие и ничто, а в некотором дальнейшем определении и
понимаются, например, как {\em положительное} и
{\em отрицательное}; первое есть положенное,
рефлектированное бытие, а последнее есть положенное, рефлектированное
ничто; но положительное и отрицательное содержат в себе как свою
абстрактную основу первое "---~бытие, а второе "---~ничто. "--- Так, например, в
самом боге качество, {\em деятельность},
{\em творение}, {\em могущество}
и~т.~д. содержат в себе по существу определение отрицательного, "--- они суть
продуцирование некоторого {\em другого}. Но
эмпирическое пояснение указанного положения примерами было бы здесь
совершенно излишне. Так как мы теперь знаем раз навсегда, что это единство
бытия и ничто лежит в основании в качестве первой истины и составляет
стихию всего последующего, то помимо самого становления все дальнейшие
логические определения: наличное бытие, качество, да и вообще все понятия
философии служат примерами этого единства. А так называющий себя
обыкновенный или здравый человеческий рассудок, поскольку он отвергает
нераздельность бытия и ничто, мы можем пригласить сделать попытку отыскать
пример, в котором мы могли бы найти одно отделенным от другого (нечто от
границы, предела, или бесконечное, бог, как мы только что упомянули, от
деятельности). Только пустые, сочиненные мыслью вещи (Gedankendinge)
"---~бытие и ничто "---~только сами они и суть такого рода раздельные, и их-то
вышеуказанный рассудок предпочитает истине, нераздельности, которую мы
повсюду имеем перед собой.

Нашим намерением не может быть всесторонне предупредить сбивчивые
возражения, путаные соображения, выдвигаемые обыденным сознанием, когда оно
имеет дело с таким логическим положением, ибо они неисчислимы. Мы можем
упомянуть лишь о некоторых из них. Одной из причин такой путаницы служит,
между прочим, то обстоятельство, что сознание привносит в такие абстрактные
логические положения представления о некотором конкретном нечто и забывает,
что речь идет вовсе не о таковом, а лишь о чистых абстракциях бытия и ничто
и что следует твердо держаться исключительно лишь этих последних.

Бытие и небытие суть одно и то же; {\em следовательно},
все равно, существую ли я или не существую, существует ли или не существует
этот дом, обладаю ли я или не обладаю ста талерами. Это умозаключение или
применение указанного положения совершенно меняет его смысл. В указанном
положении говорится о чистых абстракциях бытия и ничто; применение же
делает из них определенное бытие и определенное ничто. Но об определенном
бытии, как уже сказано, здесь нет речи. Определенное, конечное бытие есть
такое бытие, которое соотносится с чем-либо другим; оно есть содержание,
находящееся в отношении необходимости с другим содержанием, со всем миром.
Имея в виду взаимоопределяющую связь целого, метафизика могла выставить "---~в
сущности говоря, тавтологическое "---~утверждение, что если бы была разрушена
одна пылинка, то обрушилась бы вся вселенная. В примерах, приводимых против
рассматриваемого нами положения, представляется небезразличным, есть ли
нечто или его нет, не из-за бытия или небытия, а из-за его
{\em содержания}, связывающего его с другими
содержаниями. Когда {\em предполагается} некое
определенное содержание, какое-либо определенное существование, то это
существование, именно потому, что оно
"---~{\em определенное}, находится в многообразном
соотношении с другим содержанием. Для него небезразлично, есть ли известное
другое содержание, с которым оно находится в соотношении, или его нет, ибо
только благодаря такому соотношению оно существенно есть то, что оно есть.
То же самое имеет место и в {\em представлении}
(поскольку мы берем небытие в более определенном смысле, в котором оно
означает то, что мы представляем себе, в противоположность тому, что
действительно существует), в связи которого небезразлично, имеется ли бытие
или отсутствие некоторого содержания, которое как определенное
представляется нами в его соотношении с другим содержанием.

Это соображение захватывает то, что составляет один из главных моментов в
кантовской критике онтологического доказательства бытия божия, которого,
однако, мы здесь касаемся лишь в отношении встречающегося в нем различения
между бытием и ничто вообще и между {\em определенными}
бытием и небытием. "--- Как известно, в этом так называемом доказательстве
предпосылается, понятие некоторого существа, обладающего всякой реальностью
и, следовательно, также и существованием, которое также принималось за одну
из реальностей. Кантова критика напирает, главным образом, на то, что
{\em существование} или бытие (которые здесь считаются
равнозначными) не составляет {\em свойства} или
{\em реального предиката},~т.~е. не составляет понятия
чего-то такого, что может прибавиться к {\em понятию}
некоторой вещи\footnote{ <<Kants Kritik der r. Vern.>>, 2-te Aufl., 628
[ссылка!].}. --- Кант хочет этим сказать, что бытие не
есть определение содержания. --- Стало быть, продолжает он дальше,
действительное не содержит в себе чего-либо большего, чем возможное; сто
действительных талеров не содержат в себе ни капельки больше, чем сто
возможных талеров, а именно, первые не имеют другого определения
содержания, чем последние.

Для этого, рассматриваемого как изолированное, содержания в самом деле
безразлично быть или не быть; в нем не имеется различия бытия или небытия,
это различие вообще совсем не затрагивает его: сто талеров не сделаются
меньше, если их нет, и больше, если они есть. Различие должно притти из
другой сферы. "--- <<Напротив, "--- напоминает Кант, "--- мое имущественное состояние
больше при ста действительных талерах, чем при голом их понятии, или, иначе
говоря, чем при их возможности. Ибо {\em предмет},
когда он действителен, не содержится только аналитически в моем понятии, а
{\em присоединяется к моему понятию} (которое есть
некоторое {\em определение} моего
{\em состояния}) {\em синтетически}
без того, чтобы через это бытие вне моего понятия сами эти мыслимые сто
талеров хотя бы сколько-нибудь увеличились>>.

Здесь {\em предполагается} "---~если сохранить выражения
Канта, не свободные от тяжеловесной запутанности "---~двоякого рода состояния:
одно, которое Кант называет понятием, и под которым следует разуметь
представление, и другое "---~состояние имущества. Для одного, как и для
другого, для имущества, как и для представления, сто талеров суть некоторое
определение содержания, или, как выражается Кант, <<они присоединяются к
таковому {\em синтетически}>>. Я как
{\em обладатель} ста талеров или как необладатель их
или также я как {\em представляющий} себе сто талеров
или не представляющий себе их есть, во всяком случае, разное содержание.
Сформулируем в более общем виде: абстракции бытия и ничто перестают быть
абстракциями, когда они получают определенное содержание; бытие есть тогда
реальность, определенное бытие ста талеров, ничто есть отрицание,
определенное небытие последних. Самое же это определение содержания, сто
талеров, рассматриваемое равным образом абстрактно, само по себе, остается
неизменным, тем же самым как в том, так и в другом случае. Но когда, далее,
бытие берется как имущественное состояние, сто талеров вступают в связь с
некоторым состоянием, и для последнего такого рода определенность, которую
они составляют, не безразлична; их бытие или небытие есть лишь
{\em изменение}; они перемещены в сферу
{\em наличного бытия}. Если, поэтому, против единства
бытия и ничто выдвигается возражение, что ведь не все равно, существует ли
то-то (100 талеров) или не существует, то возражающие впадают в
заблуждение, относя различие между моим
{\em обладанием} и
{\em необладанием} ста талерами только за счет бытия
или небытия. Это заблуждение, как мы показали, основано на односторонней
абстракции, опускающей имеющееся в такого рода примерах
{\em определенное наличное бытие} и удерживающей лишь
бытие и небытие, так же как и, наоборот, превращающей то абстрактное бытие
и ничто, которое мы здесь должны мыслить, в некоторое определенное бытие и
ничто, в некоторое наличное бытие. Лишь наличное бытие содержит в себе
реальное различие между бытием и ничто, а именно, некое
{\em нечто} и некое {\em другое}. "---
Это реальное различие предносится представлению вместо абстрактного бытия и
чистого ничто и лишь мнимого различия между ними.

Как выражается Кант, <<через существование нечто вступает в контекст
совокупного опыта>>. <<Благодаря этому мы получаем одним предметом
{\em восприятия} больше, но наше
{\em понятие} о предмете этим не
умножается>>~[ссылка!]. "--- Это, как вытекает из предыдущего
разъяснения, означает следующее: через существование, именно потому, что
нечто есть определенное существование, оно находится в связи с
{\em другим} и, между прочим, также и с неким
воспринимающим. "--- <<Понятие ста талеров, "--- говорит Кант, "--- не умножается
вследствие того, что их воспринимают>>. {\em Понятием}
Кант здесь называет вышеозначенные представляемые
{\em изолированно} сто талеров. В такой изолированности
они, правда, суть некоторое эмпирическое содержание, но содержание
оторванное, не связанное с {\em другим} и лишенное
определенности в отношении {\em другого}. Форма
тождества с собою лишает их отношения к другому и делает их безразличными к
тому, воспринимаются ли они или нет. Но это так называемое
{\em понятие} ста талеров есть ложное понятие; форма
простого соотношения с собою не принадлежит самому подобного рода
ограниченному, конечному содержанию; она есть форма, прибавленная и
внесенная в него субъективным рассудком; сто талеров суть не некоторое
соотносящееся с собою, а некоторое изменчивое и преходящее.

Мышление или представление, которому предносится лишь некое определенное
бытие "---~наличное бытие, "--- следует отослать к вышеупомянутому первому шагу
науки, сделанному Парменидом, который очистил свое представление и,
следовательно, тем самым также и представление последующих времен, возвысил
его до {\em чистой мысли}, до бытия как такового, и
этим создал стихию науки. "--- То, что является
{\em первым в науке}, должно было явить себя первым
также и {\em исторически}. И элеатское
{\em единое} или {\em бытие} мы
должны рассматривать как первый шаг знания о мысли:
{\em вода}~[ссылка!] и тому
подобные материальные начала, хотя и {\em должны}, по
мысли выдвигавших их философов, представлять собою всеобщее, однако как
материи они не суть чистые мысли;
{\em числа}~[ссылка!] же не суть ни первая
простая, ни остающаяся у себя мысль, а мысль, всецело внешняя самой себе.

В отсылке от {\em особенного конечного} бытия к бытию,
как таковому, взятому в его совершенно абстрактной всеобщности, следует
видеть наипервейшее как теоретическое, так даже и практическое требование.
А именно, если так носятся с этими ста талерами, придают такую важность
тому указанию, что для моего имущественного состояния составляет разницу,
{\em обладаю} ли я ими или
{\em нет}, и что еще больше разницы, существую ли я или
нет, существует ли другое или нет, то "---~не говоря уже о том, что могут
существовать такие имущественные состояния, для которых такое обладание ста
талерами будет безразлично, "--- можно напомнить, что человек должен подняться
в своем умонастроении до такой абстрактной всеобщности, стоя на точке
зрения которой ему в самом деле будет все равно, существуют ли или не
существуют эти сто талеров, каково бы ни было их количественное отношение к
его имущественному состоянию, а также ему будет все равно, существует ли он
или нет,~т.~е. существует ли он или нет в конечной жизни (ибо имеется в
виду некое состояние, определенное бытие) и~т.~д. Даже si fractus illabatur
orbis, impavidum ferient ruinae (если бы на него обрушился весь мир, он без
страха встретит смерть под его развалинами), сказал один
римлянин~[ссылка!], а тем паче должно быть присуще такое
безразличие христианину.

Следует еще вкратце отметить непосредственную связь, в которой находится
возвышение над ста талерами и вообще над конечными вещами с онтологическим
доказательством и вышеприведенной кантовской критикой последнего. Эта
критика показалась всем убедительной благодаря приведенному ею популярному
примеру; кто же не знает, что сто действительных талеров отличны от ста
лишь возможных талеров, кто не знает, что это составляет разницу в моем
имущественном состоянии. Так как таким образом на примере ста талеров
явственно видна эта разница, то понятие,~т.~е. определенность содержания
как пустая возможность, и бытие разнятся друг от друга;
{\em стало быть}, понятие бога и его бытие также
различны, и сколь мало я могу из возможности ста талеров вывести их
действительность, столь же мало я могу из понятия бога <<выколупать>>
(herausklauben) его существование; а в таком выколупывании существования
бога из его понятия и состоит-де онтологическое доказательство. Но если
несомненно правильно, что понятие отлично от бытия, то бог еще более
отличен от ста талеров и других конечных вещей. В том и состоит
{\em дефиниция конечных вещей}, что в них понятие и
бытие различны, понятие и реальность, душа и тело отделены друг от друга, и
они, значит, преходящи и смертны; напротив, абстрактной дефиницией бога
является именно то, что его понятие и его бытие
{\em нераздельны} и
{\em неотделимы}. Истинная критика категорий и разума и
заключается как раз в том, чтобы просветить познание относительно этого
различия и удерживать его от применения определений и отношений конечного к
богу.

\subsection*{Примечание 2. Неудовлетворительность выражения: единство, тождество бытия и ничто}

Следует далее указать на другое основание, способствующее усилению антипатии
к положению о бытии и ничто. Это основание заключается в том, что выражение
вывода, получающегося из рассмотрения бытия и ничто, предложением:
{\em бытие и ничто суть одно и то же}, страдает
несовершенством. Ударение падает преимущественно на
<<{\em суть одно и то же}>>, как это и вообще происходит
в суждении, поскольку в нем лишь предикат впервые высказывает, что
представляет собою субъект суждения. Получается поэтому видимость, будто
смысл вывода заключается в отрицании различия, которое, однако, вместе с
тем ведь непосредственно имеется в предложении, ибо оно высказывает
{\em оба} определения, бытие и ничто, и содержит их в
себе как различные. --- Смысл предложения не может также быть тот, что
следует от них абстрагироваться и удерживать лишь единство. Этот смысл сам
выдавал бы себя за односторонний, так как то, от чего якобы должно
отвлекаться, все же имеется и называется в предложении. --- Итак, поскольку
предложение: {\em бытие и ничто есть одно и то же},
высказывает тождество этих определений, но на самом деле также и содержит в
себе эти два определения как различные, постольку оно противоречиво в самом
себе и разлагает себя. Если мы это строже зафиксируем, то окажется, что
здесь, следовательно, дано предложение, относительно которого при более
близком рассмотрении мы должны признать, что оно ведет к тому, чтобы
заставить само себя исчезнуть. Но этим в нем самом совершается то, что
затем составит его настоящее содержание, а именно
{\em становление}.

Рассматриваемое нами предложение, таким образом,
{\em содержит} в себе результат, оно есть
{\em в самом себе} этот результат. Но здесь мы должны
обратить внимание на тот недостаток, что результат сам
{\em не выражен} в предложении; только внешняя
рефлексия познает его в последнем. --- По этому поводу мы должны уже в самом
начале сделать то общее замечание, что предложение
{\em в форме суждения} не пригодно для выражения
спекулятивных истин. Знакомство с этим обстоятельством могло бы устранить
многие недоразумения касательно спекулятивных истин. Суждение есть
отношение {\em тождества} между субъектом и предикатом,
и при этом отвлекаются от того, что субъект обладает еще многими другими
определенностями, чем те, которыми обладает предикат, равно как и от того,
что предикат шире, субъекта. Но если содержание спекулятивно, то
{\em нетождественное} в субъекте и предикате также
составляет существенный момент. Однако это не выражено в суждении.
Парадоксальный и странный свет, в котором не освоившимся со спекулятивным
мышлением представляются многие положения новейшей философии, очень часто
зависит от формы простого суждения, когда она употребляется для выражения
спекулятивных выводов.

Чтобы выразить спекулятивную истину, указанный недостаток устраняют
ближайшим образом тем, что восполняют предложение, прибавляя к нему
противоположное предложение: {\em бытие и ничто не суть
одно и то же}, каковое предложение было равным образом высказано выше. Но,
таким образом, возникает дальнейший недостаток, а именно: эти предложения
не связаны между собою и, стало быть, излагают содержание лишь в антиномии,
между тем как их содержание ведь относится к одному и тому же, и
определения, выраженные в этих двух предложениях, должны быть безусловно
соединены, --- тогда получится соединение, которое может быть высказано лишь
как некое {\em беспокойство несовместимых} вместе
определений, как {\em некое движение}. Обычнейшая
несправедливость, совершаемая по отношению к спекулятивному содержанию,
заключается в том, что его делают односторонним,~т.~е. выпячивают лишь одно
из тех предложений, на которое оно может быть разложено. Нельзя в таком
случае отрицать, что это предложение действительно утверждается,
{\em но насколько даваемое им указание правильно},
{\em настолько же оно} и
{\em ложно}, ибо раз из области спекулятивного берут
{\em одно} предложение, то следовало бы по меньшей мере
в той же степени обратить внимание также и на другое предложение и указать
его. --- При этом нужно еще особо отметить так сказать неудачное слово
<<{\em единство}>>. <<{\em Единство}>>
еще больше, чем <<{\em тождество}>>, обозначает
субъективную рефлексию. Оно берется преимущественно как соотношение,
получающееся {\em из сравнивания}, из внешней
рефлексии. Поскольку последняя находит в двух
{\em разных предметах} одно и то же, единство имеется
таким образом, что при этом предполагается полнейшее
{\em равнодушие} самих сравниваемых предметов к этому
единству, так что это сравнивание и единство вовсе не касаются самих
предметов и представляют собою некое внешнее для них действование и
определение. <<Единство>> выражает поэтому совершенно
{\em абстрактное} <<одно и то же>> и звучит тем жестче и
удивительнее, чем больше те предметы, о которых оно высказывается, являют
себя без всяких затей различными. Постольку было бы лучше поэтому вместо
<<единства>> говорить лишь <<{\em нераздельность}>> и
<<{\em неотделимость}>>; но эти термины не выражают
{\em утвердительной} стороны отношения целого.

Таким образом, полным, истинным выводом, получившимся здесь, является
{\em становление}, которое не есть исключительно лишь
одностороннее или абстрактное единство бытия и ничто. Становление состоит в
следующем движении: чистое бытие непосредственно и просто; оно поэтому в
такой же мере есть чистое ничто; различие между ними
{\em есть}, но в такой же мере
{\em упраздняет себя} и {\em не
есть}. Результат следовательно утверждает также и различие между бытием и
ничто, но как такое различие, которое лишь {\em имеется
в виду} (gemeinten).

Мы имеем в виду, что бытие есть нечто безоговорочно другое, чем ничто, и
ничего нет яснее того, что они абсолютно различны, и, кажется, ничего нет
легче, чем указать это различие. Но столь же легко убедиться в том, что это
невозможно, что оно {\em несказуемо}.
{\em Пусть те, которые упорно}
{\em настаивают на различии между бытием и ничто,
потрудятся указать, в чем же оно состоит}. Если бы бытие и ничто имели в
себе какую-либо определенность, которой они отличались бы друг от друга, то
они, как мы уже говорили, были бы определенным бытием и определенным ничто,
а не чистым бытием и чистым ничто, каковы они еще суть здесь. Поэтому
различие между ними есть совершенно пустое, каждое из них есть в равной
мере неопределенное. Это различие имеется поэтому
{\em не} в них самих, а лишь в некотором третьем, в
{\em имении в виду}. Но имение в виду есть некая форма
субъективного, которое не должно находить себе место в этом контексте.
Напротив того, то третье, в котором имеют свое существование бытие и ничто,
должно найти себе место также и здесь; и оно, действительно, нашло себе
здесь место; это ---~{\em становление}. В нем они суть
как различные; становление имеется лишь постольку, поскольку они различны.
Это третье есть нечто другое, чем они. Они существуют лишь в некотором
другом. Это вместе с тем означает, что они не существуют особо (für sich).
Становление есть данность (das Bestehen) в одинаковой мере как бытия, так и
небытия, или, иначе говоря, их данность есть лишь их бытие в
{\em одном}; как раз эта их данность и есть то, что
также и снимает их различие.

Требование указать различие между бытием и ничто заключает в себе также и
требование сказать, что же такое {\em бытие} и что
такое {\em ничто}. Пусть те, которые упираются, не
желая признать, что и первое и второе есть лишь
{\em переход} одного в другое, и утверждают о бытии и
ничто то и сё, --- пусть они укажут, о {\em чем} они
говорят,~т.~е. пусть дадут {\em дефиницию} бытия и
ничто и докажут, что она правильна. Без удовлетворения этого первого
требования старой науки, логические правила которой они во всех других
случаях признают и применяют, все указанные их утверждения о бытии и ничто
представляют собою только заверения, нечто лишенное научной значимости.
Если, например, в прежнее время говорили, что существование, поскольку его
в данном случае считают равнозначным бытию, есть
{\em дополнение} к
{\em возможности}, то этим предполагается другое
определение, возможность, и бытие высказывается не в его непосредственности
и даже не как нечто самостоятельное, а как обусловленное. Для обозначения
такого бытия, которое {\em опосредствовано}, мы
сохраним выражение {\em существование}. Но, могут
сказать, мы ведь {\em представляем себе} бытие, ---
представляем его себе, примерно, под образом чистого света, как ясность
непомутненного видения, а ничто, как чистую ночь, и связываем их различие с
этой хорошо знакомой чувственной разницей. Однако на самом деле, если мы
точнее представим себе также и это видение, то мы сможем легко заметить,
что в абсолютной ясности мы так же много и так же мало видим, как и в
абсолютной тьме, что и то и другое видение есть чистое видение,~т.~е.
ничего не видящее видение. Чистый свет и чистая тьма представляют собою две
пустоты, которые суть одно и то же. Лишь в определенном свете ---~а свет
определяется тьмой, --- следовательно, в помутненном свете, и точно так же
лишь в определенной тьме ---~а тьма определяется светом, --- в освещенной тьме
получается впервые возможность что-то различать, так как лишь помутненный
свет и освещенная тьма имеют различие в самих себе и, следовательно, суть
определенное бытие, {\em наличное бытие}.

\subsection*{Примечание 3. Изолирование этих абстракций}

Единство, моменты которого, бытие и ничто, даны как нераздельные, вместе с
тем отлично от них самих, есть в отношении их некое
{\em третье}, которое в своей своеобразнейшей форме
есть {\em становление}.
<<{\em Переход}>> есть то же самое, что и становление, с
той только разницей, что в первом мы представляем себе те два определения,
от одного из которых совершается переход к другому, больше находящимися в
покое друг вне друга, а переход ---~совершающимся
{\em между} ними. Где бы и как бы ни шла речь о бытии
или ничто, там непременно должно наличествовать это третье; ибо бытие и
ничто имеются не сами по себе, а суть лишь в становлении, в этом третьем.
Но это третье имеет многоразличные эмпирические образы, которые абстракция
оставляет в стороне или которыми она пренебрегает, чтобы фиксировать каждый
из указанных ее продуктов, бытие и ничто, особо и показать их защищенными
от перехода. В противовес такому простому поведению абстракции мы должны
столь же просто напомнить лишь об эмпирическом существовании, в котором
сама эта абстракция есть лишь нечто, обладает некоторым наличным бытием.
Или же задача фиксировать разделение неразделимых выпадает на долю
каких-нибудь других форм рефлексии. В таком определении уже само по себе
имеется его собственная противоположность, так что и без того, чтобы
восходить к природе самой вещи и апеллировать к ней, можно изобличить это
определение рефлексии в нем же самом, беря его так, как оно само себя дает,
и обнаруживая его другое в нем самом. Было бы напрасным трудом стараться
как бы изловить все извороты, все шальные мысли рефлексии и ее рассуждений,
чтобы лишить ее возможности пользоваться теми лазейками и прыжками в
сторону, которыми она скрывает от себя свое противоречие с самой собой.
Поэтому я и воздерживаюсь от того, чтобы принимать во внимание те
многочисленные, так называющие себя возражения и опровержения, которые
выдвигались против положения, гласящего, что ни бытие, ни ничто не суть
нечто истинное, а что их истиной является лишь становление. Культура мысли,
требующаяся для того, чтобы усмотреть ничтожность этих опровержений, или,
вернее, чтобы отогнать от самого себя такие шальные мысли, дается лишь
критическим познанием форм рассудка. Но те, которые щедрее всего на
подобного рода возражения, сразу набрасываются со своими соображениями уже
на первые положения, не давая себе труда после этого или до этого путем
дальнейшего изучения логики помочь себе осознать природу этих плоских
соображений.

Здесь следует рассмотреть некоторые явления, получающиеся от того, что
изолируют друг от друга бытие и ничто и помещают одно вне области другого,
так что тем самым отрицается переход.

{\em Парменид} фиксировал бытие и был как нельзя более
последователен, говоря вместе с тем о ничто, что его
{\em вовсе нет}; лишь бытие, говорит он, есть. Бытие,
взятое совершенно отдельно, есть неопределенное, не находится,
следовательно, ни в каком соотношении с другим; кажется поэтому, что,
исходя {\em из этого начала}, нельзя
{\em двигаться} {\em дальше}, что,
для того, чтобы двинуться дальше, надо присоединить к нему
{\em извне} нечто чуждое. Дальнейшее движение,
заключающееся в положении, гласящем, что бытие есть то же самое, что ничто,
представляется, стало быть, вторым, абсолютным началом, переходом, стоящим
отдельно и внешне привходящим к бытию. Бытие не было бы вообще абсолютным
началом, если бы оно обладало некоторой определенностью; оно тогда зависело
бы от другого и не было бы непосредственным, не было бы началом. Если же
оно неопределенно и тем самым есть истинное начало, то оно не обладает
ничем таким, с помощью чего оно перевело бы себя к некоторому другому, оно
есть вместе с тем и {\em конец}. Из него столь же мало
может что-либо вырваться, как и ворваться в него; у Парменида, как и у
{\em Спинозы}, не должно быть поступательного движения
от бытия или абсолютной субстанции к отрицательному, конечному. Если же
все-таки совершают такой переход (что, как мы заметили, если брать исходным
пунктом лишенное соотношений и, стало быть, лишенное перехода бытие, можно
осуществить только внешним образом), то этот переход есть второе, новое
начало. Так, например, у {\em Фихте} его абсолютнейшее,
безусловное основоположение (А=А) есть {\em полагание};
второе основоположение есть {\em противополагание}; это
второе основоположение, согласно ему, {\em отчасти}
обусловлено, {\em отчасти} безусловно (оно есть,
следовательно, противоречие внутри себя). Это ---~поступательное движение
внешней рефлексии, которая столь же отрекается от того, с чего она начинает
как с абсолютного положения, --- противоположение есть отрицание первого
тождества, --- сколь и вместе с тем тотчас же нарочито делает свое второе
безусловное условным. Но если бы здесь поступательное движение,~т.~е.
снятие первого начала, было вообще правомерно, то в самом этом первом
должна была бы заключаться возможность того, чтобы некое другое могло
соотноситься с ним; оно, стало быть, должно было бы быть
{\em определенным}. Однако
{\em бытие} или даже абсолютная субстанция себя не
выдает за таковое. Напротив. Оно есть
{\em непосредственное}, еще всецело
{\em неопределенное}.

Красноречивейшие, быть может, забытые описания невозможности, начиная с
некоторой абстракции, притти к чему-то дальнейшему, а затем к их
объединению дает {\em Якоби} в интересах своей полемики
против кантовского априорного {\em синтеза}
самосознания в своей статье <<О предприятии критицизма вернуть разуму
рассудок>> (Jac. Werke, Bd. III). Он ставит (стр. 113) задачу так, что
требуется в некотором {\em чистом}, будь то чистое
сознание, чистое пространство или чистое время, обнаружить возникновение
или порождение синтеза. <<Пространство есть {\em одно},
время есть {\em одно}, сознание есть
{\em одно}; скажите же мне, каким образом какое-либо из
этих трех <<одно>> в самом себе, в своей {\em чистоте}
приобретает характер некоторого многообразия? Ведь каждое из них есть лишь
некоторое {\em одно} и не заключает в себе
{\em никакого другого}: одинаковость (eine
Einerleiheit), {\em самотождество} без определенного
<<того-то>> (eine
\foreignlanguage{english}{Der}{}-\foreignlanguage{english}{Die}{}-\foreignlanguage{english}{Das}{}-\foreignlanguage{english}{Selbigkeit}!
\foreignlanguage{english}{ohne} \foreignlanguage{english}{Derheit},
\foreignlanguage{english}{Dieheit}, \foreignlanguage{english}{Dasheit}),
ибо последнее еще дремлет вместе с <<{\em этот}>>,
<<{\em эта}>>, <<{\em это}>> в
бесконечности = 0 неопределенного, из которой еще только должно, произойти
все и всякое {\em определенное}. Чем вносится
{\em конечность} в эти три бесконечности? Что
оплодотворяет à priori пространство и время числом и мерой и превращает их
в некоторое {\em чистое многообразие}? Что приводит в
колебание {\em чистую спонтанейность} (<<я>>)? Каким
образом его чистая гласная получает согласную, или, лучше сказать, каким
образом приостанавливается, прерывая само себя,
{\em беззвучное} непрерывное дуновение этого <<я>>, чтобы
приобрести по крайней мере некоторый род гласного звука, некоторый
{\em акцент}?>> ---~Как видим,
{\em Якоби} очень ясно познал абсурдность (das Unwesen)
абстракции, будь она так называемое абсолютное,~т.~е. лишь абстрактное,
пространство, или такое же время, или такое же чистое сознание, <<я>>. Он
настаивает на этом, чтобы доказать свое утверждение о невозможности
дальнейшего перехода к другому, являющемуся условием синтеза, и к самому
синтезу. Этот интересующий нас в данном случае синтез не следует понимать,
как соединение {\em внешне} уже имеющихся определений,
--- отчасти дело идет о порождении некоторого второго, присоединяющегося к
некоторому первому, о порождении некоторого определенного,
присоединяющегося к неопределенному первоначальному, отчасти же об
{\em имманентном} синтезе, синтезе à priori, --- о
в-себе-и-для-себя-сущем единстве различных.
{\em Становление} и есть этот имманентный синтез бытия
и ничто. Но так как со словом <<синтез>> мы ближайшим образом соединяем смысл
внешнего сведения вместе таких определений, которые находятся в чисто
внешнем отношении друг к другу, то справедливо перестали пользоваться
названиями <<синтез>>, <<синтетическое единство>>. --- Якоби спрашивает,
{\em каким} {\em образом} чистая
гласная чистого <<я>> доходит до получения согласной,
{\em что} вносит определенность в неопределенность? На
вопрос: {\em что?} ---~было бы нетрудно ответить, и Кант
по-своему дал ответ на этот вопрос. А вопрос:
{\em как?} ---~означает: каким родом и способом, по каким
отношениям и~т.~п., и требует таким образом указания некоторой особой
категории; но о роде и способе, о рассудочных категориях здесь не может
быть речи. Вопрос: {\em как?} сам представляет собою
одну из дурных манер рефлексии, которая спрашивает о постижимости (nach der
Begreiflichkeit), но при этом берет предпосылкой свои застывшие категории и
тем самым знает наперед, что она вооружена против ответа на то, о чем она
спрашивает. Высшего смысла, смысла вопроса о
{\em необходимости} синтеза, он не имеет также и у
Якоби, ибо последний, как сказано, крепко держится за абстракции, защищая
утверждение о невозможности синтеза. С особенной наглядностью он описывает
(стр. 147) процедуру, посредством которой доходят до абстракции
пространства. <<Я должен на столь долгое время стараться начисто забыть, что
я когда-либо что-нибудь видел, слышал, к чему-либо прикасался, причем я
определенно не должен делать исключения и для самого себя. Я должен
начисто, начисто, начисто забыть всякое движение, и это последнее
{\em забвение} я должен проделать наиболее старательным
образом именно потому, что оно всего труднее. И все вообще я должен всецело
и полностью {\em удалить}, как я его уже отмыслил, и
ничего не должен сохранить, кроме {\em насильственно}
оставшегося созерцания одного лишь бесконечного
{\em неизменного пространства}. Я поэтому не имею права
{\em снова в него вмысливать} даже самого себя как
нечто отличное от него и, однако, связанное с ним; я даже не смею просто
представлять себя {\em окруженным} и
{\em проникнутым} им, а должен полностью
{\em перейти} в него, стать с ним единым, превратиться
в него; я не должен ничего оставить от себя, кроме самого
{\em этого моего созерцания}, чтобы рассматривать это
последнее, как истинно самостоятельное, независимое, единое и всеединое
представление>>.

При такой совершенно абстрактной чистоте непрерывности,~т.~е. при этой
неопределенности и пустоте представления, является совершенно безразличным,
будем ли мы называть эту абстракцию пространством, чистым созерцанием или
чистым мышлением; все это ---~то же самое, что индус называет
{\em брамой}, когда он, оставаясь внешне в состоянии
неподвижности, а также в состоянии неподвижности чувствования,
представления, фантазии, желаний и~т.~д., годами смотрит лишь на кончик
своего носа и лишь говорит внутренне, самому себе, <<ом, ом, ом>> или даже
ничего не говорит. Это заглушённое, пустое сознание, понимаемое как
сознание, есть {\em бытие}.

В этой пустоте, говорит далее Якоби, с ним происходит противоположное тому,
что должно было бы произойти с ним, согласно уверению Канта; он находит
себя не некоторым {\em множественным} и
{\em многообразным}, а, наоборот, единым без всякой
множественности, без всякого многообразия; даже больше того: <<...я есть
сама {\em невозможность}, само
{\em уничтожение} всякого многообразного и
множественного... Исходя из своей чистой, безоговорочно простой и
неизменной сущности, я {\em не в состоянии
восстановить} в себе или ввести в себя как признак хотя бы <<малейшую часть
того, что я отмыслил. Таким образом, в этой чистоте все внеположное и
рядоположное, всякое покоящееся на нем многообразие и множественность
открывает себя как {\em чистую невозможность}>> (стр.
149).

Эта невозможность есть не что иное, как тавтология, означает лишь то, что я
крепко держусь абстрактного единства и исключаю всякую множественность,
всякое многообразие, пребываю в лишенном различий и неопределенном и
отвлекаюсь от всего различенного и определенного. Кантовский априорный
синтез самосознания,~т.~е. деятельность этого единства, состоящую в том,
что оно расщепляет себя и в этом расщеплении сохраняет само себя, Якоби
превращает в такую же абстракцию. Этот <<синтез {\em в
себе}>>, <<{\em первоначальное суждение}>> он односторонне
превращает (стр.\foreignlanguage{english}{~}125) в
<<{\em связку в себе} ---~в словечко
<<{\em есть}>>, <<{\em есть}>>,
<<{\em есть}>>, без начала и конца и без <<что>>, <<кто>> и
<<какие>>. Это продолжающееся до бесконечности повторение повторения
составляет единственное занятие, функцию и произведение наичистейшего
синтеза; сам он есть само голое, чистое, абсолютное повторение>>. Или, в
самом деле, так как в нем нет никакого перерыва,~т.~е. никакого отрицания,
различения, то он есть не повторение, а только неразличенное простое бытие.
--- Но есть ли это еще синтез, если Якоби опускает как раз то, благодаря чему
единство есть синтетическое единство?

Если Якоби так прочно устраивается в абсолютном,~т.~е. абстрактном
пространстве, времени, а также и сознании, то, прежде всего, следует
сказать, что он таким образом избирает себе помещение и устраивается в
чем-то {\em эмпирически} ложном. Нет,~т.~е. эмпирически
не существует, таких пространств и времен, которые были бы неограниченными
пространствами и временами, не были бы наполнены в своей непрерывности
многообразно ограниченным наличным бытием и изменением, так что эти границы
и изменения нераздельно и неотделимо принадлежат пространственности и
временности. И точно так же и сознание наполнено определенными чувствами,
представлениями, желаниями и~т.~д.; оно существует нераздельно от
какого-нибудь особенного содержания. --- Эмпирический
{\em переход}, впрочем, и без того сам собою понятен;
сознание может, правда, сделать своим предметом и содержанием пустое
пространство, пустое время и само пустое сознание, или чистое бытие, но оно
на этом не останавливается, а вырывается из такой пустоты, устремляется к
некоторому лучшему,~т.~е. к каким бы то ни было образом более конкретному
содержанию, и, как бы ни было плохо в других отношениях то или иное
содержание, оно постольку лучше и истиннее; именно такого рода содержание
есть синтетическое содержание вообще, синтетическое в более общем смысле.
Так например, {\em Пармениду} приходится иметь дело с
видимостью и мнением, с противоположностью бытия и истины; и таким же
образом {\em Спинозе} приходится иметь дело с
атрибутами, модусами, протяжением, движением, рассудком, волей и~т.~д.
Синтез содержит в себе и показывает неистинность указанных выше абстракций;
в нем они находятся в единстве со своим другим, даны, следовательно, не как
сами по себе существующие, не как абсолютные, а всецело как относительные.

Но речь идет не об обнаружении эмпирической ничтожности пустого пространства
и~т.~д. Сознание может, конечно, путем абстрагирования наполнить себя также
и этими неопределенными вещами, и фиксированные абстракции суть мысли о
чистом пространстве, чистом времени, чистом сознании, чистом бытии. Должна
быть обнаружена ничтожность мысли о чистом пространстве и~т.~д.,~т.~е.
чистого пространства и~т.~д., взятого в {\em нем
самом},~т.~е. должно быть показано, что оно как таковое уже есть своя
противоположность, что в него, взятого в нем самом, уже проникла его
противоположность, что оно уже само по себе есть нечто вышедшее за пределы
самого себя ---~определенность.

Но это получается непосредственно в них же. Они, как подробно описывает
Якоби, суть результаты абстракции, ясно определены как
{\em неопределенное}, которое ---~если обратиться к его
простейшей форме ---~есть бытие. Но именно эта
{\em неопределенность} как раз есть то, что составляет
его определенность; ибо неопределенность противоположна определенности;
она, стало быть, как противоположное, сама есть нечто определенное или,
иначе говоря, отрицательное, и притом чистое, совершенно абстрактное
отрицательное. Эта-то неопределенность или абстрактное отрицание, которое
бытие, таким образом, имеет в самом себе, и есть то, что высказывает как
внешняя, так и внутренняя рефлексия, приравнивая его (бытие) к ничто,
объявляя его пустой, сочиненной мыслью вещью, --- ничем. --- Или можно это
выразить иначе: так как бытие есть нечто лишенное определений, то оно есть
не (утвердительная) определенность, не бытие, а ничто.

В чистой рефлексии начала, каковым в этой логике служит
{\em бытие} как таковое, переход еще скрыт. Так как
{\em бытие} положено лишь как непосредственное, то
{\em ничто} выступает в нем наружу лишь
непосредственно. Но все последующие определения, как, например,
появляющееся тотчас же {\em наличное бытие}, более
конкретны; в последнем уже {\em положено} то, что
содержит в себе и порождает противоречие вышеуказанных абстракций, и
поэтому содержит в себе и порождает также и их переход. Относительно бытия,
как указанного простого, непосредственного, воспоминание о том, что оно
есть результат полной абстракции и, стало быть, уже потому есть абстрактная
отрицательность, ничто, --- это воспоминание оставлено за порогом науки,
которая в своих пределах, особенно в отделе о
{\em сущности}, изобразит эту одностороннюю
{\em непосредственность}, как некое опосредствованное,
где будет {\em положено} бытие как
{\em существование}, а также и опосредствующее это
бытие основание.

С помощью этого воспоминания можно представить или даже, как это называют,
{\em объяснить и сделать постижимым} переход бытия в
ничто, как нечто такое, что само является легким и тривиальным, а именно
так, что бытие, сделанное нами началом науки, есть, разумеется, ничто, ибо
ведь можно от всего абстрагироваться, а когда мы от всего абстрагировались,
то остается ничто. Но, можно продолжать далее, стало быть, началом здесь
служит не некое утвердительное, не бытие, а как раз ничто, и ничто
оказывается также и {\em концом}; оно оказывается этим
концом с таким же правом, как непосредственное бытие, и даже с еще большим
правом, чем последнее. Короче всего опровергнем такое рассуждение, если
дадим ему развернуться до конца и посмотрим, каков же характер того вывода,
которым оно так чванится. Нечего беспокоиться, что согласно этому оказалось
бы, что ничто составляет конечный вывод указанного рассуждения и что нам
теперь следует начинать (как в китайской философии) с ничто; из-за этого не
стоит даже шевельнуть рукой, ибо раньше, чем мы повернули бы ею, это ничто
превратилось бы с одинаковым правом в бытие (см. выше: В. Ничто). Но,
далее, если берется предпосылкой указанное абстрагирование от
{\em всего}, каковое <<все>> ведь все же есть
{\em сущее}, то мы должны отнестись серьезнее к этому
абстрагированию; результатом абстрагирования от всего сущего является
ближайшим образом абстрактное бытие, {\em бытие}
вообще; так, в космологическом доказательстве бытия божия из случайного
бытия мира, выше которого (случайного бытия) мы поднимаемся в этом
доказательстве, {\em бытие} также поднимается нами выше
и приобретает определение {\em бесконечного бытия}. Но,
конечно, {\em можно} абстрагироваться также и от этого
чистого бытия, присоединить также и бытие ко всему тому, от чего мы уже
абстрагировались; тогда остается ничто. {\em Можно},
затем, если решить забыть о {\em мышлении} этого
ничто,~т.~е. о его переходе в бытие, или если бы мы ничего не знали об этом
---~{\em можно} продолжать далее в стиле этого <<можно>>; а
именно можно (слава богу) абстрагироваться также и от этого ничто
(сотворение мира и в самом деле есть абстрагирование от ничто), и тогда
остается не ничто, ибо как раз от него мы абстрагировались, а мы снова
прибыли в бытие. --- Эта <<{\em возможность}>> дает внешнюю
игру абстрагирования, причем само абстрагирование есть лишь одностороннее
дело отрицательного. Сама эта <<возможность>> подразумевает ближайшим
образом, что для нее бытие так же безразлично, как и ничто, и что в какой
мере они оба исчезают, в такой же мере они также и возникают; но столь же
безразлично, будем ли мы отправляться от действия ничто или от ничто;
действие ничто,~т.~е. голое абстрагирование, есть нечто не более и не менее
истинное, чем голое ничто.

Та диалектика, придерживаясь которой {\em Платон}
трактует в <<Пармениде>> единое, равным образом должна быть признана больше
диалектикой внешней рефлексии. Бытие и единое суть оба элеатские формы,
представляющие собою одно и то же. Но их следует также и различать друг от
друга. Такими и берет их Платон в упомянутом диалоге. Удалив из единого
разнообразные определения целого и частей, бытия в себе и бытия в другом
и~т.~д., определения фигуры, времени и~т.~д., он приходит к выводу, что
единому не присуще бытие, ибо бытие присуще некоторому нечто не иначе, как
по одному из указанных видов определения (Сочинения Платона, издание
Стефана,\foreignlanguage{english}{~}т.\foreignlanguage{english}{~}III,
стр.\foreignlanguage{english}{~}141, е). Затем Платон рассматривает
положение, гласящее: {\em единое}
{\em есть}; и следует читать у него, чтобы увидеть,
каким образом он, исходя из этого положения, получает переход к
{\em небытию} единого. Этот переход совершается путем
{\em сравнения двух} определений предпосылаемого
положения: {\em единое есть}. В этом положении
содержится единое и бытие, и <<единое {\em есть}>>
содержит в себе больше, чем если бы мы только сказали: <<единое>>. Тем
обстоятельством, что они различны, мы показываем содержащийся в положении
момент отрицания. Ясно, что этот путь имеет некую предпосылку и есть
некоторая внешняя рефлексия.

Как здесь единое приведено в связь с бытием, так и бытие, которое должно
быть фиксировано абстрактно, {\em особо}, --- простейшим
образом, не пускаясь в мышление, обнаруживается в связи, содержащей в себе
противоположность тому, что должно утверждаться. Бытие, взятое так, как оно
есть непосредственно, принадлежит некоторому
{\em субъекту}, есть нечто высказанное, обладает вообще
некоторым эмпирическим {\em существованием} (Dasein) и
поэтому стоит на почве ограниченного и отрицательного. В каких бы
выражениях или оборотах ни формулировал себя рассудок, когда он не хочет
признать единство бытия и ничто и ссылается на то, что, дескать,
непосредственно налично, он все же как раз в этом опыте не найдет ничего
другого, кроме {\em определенного} бытия, бытия с
некоторым пределом или отрицанием, --- не найдет ничего другого, кроме того
единства, которое он отвергает. Утверждение непосредственного бытия
сводится таким образом к некоторому эмпирическому существованию,
{\em показать} которое оно не может отказаться, так как
оно ведь хочет держаться находящейся вне мышления непосредственности.

Точно так же обстоит дело и с {\em ничто}, только
противоположным образом, и это соображение (Reflexion) известно и довольно
часто применялось к нему. Ничто, взятое в его непосредственности,
оказывается {\em сущим}, ибо по своей природе оно то же
самое, что и бытие. Мы мыслим, представляем себе ничто, мы о нем говорим;
оно, стало быть, {\em есть}; ничто имеет свое бытие в
мышлении, представлении, речи и~т.~д. Но, далее, это бытие также и отлично
от него; поэтому, хотя и говорят, что ничто имеет бытие в мышлении,
представлении, но это означает, что не {\em оно есть},
не ему как таковому присуще бытие, а лишь мышление или представление есть
это бытие. Проводя это различение, нельзя также отрицать, что ничто
находится в {\em соотношении} с некоторым бытием; но в
этом соотношении, хотя оно содержит в себе также и различие, имеется
единство с бытием. Каким бы образом мы ни высказывали или ни обнаруживали
ничто, оно оказывается находящимся в связи или, если угодно, в
соприкосновении с некоторым бытием, оказывается неотделимым от некоторого
бытия, оказывается именно находящимся в некотором
{\em наличном бытии}.

Но, обнаруживая таким образом ничто в некотором наличном бытии, уму
обыкновенно все еще предносится то его отличие от бытия, что наличное бытие
ничто (das Dasein des Nichts) вовсе, дескать, не присуще самому ему, что
оно, само по себе взятое, не имеет в себе бытия, что оно не
{\em есть} бытие, как таковое; ничто есть, дескать,
лишь отсутствие бытия; тьма таким образом есть лишь
{\em отсутствие} света, холод ---~отсутствие тепла
и~т.~д. Тьма, дескать, имеет значение лишь в отношении к глазу, во внешнем
сравнении с положительным, со светом, и точно так же холод есть нечто лишь
в нашем ощущении; свет же, тепло, как и бытие, суть, напротив, сами по себе
нечто объективное, реальное, действенное, обладают совершенно другим
качеством и достоинством, чем указанные отрицательные, чем ничто. Часто
можно встретить под видом очень важного соображения и значительного
познания утверждение, что тьма есть {\em лишь
отсутствие} света, холод ---~{\em лишь отсутствие} тепла.
На это остроумное соображение можно в этой области эмпирических предметов
эмпирически возразить, что тьма во всяком случае являет себя в свете как
нечто действенное, так как она его определяет, делает его цветом и только
этим впервые сообщает ему самому видимость, ибо, как мы сказали раньше, в
чистом свете столь же мало видно, как и в чистой тьме. А видимость есть
такая действенность в глазу, в которой указанное отрицательное имеет такую
же долю, как и признаваемый реальным, положительным свет; и точно так же
холод дает себя достаточно знать воде, нашему ощущению и~т.~д., и если мы
ему отказываем в так называемой объективной реальности, то от этого в нем
ничего не убывает. Но, далее, достойно порицания, что здесь так же, как и
выше, говорят об отрицательном, обладающем определенным содержанием, идут
дальше самого ничто, сравнительно с которым бытие не характеризуется ни
большей, ни меньшей абстрактностью. --- Однако возьмем сразу же самих по себе
холод, тьму и тому подобные определенные отрицания и посмотрим, что этим
полагается в отношении того их всеобщего определения, с которым мы теперь
имеем дело. Они должны быть не ничто вообще, а ничто света, тепла и~т.~д.,
ничто чего-то определенного, какого-то содержания; таким образом они, если
можно так сказать, суть определенные, содержательные ничто. Но
определенность, как мы это еще увидим дальше, сама есть отрицание; таким
образом они суть отрицательные ничто; но отрицательное ничто есть нечто
утвердительное. Превращение ничто благодаря его определенности (которая нам
раньше представилась как некоторое {\em наличное бытие}
в субъекте или в чем бы то ни было другом) в некоторое утвердительное
представляется сознанию, прочно застревающему в рассудочной абстракции,
верхом парадоксальности; как ни просто усмотрение того, что отрицание
отрицания есть положительное, оно, несмотря на это, а, может быть, именно
вследствие самой этой его простоты, представляется чем-то тривиальным, с
которым гордому рассудку поэтому нет надобности считаться, хотя в этом
усмотрении и есть что-то правильное, --- а между тем оно не только правильно,
а благодаря всеобщности таких определений еще кроме того обладает
бесконечным протяжением и всеобщим применением, так что все же следовало бы
с ним считаться.

Относительно определения перехода друг в друга бытия и ничто можно еще
заметить, что мы должны его мыслить равным образом без всякого дальнейшего
определения рефлексии. Он непосредственно и всецело абстрактен вследствие
абстрактности переходящих моментов,~т.~е. вследствие того, что в этих
моментах еще не положена определенность другого, посредством чего они
переходили бы друг в друга, ничто еще не {\em положено}
в бытии, хотя бытие есть {\em по существу} ничто, и
наоборот. Поэтому недопустимо применять здесь дальнейшие определенные
опосредствования и мыслить бытие и ничто находящимися в каком-нибудь
отношении, --- этот переход еще не есть отношение. Недозволительно, стало
быть, говорить: ничто есть {\em основание} бытия или
бытие есть {\em основание} ничто; ничто есть
{\em причина} бытия и~т.~д.; или сказать: можно
переходить в ничто лишь {\em при том условии}, что
нечто {\em есть}, или: можно переходить в бытие лишь
{\em при том условии}, что есть небытие. Род
соотношения не может получить дальнейшего определения без того, чтобы
вместе с тем не были далее определены соотносящиеся
{\em стороны}. Связь основания и следствия и~т.~д.
имеет теми сторонами, которые она связывает, уже не голое бытие и голое
ничто, а явным образом такое бытие, которое есть основание, и нечто такое,
что, хотя оно и есть лишь некое положенное, несамостоятельное, все же не
есть абстрактное ничто.

\subsection*{Примечание 4. Непостижимость начала}

Из предшествующего ясно видно, как обстоит дело с направленной против
{\em начала мира}, а также против его гибели
диалектикой, которая должна доказать {\em вечность}
материи,~т.~е. с диалектикой, направленной вообще против
{\em становления}, против возникновения и прехождения.
--- Кантовскую антиномию конечности или бесконечности мира в пространстве и
времени мы ближе рассмотрим ниже, когда будем трактовать понятие
количественной бесконечности. --- Указанная простая, обычная диалектика
основана на удержании противоположности между бытием и ничто. Невозможность
начала мира или чего бы то ни было другого доказывается следующим образом.

Нет ничего такого, что могло бы иметь начало, ни поскольку нечто есть, ни
поскольку его нет; ибо, поскольку оно есть, оно не начинается теперь
впервые, а, поскольку его нет, оно также не начинается. Если бы мир или
нечто имели начало, то он имел бы начало в ничто, но в ничто нет начала
или, иначе говоря, ничто не есть начало, ибо начало заключает в себе некое
бытие, а ничто не содержит в себе никакого бытия. Ничто есть лишь ничто. А
в причине, основании и~т.~д. --- если ничто получает эти определения
---~содержится некое утверждение, бытие. По тому же основанию нечто не может
также и прекратиться. Ибо в таком случае бытие должно было бы содержать в
себе ничто, но бытие есть лишь бытие, а не противоположность самого себя.

Ясно, что здесь против становления или начала и прекращения, против этого
{\em единства} бытия и ничто не приводится никакого
доказательства, а его лишь ассерторически отрицают и приписывают истинность
бытию и ничто в их отдельности друг от друга. --- Однако эта диалектика по
крайней мере последовательнее рефлектирующего представления. Последнее
считает полной истиной, что бытие и ничто существуют лишь раздельно, а, с
другой стороны, признает начинание и прекращение столь же истинными
определениями; но, признавая последние, оно фактически принимает
нераздельность бытия и ничто.

При предположении абсолютной раздельности бытия и ничто начало или
становление есть, конечно, --- это приходится столь часто слышать ---~нечто
{\em непонятное}. Ибо в этом случае делают предпосылку,
упраздняющую начало или становление, затем все же
{\em снова} допускают становление, и это противоречие,
которое они сами же создают и разрешение которого они делают невозможным,
они называют {\em непостижимостью}.

Вышеизложенное представляет собою также и ту диалектику, которою пользуется
рассудок против даваемого высшим анализом понятия
{\em бесконечно-малых величин}. Об этом понятии мы
подробное будем трактовать ниже. --- Эти величины определяются как величины,
{\em существующие} {\em в своем
исчезновении}: не до своего исчезновения, ибо тогда они являются конечными
величинам, но и не {\em после} своего исчезновения, ибо
тогда они суть ничто. --- Против этого чистого понятия было выдвинуто
постоянно повторявшееся возражение, что такие величины суть
{\em либо} нечто, {\em либо} ничто,
и что нет {\em промежуточного состояния} (<<состояние>>
есть здесь неподходящее, варварское выражение) между бытием и небытием. ---
При этом опять-таки принимают абсолютную раздельность бытия и ничто. Но мы,
напротив того, показали, что бытие и ничто на самом деле суть одно и то же
или, говоря на языке выдвигающих это возражение, что не
{\em существует} ничего такого, что не было бы
{\em промежуточным состоянием между бытием и ничто}.
Математика обязана своими самыми блестящими успехами принятию ею того
определения, которого не допускает рассудок.

Приведенное рассуждение, берущее ложную предпосылку об абсолютной
раздельности бытия и небытия и не идущее дальше этой предпосылки, должно
быть названо не {\em диалектикой}, а
{\em софистикой}. Ибо софистика есть рассуждение,
исходящее из необоснованной предпосылки, истинность которой признается без
критики и необдуманно. Диалектикой же мы называем высшее разумное движение,
в котором такие кажущиеся безоговорочно раздельными определения сами через
самих себя, через то, что они суть, переходят друг в друга, и предпосылка
[о их раздельности] снимается. Диалектическая, имманентная природа самих
бытия и ничто состоит в том, что они свое единство ---~становление
---~обнаруживают как свою истину.

\subsection*{2. Моменты становления: возникновение и прехождение}
[ссылка!]
Становление есть нераздельность бытия и~ничто, "---
не единство, абстрагирующееся от бытия и~ничто, а~как единство
{\em бытия} и~{\em ничто} оно есть
это {\em определенное} единство, или, иначе говоря,
такое единство, в котором {\em есть} как бытие, так и~ничто.
Но так~как каждое из~них, и~бытие, и~ничто, нераздельно от своего
другого, то {\em их нет}. Они, следовательно,
{\em суть} в~этом~единстве, но~как исчезающие, лишь как
{\em снятые}. Они понижаются в~своем ранге, теряют ту
свою {\em самостоятельность}, которая, как
первоначально представлялось, была им присуща, и~превращаются
в~{\em моменты}, {\em еще~различенные}, но вместе~с~тем снятые.

Взятые со стороны этой своей различенности, каждый из них есть
{\em в~этой~же различенности} единство с~{\em другим}.
Становление содержит в~себе, следовательно, бытие
и~ничто как два таких единства, каждое из~которых само
в~свою очередь есть единство бытия и~ничто. Одно из~них есть бытие,
как~непосредственное и~как~соотношение с~ничто; другое есть ничто,
как~непосредственное и~как~соотношение с~бытием. Определения обладают
в~этих единствах неодинаковою ценностью.

Становление, таким образом, дано в~двояком определении; в~одном определении
ничто есть непосредственное,~т.~е. оно~(определение) начинает с~ничто,
соотносящегося с~бытием,~т.~е. переходящего в~это~последнее; в~другом
"--- бытие дано как~непосредственное,~т.~е. оно (определение) начинает
с~бытия, которое переходит в~ничто, "--- {\em возникновение}
и~{\em прехождение}.

Оба суть одно~и~то~же, становление, и~даже как~эти столь различные
направления они взаимно проникают и~парализуют друг друга. Одно есть
прехождение; бытие переходит в~ничто; но~ничто есть~также
и~противоположность самого себя, переход в~бытие, возникновение. Это
возникновение есть другое направление; ничто переходит в бытие, но бытие
также и упраздняет само себя и есть, наоборот, переход в ничто, есть
прехождение. Они не упраздняют друг друга, одно не упраздняет другое извне,
а~каждое из~них упраздняет себя в~самом себе и~есть в~самом~себе
своя противоположность.

\subsection*{3. Снятие становления}
[ссылка!]
Равновесие, в~которое приводят себя
возникновение и~прехождение, есть ближайшим образом само становление.
Но~последнее также и~оседает, переходит в~{\em спокойное
единство}. Бытие и~ничто находятся в~становлении лишь как исчезающие;
становление~же как таковое имеется лишь благодаря их разности. Их исчезание
есть поэтому исчезание становления или, иначе~говоря, исчезание самого
исчезания. Становление есть неустойчивое беспокойство, которое оседает,
переходит в~некоторый спокойный результат.

Можно было~бы это выразить также~и~так: становление есть исчезание бытия
в~ничто и~ничто в~бытие, и~исчезание бытия~и~ничто вообще; но~оно
вместе~с~тем покоится на~различии последних. Оно, следовательно,
противоречит {\em себе} в~самом {\em себе}, так~как
оно соединяет в~себе нечто~такое, что противоположно самому себе;
но~такое соединение разрушает себя.

Этот результат есть происшедшее исчезновение (das Verschwundensein),
но~не~как {\em ничто}; в~последнем случае он~был~бы лишь
рецидивом, впадением в~одно~из уже снятых определений, а~не результатом
ничто {\em и бытия}. Этот результат есть ставшее
спокойной простотой единство бытия и~ничто. Но~спокойная простота
есть~{\em бытие}, однако вместе~с~тем такое~бытие, которое
уже больше не~стоит особо, а~есть~бытие как~определение~целого.

Становление, как переход в~такое единство бытия и~ничто, которое есть
как~{\em сущее} или, иначе~говоря, имеет вид одностороннего
{\em непосредственного} единства этих моментов, есть
{\em наличное бытие}.

\subsection*{Примечание. Выражение: <<снятие>>}

{\em Снятие} и {\em снятое}
({\em идеализованное}) есть одно из важнейших понятий
философии, основное определение, которое возвращается решительно повсюду,
определение, смысл которого мы должны точно понять и, в~особенности,
должны различать между снятым и~ничто. ---~То, что снимает себя, еще
не~превращается вследствие этого в ничто. Ничто есть
{\em непосредственное}; снятое же, напротив, есть некое
{\em опосредствованное}: это "--- не-сущее, но как
{\em результат}, имевший своим исходным пунктом
некоторое бытие; оно, поэтому, еще имеет {\em в~себе ту
определенность, из которой оно произошло}.

Aufheben (снятие) имеет в языке двоякий смысл: оно означает сберечь,
{\em сохранить} и вместе с тем прекратить,
{\em положить конец}. Само сбережение уже заключает в~себе
тот отрицательный смысл, что нечто изымается из своей
непосредственности и, значит, из открытой внешним воздействиям [сферы]
наличного бытия для того, чтобы сохранить его. ---~Таким образом снятое
есть некое вместе~с~тем и~сбереженное, которое лишь потеряло свою
непосредственность, но~отнюдь не~уничтожено вследствие этого.
---~Указанные два определения {\em снятия} могут быть приведены
лексически как два {\em значения} этого слова, но~удивительным
при этом должно представляться то~обстоятельство, что имеется
язык, пришедший к~тому, чтобы употреблять одно~и то~же слово для
обозначения противоположных определений. Для спекулятивного мышления
отрадно находить в~языке слова, имеющие в~самих~себе спекулятивное
значение; немецкий язык имеет много слов такого рода. Двоякий смысл
латинского слова tollere (ставший знаменитым благодаря остроте Цицерона:
tollendum esse Octavium)
[ссылка!]
не~идет так~далеко: утвердительное определение доходит лишь до поднятия,
возвышения.

Нечто снято лишь постольку, поскольку оно вступило в~единство со~своей
противоположностью; взятое в~этом более точном определении, как некоторое
рефлектированное, оно может быть подходяще названо
{\em моментом}. {\em Вес} и~{\em расстояние} от известной точки называются
в~рычаге его механическими {\em моментами} из-за
{\em тождественности} оказываемого ими действие при
всем прочем различии между такой реальной вещью, какой является вес,
и~такой идеализованной, как голое пространственное определение, линия
(см. <<Энциклопедию философских наук>>, изд.~3-е, \S 261, примечание).
---~Часто еще нам невольно будет приходить на ум мысль, что философский,
технический язык употребляет для обозначения рефлектированных
определений латинские выражения
[ссылка!]
либо потому, что в~родном языке нет выражений для их~обозначения, либо~же,
если~он, как~в~данном случае, обладает такими выражениями, потому, что
выражение, которым располагает родной язык, больше напоминает
о~непосредственном, а~выражения, заимствованные из~чужого языка,
больше о~рефлектированном.

Более точные смысл и~выражение, которые бытие и~ничто получают, поскольку
они стали теперь {\em моментами}, должны выступить при
рассмотрении наличного бытия, как того единства, в~котором они сохранены.
Бытие есть~бытие и~ничто есть~ничто лишь в~их~различии друг от друга; но
в~их истине, в~их единстве, они исчезли как эти определения, и~суть теперь
нечто другое. Бытие~и~ничто суть одно~и~то же;
{\em именно потому, что~они "--- одно~и~то же, они уже не~суть
бытие и~ничто}, и~имеют другое определение: в~становлении они были
возникновением и~прехождением; в~наличном бытии как некоем иначе
определенном единстве, они суть опять-таки иначе определенные моменты. Это
единство остается отныне их основой, из которой они уже больше не~будут
выходить, чтобы возвращаться к~абстрактному значению бытия~и~ничто.


\chapter*{Вторая глава. Наличное бытие.}
Наличное бытие есть {\em определенное} бытие; его
определенность есть {\em сущая} определенность,
{\em качество}. Через свое качество
{\em нечто} есть наряду с некоторым
{\em другим}, оно {\em изменчиво}
(veränderlich) [точнее "--- {\em способно стать другим}. —
{\em Перев}.] и {\em конечно},
определено безоговорочно отрицательно не только в отношении некоторого
другого, но и в нем же. Это его отрицание по отношению прежде всего к
конечному нечто есть {\em бесконечное}; абстрактная
противоположность, в которой выступают эти определения, разрешается в не
имеющую противоположности бесконечность, в
{\em для-себя-бытие}.

Таким образом рассмотрение наличного бытия распадается на следующие три
раздела:

A) {\em Наличное бытие как таковое},

B) {\em Нечто и другое, конечность},

С) {\em Качественная бесконечность}.

\section*{А. Наличное бытие как таковое}
В наличном бытии

a) {\em как таковом} следует прежде всего различать его
определенность

b) как {\em качество}. Последнее же следует брать и в
одном, и в другом определении наличного бытия, как
{\em реальность} и как
{\em отрицание}. Но в этих определённостях наличное
бытие также и рефлектировано в себя, и положенное как таковое оно есть

c) {\em нечто}, налично сущее.

\subsection*{a) Наличное бытие вообще}
Из становления происходит наличное бытие. Наличное бытие есть простая
единость бытия и ничто. Из-за этой простоты оно имеет форму некоего
{\em непосредственного}. Его опосредствование,
становление, лежит позади него; оно сняло себя, и наличное бытие
представляется поэтому некиим первым, из которого исходят. Оно выступает
прежде всего в одностороннем определении {\em бытия};
другое содержащееся в нем определение, {\em ничто},
равным образом проявится в нем, проявится в противоположность первому.

Оно есть не голое бытие, а {\em наличное бытие}; взятое
этимологически, Dasein означает бытие в известном
{\em месте}; но представление о пространстве здесь не
приложимо. Наличное бытие есть вообще по своему становлению
{\em бытие} с некоторым
{\em небытием}, так что это небытие принято в простое
единство с бытием. {\em Небытие}, принятое в бытие,
таким образом, что конкретное целое имеет форму бытия, непосредственности,
составляет {\em определенность} как таковую.

Это {\em целое} имеет равным образом форму~т.~е.
{\em определенность} бытия, ибо бытие равным образом
явило себя в становлении имеющим характер всего лишь момента,
представляющим собой некое снятое,
отрицательно-определенное
[ссылка!]; но таково оно
{\em для нас, в нашей рефлексии}; оно еще не
{\em положено} в себе самом. Определенность же
наличного бытия как таковая есть положенная определенность, на что
указывает также и выражение «{\em наличное бытие}». —
Следует всегда строго различать между тем, что есть для нас, и тем, что
положено; лишь то, что {\em положено} в известном
понятии, входит в развертывающее рассмотрение его, в состав его содержания.
Определенность же, еще не положенная в нем самом "--- все равно, касается ли
она природы самого понятия или она есть внешнее сравнение, — принадлежит
нашей рефлексии; обращение внимания читателя на определенность последнего
рода может лишь служить к уяснению того пути, который представится нам в
самом ходе развития понятия, или же являться предварительным намеком на
этот путь. Что целое, единство бытия и ничто, имеет
{\em одностороннюю определенность} бытия, — это
является внешней рефлексией. В отрицании же, в нечто и
{\em другом} и~т.~д. эта односторонняя определенность
дойдет до того, чтобы выступить как {\em положенная}. —
Мы должны были здесь обратить внимание на это различие; но давать отчет обо
всем, что рефлексия может позволить себе заметить, не следует, ибо это
привело бы к пространности изложения, к предвосхищению того, что должно
получиться в самом предмете. Если такого рода рефлексии могут служить к
облегчению обозревания и тем самым и понимания, то они, однако, также
влекут за собой ту невыгоду, что они выглядят неоправданными утверждениями,
основаниями и основами последующего. Не следует поэтому придавать им
большее значение, чем то, которое они должны иметь, и надлежит отличать их
от того, что составляет момент в дальнейшем ходе развития самого предмета.

Наличное бытие соответствует {\em бытию} предшествующей
сферы; однако, бытие есть неопределенное, в нем вследствие этого не
получается никаких определений. Наличное же бытие есть некоторое
определенное бытие, некоторое {\em конкретное}; поэтому
в нем сразу же открываются несколько определений, различные отношения его
моментов.

\subsection*{b) Качество}
Ввиду непосредственности, в которой бытие и ничто
{\em едины} в наличном бытии, они не выходят за пределы
друг друга; сколь далеко наличное бытие есть сущее, столь же далеко оно
есть небытие, определено. Бытие не есть {\em всеобщее},
определенность не есть {\em особенное}. Определенность
еще {\em не отделилась от бытия}; она, правда, уже
больше и не будет отделяться от него, ибо лежащее отныне в основании
истинное есть единство небытия с бытием; на нем как на основании получаются
все дальнейшие определения. Но здесь то соотношение, в котором
определенность находится с бытием, есть непосредственное единство обоих,
так что еще не положено никакого различения между ними.

Определенность, так самодовлеюще (für sich) изолированная, как
{\em сущая} определенность, есть
{\em качество} "--- некое совершенно простое,
непосредственное. {\em Определенность} вообще есть
более всеобщее, которое в одинаковой мере может быть также и
количественным, равно как и определенным еще далее. Ввиду этой простоты
нечего более сказать о качестве как таковом.

Но наличное бытие, в котором содержатся как ничто, так и бытие, само
является масштабом для односторонности качества как лишь
{\em непосредственной} или
{\em сущей} определенности. Качество должно быть
положено также и в определении ничто, благодаря чему непосредственная или
{\em сущая} определенность полагается как некая
различенная, рефлектированная определенность, и таким образом, ничто как
определенность некоторой определенности есть также некое рефлектированное,
некое {\em отрицание}. Качество, взятое с той стороны,
что оно, будучи различенным, признается {\em сущим},
есть {\em реальность}; оно же, обремененное некоторым
отрицанием, есть {\em отрицание} вообще; это "--- также
некоторое качество, но такое, которое признается недостатком и определится
в дальнейшем как граница, предел.

Оба суть наличное бытие; но в {\em реальности} как
качестве с ударением на то, что оно есть {\em сущее},
запрятано то обстоятельство, что оно содержит в себе определенность и,
следовательно, также и отрицание; реальность признается поэтому чем-то
только положительным, из которого исключены отрицание, ограниченность,
недостаток. Отрицание, взятое как голый недостаток, было бы то же, что
ничто; но оно есть некоторое наличное бытие, некоторое качество, только с
ударением на небытие.

\subsection*{Примечание 1. Реальность и отрицание}

Реальность может показаться словом, имеющим разнообразные значения, так как
оно употребляется для выражения разных и даже противоположных определений.
В философском смысле говорят, например, об
{\em исключительно эмпирической} реальности, как о
лишенном ценности существовании (Dasein). Но когда говорят о мыслях,
понятиях, теориях, что они {\em лишены реальности}, то
это означает, что они не обладают
{\em действительностью}, хотя
{\em в себе} или в понятии идея, например, платоновской
республики может, дескать, быть истинной. Здесь не отрицается за идеей ее
ценность, и оставляют стоять {\em наряду} с реальностью
также и ее. Но сравнительно с так называемыми
{\em голыми} идеями, с {\em голыми}
понятиями, реальное признается единственно истинным. — Смысл, в котором
внешнему существованию приписывается решение вопроса об истинности
некоторого содержания, столь же односторонен, как односторонни те, которые
представляют себе, что для идеи, сущности или даже внутреннего чувства
безразлично внешнее существование (Dasein), и которые даже считают, что они
тем превосходнее, чем более они отдалены от реальности.

По поводу выражения «реальность» мы должны коснуться прежнего
метафизического {\em понятия бога}, которое
преимущественно клали в основание так называемого онтологического
доказательства бытия божия. Бог определялся как
{\em совокупность всех реальностей}, и об этой
совокупности говорилось, что она не заключает в себе противоречия, что ни
одна из реальностей не упраздняет другую; ибо реальность следует понимать
лишь как некоторое совершенство, как некое
{\em утвердительное}, не содержащее в себе никакого
отрицания. Реальности, стало быть, не противоположны и не противоречат друг
другу.

Выставляющие это понятие реальности предполагают, что она остается еще и
тогда, когда мы отмыслим всякое отрицание; но, отмыслив отрицание, мы тем
самым упраздняем всякую определенность реальности. Реальность есть
качество, наличное бытие; тем самым она содержит в себе момент
отрицательного, и лишь благодаря этому она есть то определенное, которое
она есть. В так называемом
{\em эминентном}[ссылка!]
{\em смысле} или как
{\em бесконечная} "--- в обычном значении этого слова
—~т.~е. в том смысле, в котором ее якобы следует понимать, она расширяется
до неопределенности и теряет свое значение. Божественная благость,
утверждали, не есть благость в обычном смысле, а в эминентном; она не
отлична от правосудия, а {\em умеряется}
({\em лейбницевское} примиряющее выражение), ею, равно
как и, наоборот, правосудие умеряется благостью; таким образом, благость
уже перестает быть благостью и правосудие "--- правосудием. Могущество бога,
говорят, умеряется его мудростью, но, таким образом, оно уже не могущество
как таковое, ибо оно подчинено мудрости; мудрость бога, утверждают,
расширяется до могущества, но, таким образом, она исчезает как определяющая
цель и меру мудрость. Истинное понятие бесконечного и его
{\em абсолютное} единство "--- то понятие, к которому мы
придем позднее, — нельзя понимать как {\em умерение,
взаимное ограничение} или {\em смешение}; это
—~поверхностное, остающееся неопределенно туманным соотношение, которым
может удовлетворяться лишь чуждое понятию представление. Реальность, как ее
берут в вышеуказанной дефиниции бога,~т.~е. реальность как определенное
качество, выведенное за пределы его определенности, перестает быть
реальностью; оно превращается в абстрактное бытие; бог как
{\em чисто} реальное во всем реальном или как
{\em совокупность} всех реальностей есть то же самое
лишенное определения и содержания, что и пустое абсолютное, в котором все
есть одно.

Если же, напротив, брать реальность в ее определенности, то ввиду того, что
она по существу содержит в себе момент отрицательного, совокупность всех
реальностей оказывается также совокупностью всех отрицаний, совокупностью
всех противоречий; она, скажем примерно, превращается в абсолютное
{\em могущество}, в котором все определенное
поглощается; но так как само оно существует лишь постольку, поскольку оно
имеет рядом с собою нечто еще не упраздненное им, то, когда его мыслят
расширенным до осуществленного, беспредельного могущества, оно превращается
в абстрактное ничто. То реальное во всяком реальном,
{\em бытие} во всяком {\em наличном
бытии}, которое якобы выражает понятие бога, есть не что иное, как
абстрактное бытие, есть то же самое, что и ничто.

Определенность есть отрицание, положенное как утвердительное, — это и есть
положение Спинозы: Omnis determinatio est negatio (всякое определение есть
отрицание)[ссылка!]. Это положение имеет
бесконечную важность; только следует сказать, что отрицание как таковое
есть бесформенная абстракция. Но не следует обвинять спекулятивную
философию в том, что для нее отрицание или ничто есть последнее слово; оно
является для нее столь же мало последним словом, как и реальность
—~последней истиной.

Необходимым выводом из положения, гласящего, что определенность есть
отрицание, является {\em единство спинозовской
субстанции} или существование лишь одной субстанции.
{\em Мышление} и {\em бытие} или
протяжение, эти два определения, которые Спиноза именно имеет перед собою,
он должен был слить воедино (in Eins setzen) в этом единстве, ибо как
определенные реальности они суть отрицания, бесконечность которых есть их
единство; согласно дефиниции, даваемой Спинозой, о чем будет сказано далее,
бесконечность чего-либо есть его утверждение. Он поэтому их понимал как
атрибуты,~т.~е. как такие, которые не обладают особым существованием,
бытием в себе и для себя, а имеют бытие лишь как снятые, как моменты; или,
правильнее сказать, они для него даже и не моменты, ибо субстанция
совершенно лишена определений в самой себе, а атрибуты, равно как и модусы,
суть различения, делаемые внешним рассудком. — Это положение точно так же
не допускает субстанциальности индивидуумов. Индивидуум есть соотношение с
собою благодаря тому, что он ставит границы всему другому; но эти границы
суть тем самым также и границы его самого, суть соотношения с другим; он не
имеет своего наличного бытия в самом себе. Индивидуум, правда, есть нечто
{\em большее}, чем только всесторонне ограниченное, но
это «большое» принадлежит другой сфере понятия; в метафизике бытия он есть
некое всецело определенное; и против того, чтобы нечто подобное, чтобы
конечное как таковое существовало в себе и для себя, выступает, предъявляя
свои права, определенность именно как отрицание и увлекает его в то же
отрицательное движение рассудка, которое заставляет все исчезать в
абстрактном единстве, в субстанции.

Отрицание непосредственно противостоит реальности; в дальнейшем, в сфере
собственно рефлектированных определений, оно противопоставляется
{\em положительному}, которое есть рефлектирующая на
отрицание реальность, — реальность, в которой как бы
{\em светится} то отрицательное, которое в реальности
как таковой еще запрятано.

Качество есть преимущественно лишь с той стороны
{\em свойство}, с какой оно в некотором
{\em внешнем соотношении} показывает себя
{\em имманентным определением}. Под свойствами,
например, трав понимают определения, которые не только вообще
{\em свойственны} некоторому нечто, а свойственны ему
как раз постольку, поскольку это нечто через них своеобразно
{\em сохраняет} себя в отношении к другим нечто, не
дает внутри себя воли чужим положенным в нем воздействиям, а само
{\em показывает} в другом
{\em силу} своих собственных определений, хотя оно и не
отстраняет от себя этого другого. Напротив, таких более покоящихся
определенностей, как, например, фигура, внешний вид, не называют
свойствами, впрочем, и не качествами, поскольку их представляют себе
изменчивыми, не тождественными с {\em бытием}.

{\em Qualierung} (качествование) или
{\em Inqualierung} (вкачествование) "--- специфическое
выражение философии Якова {\em Беме}, философии,
проникающей вглубь, но в смутную глубь, — означает движение некоторого
качества (кислого, терпкого, огненного качества и~т.~д.) в самом себе,
поскольку оно в своей отрицательной природе (в своей
{\em Qual}[ссылка!],
м\`{у}ке) выделяется из другого и укрепляется, поскольку оно
вообще есть свое собственное беспокойство в самом себе, в соответствии с
которым оно порождает и сохраняет себя лишь в борьбе.

\subsection*{c) Нечто}
В наличном бытии мы различили его определенность, как качество; в последнем,
как налично сущем, {\em есть} различие, — различие
реальности и отрицания. Насколько эти различия имеются в наличном бытии,
настолько же они вместе с тем ничтожны и сняты. Реальность сама содержит в
себе отрицание, есть наличное, а не неопределенное, абстрактное бытие. И
точно так же отрицание есть наличное бытие; оно "--- не то ничто, которое
должно было оставаться абстрактным, а оно здесь положено, как оно есть в
себе, как сущее, принадлежащее к наличному бытию. Таким образом, качество
вообще не отделено от наличного бытия, которое есть лишь определенное,
качественное бытие.

Это снятие различения есть больше, чем голый отказ от него и внешнее новое
отбрасывание его или простой возврат к простому началу, к наличному бытию
как таковому. Различие не может быть отброшено, ибо оно
{\em есть}. Следовательно, фактически имеющимся
оказывается наличное бытие вообще, различие в нем, и снятие этого различия;
не наличное бытие, лишенное различий, как вначале, а наличное бытие как
{\em снова} равное самому себе
{\em благодаря снятию различия}, как простота наличного
бытия, {\em опосредствованная} этим снятием. Эта
снятость различия есть собственная определенность наличного бытия. Таким
образом, оно есть {\em внутри-себя-бытие}; наличное
бытие есть {\em налично сущее, нечто}.

Нечто есть {\em первое отрицание отрицания} как простое
сущее соотношение с собою. Наличное бытие, жизнь, мышление и~т.~д.
существенно определяются в {\em налично сущее},
{\em живое}, {\em мыслящее} (в «я»)
и~т.~д. Это определение имеет величайшую важность: благодаря ему не
останавливаются, как на всеобщностях, на наличном бытии, жизни, мышлении
и~т.~д.; не останавливаются также и на божестве (вместо бога).
Представление справедливо считает {\em нечто} некоторым
{\em реальным}. Однако {\em нечто}
есть еще очень поверхностное определение, подобно тому, как
{\em реальность} и {\em отрицание},
наличное бытие и его определенность, хотя уже более не суть пустые бытие и
ничто, все же суть совершенно абстрактные определения. Вследствие этого они
и являются самыми ходячими выражениями, и философски необразованная
рефлексия чаще всего пользуется ими, вливает в них свои различения и мнит,
что в них она обладает чем-то вполне хорошо и твердо определенным. —
Отрицание отрицания есть как {\em нечто} лишь начало
субъекта, — внутри-себя-бытие, пока что лишь совершенно неопределенное. Оно
определяет себя в дальнейшем прежде всего как сущее для себя, затем
продолжает определять себя и далее до тех пор, пока оно не получит впервые
в понятии конкретную напряженность субъекта. В основании всех этих
определений лежит отрицательное единство с собою. Но при этом следует
различать между отрицанием как {\em первым}, как
отрицанием {\em вообще}, и вторым, отрицанием
отрицания, которое есть конкретная, {\em абсолютная}
отрицательность, точно так же, как первое отрицание есть, напротив, лишь
{\em абстрактная} отрицательность.

{\em Нечто} есть {\em сущее} как
отрицание отрицания; ибо последнее есть восстановление простого соотношения
с собою; но тем самым нечто есть точно так же и
{\em опосредствование себя с самим собою}. Уже в
простоте [категории] нечто, а затем еще определеннее в для-себя-бытии,
субъекте и~т.~д. имеется опосредствование себя с самим собою; оно имеется
уже и в становлении, но в нем оно есть лишь совершенно абстрактное
опосредствование. В нечто опосредствование с {\em собою
положено}, поскольку нечто определено как простое
{\em тождественное}. — Можно обратить внимание читателя
на присутствие опосредствования вообще в противовес утверждению о якобы
голой непосредственности знания, в которой опосредствования якобы
совершенно нет; но в дальнейшем нет нужды обращать особенное внимание
читателя на момент опосредствования, ибо он находится везде и повсюду, в
каждом понятии.

Это опосредствование с собою, которым нечто является
{\em в себе}, взятое лишь как отрицание отрицания, не
имеет своими сторонами каких-либо конкретных определений; таким образом,
оно сжимается в простое единство, которое есть
{\em бытие}. Нечто {\em есть}, и
оно ведь {\em есть} также и налично сущее; оно, далее,
есть {\em в себе} также и
{\em становление}, которое, однако, уже более не имеет
своими моментами только бытие и ничто. Один из них "--- бытие "--- есть теперь
наличное бытие и, далее, налично сущее; второй есть также некое
{\em налично сущее}, но определенное как отрицание
нечто (\foreignlanguage{english}{Negatives} \foreignlanguage{english}{des}
\foreignlanguage{english}{Etwas}), — как {\em другое}.
Нечто как становление есть переход, моменты которого сами суть нечто и
который поэтому есть {\em изменение}, — есть ставшее
уже {\em конкретным} становление. — Но нечто изменяется
сначала лишь в своем понятии; оно, таким образом, еще не
{\em положено} как опосредствующее и опосредствованное;
пока что оно положено, как просто сохраняющее себя в своем соотношении с
собою, а отрицание его "--- как некоторое также качественное, как только
некоторое {\em другое} вообще.

\section*{B. Конечность}
a) Нечто {\em и} другое; они ближайшим образом
равнодушны друг к другу; другое есть тоже некоторое непосредственно налично
сущее, некоторое нечто; отрицание, таким образом, имеет место вне их обоих.
Нечто есть {\em в себе} в противоположность к своему
{\em бытию-для-другого}. Но определенность принадлежит
также и к его «{\em в себе}» и есть

b) его {\em определение}, переходящее также в
{\em характер} (Beschaffenheit), который, будучи
тождественным с первым, составляет имманентное и вместе с тем подвергшееся
отрицанию бытие-для-другого, составляет {\em границу}
нечто, которая

c) есть имманентное определение самого нечто, и последнее есть,
следовательно, {\em конечное}.

В первом отделе, в котором мы рассматривали
{\em наличное бытие} вообще, последнее как взятое в
начальной стадии рассмотрения, имело определение
{\em сущего}. Моменты его развития, качество и нечто,
суть поэтому также утвердительные определения. Напротив, в этом отделе
развивается заключающееся в наличном бытии отрицательное определение,
которое там еще было только отрицанием вообще,
{\em первым} отрицанием, а теперь определилось далее до
{\em внутри-себя-бытия} нечто, до отрицания отрицания.

\subsection*{a) Нечто и иное}
1. Нечто и иное суть, {\em во-первых}, оба налично
сущие или {\em нечто}.

{\em Во-вторых}, каждое из них есть также некоторое
{\em другое}. Безразлично, которое из них мы называем
сначала и лишь потому именуем {\em нечто} (по-латыни,
когда они встречаются в предложении вместе, оба называются aliud, или «один
другого» "--- alius alium, а когда идет речь об отношении взаимности,
аналогичным выражением служит alter alterum). Если мы некоторое наличное
бытие называем А, а другое В, то В определено ближайшим образом как другое.
Но А есть также и столь же другое этого В. Оба суть одинаковым образом
{\em другие}. Для фиксирования различия и того нечто,
которое следует брать как утвердительное, служит слово
«{\em это}». Но «{\em это}» именно
и выражает, что это различение и выделение одного нечто есть субъективное
обозначение, имеющее место вне самого нечто. В этом внешнем показывании и
заключается вся определенность; даже выражение
«{\em это}» не содержит в себе никакого различия;
каждое нечто есть столь же «{\em это}», сколь и другое.
Мы {\em мним}, что словом
«{\em это}» мы выражаем нечто совершенно определенное;
но мы при этом упускаем из виду, что язык как произведение рассудка
выражает лишь всеобщее; исключение составляет только
{\em имя} некоторого единичного предмета, но
индивидуальное имя есть нечто бессмысленное в том смысле, что оно не есть
выражение всеобщего, и по этой же причине оно представляется чем-то лишь
положенным, произвольным, как и на самом деле собственные имена могут быть
произвольно приняты, даны или также заменены другими.

Таким образом, {\em инобытие} представляется
определением, чуждым определенному таким образом наличному бытию, или,
иначе говоря, другое выступает {\em вне} данного
наличного бытия; это представляют себе так, что отчасти некоторое наличное
бытие определяется нами как другое только через
{\em сравнение}, производимое некоторым третьим,
отчасти же это наличное бытие определяется нами как другое только из-за
другого, находящегося вне его, но само по себе
{\em оно} не таково. Вместе с тем, как мы уже заметили,
каждое наличное бытие определяется также и для представления в равной мере
и как некоторое другое наличное бытие, так что не остается ни одного
наличного бытия, которое было бы определено лишь как наличное бытие, не
было бы вне некоторого наличного бытия, и, следовательно, само не было бы
некоторым другим.

Оба определены и как {\em нечто} и как
{\em другое}, суть, значит,
{\em одно и то же}, и между ними еще нет никакого
различия. Но эта {\em однозначность} (Dieselbigkeit)
определений также имеет место только во внешней рефлексии, в
{\em сравнении} их друг с другом; но в том виде, в
каком пока что положено {\em другое}, оно само по себе,
правда, находится в соотношении с нечто, однако вместе с тем оно есть также
и {\em само по себе вне последнего}.

{\em В-третьих}, следует поэтому брать
{\em другое} как изолированное, в соотношении с самим
собою, брать {\em абстрактно} как другое ($\tau o$
$'{\varepsilon} \tau \varepsilon \rho o\nu $) Платона, который
противопоставляет его {\em единому} как один из
моментов тотальности и, таким образом, приписывает
{\em другому} собственно ему принадлежащую
{\em природу}. Таким образом,
{\em другое}, понимаемое единственно как таковое, есть
не другое некоторого нечто, а другое в нем самом,~т.~е. другое самого себя.
— {\em Физическая природа} есть такое по своему
определению другое; она есть {\em другое духа}. Это ее
определение есть, таким образом, пока что голая относительность, которой
выражается не качество самой природы, а лишь внешнее ей соотношение. Но так
как дух есть истинное нечто, а природа поэтому есть в себе же самой лишь
то, что она есть в отношении к духу, то ее качество постольку, поскольку
она берется сама по себе, именно и состоит в том, что она есть в самой себе
другое, {\em вне себя сущее} (в определениях
пространства, времени, материи).

Другое само по себе есть другое в самом себе и, следовательно, другое самого
себя есть, таким образом, другое другого, — стало быть, всецело неравное
внутри себя, отрицающее себя, {\em изменяющееся}. Но
оно вместе с тем также и остается тождественным с собою, ибо то, во что оно
изменилось, есть {\em другое}, которое помимо этого не
имеет никаких других дальнейших определений. А то, что изменяется,
определено быть другим не каким-нибудь иным образом, а тем же самым; оно
поэтому {\em сливается} в том другом, в которое оно
переходит, лишь {\em с самим собою}. Таким образом, оно
положено как рефлектированное в себя со снятием инобытия; оно есть
{\em тождественное} с собою нечто, по отношению к
которому, следовательно, инобытие, составляющее вместе с тем его момент,
есть некое отличное от него, не принадлежащее ему самому как такому нечто.

2. Нечто {\em сохраняется} в своем неимении наличного
бытия (Nichtdasein), оно существенно {\em едино} с ним
и существенно {\em не едино} с ним. Оно, следовательно,
находится в {\em соотношении} со своим инобытием; оно
не есть просто свое инобытие. Инобытие в одно и то же время и содержится в
нем и еще {\em отделено} от него. Оно есть
{\em бытие для другого}.

Наличное бытие как таковое есть непосредственное, безотносительное; или,
иначе говоря, оно есть в определении {\em бытия}. Но
наличное бытие, как включающее в себя небытие, есть
{\em определенное}, подвергшееся внутри себя отрицанию
бытие, а затем, ближайшим образом "--- другое; но так как оно вместе с тем
также и сохраняется в своей подвергнутости отрицанию, то оно есть лишь
{\em бытие для другого}.

Оно сохраняется в своем неимении наличного бытия и есть бытие; но не бытие
вообще, а как соотношение с собою {\em в
противоположность} своему соотношению с другим, как равенство с собою в
противоположность своему неравенству. Таковое бытие есть
{\em в-себе-бытие}.

Бытие для другого и в-себе-бытие составляют {\em два
момента} нечто. Здесь мы имеем перед собою {\em две
пары} определений: 1) {\em нечто} и
{\em другое}; 2) {\em бытие для
другого} и {\em в-себе-бытие}. В первых имеется
безотносительность их определенности: нечто и другое не связаны друг с
другом. Но их истиной служит соотношение между ними; бытие-для-другого и
в-себе-бытие суть поэтому указанные определения, положенные как
{\em моменты} одного и того же, как определения,
которые суть соотношения и остаются в своем единстве, в единстве наличного
бытия. Каждое из них, следовательно, само содержит в себе вместе с тем
также и свой разнствующий от него момент.

Бытие и ничто в том их единстве, которое есть наличное бытие, уже более не
суть бытие и ничто. Таковы они только вне своего единства. В их беспокойном
единстве, в становлении, они суть возникновение и прехождение. — Бытие в
нечто есть {\em в-себе-бытие}. Бытие, соотношение с
собою, равенство с собою, теперь уже более не непосредственно, а есть
соотношение с собою лишь как небытие инобытия (как рефлектированное в себя
наличное бытие). И точно так же небытие, как момент нечто в этом единстве
бытия и небытия, есть не неимение наличного бытия вообще, а другое, и,
говоря определеннее, по {\em различению} от него бытия
оно есть вместе с тем {\em соотношение} со своим
неимением наличного бытия, бытие-для-другого.

Тем самым {\em в-себе-бытие} есть, во-первых,
отрицательное соотношение с неимением наличного бытия, оно имеет инобытие
вне себя и противоположно ему; поскольку нечто есть
{\em в себе}, оно изъято из инобытия и бытия для
другого. Но, во-вторых, оно имеет инобытие также и в самом себе, ибо оно
само {\em есть небытие} бытия-для-другого.

Но {\em бытие-для-другого} есть, во-первых, отрицание
простого соотношения бытия с собою, соотношения, которым ближайшим образом
должно быть наличное бытие и нечто; поскольку нечто есть в некотором другом
или для некоторого другого, оно лишено собственного бытия. Но, во-вторых,
оно не есть неимение наличного бытия (Nichtdasein), как чистое ничто. Оно
есть неимение наличного бытия (Nichtdasein), указующее на в-себе-бытие, как
на свое рефлектированное в себя бытие, равно как и наоборот, в-себе-бытие
указует на бытие-для-другого.

3. Оба момента суть определения одного и того же, а именно определения
нечто. Нечто есть {\em в себе}, поскольку оно ушло из
бытия-для-другого, возвратилось в себя. Но нечто имеет также некоторое
определение или обстоятельство {\em в себе} (здесь
ударение падает на «в») или {\em в нем}, поскольку это
обстоятельство есть {\em в нем} внешним образом, есть
бытие-для-другого.

Это ведет к некоторому дальнейшему определению.
{\em В-себе-бытие} и бытие-для-другого ближайшим
образом разны, но то обстоятельство, что нечто имеет
{\em то же самое}, {\em что оно
есть в себе}, также и {\em в нем}, и что, наоборот, то,
что оно есть как бытие-для-другого, оно есть также и в себе "--- в этом
состоит тождество в-себе-бытия и бытия-для-другого, согласно тому
определению, что само нечто есть тождество (ein und dasselbe) обоих
моментов, и что они, следовательно, в нем нераздельны. — Формально это
тождество получается уже в сфере наличного бытия, но более определенное
выражение оно получит при рассмотрении сущности и затем "--- отношения
{\em внутреннего} (der Innerlichkeit) и
{\em внешнего} (Äusserlichkeit), а определеннее всего
оно выявится при рассмотрении идеи как единства понятия и действительности.
— Обыкновенно мнят, что словами «{\em в себе}» мы
высказываем нечто высокое, точно так же, как словом
«{\em внутреннее}»; но на самом деле то, что нечто есть
{\em только в себе}, есть также
{\em только в нем}; «в себе» есть только абстрактное и,
следовательно, внешнее определение. Выражения: «в нем ничего нет», «в этом
что-то есть», содержат в себе, хотя и смутно, тот смысл, что то, что
{\em в человеке} есть (an einem), принадлежит также и к
его {\em в-себе-бытию}, к его внутренней истинной
ценности.

Можно указать, что здесь уясняется смысл
{\em вещи-в-себе}, которая есть очень простая
абстракция, но в продолжение долгого времени слыла очень важным
определением, как бы чем-то аристократическим, точно так же, как положение,
гласящее, что мы не знаем, каковы вещи в себе, признавалось
многозначительной мудростью. — Вещи называются вещами-в-себе, поскольку мы
абстрагируемся от всякого бытия-для-другого, поскольку мы их мыслим без
всякого определения, как представляющие собою ничто. В этом смысле нельзя,
разумеется, знать, {\em что такое}
вещь-{\em в-себе}. Ибо вопрос,
{\em что такое?} требует, чтобы были указаны
{\em определения}; но так как те вещи, относительно
которых требуется, чтобы были указаны определения, должны быть вместе с тем
{\em вещами-в-себе},~т.~е. как раз не обладать никакими
определениями, то в вопрос бессмысленным образом вложена невозможность
ответить на него или же (если все-таки пытаются ответить) на него дают
только нелепый ответ. — Вещь в себе есть то же самое, чтó то абсолютное, о
котором знают только то, что все в нем едино. Мы поэтому знаем очень
хорошо, чтó представляют из себя эти вещи-в-себе; они как таковые суть не
что иное, как не имеющие истинности, пустые абстракции. Но что такое
поистине вещь в себе, чтó поистине есть в себе, — изложением этого является
логика, причем, однако, под «{\em в-себе}» понимается
нечто лучшее, чем абстракция, а именно, то, чтó нечто есть в своем понятии;
но последнее конкретно внутри себя, как понятие вообще постижимо и, как
определенное и связь своих определений, внутри себя познаваемо.

В-себе-бытие имеет прежде всего своим противостоящим моментом
бытие-для-другого; но в-себе-бытию противопоставляется также и
{\em положенность}. Это выражение, правда,
подразумевает также и бытие-для-другого, но оно определительно разумеет уже
происшедший поворот назад того, чтó не есть в себе, в то, чтó есть его
в-себе-бытие, в чем оно {\em положительно}.
{\em В-себе-бытие} должно быть обычно понимаемо как
абстрактный способ выражения понятия; {\em полагание},
собственно говоря, относится уже к сфере сущности, объективной рефлексии;
основание {\em полагает} то, чтó им обосновывается;
причина, больше того, {\em производит} некоторое
действие, некоторое наличное бытие, самостоятельность которого
{\em непосредственно} отрицается и смысл которого
заключается в том, что оно имеет свою {\em суть}
(Sache), свое бытие в некотором другом. В сфере бытия наличное бытие лишь
{\em происходит} из становления или вместе с нечто
положено некоторое другое, вместе с конечным "--- бесконечное, но конечное не
производит бесконечного, не {\em полагает} его. В сфере
бытия {\em самоопределение} понятия само есть лишь
{\em в себе} "--- и соответственно с этим оно называется
переходом. Рефлектирующие определения бытия, как, например, нечто и другое
или конечное и бесконечное, хотя они по существу указывают друг на друга,
или суть как бытие-для-другого, также считаются как
{\em качественные} существующими особо;
{\em другое есть}, конечное считается так же
{\em непосредственно сущим} и прочно стоящим особо, как
и бесконечное; их смысл представляется завершенным также и без их другого.
Напротив, положительное и отрицательное, причина и действие, хотя они также
берутся как сущие изолированно, все же не имеют вместе с тем никакого
смысла друг без друга; {\em в них самих} имеется
отблеск своего другого, каждое из них как бы светится в своем другом. — В
разных кругах определения и в особенности в поступательном движении
экспозиции, или, точнее, в поступательном движении понятия по направлению к
своей экспозиции, главное заключается в том, чтобы всегда вполне различать
между тем, чтó еще есть {\em в себе}, и тем, чтó
{\em положено}, каковы определения, как они суть в
понятии, и каковы они, как положенные или сущие для других. Это
—~различение, которое принадлежит только диалектическому развитию,
различение, которого не знает метафизическое философствование "--- к
последнему принадлежит также и критическая философия; дефиниции метафизики,
равно как и ее предпосылки, различения и следствия, имеют в виду делать
утверждения и выводы лишь относительно {\em сущего} и
притом {\em в-себе-сущего}.

В единстве нечто с собою {\em бытие-для-другого}
тождественно со своим «{\em в себе}»; бытие-для-другого
есть, {\em таким образом}, в [самом] нечто.
Рефлектированная таким образом в себя определенность тем самым есть снова
{\em простое сущее}, есть, следовательно, снова
качество, — {\em определение}.

\subsection*{b) Определение, характер (Beschaffenheit) и граница}
«{\em В себе}», в которое нечто рефлектировано внутри
себя из своего бытия-для-другого, уже более не есть абстрактное «в себе», а
как отрицание его бытия-для-другого, опосредствовано последним, которое
таким образом составляет его момент. Оно есть не только непосредственное
тождество нечто с собою, а то тождество, через которое нечто есть также и
{\em в~нем} то, чтó оно есть {\em в
себе}; бытие-для-другого есть {\em в нем}, потому что
«{\em в себе}» есть его снятие, есть выхождение
{\em из него} в себя; но оно есть в нем также уже и
потому, что оно абстрактно, следовательно, существенно обременено
отрицанием, бытием-для-другого. Здесь имеется не только качество и
реальность, сущая определенность, но и
{\em в-себе-сущая} определенность, и ее развертывание
состоит в том, чтобы {\em положить} ее как эту
рефлектированную в себя определенность.

1. Качество, которое есть «в себе» (das Ansich) в простом нечто, находящееся
существенно в единстве с другим моментом последнего, с
{\em в-нем-бытием}, может быть названо его
{\em определением}, поскольку различают это слово в
точном его значении от {\em определенности} вообще.
Определение есть утвердительная определенность как в-себе-бытие, которому
нечто в своем наличном бытии, борясь со своей переплетенностью с тем
другим, которым оно могло быть определено, остается соответственным,
удерживаясь в своем равенстве с собою и проявляя это последнее в своем
бытии-для-другого. Нечто {\em исполняет} свое
определение (назначение)\footnote{Немецкое слово Bestimmung
означает как определение, так и назначение. "--- {\em Перев}.},
поскольку дальнейшая определенность,
многообразно вырастающая ближайшим образом на почве его отношения к
другому, делается соответственной его в-себе-бытию, становится его
полнотой. Определение подразумевает, что то, что нечто есть
{\em в себе}, есть также и {\em в
нем}.

{\em Определением человека} служит мыслящий разум;
мышление вообще есть его простая {\em определенность},
ею он отличается от животного; он есть мышление {\em в
себе}, поскольку оно отличается также и от его бытия-для-другого, от его
собственной природности и чувственности, которыми он непосредственно связан
с другим. Но мышление есть также и {\em в нем}: сам
человек есть мышление, он {\em налично сущ} как
мыслящий, оно есть его существование и действительность; и далее: так как
мышление есть в его наличном бытии и его наличное бытие есть в мышлении, то
оно {\em конкретно}, его следует брать с содержанием и
наполнением, оно есть мыслящий разум и таким образом оно есть
{\em определение} человека. Но даже это определение
есть опять-таки лишь {\em в себе} как некоторое
{\em долженствование},~т.~е. оно вместе с
инкорпорированным в его в-себе-бытии исполнением дано в форме «в себе»
вообще, в {\em противоположность} не инкорпорированному
в нем наличному бытию, которое вместе с тем еще есть внешне противостоящая
ему чувственность и природа.

2. Наполнение в-себе-бытия определенностью также отлично от той
определенности, которая есть лишь бытие-для-другого и остается вне
определения. Ибо в области [категорий] качества различия сохраняют даже в
их снятости непосредственное качественное бытие в отношении друг друга. То,
чтó нечто имеет {\em в нем}, таким образом разделяется,
и оно есть с этой стороны внешнее наличное бытие нечто, каковое наличное
бытие также есть {\em его} наличное бытие, но не
принадлежит его в-себе-бытию. Определенность, таким образом, есть
{\em характер}.

Нося тот или другой характер, нечто подвергается воздействию внешних влияний
и обстоятельств. Это внешнее соотношение, от которого зависит характер, и
определяемость некоторым другим представляются чем-то случайным. Но
качество какого-нибудь нечто в том-то и состоит, чтобы быть предоставленным
этой внешности и обладать некоторым характером.

Поскольку нечто изменяется, изменение имеет место в характере; последний
есть {\em в} нечто то, чтó становится некоторым другим.
Само нечто сохраняет себя в изменении, которое затрагивает только эту
непостоянную поверхность его инобытия, а не его определение.

Определение и характер таким образом отличны друг от друга; со стороны
своего определения нечто безразлично к своему характеру. Но то, чтó нечто
имеет {\em в нем}, есть связующий их средний термин
этого силлогизма. Но бытие-{\em в-нечто}
(Am-Еtwas-Sein) оказалось, наоборот, распадающимся на указанные два крайних
термина. Простой средний термин есть
{\em определенность} как таковая; к ее тождеству
принадлежит как определение, так и характер. Но определение переходит само
по себе в характер и характер сам по себе "--- в определение. Это вытекает из
предыдущего; связь мыслей, говоря более точно, такова: поскольку то, чтó
нечто {\em есть в себе}, есть также и
{\em в нем}, оно обременено бытием-для-другого;
определение как таковое открыто, следовательно, отношению к другому.
Определенность есть вместе с тем момент, но вместе с тем содержит
качественное различие, состоящее в том, что оно разнится от в-себе-бытия,
есть отрицание нечто, некоторое другое наличное бытие. Определенность,
включающая в себя таким образом другое, соединенная с в-себе-бытием, вводит
инобытие во в-себе-бытие или, иначе говоря, в определение, которое, таким
образом, понижается до характера. — Наоборот, бытие-для-другого,
изолированное и положенное само по себе в форме характера, есть в нем (в
нечто) то же, что другое как таковое, другое в нем (в другом) самом,~т.~е.
другое самого себя; но, таким образом, оно есть
{\em соотносящееся с собою} наличное бытие, есть, таким
образом, в-себе-бытие с некоторой определенностью, стало быть,
{\em определение}. — Следовательно, поскольку оба
должны быть вместе с тем удержаны друг вне друга, характер, являющийся
обоснованным в некотором внешнем, в некотором вообще другом,
{\em зависит} также и от определения, и идущий от
чужого процесс определения определен вместе с тем собственной имманентной
определенностью данного нечто. Но, далее, характер принадлежит к тому, чтó
нечто есть в себе; вместе со своим характером изменяется и нечто.

Это изменение нечто уже более не есть первое изменение нечто, изменение
исключительно по своему бытию-для-другого; то первое изменение было только
в себе сущим, принадлежащим внутреннему понятию; теперь же изменение есть
также и положенное в нечто. — Само нечто определено далее, и отрицание
положено как имманентное ему, как его развитое
{\em внутри-себя-бытие}.

Переход определения и характера друг в друга есть ближайшим образом снятие
их различия; тем самым положено наличное бытие или нечто вообще, а так как
оно есть результат указанного различия, обнимающего собою также и
качественное инобытие, то имеются два нечто, но не только вообще другие по
отношению друг к другу, так что это отрицание оказалось бы в таком случае
еще абстрактным и находило бы место лишь в нашем сравнивании их между
собою, а это отрицание теперь имеется как
{\em имманентное} этим ничто. Они как
{\em налично сущие} безразличны друг к другу. Но теперь
это их утверждение уже более не есть непосредственное, каждое из них
соотносится с самим собою через {\em посредство} снятия
того инобытия, которое в определении рефлектировано во в-себе-бытие.

Таким образом, нечто относится к другому {\em из самого
себя} [спонтанейно], ибо инобытие положено в нем как его собственный
момент; его внутри-себя-бытие объемлет собою отрицание, через посредство
которого оно теперь вообще обладает своим утвердительным наличным бытием.
Но это другое также и качественно отлично от последнего и, следовательно,
положено вне нечто. Отрицание своего другого есть лишь качество данного
нечто, ибо оно есть нечто именно как это снятие своего другого. Итак,
собственно говоря, только теперь другое настоящим образом само
противополагает себя некоторому наличному бытию; первому нечто другое
противополагается лишь внешним образом или, иначе говоря, так как они на
самом деле находятся во взаимной связи безоговорочно,~т.~е. по своему
понятию, то эта связь заключается в том, что наличное бытие
{\em перешло} в инобытие, нечто
{\em перешло} в другое, заключается в том, что нечто,
как и другое, есть другое. Поскольку же внутри-себя-бытие есть небытие
инобытия, которое в нем содержится, но вместе с тем, как сущее, отлично от
него, постольку само нечто есть отрицание,
{\em прекращение в нем некоторого другого}; оно
положено, как относящееся к нему отрицательно и тем самым сохраняющее себя;
—~это другое, внутри-себя-бытие данного нечто, как отрицание отрицания,
есть его {\em в-себе-бытие}, и, вместе с тем, это
снятие есть {\em в нем} как простое отрицание, а
именно, как отрицание им внешнего ему другого нечто. Одна и та же единая их
определенность, с одной стороны, тождественна с внутри-себя-бытием этих
нечто как отрицание отрицания, а, с другой стороны, вместе с тем, поскольку
эти отрицания противостоят одно другому как другие нечто, — из них же самих
смыкает их и также отделяет их друг от друга, так как каждое из них
отрицает другое; это "--- {\em граница}.

3. {\em Бытие-для-другого} есть неопределенная,
утвердительная общность нечто со своим другим; в границе же выдвигается
{\em небытие}{}-для-другого, качественное отрицание
другого, которое (другое) благодаря этому не подпускается к
рефлектированному в себя нечто. Мы должны присмотреться к развертыванию
этого понятия, каковое развертывание впрочем скорее оказывается
запутанностью и противоречием. Последнее сразу же сказывается в том, что
граница, как рефлектированное в себя отрицание данного нечто, содержит в
себе {\em идеализованно} моменты нечто и другого, и они
же как различенные моменты вместе с тем положены в сфере наличного бытия
как {\em реальные},
{\em качественно различные}.

\textbf{$\alpha $.} Нечто, следовательно, есть непосредственное
соотносящееся с собою наличное бытие и имеет границу ближайшим образом как
границу в отношении другого; она есть небытие другого, а не самого нечто;
последнее ограничивает в ней свое другое. — Но другое само есть некоторое
нечто вообще; стало быть, граница, которую нечто имеет в отношении к
другому, есть граница также и другого как нечто, граница этого нечто,
которой оно не подпускает к себе первое нечто, как
{\em свое} другое, или, иначе говоря, она есть
{\em небытие этого первого нечто}; таким образом, она
не есть только небытие другого, а есть небытие как одного, так и другого
нечто и, значит, небытие [всякого] {\em нечто} вообще.

Но она есть существенно также и небытие другого; таким образом, нечто вместе
с тем {\em есть} благодаря своей границе. Будучи
ограничивающим, нечто, правда, понижается до того, что само оно оказывается
ограничиваемым, — однако его граница как прекращение другого в нем, вместе
с тем сама есть лишь бытие этого нечто; {\em последнее
есть благодаря ей то, чт}ó{\em  оно есть, имеет в ней
свое качество}. — Это отношение есть внешнее проявление того
обстоятельства, что граница есть простое или
{\em первое} отрицание, другое же есть вместе с тем
отрицание отрицания, внутри-себя-бытие данного нечто.

Нечто как непосредственное наличное бытие есть, следовательно, граница в
отношении другого нечто, но оно имеет ее {\em в себе
самом} и есть нечто через ее опосредствование, которое есть также и его
небытие. Она есть то опосредствование, через которое нечто и другое
{\em столь же суть, сколь и не суть}.

\textbf{$\beta $.} Поскольку нечто {\em и есть} и
{\em не есть} в своей границе и эти моменты суть
некоторое непосредственное, качественное различие, постольку неимение
наличного бытия (Nichtdasein) нашим нечто и его наличное бытие оказываются
друг вне друга. Нечто имеет свое наличное бытие
{\em вне} (или, как это себе также представляют,
{\em внутри}) своей границы; и точно так же и другое
есть вне ее, так как оно есть нечто. Она есть
{\em середина между} ними, в которой они прекращаются.
Они имеют свое {\em наличное бытие по ту сторону} друг
друга и {\em их границы}; граница как небытие каждого
из них есть другое в отношении обоих.

В силу такого различия между нечто и его границей
{\em линия} представляется линией лишь вне своей
границы, точки; {\em плоскость} представляется
плоскостью вне линии; {\em тело} представляется телом
лишь вне ограничивающей его плоскости. — Это есть тот аспект, в котором
граница прежде всего воспринимается представлением, этим вне-себя-бытием
понятия и в этом же аспекте она берется преимущественно в пространственных
предметах.

\textbf{$\gamma $.} Но, далее, нечто, как оно есть вне границы, есть
неограниченное нечто, лишь наличное бытие вообще. Таким образом, оно не
отлично от своего другого; оно есть лишь наличное бытие, имеет,
следовательно, одно и то же определение со своим другим; каждое из них есть
лишь нечто вообще или, иначе говоря, каждое есть другое; оба суть, таким
образом, {\em одно и то же}. Но это их сначала лишь
непосредственное наличное бытие теперь положено с определенностью, как
границей, в которой оба суть то, что они суть, в различенности друг от
друга. Но она точно так же, как и наличное бытие, есть
{\em общее} им обоим различие, их единство и различие.
Это двоякое тождество обоих "--- наличное бытие и граница "--- подразумевает, что
нечто имеет свое наличное бытие только в границе и что, так как и граница и
непосредственное наличное бытие вместо с тем суть отрицания друг друга, то
нечто, которое есть только в своей границе, в такой же мере отделяет себя
от самого себя, указует дальше себя, на свое небытие, и высказывает
последнее как свое бытие, переходя, таким образом, в последнее. Чтобы
применить это к предыдущему примеру, следует сказать, что
{\em одно} определение нашего нечто состоит в том, что
нечто есть то, чтó оно есть, только в своей границе; следовательно,
{\em точка} есть граница
{\em линии} не только таким образом, что последняя лишь
прекращается в точке, и что линия как наличное бытие есть вне точки;
{\em линия} есть граница
{\em плоскости} не только таким образом, что последняя
лишь прекращается в линии (это точно так же применимо к
{\em плоскости}, как к границе
{\em тела}). А в точке линия также и
{\em начинается}; точка есть абсолютное начало линии.
Даже и в том случае, когда линию представляют себе продолженной в обе ее
стороны безгранично, или, как обыкновенно выражаются, бесконечно, точка
составляет ее {\em элемент}, подобно тому как линия
составляет элемент плоскости, а плоскость "--- элемент тела. Эти
{\em границы} суть {\em принцип}
того, чтó они ограничивают, подобно тому, как единица, например, как сотая,
есть граница, но вместе с тем также и элемент всей сотни.

{\em Другим} определением служит беспокойство нашего
нечто "--- беспокойство, состоящее в том, что оно в своей границе, в которой
оно пребывает, представляет собою {\em противоречие},
заставляющее его выходить дальше самого себя. Так например, точка есть
диалектика самой себя, заставляющая ее стать линией; линия "--- диалектика,
заставляющая стать плоскостью, плоскость "--- диалектика, заставляющая стать
целостным пространством. Вторая дефиниция, которую дают линии, плоскости и
всему пространству, гласит поэтому, что через
{\em движение} точки возникает линия, через движение
линии возникает плоскость и~т.~д. Но на это
{\em движение} точки, линии и~т.~д. смотрят как на
нечто случайное или как на нечто такое, чтó мы только представляем себе.
Однако от этого взгляда, собственно говоря, отказываются уже тогда, когда
признают, что определения, из которых, согласно этой дефиниции, возникают
линии и~т.~д., суть их {\em элементы} и принципы, а
последние суть не что иное, как вместе с тем и их границы; возникновение,
таким образом, рассматривается не как случайное или лишь представляемое.
Что точка, линия, поверхность сами по себе, противореча себе, суть начала,
которые сами отталкиваются от себя, и что точка, следовательно, сама собою,
через свое понятие, переходит в линию, {\em движется в
себе} и заставляет возникнуть линию и~т.~д., — это лежит в понятии
имманентной данному нечто границы. Однако само применение должно
рассматриваться не здесь, а там, где будем трактовать о пространстве; чтобы
здесь только намекнуть на это применение, скажем, что точка есть совершенно
абстрактная граница, но {\em в некотором наличном
бытии}; последнее берется здесь еще совершенно неопределенно; оно есть так
называемое абсолютное,~т.~е. абстрактное
{\em пространство}, безоговорочно непрерывная
внеположность. Тем самым, что граница не есть абстрактное отрицание, а есть
отрицание {\em в этом наличном бытии}, тем самым, что
она есть {\em пространственная} определенность, — точка
пространственна и представляет собою противоречие между абстрактным
отрицанием и непрерывностью и, значит, совершающийся и совершившийся
переход в линию и~т.~д., как и на самом деле [в реальном мире]
{\em нет} ни точки, ни линии, ни поверхности.

Нечто вместе со своей имманентной границей, положенное, как противоречие
самого себя, в силу которого оно выводится и гонится вне себя, есть
{\em конечное}.

\subsection*{c) Конечность}
Наличное бытие определено; нечто имеет некоторое качество, и оно в последнем
не только определено, но и ограничено; его качество есть его граница,
обремененное которой оно сначала остается утвердительным, спокойным
наличным бытием. Но когда это отрицание развито так, что противоположность
между его наличным бытием и отрицанием как имманентной ему границей сама
есть внутри-себя-бытие этого нечто, и последнее, таким образом, есть лишь
становление в нем самом, — когда это отрицание так развито, оно составляет
его (этого нечто) конечность.

Когда мы говорим о вещах, что {\em они конечны}, то мы
разумеем под этим, что они не только имеют некоторую определенность, что
качество есть не только реальность и сущее-в-себе определение, что они не
только ограничены, — а как таковые они еще обладают наличным бытием вне
своей границы, — но что, наоборот, небытие составляет их природу, их бытие.
Конечные вещи {\em суть}, но их соотношение с самими
собою состоит в том, что они соотносятся с самими собою как
{\em отрицательные}, что они именно в этом соотношении
с самими собою гонят себя дальше себя, дальше своего бытия. Они
{\em суть}, но истиной этого бытия служит их
{\em конец}. Конечное не только изменяется, как нечто
вообще, а {\em преходит}; и это не только возможно, что
оно преходит, так что оно могло бы быть, не преходя, а бытие конечных вещей
как таковое состоит в том, что они носят в себе зародыш прехождения, как
свое внутри-себя-бытие, что час их рождения есть час их смерти.

\subsection*{$\alpha $. Непосредственность конечности}

Мысль о конечности вещей влечет за собой эту скорбь по той причине, что эта
конечность есть доведенное до последнего заострения качественное отрицание
и что в простоте такого определения им уже более не оставлено никакого
утвердительного бытия, {\em отличного} от их
определения к гибели. Вследствие этой качественной простоты отрицания,
возвратившегося к абстрактной противоположности ничто и прехождения, с
одной стороны, и бытия "--- с другой, конечность есть наиболее упрямая
категория рассудка; отрицание вообще, характер, граница уживаются со своим
другим, с наличным бытием; даже от абстрактного ничто, взятого само по
себе, как абстракция, готовы отказаться; но конечность есть
{\em фиксированное в себе} отрицание и поэтому резко
противостоит своему утвердительному. Конечное, правда, не сопротивляется
тому, чтобы его приводили в движение, оно само и состоит в том, что оно
предназначено к своему концу, но лишь к своему концу; оно есть упорный
отказ от того, чтобы его утвердительно приводили к его утвердительному, к
бесконечному, недопущение того, чтобы его приводили в связь с последним.
Оно, следовательно, положено нераздельным со своим ничто, и этим отрезан
путь к какому бы то ни было его примирению со своим другим, с
утвердительным. Определение конечных вещей не простирается далее их
{\em конца}. Рассудок никак не хочет отказаться от этой
скорби о конечности, делая небытие определением вещей и вместе с тем
{\em непреходящим} и
{\em абсолютным}. Их преходимость могла бы прейти лишь
в их другом, в утвердительном; тогда их конечность отделилась бы от них. Но
она есть их неизменное качество,~т.~е. не переходящее в свое другое,~т.~е.
в свое утвердительное; {\em таким образом, она вечна}.

Это "--- весьма важное соображение; но что конечное абсолютно, — это такая
точка зрения, которую, разумеется, вряд ли какое-либо философское учение
или какое-либо воззрение или рассудок позволят навязать себе; можно
сказать, что, наоборот, в утверждении о конечном определенно заключается
противоположный взгляд: конечное есть ограниченное преходящее; конечное
есть {\em только} конечное, а не непреходящее; это
заключается непосредственно в его определении и выражении. Но важно знать,
настаивает ли это воззрение на том, чтобы мы не шли дальше
{\em бытия конечности} и рассматривали
{\em преходимость} как остающуюся существовать, или же
оно признает, что {\em преходимость} и
{\em прехождение преходят}? Что последнее не имеет
места, это как раз фактически утверждается тем воззрением на конечное,
которое делает {\em прехождение последним словом} о
конечном. Оно определенно утверждает, что конечное непримиримо и
несоединимо с бесконечным, что конечное безоговорочно противоположно
бесконечному. Бесконечному это воззрение приписывает бытие, абсолютное
бытие; конечное, таким образом, остается в отношении к нему фиксированным
как его отрицательное; несоединимое с бесконечным, оно остается абсолютным
на своей собственной стороне; оно могло бы получить утвердительность от
утвердительного, от бесконечного и, таким образом, оно прешло бы; но
соединение-то с последним именно и объявляется невозможным. Если верно, что
оно пред лицом бесконечного не пребывает, а преходит, то, как мы сказали
раньше, последнее слово о нем есть прехождение, а не утвердительное,
которым могло бы быть лишь прехождение прехождения. Если же конечное
преходит не в утвердительном, а его конец понимается как
{\em ничто}, то мы снова оказываемся у того первого,
абстрактного ничто, которое само давно прешло.

Однако в этом ничто, которое должно быть {\em только}
ничто и которому вместе с тем приписывают некоторое существование, а
именно, существование в мышлении, представлении или речи, мы встречаем то
же самое противоречие, которое только что было указано в конечном, с той
только разницей, что в абстрактном ничто это противоречие только
{\em встречается}, а в конечности оно
{\em решительно выражено}. Там оно представляется
субъективным, здесь же утверждают, что конечное
{\em противостоит} бесконечному
{\em во веки веков}, {\em есть} в
себе ничтожное и есть {\em как} в себе ничтожное. Это
нужно осознать; и развертывание конечного показывает, что оно в самом себе,
как это внутреннее противоречие, рушится внутри себя, но при этом
действительно разрешает указанное противоречие, обнаруживая, что оно не
только преходяще и преходит, но что прехождение, ничто не есть нечто
окончательное, а само преходит.

\bigskip

\subsection*{$\beta $. Предел и долженствование}

Хотя абстрактно это противоречие сразу же содержится в том, что
{\em нечто} конечно, или, иначе говоря, что конечное
есть, однако {\em нечто} или бытие теперь уже более не
положено абстрактно, а рефлектировано в себя и развито, как
внутри-себя-бытие, имеющее в себе некоторое определение и характер, и, еще
определеннее, оно развито так, что имеет границу в нем самом, которая,
будучи имманентной этому нечто и составляя качество его внутри-себя-бытия,
есть конечность. Мы должны посмотреть, какие моменты содержатся в этом
понятии конечного нечто.

Определение и характер оказались {\em сторонами} для
внешней рефлексии. Но первое уже содержало инобытие, как принадлежащее к
«{\em в себе}» данного нечто. Внешность инобытия есть,
с одной стороны, в собственной внутренности нечто, а, с другой стороны, она
как внешность остается отличной от последней, она еще есть внешность как
таковая, но {\em в} (an) нечто. Но так как, далее,
инобытие как {\em граница} само определено как
отрицание отрицания, то имманентное нашему нечто инобытие положено как
соотношение обеих сторон, и единство нашего нечто с собою, которому (нечто)
принадлежит как определение, так и характер, оказывается его обращенным
против самого себя соотношением, отрицающим в нем его имманентную границу
соотнесением его в-себе-сущего определения с этой границей. Тождественное с
собою внутри-себя-бытие соотносится, таким образом, с самим собою как со
своим собственным небытием, однако как отрицание отрицания, как отрицающее
это свое небытие, которое вместе с тем сохраняет в нем наличное бытие, ибо
оно есть качество его внутри-себя-бытия. Собственная граница данного нечто,
положенная, таким образом, им как такое отрицательное, которое вместе с тем
существенным образом есть, есть не только граница как таковая, а
{\em предел}. Но предел не есть только положенное как
подвергнутое отрицанию. Отрицание обоюдоостро, поскольку положенное им как
отрицаемое есть {\em граница}. А именно, последняя есть
вообще нечто такое, что обще данному нечто и его другому; она есть также
определенность {\em в-себе-бытия} определения как
такового. Это в-себе-бытие, следовательно, как отрицательное соотношение со
своей также и отличной от него границей, с собою как пределом, есть
{\em долженствование}.

Для того, чтобы граница, которая есть вообще в нечто, была пределом, оно
необходимо должно вместе с тем внутри самого себя переступать ее, в самом
себе соотноситься {\em с нею как с некоторым не-сущим}.
Наличное бытие нашего нечто лежит спокойно-равнодушно, как бы
{\em подле} своей границы. Но нечто переступает свою
границу лишь постольку, поскольку оно есть ее снятость, отрицательное по
отношению к ней в-себе-бытие. А так как она в самом
{\em определении} имеет бытие как предел, то нечто тем
самым {\em переступает через самого себя}.

Долженствование содержит, следовательно, двоякое определение: содержит,
{\em во-первых}, его как в-себе-сущее определение,
противостоящее отрицанию, но, {\em во-вторых}, содержит
это же определение как некоторое небытие, которое как предел отлично от
него, но вместе с тем само есть в-себе-сущее определение.

Итак, конечное определилось как соотношение его определения с границей,
первое есть в этом соотношении {\em долженствование}, а
последняя есть {\em предел}. Оба суть, таким образом,
моменты конечного; тем самым оба, как долженствование, так и предел, сами
конечны. Но лишь предел {\em положен} как конечное;
долженствование ограничено лишь в себе, следовательно, лишь для нас.
Благодаря своему соотношению с ему самому уже имманентной границей оно
ограничено, но эта его ограниченность закутана во в-себе-бытие, ибо по
своему наличному бытию,~т.~е. по своей определенности, противостоящей
пределу, долженствование положено как в-себе-бытие.

То, чтó должно быть, {\em есть} и, вместе с тем,
{\em не есть}. Если бы оно
{\em было}, оно тогда не только
{\em должно было бы быть}. Следовательно,
долженствование имеет по существу своему некоторый предел. Этот предел не
есть некое чуждое; {\em то}, чтó
{\em лишь} должно быть, есть
{\em определение}, которое теперь положено таковым,
каково оно есть в самом деле, а именно, как то, чтó есть вместе с тем лишь
некоторая определенность.

В-себе-бытие, присущее нашему нечто в его определении, низводит себя,
следовательно, до уровня {\em долженствования} тем, что
то самое, чтó составляет его в-себе-бытие, дано (ist) в одном и том же
отношении как {\em небытие} и притом таким образом, что
во внутри-себя-бытии, в отрицании отрицания, означенное в-себе-бытие как
одно отрицание (отрицающее) есть единство с другим отрицанием, которое как
качественно другое есть вместе с тем граница, благодаря чему указанное
единство дано как {\em соотношение} с нею. Предел
конечного не есть некое внешнее, а его собственное определение есть также и
его предел; и последний есть как сам он, так и долженствование; он есть
общее обоим, или, вернее, то, в чем оба тождественны.

Но, далее, как долженствование конечное выходит {\em за}
свой предел; та же самая определенность, которая есть его отрицание, также
и снята и, таким образом, есть его в-себе-бытие; его граница также и не
есть его граница.

Как {\em долженствование} нечто, следовательно,
{\em выше своего предела}, но и наоборот, лишь
{\em как долженствование} оно имеет свой
{\em предел}; оба нераздельны. Нечто имеет предел
постольку, поскольку оно в своем определении имеет отрицание, а определение
есть также и снятость предела.

\subsection*{Примечание. Долженствование}

Долженствование играло недавно большую роль в философии, преимущественно в
том, что касается морали, а также и в метафизике вообще, как последнее и
абсолютное понятие о тождестве в-себе-бытия или соотношения с
{\em самим собою} и
{\em определенности} или границы.

«{\em Ты можешь, потому что ты
должен}»[ссылка!] "--- это выражение, которое должно
было много говорить уму, содержится в понятии долженствования. Ибо
долженствование есть выход за предел; граница в нем снята, в-себе-бытие
долженствования есть, таким образом, тождественное соотношение с собою, и,
следовательно, есть абстракция представления: «мочь, быть в состоянии»
(Abstraktion des Könnens). — Но столь же правильно и обратное утверждение:
{\em ты не можешь именно потому, что ты должен}. Ибо в
долженствовании содержится также и предел как предел; вышеуказанный
формализм возможности имеет в этом пределе некоторую противостоящую ему
реальность, некоторое качественное инобытие, и их взаимоотношение есть
противоречие, означает, следовательно, не быть в состоянии или, вернее,
невозможность.

В долженствовании начинается выхождение за конечность, бесконечность.
Долженствование есть то, чтó в дальнейшем [логическом] развитии оказывается
со стороны вышеуказанной невозможности прогрессом в бесконечность.

Мы можем здесь ближе подвергнуть критике два предрассудка касательно формы
{\em предела} и
{\em долженствования}. Во-первых, обыкновенно придают
{\em большое} значение пределам мышления, разума
и~т.~д. и утверждают, что наш разум, наше мышление,
{\em не в состоянии} выйти за эти пределы. В этом
утверждении сказывается несознавание того, что, определяя нечто как предел,
мы тем самым уже вышли за него. Ибо некоторая определенность, граница,
определяется как предел лишь в противоположность к его другому вообще, как
к его {\em неограниченному}; другое некоторого предела
именно и есть {\em выход} за него. Камень, металл не
выходят за свой предел, потому что {\em для них} он не
есть предел. Но если при таких всеобщих положениях рассудочного мышления,
как утверждение о невозможности выйти за предел, мышление не хочет
потрудиться рассмотреть, что содержится в понятии, то можно отослать
читателя к действительности, в которой подобного рода положения оказываются
самым что ни на есть недействительным. Именно вследствие того, что мышление
{\em должно} быть чем-то более высоким, чем
действительность, {\em должно} оставаться вдали от нее,
витать в высших областях, вследствие того, следовательно, что само оно
определено как некоторое {\em долженствование}, —
именно вследствие этого оно, с одной стороны, не движется вперед по
направлению к понятию, а, с другой стороны, оказывается в такой же мере
неистинным по отношению к действительности, в какой оно неистинно по
отношению к понятию. — Так как камень не мыслит и даже не ощущает, то его
ограниченность не есть {\em для него} предел,~т.~е. она
не есть в нем отрицание для ощущения, представления, мышления и~т.~д.,
которыми он не обладает. Но даже и камень как некоторое нечто заключает в
себе различие между своим определением или своим в-себе-бытием и своим
наличным бытием, и постольку он тоже выходит за свой предел; понятие,
которое он есть в себе, содержит в себе тождество с его другим. Если он
есть способное к окислению [химическое] основание, то он окисляется,
нейтрализуется и~т.~д. В окислении, нейтрализации и~т.~д. его предел,
состоящий в том, что он имеет наличное бытие лишь как [химическое]
основание, снимается; он выходит за него; и точно так же и кислота снимает
свой предел, состоящий в том, что она имеет бытие только как кислота, и в
ней, равно как и в щелочном основании, имеется в такой мере
{\em долженствование} выйти за свой предел, что лишь
силой их можно заставить оставаться "--- безводными,~т.~е. в чистом виде, не
нейтральными "--- кислотой и щелочным основанием.

Но если некоторое существование содержит понятие не только как абстрактное
в-себе-бытие, а как для-себя-сущую целостность, как влечение, как жизнь,
ощущение, представление и~т.~д., то оно само изнутри самого себя
осуществляет бытие за своим пределом и переход дальше его. Растение выходит
за предел "--- быть зародышем, и точно так же оно переходит за предел "--- быть
цветком, плодом, листом; зародыш становится развитым растением, цветок
отцветает и~т.~д. Чувствующее существо в пределах голода, жажды и~т.~д.
есть стремление выйти за эти пределы, и оно осуществляет этот выход. Оно
ощущает {\em боль}, и ощущение боли есть прерогатива
чувствующей природы. В его самости есть некоторое отрицание, и это
отрицание определено в его чувстве {\em как некоторый
предел} именно потому, что ощущающее существо обладает чувством своей
{\em самости}, которая есть целостность, выходящая за
пределы указанной определенности. Если бы оно не выходило за эту
определенность, оно не ощущало бы ее как свое отрицание и не испытывало бы
боли. — Но разум, мышление не может, дескать, выйти за предел, он, который
есть {\em всеобщее}, само по себе вышедшее за пределы
особенности,~т.~е. {\em всякой} особенности,
представляющее собою лишь выхождение за предел? "--- Правда, не всякий переход
за предел и не всякое бытие за пределом есть истинное освобождение от него,
истинное утверждение; уже само долженствование есть такое несовершенное
выхождение за предел, есть вообще абстракция. Но указания на совершенно
абстрактное всеобщее достаточно, чтобы служить противовесом такому же
абстрактному заверению, что нельзя выйти за предел, или, пожалуй,
достаточно уже указания на бесконечное вообще, чтобы служить противовесом
заверению, что нельзя выйти за пределы конечного.

Можно при этом упомянуть об одном кажущемся остроумным замечании
{\em Лейбница}, что если бы магнит обладал сознанием,
то он считал бы свое направление к северу определением своей воли, законом
своей свободы[ссылка!]. Верно как раз
обратное. Если бы магнит обладал сознанием и, значит, волей и свободой, то
он был бы мыслящим; и тем самым пространство как
{\em всеобщее} было бы для него объемлющим все
направления, и потому {\em одно} направление к северу
было бы для него пределом его свободы, подобно тому как для человека быть
удерживаемым на одном месте есть предел, а для растения "--- нет.

С другой стороны, {\em долженствование} есть выхождение
за предел, но такое, которое само есть лишь
{\em конечное выхождение}. Оно имеет поэтому свое место
и свою значимость в области конечного, где оно твердо держится в-себе-бытия
против ограниченного и отстаивает его как правило и существенное против
ничтожного. Обязанность (die Pflicht) есть некоторое
{\em долженствование}, обращенное против особенной
воли, против эгоистического вожделения и произвольного интереса; воле,
поскольку она в своей подвижности может изолироваться от истинного,
напоминают о последнем как о некотором долженствовании. Те, которые ставят
долженствование как принцип морали так высоко и полагают, будто непризнание
долженствования последним словом и истинным приводит к разрушению морали,
равно как резонеры, рассудок которых доставляет себе постоянное
удовлетворение тем, что он имеет возможность выставлять против всего
существующего какое-нибудь долженствование и тем самым проявлять свое
лучшезнайство, и которые поэтому тоже не желают, чтобы их лишили
долженствования, не замечают того, что для интересующих их конечных
областей жизни долженствование признается вполне имеющим силу. — Но в самой
действительности вовсе не обстоит так печально с разумностью и законом,
чтобы им приходилось быть только {\em долженствующими}
быть, — на этом останавливается лишь абстракция в-себе-бытия, — равно как и
неверно, что долженствование, взятое в самом себе, есть нечто пребывающее
во веки веков, что было бы тем же самым, как если бы конечность была
абсолютной. Кантовская и фихтевская философии выдают
{\em долженствование} за высший пункт разрешения
противоречий разума, но это, наоборот, есть точка зрения, не желающая выйти
из области конечного и, следовательно, из противоречия.

\subsection*{$\gamma $. Переход конечного в бесконечное}

Долженствование, взятое само по себе, содержит в себе предел, а предел
—~долженствование. Их взаимоотношение есть само конечное, содержащее их оба
в своем внутри-себя-бытии. Эти моменты его определения качественно
противоположны; предел определен как отрицание долженствования, а
долженствование "--- как отрицание предела. Таким образом конечное есть
внутреннее самопротиворечие; оно снимает себя, преходит. Но этот его
результат, отрицательное вообще, есть ($\alpha $) самое его
{\em определение}; ибо оно есть отрицательное
отрицательного. Конечное, таким образом, не прешло в прехождении; оно
ближайшим образом лишь стало некоторым {\em другим}
конечным, которое, однако, есть также прехождение как переход в некоторое
другое конечное и~т.~д., можно сказать "--- до
{\em бесконечности}. Но ($\beta $) рассматривая ближе
этот результат, мы убеждаемся, что в своем прехождении, этом отрицании
самого себя, конечное достигло своего в-себе-бытия, оно в этом прехождении
{\em слилось с самим собою}. Каждый из его моментов
содержит в себе именно этот результат; долженствование выходит за
предел,~т.~е. за себя само; но вне этого долженствования или его другое
есть лишь сам предел. Предел же указует непосредственно вовне на свое
другое, которое есть долженствование, а последнее есть то же самое
раздвоение {\em в-себе-бытия} и
{\em наличного бытия}, что и предел, есть то же самое;
выходя вне себя, оно поэтому точно так же лишь сливается с собою. Это
{\em тождество с собою}, отрицание отрицания, есть
утвердительное бытие, есть, таким образом, другое конечного,
долженствующего иметь своей определенностью первое отрицание; это другое
есть {\em бесконечное}.

\section*{C. Бесконечность}
Бесконечное в его простом понятии можно, прежде всего, рассматривать как
новую дефиницию абсолютного; как лишенное определений соотношение с собою,
оно (абсолютное) положено как {\em бытие} и
{\em становление}. Формы
{\em наличного бытия} выпадают из ряда определений,
которые могут быть рассматриваемы как дефиниции абсолютного, ибо формы
указанной сферы, взятые сами по себе, непосредственно положены лишь как
определенности, как конечные вообще. Бесконечное же признается
безоговорочно абсолютным, так как оно явно определено как отрицание
конечного, и в бесконечном, следовательно, явно выраженным образом
принимается во внимание ограниченность, которой могли бы обладать бытие и
становление, хотя сами в себе они не обладают никакой ограниченностью и не
обнаруживают таковой, — принимается во внимание ограниченность и отрицается
наличие таковой в нем.

Но тем самым бесконечное на самом деле отнюдь еще не изъемлется из области
ограниченности и конечности. Главное состоит в том, чтобы отличить истинное
понятие бесконечности от дурной бесконечности, бесконечное разума от
бесконечного рассудка; однако последнее есть
{\em оконеченное} бесконечное, и мы увидим, что,
удерживая бесконечное чистым от конечного и вдали от него, мы его как раз
лишь оконечиваем.

Бесконечное есть:

a) в {\em простом определении} утвердительное как
отрицание конечного;

b) но оно тем самым находится во {\em взаимоопределении
с конечным} и есть абстрактное, {\em одностороннее
бесконечное};

c) оно есть само снятие этого бесконечного, а равно и конечного, как
{\em единый} процесс, — есть
{\em истинное бесконечное}.

\subsection*{a) Бесконечное вообще}
Бесконечное есть отрицание отрицания, утвердительное,
{\em бытие}, которое, выйдя из ограниченности, вновь
восстановило себя. Бесконечное {\em есть}, и оно есть в
более интенсивном смысле, чем первое непосредственное бытие; оно есть
истинное бытие, восстание из предела. При слове «бесконечное» для души и
для духа {\em восходит} его свет, ибо в нем дух не
только находится абстрактно у себя, а поднимается к самому себе, к свету
своего мышления, своей всеобщности, своей свободы.

Сначала выяснилось по отношению к понятию бесконечного, что наличное бытие в
своем в-себе-бытии определяет себя как конечное и выходит за предел.
Природа самого конечного в том и состоит, чтобы выходить за себя, отрицать
свое отрицание и становиться бесконечным. Бесконечное, стало быть, не стоит
над конечным, как нечто само по себе готовое, так что выходило бы, что
конечное имеет и сохраняет место {\em вне} его или под
ним. Равным образом дело не обстоит так, что лишь
{\em мы}, как некоторый субъективный разум, выходим за
пределы конечного, переходим в бесконечное. Так например, представляют себе
дело, когда говорят, что бесконечное есть понятие разума, и мы посредством
разума возвышаемся над земным и бренным; тут выходит так, что это
совершается без всякого ущерба для конечного, которого вовсе не касается
это остающееся для него внешним возвышение. Но поскольку само конечное
поднимается до бесконечности, оно отнюдь не принуждается к этому чуждой
силой, а его собственная природа состоит в том, чтобы соотноситься с собою
как с пределом "--- и притом как с пределом как таковым, так и с пределом как
долженствованием "--- и выходить за этот предел, или, вернее, его природа
состоит в том, чтобы оно как соотношение с собою подвергло отрицанию этот
предел и вышло за него. Не в упразднении конечности вообще рождается (wird)
бесконечность вообще, а конечное только и состоит в том, что само оно через
свою природу становится бесконечным. Бесконечность есть его
{\em утвердительное определение}, то, что оно поистине
есть в себе.

Таким образом, конечное исчезло в бесконечном, и то, чтó
{\em есть}, есть лишь
{\em бесконечное}.

\subsection*{b) Взаимоопределение конечного и бесконечного}
Бесконечное {\em есть}; в этой непосредственности оно
вместе с тем есть {\em отрицание} некоторого
{\em другого}, конечного. Будучи, таким образом,
{\em сущим} и вместе с тем
{\em небытием} некоторого
{\em другого}, оно впало обратно в категорию нечто как
некоторого определенного вообще; говоря точнее, так как оно есть
рефлектированное в себя, получающееся посредством снятия определенности
вообще наличное бытие и, следовательно, {\em положено}
как отличное от своей определенности наличное бытие, то оно снова впало в
категорию нечто, имеющего некоторую границу. По этой определенности
конечное противостоит бесконечности как {\em реальное
наличное бытие}; таким образом, они находятся в качественном
{\em соотношении} как
{\em остающиеся} вне друг друга:
{\em непосредственное бытие} бесконечного снова
пробуждает {\em бытие} его отрицания, конечного,
которое, как сначала казалось, исчезло в бесконечном.

Но бесконечное и конечное не только находятся в этих категориях соотношения;
обе стороны определены далее так, чтобы быть в отношении друг друга лишь
{\em другими}. А именно, конечность есть предел,
положенный как предел, есть наличное бытие, положенное с
{\em определением} переходить в свое
{\em в-себе-бытие},
{\em становиться} бесконечным. Бесконечность есть ничто
конечного, его {\em в-себе-бытие} и
{\em долженствование}, но последнее дано (ist) вместе с
тем как рефлектированное в себя, как выполненное долженствование, как лишь
с самим собою соотносящееся, совершенно утвердительное бытие. В
бесконечности имеется то удовлетворение, что всяческая определенность,
изменение, всякий предел, а с ним и само долженствование исчезли, положены
как упраздненные, как ничто конечного. Как такое отрицание конечного
определено в-себе-бытие, которое, таким образом, как отрицание отрицания
утвердительно внутри себя. Однако это утверждение есть, как качественно
{\em непосредственное} соотношение с собою,
{\em бытие}; вследствие этого бесконечное сведено к той
категории, что ему противостоит конечное как некое
{\em другое}; его отрицательная природа положена как
{\em сущее}, следовательно, как первое и
непосредственное отрицание. — Бесконечное, таким образом, обременено
противоположностью к конечному, которое как другое остается вместе с тем
определенным, реальным наличным бытием, хотя оно в своем в-себе-бытии, в
бесконечном, положено вместе с тем как упраздненное; последнее есть
не-конечное, — некое бытие в определенности отрицания. В сопоставлении с
конечным, с кругом сущих определенностей, реальностей, бесконечное есть
неопределенное пустое, потустороннее конечного, имеющего свое в-себе-бытие
не в своем наличном бытии, которое есть некоторое определенное бытие.

Бесконечное, сопоставленное таким образом с конечным, положенное во взаимном
качественном соотношении {\em других}, должно быть
названо {\em дурным бесконечным}, бесконечным
{\em рассудка}, который считает его высшей, абсолютной
истиной. Те противоречия, в которые он впадает со всех сторон, как только
он берется за применение и объяснение этих своих категорий, должны были бы
заставить его осознать, что, полагая, что он достиг своего удовлетворения в
примирении истины, он на самом деле пребывает в непримиренном,
неразрешенном, абсолютном противоречии.

Это противоречие сразу же сказывается в том, что наряду с бесконечным
остается конечное как наличное бытие; имеются, таким образом,
{\em две} определенности;
{\em даны} (имеются) два мира, бесконечный и конечный,
и в их соотношении бесконечное есть лишь {\em граница}
конечного и, следовательно, {\em само} есть лишь
определенное, {\em конечное бесконечное}.

Это противоречие развивает свое содержание до более выразительных форм. —
Конечное есть реальное наличное бытие, которое, таким образом, остается и
тогда, когда мы переходим к его небытию, к бесконечному. Последнее, как мы
показали, имеет своей определенностью в отношении конечного лишь первое,
непосредственное отрицание, равно как и конечное в отношении указанного
отрицания имеет, как подвергшееся отрицанию, лишь значение некоторого
{\em другого} и поэтому еще есть нечто. Следовательно,
когда поднимающийся над этим конечным миром рассудок восходит к своему
наивысшему, к бесконечному, этот конечный мир остается существовать как
некое посюстороннее, так что бесконечное лишь становится над конечным,
{\em отделяется} от него, и тем самым конечное как раз
отделяется от бесконечного. Они {\em ставятся в
различные} места; конечное как здешнее наличное бытие, а бесконечное, хотя
оно и есть «{\em в себе}» конечного, все же как некое
потустороннее перемещается в смутную, недостижимую даль,
{\em вне} которой находится и остается конечное.

Отделенные таким образом друг от друга, они столь же существенно
{\em соотнесены} друг с другом как раз разлучающим их
отрицанием. Это соотносящее их "--- рефлектированные в себя нечто "--- отрицание
есть их взаимная граница одного относительно другого и притом таким
образом, что каждое из них имеет ее не только в отношении другого в нем, а
отрицание есть их {\em в-себе-бытие}; каждое из них,
таким образом, имеет границу в самом себе, взятом особо, в его отделенности
от другого. Но эта граница имеет бытие как первое отрицание; таким образом,
оба суть ограниченные, конечные в самих себе. Однако каждое из них, как
утвердительно соотносящееся с собою, есть также и отрицание своей границы.
Таким образом, оно непосредственно отталкивает ее от себя как свое небытие,
и, будучи качественно отделенным от нее, оно ее полагает как некоторое
{\em другое бытие}, вне себя; конечное полагает свое
небытие как это бесконечное, а последнее полагает таким же образом
конечное. Что от конечного необходимо,~т.~е. благодаря определению
конечного, совершается переход к бесконечному и что конечное тем самым
возводится во в-себе-бытие, с этим легко соглашаются, поскольку конечное,
хотя и определено как устойчивое наличное бытие, определено, однако, вместе
с тем {\em также} и как ничтожное
{\em в себе}, следовательно, по самому своему
определению разлагающееся, а бесконечное, хотя и определено как
обремененное отрицанием и границей, определено, однако, вместе с тем также
и как сущее {\em в себе}, так что эта абстракция
соотносящегося с собою утверждения составляет его определение и,
следовательно, согласно последнему в нем не заключено конечное наличное
бытие. Но мы показали выше, что само бесконечное получает утвердительное
бытие лишь {\em посредством} отрицания как отрицания
отрицания и что это его утверждение, взятое как лишь простое, качественное
бытие, понижает содержащееся в нем отрицание до простого, непосредственного
отрицания и тем самым "--- до определенности и границы, которая затем как
противоречащая его в-себе-бытию вместе с тем исключается из него,
полагается как не ему принадлежащая, а, наоборот, противоположная его
в-себе-бытию, полагается как конечное. Таким образом, поскольку каждое из
них в самом себе и в силу своего определения есть полагание своего другого,
они {\em нераздельны}. Но это их единство
{\em скрыто} в их качественной инаковости; оно есть
{\em внутреннее}, которое только лежит в основании.

Этим определен способ проявления указанного единства; положенное в
{\em наличном бытии}, оно дано (ist) как превращение
или переход конечного в бесконечное, и наоборот; так что бесконечное в
конечном и конечное в бесконечном, другое в другом, лишь
{\em выступает},~т.~е. каждое из них есть некое
собственное {\em непосредственное} возникновение в
другом и их соотношение есть лишь внешнее.

Процесс их перехода [друг в друга] имеет детально следующий вид. Совершается
выхождение за пределы конечного, переход в бесконечное. Это выхождение
представляется внешним действием. Что возникает в этой потусторонней для
конечного пустоте? Что в ней положительного? Вследствие нераздельности
бесконечного и конечного (или, иначе говоря, вследствие того, что это
находящееся на своей стороне бесконечное само ограничено) возникает
граница; бесконечное исчезло, и появилось его другое, конечное. Но это
появление конечного представляется некиим внешним бесконечному событием, а
новая граница "--- чем-то таким, что не возникает из самого бесконечного, а
само есть такое же преднайденное. Перед нами, таким образом, впадение снова
в прежнее, тщетно снятое определение. Но эта новая граница сама в свою
очередь есть лишь нечто такое, что должно быть снято или, иначе говоря, что
следует преступить. Стало быть, снова возникла пустота, ничто, в котором мы
равным образом встречаем указанную определенность, некоторую новую границу
—~{\em и так далее до бесконечности}.

Имеется {\em взаимоопределение конечного и
бесконечного}; конечное конечно лишь в соотношении с долженствованием или с
бесконечным, а бесконечное бесконечно лишь в соотношении с конечным. Они
нераздельны и вместе с тем суть безоговорочно другие в отношении друг
друга; каждое из них имеет в нем самом свое другое; таким образом, каждое
есть единство себя и своего другого, и есть в своей определенности наличное
бытие, состоящее в том, чтобы {\em не} быть тем, что
оно само есть и что есть его другое.

Это взаимоопределение, отрицающее само себя и свое отрицание, и есть то, что
выступает как {\em прогресс в бесконечность}, который в
столь многих образах и применениях признается
{\em последним словом}, дальше которого уже не идут,
ибо, дойдя до этого «{\em и так далее} до
бесконечности», мысль имеет обыкновение останавливаться, достигнув своего
конца. — Этот прогресс проявляется всюду, где
{\em относительные} определения доводятся до их
противопоставления, так что они находятся в нераздельном единстве, и тем не
менее каждому в отношении другого приписывается самостоятельное
существование (Dasein). Этот прогресс есть поэтому
{\em противоречие}, которое не разрешено, а лишь
высказывается постоянно, как просто {\em имеющееся
налицо}.

Имеется некое абстрактное выхождение, которое остается неполным, так как
{\em не выходят дальше самого этого выхождения}.
Имеется бесконечное; за него, правда, выходят, ибо полагают некоторую новую
границу, но тем самым, как раз наоборот, лишь возвращаются к конечному. Эта
дурная бесконечность есть в себе то же самое, что продолжающееся во веки
веков {\em долженствование}; она, хотя и есть отрицание
конечного, не может, однако, истинно освободиться от него; это конечное
снова выступает {\em в ней же самой} как ее другое,
потому что это бесконечное имеет бытие лишь как находящееся
{\em в соотношении} с другим для него конечным.
Прогресс в бесконечность есть поэтому лишь повторяющаяся одинаковость, одно
и то же скучное чередование этого конечного и бесконечного.

Бесконечность бесконечного прогресса остается обремененной конечным как
таковым, ограничена им и сама {\em конечна}. Но этим
она на самом деле была бы положена как единство конечного и бесконечного.
Однако указанное единство не делается предметом размышления. Тем не менее,
только оно-то и вызывает в конечном бесконечное и в бесконечном конечное;
оно есть, так сказать, движущая пружина бесконечного прогресса. Он есть
{\em внешняя сторона} сказанного единства, в которой
застревает представление; последнее застревает в этом продолжающемся во
веки веков повторении одного и того же чередования, в пустом беспокойстве
выхождения во вне, за границу к бесконечности, выхождения, которое
{\em находит} в этом бесконечном новую границу, но
столь же мало может удержаться на этой границе, как и на бесконечном. Это
бесконечное имеет твердую детерминацию некоего
{\em потустороннего}, которое не может быть достигнуто
потому, что оно не {\em должно} быть достигнуто, так
как не хотят отказаться от определенности потустороннего, от
{\em сущего} отрицания. По этому определению оно имеет
на противоположной стороне конечное как некое
{\em посюстороннее}, которое столь же мало может
возвыситься до бесконечности именно потому, что оно имеет эту детерминацию
некоторого {\em другого} и, следовательно, детерминацию
продолжающегося во веки веков, все снова и снова порождающего себя в своем
потустороннем (и притом порождающего себя как отличное от него)
{\em наличного бытия}[ссылка!].

\subsection*{c) Утвердительная бесконечность}
В показанном нами переходящем туда и сюда взаимоопределении конечного и
бесконечного их истина уже {\em имеется} в себе, и
требуется лишь воспринять то, что имеется. Это качание туда и сюда
составляет внешнюю реализацию понятия. В ней
{\em положено}, но {\em внешним
образом}, одно вне другого, то, что содержится в понятии; требуется лишь
сравнение этих разных моментов, в котором получается
{\em единство}, дающее само понятие.
{\em Единство} бесконечного и конечного "--- мы на это
часто указывали, но здесь следует в особенности напомнить об этом "--- есть
неудачное выражение для единства, каково оно есть поистине; но и устранение
этого неудачного определения должно иметься в этом лежащем перед нами
внешнем проявлении (Äusserung) понятия.

Взятое по своему ближайшему, лишь непосредственному определению, бесконечное
имеет бытие (ist) только как {\em выход за конечное};
оно есть по своему определению отрицание конечного; таким образом, и
конечное имеет бытие (ist) только как то, за что следует выйти, как
отрицание себя в самом себе, отрицание, которое есть бесконечность.
{\em В каждом из них заключается}, следовательно,
{\em определенность другого}, причем по смыслу
бесконечного прогресса они исключены друг из друга и лишь попеременно
следуют одно за другим; одно не может быть положено и мыслимо (gefasst) без
другого, бесконечное "--- без конечного и конечное "--- без бесконечного. Когда
{\em высказывают}, что такое бесконечное, а именно, что
оно есть отрицание {\em конечного}, то одновременно
{\em высказывается} само конечное; и
{\em обойтись} без него при определении бесконечного
{\em нельзя}. Нужно только
{\em знать, что высказываешь}, чтобы найти в
бесконечном определение конечного. Относительно же конечного, с другой
стороны, сразу соглашаются, что оно есть ничтожное; но именно его
ничтожность и есть бесконечность, от которой оно равным образом неотделимо.
— Может показаться, что это понимание берет их по их
{\em соотношению с их другим}. Следовательно, если их
брать {\em безотносительно}, так что они будут
соединены лишь союзом «и», то они будут противостоять друг другу, как
самостоятельные, каждое из которых есть только в самом себе. Посмотрим
какой характер они, взятые таким способом, будут носить. Бесконечное,
поставленное таким образом, есть {\em одно из этих
двух}; но как {\em лишь} одно из двух, оно само
конечно, оно "--- не целое, а лишь одна сторона; оно имеет свою границу в
противостоящем; таким образом, оно есть {\em конечное
бесконечное}. Имеются лишь {\em два конечных}. Как раз
в том обстоятельстве, что бесконечное, таким образом,
{\em отделено} от конечного, поставлено, следовательно,
как {\em одностороннее}, и заключается его конечность
и, стало быть, его единство с конечным. — Конечное со своей стороны, как
поставленное само по себе, в отдалении от бесконечного, есть
{\em то соотношение с собою}, в котором удалена его
относительность, зависимость, его преходимость; оно есть те же самые
самостоятельность и утверждение себя, которыми должно быть бесконечное.

Оба способа рассмотрения, имеющие, как кажется сначала, своим исходным
пунктом разные определенности, поскольку первый брал лишь
{\em соотношение} друг с другом конечного и
бесконечного, каждого с его другим, а второй якобы удерживает их в их
полной отделенности друг от друга, приводят к одному и тому же результату.
Бесконечное и конечное, взятые по их {\em соотношению}
друг с другом, которое как будто внешне для них, но на самом деле для них
существенно, и без которого ни одно из них не есть то, что оно есть,
содержат, таким образом, свое другое в своем собственном определении, и
точно так же каждое, взятое {\em особо},
рассматриваемое в самом {\em себе}, заключает в себе
свое другое, как свой собственный момент.

Это и дает приобретшее дурную славу единство конечного и бесконечного
—~единство, которое само есть бесконечное, охватывающее собою само себя и
конечность, — следовательно, бесконечное в другом смысле, чем в том,
согласно которому конечное отделено от него и поставлено на другой стороне;
так как они должны быть также и различны, то каждое, как мы показали
раньше, есть само в себе единство обоих; таким образом, получаются два
таких единства. То, что обще тому и другому,~т.~е. единство этих двух
определенностей, полагает их, как единство, ближайшим образом подвергшимися
отрицанию, так как каждое берется как долженствующее быть тем, что оно есть
в их различности; в своем единстве они, следовательно, теряют свою
качественную природу. Это "--- очень важное соображение против представления,
которое не хочет отказаться от того, чтобы в единстве бесконечного и
конечного удерживать их в том качестве, которое они должны иметь, взятые
вне друг друга, и которое (представление) поэтому видит в сказанном
единстве только противоречие, а не также и разрешение последнего путем
отрицания качественной определенности их обоих. Таким образом,
фальсифицируется это ближайшим образом простое, всеобщее единство
бесконечного и конечного.

Но, далее, так как они должны быть взяты также и как различные, то единство
бесконечного [и конечного], которое (единство) каждый из этих моментов есть
сам, определено в каждом из них различным образом. То, что по своему
определению есть бесконечное, имеет в себе (an ihm) отличную от себя
конечность; первое есть «{\em в себе}» (das Ansich) в
этом единстве, а конечность есть лишь определенность, граница в нем; но это
—~такая граница, которая есть его безоговорочно другое, его
противоположность. Его определение, которое есть в-себе-бытие как таковое,
портится примесью такого рода качества; оно есть, таким образом,
{\em оконеченное бесконечное}. Подобным же образом, так
как конечное как таковое есть лишь не-в-себе-бытие, но согласно сказанному
единству заключает в себе также и свою противоположность, то оно
возвеличивается превыше своей ценности и притом, можно сказать,
возвеличивается бесконечно; оно полагается, как
{\em обесконеченное} конечное.

Таким же образом, как раньше рассудок фальсифицировал простое единство, он
теперь фальсифицирует также и двоякое единство бесконечного и конечного.
Это и здесь также происходит потому, что в одном из этих двух единств
бесконечное берется не как подвергшееся отрицанию, а, наоборот, как
в-себе-бытие, в котором, следовательно, не должны быть положены
определенность и предел; в-себе- бытие этим-де унижается и портится.
Обратно, конечное равным образом фиксируется, как не подвергшееся
отрицанию, хотя в себе ничтожное, так что оно в своей связи с бесконечным
возводится в то, что оно не {\em есть}, и тем самым ему
в противоположность его не исчезнувшему, а, наоборот, вековечному
определению придается характер бесконечности.

Фальсификация, которую проделывает рассудок касательно конечного и
бесконечного и которая состоит в том, что он фиксирует их взаимоотношение
как качественную разность и утверждает, что они в своем определении
раздельны и притом абсолютно раздельны, — эта фальсификация основывается на
забвении того, что представляет собою понятие этих моментов для самого же
рассудка. Согласно этому понятию единство конечного и бесконечного не есть
ни внешнее сведение их вместе, ни ненадлежащее, противное их определению
соединение, в котором связывались бы в себе раздельные и противоположные,
самостоятельные в отношении друг друга, сущие и, стало быть, несовместимые
[определения], а каждое есть само в себе это единство, и притом лишь как
{\em снятие} самого себя, снятие, в котором ни одно не
имеет перед другим преимущества в-себе-бытия и утвердительного наличного
бытия. Как мы показали раньше, конечность имеет бытие лишь как выход за
себя; в ней, следовательно, содержится бесконечность, другое ее самой. И
точно так же бесконечность имеет бытие лишь как выход за конечное. В ней,
следовательно, существенно содержится ее другое, и она есть, следовательно,
в ней же самой другое самой себя. Конечное не снимается бесконечным как вне
его имеющейся силой, а его собственная бесконечность состоит в том, что оно
снимает само себя. Это снятие есть, стало быть, не изменение или инобытие
вообще, не снятие [данного] {\em нечто}. То, в чем
конечное снимает себя, есть бесконечное как отрицание конечности; но
последняя сама давно уже есть лишь наличное бытие, определенное как
некоторое {\em небытие}. Следовательно, это только
{\em отрицание снимает себя в отрицании}. Точно так же
бесконечность со своей стороны определена как отрицательное конечности и
тем самым определенности вообще, — как бессодержательное (leere)
потустороннее; его снятие себя в конечном есть возвращение из
бессодержательного бегства, {\em отрицание} такого
потустороннего, которое есть некоторое
{\em отрицательное} в самом себе.

Стало быть, в обоих имеется здесь налицо одно и то же отрицание отрицания.
Но это отрицание отрицания есть {\em в себе}
соотношение с самим собою, утверждение, однако как возвращение к самому
себе,~т.~е., через {\em опосредствование}, которое есть
отрицание отрицания. Эти-то определения следует по существу иметь в виду;
второе же, что следует иметь в виду, — это то, что они в бесконечном
прогрессе также и {\em положены}, и тот способ, каким
они положены, а именно, следует иметь в виду, что они положены еще не в
своей последней истине.

Здесь, {\em во-первых}, оба, как бесконечное, так и
конечное, подвергаются отрицанию, — совершается одинаковым образом выход
как за конечное, так и за бесконечное; {\em во-вторых},
они полагаются также и как различные, каждое после другого, полагаются как
сами по себе положительные. Мы выделяем, таким образом, эти два
определения, сравнивая их между собою, точно так же, как мы в сравнении,
внешнем сравнении, отделили друг от друга два способа рассмотрения
—~рассмотрение конечного и бесконечного в их соотношении и рассмотрение
каждого из них, взятого само по себе. Но бесконечный прогресс выражает еще
нечто большее: в нем положена также и {\em связь} также
и различных, однако ближайшим образом она еще положена только как переход и
чередование. Нам следует в простом размышлении лишь разглядеть то, что
здесь на самом дело имеется.

Сначала можно брать то отрицание конечного и бесконечного, которое положено
в бесконечном прогрессе, как простое, следовательно, брать их как
внеположные, лишь следующие друг за другом. Если начнем с конечного, то
совершается выход за границу, конечное подвергается отрицанию,
потустороннее этого конечного, бесконечное, имеется следовательно теперь
налицо, но в последнем снова возникает граница; таким образом, имеется
выход за бесконечное. Это двойное снятие, однако, частью положено вообще
лишь как некоторое внешнее событие (Geschehen) и чередование моментов,
частью же еще не положено как одно единство; каждое из этих выхождений есть
особый разбег, новый акт, так что они, таким образом, лишены связи друг с
другом. — Но в бесконечном прогрессе налицо также и их соотношение.
Имеется, во-первых, конечное; засим совершается выхождение за него; это
отрицательное или потустороннее конечного есть бесконечное; в-третьих,
совершают снова выход, выходят также и за это отрицание, возникает новая
граница, опять некоторое конечное. — Это "--- полное, замыкающее само себя
движение, пришедшее к тому, что составляло начало. Возникает то же самое,
из чего исходили,~т.~е. конечное восстановлено; последнее, следовательно,
слилось с самим собою, снова нашло в своем потустороннем лишь само себя.

То же самое происходит и с бесконечным. В бесконечном, в потустороннем
данной границы, возникает лишь новая граница, которую постигает та же самая
участь "--- подвергнуться отрицанию в качестве конечного. Что, таким образом,
снова имеется, это то же самое бесконечное, которое перед тем исчезло в
новой границе. Бесконечное поэтому указанным снятием его, этой новой
границей, не выталкивается дальше за последнюю, оно не удалено ни от
конечного, — ибо последнее и состоит лишь в том, что оно переходит в
бесконечное, — ни от себя самого, ибо оно прибыло к себе.

Таким образом, оба, конечное и бесконечное, суть
{\em движение}, состоящее в возвращении к себе через
свое отрицание; они имеют бытие (sind) лишь как
{\em опосредствование} внутри себя, и утвердительное
обоих содержит в себе отрицание обоих и есть отрицание отрицания. — Они,
таким образом, суть {\em результат} и, стало быть, не
то же самое, чем они были в определении их
{\em начала}, — конечное не есть со своей стороны
некоторое {\em наличное бытие}, а бесконечное не есть
некоторое {\em наличное бытие} или
{\em в-себе-бытие} по ту сторону наличного бытия,~т.~е.
определенного как конечное. Против единства конечного и бесконечного
рассудок столь энергично восстает только потому, что он предполагает предел
и конечное, равно как и в-себе-бытие {\em вековечными};
тем самым он {\em упускает из виду} отрицание обоих,
фактически имеющиеся в бесконечном прогрессе, равно как и то, что они
встречаются в последнем лишь как моменты некоторого целого и что каждое из
них выступает наружу лишь через посредство своего противоположного, а по
существу также и через посредство снятия своего противоположного.

Когда мы в предыдущем рассматривали ближайшим образом возвращение к себе
как, с одной стороны, возвращение к себе конечного, а с другой стороны
—~возвращение к себе бесконечного, то в самом этом результате
обнаруживается некоторая неправильность, находящаяся в связи с только что
порицавшейся нами неудачностью [выражения: единство бесконечного и
конечного]: в первый раз взято {\em исходным пунктом}
конечное, а во второй раз "--- бесконечное, и только благодаря этому возникают
{\em два} результата. Но на самом деле совершенно
безразлично, какое из них мы берем как начало и, следовательно, само собою
отпадает то различие, которое породило {\em двоякость}
результата. Это равным образом положено в неограниченной по направлению
обеих сторон линии бесконечного прогресса, в котором с одинаковым
чередованием имеется каждый из моментов, и является совершенно внешним
делом, за какой из них и где именно мы возьмемся, чтобы сделать его
началом. — Они различаются в этом бесконечном прогрессе, но равным образом
одно есть лишь момент другого. Поскольку они оба, конечное и бесконечное,
сами суть моменты прогресса, они суть {\em сообща
конечное}, а поскольку они столь же сообща подвергаются отрицанию и в нем и
в результате, то этот результат как отрицание указанной конечности обоих
истинно именуется бесконечным. Их различие есть, таким образом, тот
{\em двоякий смысл}, который они оба имеют. Конечное
имеет тот двоякий смысл, что оно, во-первых, есть лишь конечное
{\em наряду} с бесконечным, которое ему противостоит, и
что оно, во-вторых, есть {\em вместе} и конечное и
противостоящее ему бесконечное. Бесконечное также имеет тот двоякий смысл,
что оно есть, во-первых, {\em один} из этих двух
моментов, — таким образом, оно есть дурное бесконечное "--- и, во-вторых, оно
есть то бесконечное, в котором оба, оно само и его другое, суть лишь
моменты. Следовательно, на самом деле бесконечное, взятое таковым, как оно
подлинно имеется, есть процесс, в котором оно понижает себя до того, чтобы
быть лишь {\em одним} из своих определений,
противостоять конечному, и, значит, быть самому лишь одним из конечных, а
затем снимает это различие себя от себя самого, превращает его в
утверждение себя и есть через это опосредствование
{\em истинно бесконечное}.

Это определение истинно бесконечного не может быть облечено в уже
отвергнутую нами {\em формулу единства} конечного и
бесконечного; {\em единство} есть абстрактное,
неподвижное саморавенство, и моменты тогда также оказываются неподвижно
сущими. Бесконечное же, подобно своим двум моментам, есть, наоборот, по
существу лишь {\em становление}, но становление, теперь
{\em далее определенное} в своих моментах. Становление
имеет сначала своими определениями абстрактное бытие и ничто; затем оно как
изменение имеет своими моментами налично сущие,~т.~е. нечто и другое;
теперь же как бесконечное оно имеет своими моментами конечное и
бесконечное, которые сами суть становящиеся.

Это бесконечное как возвращенность в себя, соотношение себя с самим собою,
есть {\em бытие}, но не лишенное определений
абстрактное бытие, ибо оно положено отрицающим отрицание; оно,
следовательно, есть также и {\em наличное бытие}, ибо
оно содержит в себе отрицание вообще и, стало быть, определенность. Оно
{\em есть} и оно {\em есть здесь},
налично, присутствует. Только дурное бесконечное есть
{\em потустороннее}, ибо оно представляет собою
{\em лишь} отрицание конечного, положенного как
{\em реальное}; таким образом, оно есть абстрактное,
первое отрицание; будучи определено {\em лишь} как
отрицательное, оно не имеет в себе утверждения
{\em наличного бытия}; фиксированное как только
отрицательное, оно даже {\em не должно} быть
{\em тут} "--- оно должно быть недостижимым. Но эта
недостижимость есть не его величие (Hoheit), а его недостаток, который
имеет свое последнее основание в том, что
{\em конечное} как таковое удерживается как
{\em сущее}. Неистинное есть недостижимое; и легко
усмотреть, что такое бесконечное неистинно. — Образом прогресса в
бесконечность служит {\em прямая линия}, только на
обеих границах которой лежит бесконечное и всегда лишь там, где ее "--- а она
есть наличное бытие "--- нет, и которая выходит
{\em вовне} к этому своему неимению наличного бытия
(Nichtdasein),~т.~е. выходит вовне в неопределенность; истинная же
бесконечность, обратно в себя загибающаяся, имеет своим образом
{\em круг}, достигшую себя линию, которая замкнута и
всецело налична, не имеет ни {\em начального}, ни
{\em конечного} пункта.

Истинная бесконечность, взятая, таким образом, вообще как
{\em наличное бытие}, положенное как
{\em утвердительное} в противоположность абстрактному
отрицанию, есть {\em реальность} в более высоком
смысле, чем та реальность, которая была {\em просто}
определена раньше; она получила здесь некоторое конкретное содержание. Не
конечное есть реальное, а бесконечное. Так и в дальнейшем реальность
определяется как сущность, понятие, идея и~т.~д. Однако при рассмотрении
более конкретного излишне повторять такие более ранние, более абстрактные
категории, как реальность, и применять их для характеристики более
конкретных определений, чем то, что они суть сами в себе. Такое повторение,
как, например, в том случае, когда говорят, что сущность "--- или идея "--- есть
реальное, вызывается тем, что для некультивированного мышления самые
абстрактные категории, например бытие, наличное бытие, реальность,
конечность, суть наиболее привычные.

Здесь повторение категории реальности вызывается более определенным поводом,
так как то отрицание, в отношении которого она есть утвердительное, есть
здесь отрицание отрицания, и, стало быть, она сама противополагается той
реальности, которая есть конечное наличное бытие. — Отрицание определено,
таким образом, как идеальность;
идеализованное\footnote{
{\em\bfseries Идеальное}
имеет дальнейшее, более определенное значение (прекрасного и
того, что ведет к последнему), чем
{\em\bfseries идеализованное}; первому
здесь еще не место; поэтому мы здесь употребляем выражение:
«{\em\bfseries идеализованное}». В
отношении к реальности это различие в словоупотреблении не имеет место;
«das Reelle» и «das Reale» употребляются приблизительно в одном и том же
значении. Выяснение оттенков этих двух выражений в их отличии друг от друга
не представляет интереса.}
есть конечное, как оно есть в истинном
бесконечном "--- как некоторое определение, содержание, которое различено, но
не есть нечто {\em самостоятельно сущее}, а имеет бытие
как {\em момент}. Идеальность имеет этот более
конкретный смысл, который не вполне выражен отрицанием конечного наличного
бытия. Но в отношении реальности и идеальности противоположность между
конечным и бесконечным понимают так, что конечное считается реальным, а
бесконечное идеализованным; как и в дальнейшем, понятие рассматривается как
некоторое идеализованное и притом как некоторое
{\em лишь} идеализованное, наличное же бытие вообще
рассматривается, наоборот, как реальное. При таком понимании, разумеется,
нисколько не поможет то, что мы имеем для обозначения указанного
конкретного определения отрицания особое слово «идеализованное»; в этой
противоположности снова возвращаются к односторонности абстрактного
отрицания, которая присуща дурному бесконечному, и упорно настаивают на
утвердительном наличном бытии конечного.

\subsection*{Переход}
Идеальность может быть названа {\em качеством}
бесконечности; по существу она есть процесс
{\em становления} и тем самым некоторый переход,
подобный переходу становления в наличное бытие, и теперь следует указать
характер этого перехода. Как снятие конечности,~т.~е. и конечности как
таковой, и равным образом лишь противостоящей ей, лишь отрицательной
бесконечности, это возвращение в себя есть
{\em соотношение} с самим собой,
{\em бытие}. Так как в этом бытии есть отрицание, то
оно есть {\em наличное бытие}, но так как, далее, это
отрицание есть по существу отрицание отрицания, соотносящееся с собою
отрицание, то оно есть то наличное бытие, которое именуется
{\em для-себя-бытием}.

\subsection*{Примечание 1. Бесконечный прогресс}

Бесконечное "--- взятое в обычном смысле, в смысле дурного бесконечного "--- и
{\em прогресс в бесконечность} как
{\em долженствование} суть выражение
{\em противоречия}, которое выдает само себя за
{\em разрешение} и за последнее слово. Это бесконечное
есть первое возвышение чувственного представления над конечным, возвышение
его в область мысли, имеющей, однако, своим содержанием лишь ничто, некое
{\em нарочито} положенное как не-сущее, — есть бегство
за пределы ограниченного, не концентрирующееся на самом себе и не умеющее
возвратить отрицательное к положительному. Эта
{\em незавершенная рефлексия} имеет перед собою
полностью оба определения истинно бесконечного:
{\em противоположность} между конечным и бесконечным и
{\em единство} конечного и бесконечного, но
{\em не сводит вместе этих двух мыслей}. Одна мысль
неразлучно приводит за собою другую, эта же рефлексия лишь
{\em чередует} их. Изображение этого чередования,
бесконечный прогресс, появляется повсюду, где не хотят выбраться из
противоречия {\em единства} двух определений и их
{\em противоположности}. Конечное есть снятие самого
себя, оно заключает в себе свое отрицание, бесконечность: это "--- их
{\em единство}. Затем совершается выход
{\em вовне} за конечное к бесконечному, как к
потустороннему конечного: это "--- их {\em разъединение}.
Но за бесконечным есть другое конечное; выход за конечное, бесконечность,
содержит в себе конечность: это "--- их {\em единство}. Но
это конечное есть также некое отрицание бесконечного: это "--- их
{\em разъединение}. и~т.~д. — Так, например, в
причинном отношении причина и действие нераздельны: причина, которая не
производила бы никакого действия, не была бы причиной, равно как действие,
которое не имело бы причины, уже не было бы действием. Это отношение
приводит таким образом к бесконечному прогрессу
{\em причин и действий}. Нечто определено как причина,
но последняя как конечное (а конечна она, собственно говоря, как раз
вследствие ее отделения от действия) сама имеет причину,~т.~е. она есть
также действие; следовательно, {\em то самое}, что
раньше было определено как причина, определено также и как действие; это
—~{\em единство} причины и действия. Но определяемое
теперь как действие, опять-таки имеет некоторую причину,~т.~е. причину
следует {\em отделить} от ее действия и положить как
отличное от него нечто. Эта новая причина сама однако есть только действие;
это "--- {\em единство} причины и действия. Она имеет
своей причиной некоторое другое; это
—~{\em разъединение} сказанных двух определений,
и~т.~д. до {\em бесконечности}.

Этому прогрессу можно, таким образом, придать более своеобразную форму.
Выдвигается утверждение, что конечное и бесконечное суть одно единство; это
ложное утверждение должно быть исправлено противоположным утверждением: они
всецело разны и противоположны друг другу. Это утверждение должно быть
вновь исправлено утверждением о их единстве в том смысле, что они
неразделимы, что в одном определении заключено другое, и~т.~д., до
бесконечности. — Легко исполнимое требование, предъявляемое к тому, кто
хочет проникнуть в природу бесконечного, заключается в том, что он должен
сознавать, что бесконечный прогресс, развитое бесконечное рассудка, носит
характер {\em чередования} обоих определений,
{\em чередования единства} и
{\em раздельности} обоих моментов, а затем должен он
иметь дальнейшее сознание того, что это единство и эта раздельность сами
нераздельны.

Разрешением этого противоречия служит не признание
{\em одинаковой правильности} и одинаковой
неправильности обоих утверждений "--- это будет лишь другой формой остающегося
противоречия, — а {\em идеальность} обоих определений,
в каковой они в своем различии как взаимные отрицания суть лишь
{\em моменты}; вышеуказанное монотонное чередование
есть фактически отрицание как {\em единства}, так и
{\em раздельности} их. В нем (в этом чередовании)
фактически имеется также и показанное нами выше, а именно: конечное, выходя
за себя, впадает в бесконечное, но оно также и выходит за последнее,
находит себя порожденным снова, а, стало быть, сливается в этом выхождении
за себя лишь с самим собою, и это равным образом происходит и с
бесконечным, так что из этого отрицания отрицания получается
{\em утверждение}, каковой результат, стало быть,
оказывается их истиной и изначальным значением. Таким образом, в этом бытии
как {\em идеальности} отличных друг от друга
[определений] противоречие не исчезло абстрактно, а разрешено и примирено,
и мысли оказываются не только полными, но также и
{\em сведенными вместе}. Природа спекулятивного
мышления являет себя здесь, как на вполне развитом примере, в своем
определенном виде; она состоит единственно в схватывании противоположных
моментов в их единстве. Так как каждый из них являет себя в себе же, и
притом фактически, имеющим в самом себе свою противоположность и в ней
сливающимся с самим собою, то утвердительная истина есть это движущееся
внутри себя единство, объединение обеих мыслей, их бесконечность, — есть
соотношение с самим собою, не непосредственное, а бесконечное.

Многие, уже несколько более освоившиеся с философией, часто полагали
сущность и задачу философии в разрешении вопроса,
{\em каким образом бесконечное выходит из себя и
приходит к конечности}. Это, полагают они, не может быть сделано
{\em постижимым}. То бесконечное, к понятию которого мы
пришли, получит дальнейшие определения в ходе последующего изложения, и на
нем (на этом бесконечном) требуемое этими людьми будет показано во всем
многообразии форм, а именно, будет показано, {\em каким
образом} это бесконечное, если угодно так выражаться,
{\em приходит к конечности}. Здесь же мы рассматриваем
этот вопрос лишь в его непосредственности и имея в виду ранее рассмотренный
смысл, в котором обыкновенно понимают слово «бесконечное».

От ответа на этот вопрос зависит, как утверждают, вообще решение вопроса,
{\em существует ли философия}, и, делая вид, что еще
видят в нем вопрос, ждущий своего разрешения, задающие его полагают вместе
с тем, что они обладают в самом этом вопросе некоторого рода каверзным
вопросом, неодолимым талисманом, служащим верной и обеспечивающей защитой
от утвердительного ответа и тем самым от философии и достижения ее. — И
относительно других предметов также требуется известное развитие для того,
чтобы уметь {\em задавать вопросы}; тем паче оно
требуется в отношении философских предметов, чтобы получить другой ответ,
чем тот, что вопрос никуда не годится.

При задавании таких вопросов взывают обыкновенно к чувству справедливости,
говоря, что дело не в том, какие употребляют слова, а что, выражают ли
вопрос так или этак, — все равно понятно, о чем идет речь. Употребление в
этом вопросе выражений, заимствованных из области чувственного
представления, как, например, «выходить» и~т.~п., возбуждает подозрение,
что он возник на почве обычного представления и что для ответа на него
также ожидают представлений, имеющих хождение в обыденной жизни, и образов
чувственной метафоры.

Если вместо бесконечного взять бытие вообще, то кажется, что легче постичь
процесс {\em определения бытия}, наличие в нем
некоторого отрицания или конечности. Хотя само бытие есть неопределенное, в
нем, однако, непосредственно не выражено, что оно есть противоположность
определенного. Напротив, бесконечное содержит эту мысль в явно выраженном
виде; оно есть {\em не}{}-конечное. Единство конечного
и бесконечного кажется, следовательно, непосредственно исключенным; поэтому
незавершенная рефлексия упорнее всего не приемлет этого единства. Но мы уже
показали, да и без дальнейшего углубления в определение конечного и
бесконечного непосредственно ясно, что бесконечное в том смысле, в котором
его берет сказанная незавершенная рефлексия, — а именно в смысле чего-то
противостоящего конечному, — как раз в силу того, что оно противостоит
последнему, имеет в нем свое другое и уже потому ограничено и само конечно,
есть дурное бесконечное. Поэтому ответ на вопрос,
{\em каким образом бесконечное становится конечным},
заключается в том, что {\em нет} такого бесконечного,
которое {\em сначала} бесконечно и которому только
потом приходится стать конечным, выйти к конечности, но что оно уже само по
себе столь же конечно, сколь и бесконечно. Так как вопрос принимает, что, с
одной стороны, бесконечное стоит особо и что, с другой стороны, конечное,
которое вышло из него, чтобы стать разлученным с ним, или которое, откуда
бы оно ни пришло, обособлено и отделено от него, — что такое конечное
поистине реально, то следовало бы скорее сказать, что
{\em непостижимым} является именно эта разлученность.
Ни такое конечное, ни такое бесконечное не имеют истинности, а неистинное
{\em непостижимо}. Но нужно также сказать, что они
{\em постижимы}. Рассмотрение их, даже взятых так, как
они даны в представлении, устанавливающее, что в каждом заключено
определение другого, простое усмотрение этой их нераздельности означает
постижение их: {\em эта нераздельность} есть
{\em их понятие}. — Напротив, принимая
{\em самостоятельность} вышесказанных конечного и
бесконечного, этот вопрос выставляет неистинное содержание и уже заключает
в себе неистинное соотношение между ними. На него поэтому не следует
отвечать, а следует, наоборот, отринуть содержащиеся в нем ложные
предпосылки,~т.~е. следует отринуть самый вопрос. Вопрос об истинности
вышесказанных конечного и бесконечного изменяет точку зрения на них, и это
изменение переносит на самый первый вопрос то смущение, которое он хотел
вызвать. Наш {\em вопрос} оказывается чем-то
{\em новым} для рефлексии, являющейся источником
первого вопроса, так как в таком рефлектировании нет того спекулятивного
устремления, которое само по себе и прежде, чем соотносить между собой
определения, добивается познать, представляют ли из себя эти определения,
взятые так, как они предпосланы, нечто истинное. Но поскольку познана
неистинность вышеуказанного абстрактного бесконечного, а также и
неистинность долженствующего равным образом продолжать стоять на своей
стороне конечного, мы должны сказать относительно этого выхождения
конечного из бесконечного, что бесконечное
{\em выходит} к конечному потому, что оно, если его
понимают, как абстрактное единство, не имеет в себе истинности, не имеет
устойчивого существования, равно как и, наоборот, конечное
{\em входит} в бесконечное вследствие той же причины,
вследствие своей ничтожности. Или, правильнее будет сказать, что
бесконечное извечно выходит из себя к конечности, что его (точно так же,
как и чистого {\em бытия}) безоговорочно
{\em нет} самого по себе, без его другого
{\em в нем же самом}.

Тот вопрос, каким образом бесконечное выходит из себя к конечному, может
содержать еще ту дальнейшую предпосылку, что бесконечное
{\em в себе} включает в себя конечное и, стало быть
есть в себе единство самого себя и своего другого, так что трудность
состоит по существу в их {\em разъединении}, которое
противоречит принятому в качестве предпосылки единству обоих. В этой
предпосылке та противоположность [обоих определений], за которую крепко
держатся, получает только другой вид; {\em единство} и
{\em различение} разлучаются и изолируются друг от
друга. Но если берут первое не как абстрактное, неопределенное единство, а
(как в указанной предпосылке) уже как определенное единство
{\em конечного} и
{\em бесконечного}, то здесь уже имеется также и
различение обоих, — различение, которое, таким образом, вместе с тем не
есть предоставление им быть раздельно самостоятельными, а оставляет их в
единстве, как {\em идеализованные}. Это
{\em единство} конечного и бесконечного и их
{\em различение} суть та же самая нераздельность, что
конечность и бесконечность.

\subsection*{Примечание 2. Идеализм}

Положение, гласящее, что {\em конечное идеализованно},
составляет {\em идеализм}. Философский идеализм состоит
не в чем другом, как в том, что конечное не признается истинно сущим.
Всякая философия есть по существу идеализм или по крайней мере имеет его
своим принципом, и вопрос затем заключается лишь в том, насколько этот
принцип действительно проведен. Философия есть столь же идеализм, как и
религия, ибо религия столь же мало признает конечность истинным бытием,
некоторым окончательным, абсолютным или, иначе говоря, некоторым
неположенным, несотворенным, вечным. Противоположение идеалистической и
реалистической философии не имеет поэтому никакого значения. Философия,
которая приписывала бы конечному существованию (Dasein) как таковому
истинное, последнее, абсолютное бытие, не заслуживала бы названия
философии. Первоначала древних или новых философских учений "--- вода или
материя или атомы "--- суть {\em мысли}, всеобщее,
идеализованное, а не вещи, как мы их непосредственно преднаходим,~т.~е.
вещи в чувственной единичности; даже фалесовская вода тоже не есть такая
вещь; ибо, хотя она есть также и эмпирическая вода, она кроме того есть
вместе с тем «{\em в себе}» или
{\em сущность} всех других вещей, и эти последние суть
не самостоятельные, не обоснованные внутри себя, а
{\em положены} проистекающими из другого, из
воды,~т.~е. суть идеализованные. Назвав только что принцип, всеобщее,
{\em идеализованным}, как еще с большим правом должны
быть названы {\em идеализованными} понятие, идея, дух,
и, говоря затем, что единичные чувственные вещи в свою очередь имеют бытие
как {\em идеализованные} в принципе, в понятии, а еще
больше "--- в духе, как снятые в них, мы должны предварительно обратить
внимание читателя на ту же двусторонность, которая обнаружилась также и при
трактовании бесконечного, а именно, что {\em то}
идеализованным оказывается конкретное, истинно-сущее,
{\em то} идеализованным оказываются равным образом его
моменты в том смысле, что они сняты в нем; на самом же деле имеется только
единое конкретное целое, от которого моменты неотделимы.

Под идеализованным обыкновенно разумеют форму
{\em представления} и
{\em идеализованным} называют то, что есть вообще
{\em в} моем представлении или
{\em в} понятии, {\em в} идее,
{\em в} воображении и~т.~д., так что идеализованное
признается вообще также и фантазиями "--- представлениями, которые, как
предполагается, не только отличаются от реального, но по существу и
{\em не} должны быть реальными. Дух в самом деле есть
вообще настоящий {\em идеалист}; в нем, уже как в
ощущающем и представляющем, а еще более, поскольку он мыслит и постигает в
понятиях, содержание имеет бытие не как так называемое
{\em реальное существование} (Dasein) "--- в простоте «я»
такого рода внешнее бытие есть лишь снятое, оно есть
{\em для меня}, оно
{\em идеализованно} во мне. Этот субъективный идеализм,
высказывается ли он и устанавливается как бессознательный идеализм сознания
вообще или сознательно как принцип, имеет в виду лишь ту
{\em форму} представления, по которой некоторое
содержание есть {\em мое} содержание. Систематический
субъективный идеализм утверждает относительно этой формы, что она есть
единственно истинная, исключающая форму объективности или реальности, форму
{\em внешнего существования} сказанного содержания.
Такой идеализм формален, так как он не обращает внимания на
{\em содержание} представления или мышления, каковое
содержание может при этом оставаться в представлении или мышлении всецело в
своей конечности. С принятием такого идеализма ничего не теряется, как
потому, что сохраняется реальность этого конечного содержания, наполненное
конечностью существование, так и потому, что, поскольку абстрагируются от
него, оно {\em само по себе} не должно иметь для нас
никакой важности; с принятием этого идеализма ничего также и не
выигрывается, именно потому, что ничего не теряется, так как «я»,
представление, дух остается наполненным тем же конечным содержанием.
Противоположность формы субъективности и объективности есть, разумеется,
один из видов конечности. Но {\em содержание}, как оно
воспринимается в ощущение, созерцание или же в более абстрактную стихию
представления, мышления, содержит в себе массу видов конечности, которые с
исключением лишь сказанного одного вида, формы субъективного и
объективного, еще совершенно не устранены и тем более не отпадают сами
собою.

\bigskip


\chapter*{Третья глава. Для-себя-бытие.}
В {\em для-себя-бытии качественное бытие завершено}; оно
есть бесконечное бытие. Бытие, которым мы начали, лишено определений.
Наличное бытие есть снятое, но лишь непосредственно снятое бытие. Оно,
таким образом, содержит в себе пока что лишь первое отрицание, которое само
непосредственно. Бытие, правда, также сохранено, и в наличном бытии оба
(т.~е. бытие и отрицание) объединены в простое единство, но как раз поэтому
они сами в себе еще {\em неравны} друг другу и их
единство еще {\em не положено}. Наличное бытие есть
поэтому сфера диферентности, дуализма, область конечности. Определенность
есть определенность как таковая, некая относительная, а не абсолютная
определяемость. В для-себя-бытии различие между бытием и определенностью
или отрицанием положено и примирено; качество, инобытие, граница, как и
реальность, в-себе-бытие, долженствование и~т.~д. суть несовершенные
внедрения отрицания в бытие, в каковом внедрении еще лежит в основании
различие обоих. Но так как в конечности отрицание перешло в бесконечность,
в {\em положенное} отрицание отрицания, то оно есть
простое соотношение с собою, есть, следовательно, в самом себе примирение с
бытием "--- {\em абсолютная определенность}.

Для-себя-бытие есть, {\em во-первых}, непосредственно
для-себя-сущее, {\em одно}.

{\em Во-вторых}, одно переходит во
{\em множество одних} "--- в
{\em отталкивание}, каковое инобытие одного снимается в
идеальности последнего; это "--- {\em притяжение}.

{\em В-третьих}, оно есть взаимоопределение отталкивания
и притяжения, в котором они погружаются вместе в равновесие, и качество,
доведшее себя в для-себя-бытии до последнего заострения, переходит в
{\em количество}.

\section*{А. Для-себя-бытие как таковое}
Общее понятие для-себя-бытия получилось. Теперь дело идет только о том,
чтобы доказать, что этому понятию соответствует представление, которое мы
соединяем с выражением «для-себя-бытие», дабы мы имели право употреблять
его для обозначения сказанного понятия. И, повидимому, это так; мы говорим,
что нечто есть для себя, поскольку оно снимает инобытие, свое соотношение и
свою общность с другим, оттолкнуло их от себя, абстрагировалось от них.
Другое имеет для него бытие лишь как некое снятое, как
{\em его момент}. Для-себя-бытие состоит в
{\em таком} выходе за предел, за свое инобытие, что оно
как это отрицание есть бесконечное {\em возвращение} в
себя. — Сознание уже как таковое содержит в себе определение
для-себя-бытия, так как оно представляет себе тот предмет, который оно
ощущает, созерцает,~т.~е. имеет его содержание
{\em внутри себя}, каковое содержание, таким образом,
дано как {\em идеализованное}; в самом своем созерцании
и вообще в своей переплетенности со своим отрицательным, с другим, оно
{\em находится у самого себя}. Для-себя-бытие есть
полемическое, отрицательное отношение к ограничивающему другому и через это
отрицание последнего "--- рефлектированность в себя, хотя
{\em наряду} с этим возвращением сознания в себя и
идеальностью предмета еще сохранилась {\em также и его
реальность}, так как он {\em вместе с тем} знаем, как
некое внешнее наличное бытие. Сознание есть, таким образом,
{\em являющееся} или, иначе говоря, есть дуализм,
заключающийся в том, что оно, с одной стороны, знает о некотором другом для
него внешнем предмете, а с другой стороны, есть для себя, имеет предмет в
себе идеализованным, находится не только у такового другого, а в нем
находится также и у самого себя. Напротив,
{\em самосознание} есть
{\em для-себя-бытие} как
{\em исполненное} и
{\em положенное}; вышеуказанная сторона соотношения с
некоторым {\em другим}, с внешним предметом устранена.
Самосознание есть, таким образом, ближайший пример наличия бесконечности,
правда, все еще абстрактной бесконечности, которая, однако, вместе с тем
носит характер определения, совершенно иным образом конкретного, чем
для-себя-бытие вообще, бесконечность которого еще всецело имеет
исключительно лишь качественную определенность.

\subsection*{a) Наличное бытие и для-себя-бытие}
Для-себя-бытие есть, как мы уже указали, погрузившаяся в простое бытие
бесконечность; оно есть наличное бытие, поскольку отрицательная природа
бесконечности, которая есть отрицание отрицания в положенной теперь форме
непосредственности бытия, дана лишь как отрицание вообще, как простая
качественная определенность. Но бытие в такой определенности, в которой оно
есть наличное бытие, также и отлично "--- это сразу явно "--- от самого
для-себя-бытия, которое есть для-себя-бытие лишь постольку, поскольку его
определенность есть сказанное бесконечное. Однако наличное бытие есть
вместе с тем момент самого для-себя-бытия, ибо последнее содержит в себе,
во всяком случае, также и бытие, обремененное отрицанием. Таким образом
определенность, которая в наличном бытии как таковом есть некоторое
{\em другое} и
{\em бытие-для-другого}, повернута обратно в
бесконечное единство для-себя-бытия, и момент наличного бытия имеется в
для-себя-бытии как {\em бытие-для-одного}.

\subsection*{b) Бытие-для-одного}
Этот момент выражает тот способ, каким конечное есть в своем единстве с
бесконечным или, иначе говоря, имеет бытие как идеализованное.
Для-себя-бытие имеет отрицание не {\em в себе}, как
некоторую определенность или границу и, значит, также и не как соотношение
с некоторым другим, чем оно, наличным бытием. Обозначив этот момент как
{\em бытие-для-одного}, следует сказать, что еще нет
ничего, для которого он был бы, — еще нет того одного, момент которого он
составлял бы. И в самом деле, такого рода одно еще не фиксировано в
для-себя-бытии; то, для чего нечто (а здесь нет никакого нечто) было бы,
то, что вообще должно было бы быть другой стороной, есть равным образом
момент, само есть лишь бытие-для-одного, еще не есть одно. — Следовательно,
еще имеется неразличенность тех двух сторон, которые могут предноситься
нашему умственному взору в бытии-для-одного. Есть лишь
{\em одно} бытие-для-другого, и так как есть лишь
{\em одно} бытие-для-другого, то последнее есть также
лишь бытие-для-одного; оно есть лишь {\em одна}
идеальность того, для чего или в чем некоторое определение должно было бы
быть как момент, и того, что должно было бы быть в нем моментом. Таким
образом, {\em для-одного-бытие} и
{\em для-себя-бытие} не составляют истинных
определенностей в отношении друг друга. Поскольку мы принимаем на одно
мгновенье, что имеется различие, и говорим здесь о некотором
{\em для-себя-сущем}, то само для-себя-сущее как
снятость инобытия соотносится с собою как со снятым другим, стало быть,
есть {\em для-одного}; оно соотносится в своем другом
лишь с собою. Идеализованное необходимо есть
{\em для-одного}, но оно не есть для некоторого
{\em другого}; то одно, для которого оно есть, есть
лишь само же оно. — Следовательно, «я», дух вообще или бог суть
идеализованные, потому что они бесконечны, но они в своей идеальности, как
для-себя-сущие, не разнятся от того, что есть для-одного. Ибо, таким
образом, они были бы лишь непосредственными или, ближе, наличным бытием и
неким бытием-для-другого, потому что то, что есть для них, было бы не они
сами, а некоторое другое, если бы им не был присущ момент бытия-для-одного.
Поэтому бог есть {\em для себя}, поскольку сам он есть
то, что есть {\em для него}.

Для-себя-бытие и для-одного-бытие суть, следовательно, не разные значения
идеальности, а существенные, неразделимые ее моменты.

\subsection*{Примечание. Выражение: Was für eines?[ссылка!]}

Кажущееся сперва странным выражение нашего языка при вопросе о качестве, was
für ein Ding etwas sei (по-русски: что это за вещь, но буквально это
выражение означает: что есть нечто для одной вещи, и эту двусмысленность
данного выражения использует здесь Гегель. — \textit{Перев.}), выделяет
рассматриваемый здесь момент в его рефлексии в себя. Это выражение
идеалистично в своем происхождении, так как оно не спрашивает, что есть эта
вещь {\em А для другой вещи} В, не спрашивает, что есть
этот человек для другого человека, а спрашивает, что
{\em это для одной вещи, для одного человека} (т.~е.
что {\em это за вещь}, {\em за
человек}. — \textit{Перев.}), так что это бытие-для-одного вместе с тем
возвратилось в самое эту вещь, в самого этого человека, и то,
{\em что есть}, и то, {\em для
чего} оно есть, есть одно и то же; мы видим здесь тождество, каковым должна
рассматриваться также и идеальность.

Идеальность присуща ближайшим образом снятым определениям, как отличным от
того, {\em в чем} они сняты, каковое, напротив, можно
брать как реальное. Однако таким образом идеализованное оказывается опять
одним из моментов, а реальное "--- другим; но идеальность заключается в том,
что оба определения одинаково суть только для
{\em одного} и считаются лишь за
{\em одно}, каковая одна идеальность тем самым есть
неразличимо реальность. В этом смысле самосознание, дух, бог есть
идеализованное как бесконечное соотношение чисто с собою, — «я» есть для
«я», оба суть одно и то же; «я» названо два раза, но каждое из этих двух
есть лишь для-одного, идеализованно; дух есть лишь для духа, бог есть лишь
для бога и лишь это единство есть бог, бог как дух. — Но самосознание как
сознание вступает в различие между {\em собою} и
некоторым {\em другим} или, иными словами, между своей
идеальностью, в которой оно есть представляющее, и своей реальностью,
поскольку его представление имеет некоторое определенное содержание,
которое имеет еще ту сторону, что оно знаемо как неснятое отрицательное,
как наличное бытие. Однако называть мысль, дух, бога
{\em лишь} идеализованными, значит, исходить из той
точки зрения, на которой конечное наличное бытие представляется реальным, а
идеализованное или бытие-для-одного имеет только односторонний смысл.

В одном из предшествующих примечаний мы указали принцип идеализма и сказали,
что, зная принцип, важно знать относительно того или другого философского
учения, насколько последовательно оно проводит этот принцип. О характере
проведения указанного принципа в отношении той категории, которая нас
сейчас занимает, можно сделать дальнейшее замечание. Последовательность в
проведении этого принципа зависит ближайшим образом от того, остается ли в
данном философском учении самостоятельно существовать наряду с
для-себя-бытием еще и конечное бытие, а затем также и от того, положен ли
уже в самом бесконечном момент «{\em для-одного}»
—~отношение идеализованного к себе как к идеализованному. Так например
элеатское бытие или спинозовская субстанция суть лишь абстрактное отрицание
всякой определенности, причем в них самих идеальность еще не положена. У
{\em Спинозы}, как мы об этом скажем ниже,
бесконечность есть лишь абсолютное {\em утверждение}
некоторой вещи и, следовательно, лишь неподвижное единство; субстанция
поэтому не доходит даже до определения для-себя-бытия и тем менее до
определения субъекта и духа. Идеализм благородного
{\em Мальбранша} более развернут внутри себя; он
содержит в себе следующие основные мысли: так как бог заключает в себе все
вечные истины, идеи и совершенства всех вещей, так что они принадлежат лишь
{\em ему}, то мы их видим только в нем; бог вызывает в
нас наши ощущения предметов посредством действия, в котором нет ничего
чувственного, причем мы воображаем себе, что получаем от предмета не только
его идею, представляющую его сущность, но также и ощущение его
существования («Разыскание истины», Разъяснение относительно природы идей
и~т.~д.). Стало быть, не только вечные истины и идеи (сущности) вещей, но и
их существование есть существование в боге, идеализованное, а не
действительное существование, хотя, как наши предметы, они суть только
{\em для-одного}. Недостающий в спинозизме момент
развернутого и конкретного идеализма здесь имеется налицо, так как
абсолютная идеальность определена как знание. Как ни чист и ни глубок этот
идеализм, все же указанные отношения частью содержат еще в себе много
неопределенного для мысли, частью же их содержание сразу же оказывается
совершенно конкретным (грех и спасение и~т.~д. сразу же появляются в этой
философии). Логическое определение бесконечности, которое должно было бы
быть основой этого идеализма, не разработано самостоятельно, и, таким
образом, этот возвышенный и наполненный идеализм есть, правда, продукт
чистого спекулятивного ума, но еще не чистого спекулятивного, единственно
лишь дающего истинное обоснование мышления.

{\em Лейбницевский} идеализм движется в большей мере в
рамках абстрактного понятия. — {\em Лейбницевская
представляющая} сущность, {\em монада}, по существу
идеализованна. Представление есть некое для-себя-бытие, в котором
определенности суть не границы и, следовательно, не некоторое наличное
бытие, а лишь моменты. Представливание есть, правда, также и некое более
конкретное определение, но здесь оно не имеет никакого иного значения,
кроме значения идеальности, ибо и то, что вообще лишено сознания, есть у
Лейбница представляющее, перципирующее. В этой системе инобытие, стало
быть, снято; дух и тело или вообще монады суть не другие друг для друга,
они не ограничивают друг друга, не воздействуют друг на друга; здесь вообще
отпадают все те отношения, в основании которых лежит некоторое наличное
бытие. Многообразие есть лишь идеализованное и внутреннее, монады остаются
в нем лишь соотнесенными с самими собою, изменения развиваются внутри
монады и не суть соотношения последней с другими. То, что со стороны
реального определения берется нами как некоторое налично сущее соотношение
монад друг с другом, есть независимое, лишь
{\em одновременное} становление, заключенное в
для-себя-бытии каждой из них. — То обстоятельство, что существуют
{\em многие монады}, что их, следовательно, определяют
также и как другие, не касается самих монад; это "--- имеющее место вне них
размышление некоторого третьего; {\em в самих себе} они
не суть {\em другие по отношению друг к другу};
для-себя-бытие сохраняется чисто, без примеси некоторого находящегося
{\em рядом} существования. — Но тем самым явствует
вместе с тем и незавершенность этой системы. Монады суть представляющие
таким образом лишь {\em в себе} или
{\em в боге} как монаде монад, или
{\em также в системе}. Инобытие также имеется, где бы
оно ни имело место, в самом ли представлении, или как бы мы ни определили
то третье, которое рассматривает их как другие, как многие. Множественность
их существования лишь исключена и притом только на мгновение, монады лишь
путем абстрагирования положены как такие, которые суть не-другие. Если
некое третье полагает их инобытие, то некое третье также и снимает их
инобытие; но все это {\em движение, которое делает их
идеализованными}, совершается вне их. Однако так как нам могут напомнить о
том, что это движение мысли само имеет место лишь внутри некоторой
представляющей монады, то мы должны указать вместе с тем на то, что как раз
{\em содержание} такого мышления
{\em само в себе внешне себе}. Переход от единства
абсолютной идеальности (монады монад) к категории абстрактного (лишенного
соотношений) {\em множества} наличного бытия
совершается непосредственно, не путем постижения в понятии (совершается
посредством представления о сотворении), и обратный переход от этого
множества к тому единству совершается столь же абстрактно. Идеальность,
представливание вообще, остается чем-то формальным, равно как формальным
остается и то представливание, которое интенсифицировано до сознания. Как в
вышеприведенном замечании
Лейбница[ссылка!] о магнитной игле, которая, если
бы обладала сознанием, рассматривала бы свое направление к северу как
определение своей свободы, сознание мыслится лишь как односторонняя форма,
безразличная к своему определению и содержанию, так и идеальность в монадах
есть лишь некая остающаяся внешней для множественности форма. Идеальность,
согласно Лейбницу, имманентна им, их природа состоит в представливании; но
способ их поведения есть, с одной стороны, их гармония, не имеющая места в
их наличном бытии, — она поэтому предустановлена; с другой стороны, это их
{\em наличное бытие} не понимается Лейбницем ни как
бытие-для-другого, ни еще шире как идеальность, а определено лишь как
абстрактная множественность. Идеальность множественности и дальнейшее ее
определение в гармонию не имманентно самой этой множественности и не
принадлежит ей самой.

Другого рода идеализм, как, например, кантовский и фихтевский, не выходит за
пределы {\em долженствования} или
{\em бесконечного прогресса} и застревает в дуализме
наличного бытия и для-себя-бытия. В этих системах вещь-в-себе или
бесконечный толчок, правда, вступает непосредственно в «я» и становится
лишь неким «{\em для последнего}» (для «я»); однако
толчок этот исходит от некоторого свободного инобытия, которое пребывает во
веки веков как отрицательное в-себе-бытие. Поэтому «я», правда,
определяется в этого рода идеализме как идеализованное, как для-себя-сущее,
как бесконечное соотношение с собою; однако
{\em для-одного-бытие} не завершено до исчезновения
того потустороннего или направления в потустороннее.

\subsection*{c) Одно}
Для-себя-бытие есть простое единство самого себя и своего момента,
бытия-для-одного. Имеется лишь одно определение "--- свойственное снятию
соотношение с самим собою. {\em Моменты}
для-себя-бытия, слившись, погрузились в {\em отсутствие
различий}, которое есть непосредственность или бытие, но
{\em непосредственность}, основанная на отрицании,
положенном как ее определение. Для-себя-бытие есть, таким образом,
{\em для-себя-сущее}, и ввиду того, что в этой
непосредственности исчезает его внутреннее значение, оно есть совершенно
абстрактная граница самого себя "--- {\em одно}.

Можно здесь наперед обратить внимание читателя на ту трудность, которая
заключается в последующем изложении {\em развития}
одного, и на причину этой трудности. {\em Моменты},
составляющие {\em понятие} одного как для-себя-бытия, в
нем {\em разъединяются} (treten auseinander). Эти
моменты таковы: (1) отрицание вообще; (2) два отрицания, (3) стало быть,
отрицания двух, которые суть {\em одно и то же} и (4)
которые безоговорочно противоположны; (5) соотношение с собою, тождество
как таковое; (6) {\em отрицательное} соотношение и,
однако, с {\em самим собою}. Эти моменты здесь
разъединяются вследствие того, что в для-себя-бытии как сущем-для-себя
привходит форма {\em непосредственности},
{\em бытия}; благодаря этой непосредственности каждый
момент {\em полагается}, как
{\em некое особое (eigene) сущее определение}; и тем не
менее, они также и нераздельны. Приходится, следовательно, о каждом
определении высказывать также и ему противоположное; это-то противоречие
при абстрактном {\em характере моментов} и составляет
указанную трудность.

\section*{B. Единое и многое}
Одно есть простое соотношение для-себя-бытия с самим собою, в каковом
соотношении моменты этого для-себя-бытия совпали, и потому в сказанном
соотношении для-себя-бытие имеет форму
{\em непосредственности}, и его моменты становятся
поэтому теперь {\em налично сущими}.

Как соотношение {\em отрицательного} с собою, одно есть
процесс определения, а как соотношение {\em с собою}
оно есть бесконечное {\em самоопределение}. Но
вследствие теперешней непосредственности эти
{\em различия} уже более не положены лишь как моменты
{\em одного и того же} самоопределения, а положены
вместе с тем также и как {\em сущие}.
{\em Идеальность} для-себя-бытия как тотальность
превращается, таким образом, во-первых, в
{\em реальность} и притом в наиабстрактнейшую,
наипрочнейшую, как {\em одно}. В
{\em одном} для-себя-бытие есть
{\em положенное} единство бытия и наличного бытия как
абсолютное соединение соотношения с другим и соотношения с собою; но кроме
того появляется также и определенность бытия в
{\em противоположность} определению
{\em бесконечного отрицания}, в противоположность
самоопределению, так что то, что одно есть {\em в
себе}, оно есть теперь только {\em в нем} и, стало
быть, отрицательное есть некое другое как отличное от него. То, что
обнаруживает себя {\em имеющимся} как отличное от него,
есть его собственное самоопределение; его единство с собою, взятое как
отличное от него, понижено до {\em соотношения} и, как
{\em отрицательное} единство, оно есть отрицание самого
себя как некоторого {\em другого},
{\em исключение} одного как некоторого
{\em другого} из себя, из одного.

\subsection*{a) Одно в нём самом}
В нем самом одно вообще {\em есть}; это его бытие есть
не наличное бытие, не определенность как соотношение с другим, не характер;
оно есть состоявшееся отрицание этого круга категорий. Одно, следовательно,
не способно становиться другим; оно {\em неизменно}.

Оно неопределенно, однако уже более не таким образом, как бытие; его
неопределенность есть определенность, которая есть соотношение с самим
собою, абсолютная определенность; это
—~{\em положенное} внутри-себя-бытие.
{\em Как} то, что согласно своему понятию есть
соотносящееся с собою отрицание, оно имеет различие внутри себя "--- имеет
некоторое направление вовне, от себя к другому, каковое направление,
однако, непосредственно повернуто назад и возвратилось в себя, так как
согласно этому моменту самоопределения нет никакого другого, к которому оно
устремлялось бы.

В этой простой непосредственности исчезло опосредствование наличного бытия и
самой идеальности, исчезли, стало быть, всякие различия и всякое
многообразие. В нем нет {\em ничего}; это
{\em ничто}, абстракция соотношения с самим собою,
отлично здесь от самого внутри-себя-бытия; оно есть
{\em положенное} ничто, так как это внутри-себя-бытие
уже более не есть простое нечто, а имеет определением то, что оно как
опосредствование конкретно; ничто же как абстрактное, хотя и тождественно с
одним, разнится, однако, от его определения. Это ничто, положенное, таким
образом, как имеющее место {\em в одном}, есть ничто
как {\em пустота}. — Пустота есть таким образом
качество одного в его непосредственности.

\subsection*{b) Одно и пустота}
Одно есть пустота, как абстрактное соотношение отрицания с самим собою. Но
от простой непосредственности, от того бытия одного, которое также и
утвердительно, пустота как ничто безоговорочно разнится, а так как они
находятся в {\em одном} соотношении, а именно, в
соотношении самого одного, то их разница
{\em положена}. Но, разнствуя от сущего, ничто как
пустота находится {\em вне} сущего одного.

Для-себя-бытие, определяя себя, таким образом, как одно и пустоту, вновь
достигло некоторого {\em наличного бытия}. — Одно и
пустота имеют своей общей простой почвой отрицательное соотношение с собою.
Моменты для-себя-бытия выступают из этого единства, становятся внешними
себе; так как через {\em простое} единство моментов
привходит определение {\em бытия}, то оно (простое
единство) тем самым понижает само себя до {\em одной}
стороны и, следовательно, до наличного бытия, и тем самым его другое
определение, отрицание вообще, равным образом становится рядом как наличное
бытие ничто (des Nichts), как пустота.

\subsection*{Примечание. Атомистика}

Одно в этой форме наличного бытия есть та ступень категории, которую мы
встречаем у древних как {\em атомистический принцип},
согласно которому сущность вещей составляют {\em атом}
и {\em пустота} ($\tau \acute{o}$
$\acute{\alpha} \tau o \mu o \nu $ или $\tau \acute{\alpha}$
$\acute{\alpha} \tau o \mu \alpha $ $\chi \alpha \iota $
$\tau \acute{o}$ $\chi \varepsilon \nu \acute{o} \nu $).
Абстракция, созревшая до этой формы, получила бóльшую определенность, чем
{\em бытие} Парменида и
{\em становление} Гераклита. Насколько
{\em высоко} поднимается эта абстракция, делая эту
простую определенность одного и пустоты принципом всех вещей, сводя
бесконечное многообразие мира к этой простой противоположности и
отваживаясь познать и объяснить его из нее, настолько же
{\em легко} для представляющего рефлектирования
представлять себе, что вот {\em здесь} находятся атомы,
а {\em рядом} с ними "--- пустота. Неудивительно поэтому,
что атомистический принцип сохранялся во все времена; такое же тривиальное
и внешнее отношение составности, которое должно еще прибавиться, чтобы была
достигнута видимость некоторого конкретного и некоторого многообразия,
столь же популярно, как и сами атомы и пустота. Одно и пустота есть
для-себя-бытие, наивысшее качественное внутри-себя-бытие, опустившееся до
полной {\em внешности}; непосредственность или бытие
одного ввиду того, что оно есть отрицание всякого инобытия, положено так,
чтобы не быть более определимым и изменчивым; для его абсолютной
неподатливости всякое определение, многообразие, всякая связь остается,
следовательно, всецело внешним соотношением.

У тех мыслителей, которые впервые выдвинули указанный атомистический
принцип, он, однако, не застрял в этом внешнем своем характере, а имел
помимо своего абстрактного еще и некоторое спекулятивное определение,
заключающееся в том, что {\em пустота} была ими познана
как {\em источник движения}, что является совершенно
другим отношением между атомами и пустотой, чем голая рядоположность этих
двух определений и их безразличие друг к другу. Утверждение, что пустота
есть источник движения, имеет не тот малозначительный смысл, что нечто
может вдвинуться лишь в пустоту, а не в уже наполненное пространство, так
как в последнем оно уже не находило бы открытого для него места; в этом
смысле пустота была бы лишь предпосылкой или условием, а не
{\em основанием} движения, равно как и само движение
предполагается при этом имеющимся налицо и забывается существенное "--- его
основание. Воззрение, согласно которому пустота составляет основание
движения, заключает в себе ту более глубокую мысль, что в отрицательном
вообще лежит основание становления, беспокойства самодвижения, причем,
однако, отрицательное следует понимать как истинную отрицательность
бесконечного. — Пустота есть {\em основание движения}
лишь как {\em отрицательное} соотношение одного со
своим {\em отрицательным}, с одним,~т.~е. с самим
собою, которое, однако, положено как налично сущее.

Но помимо этого спекулятивного смысла дальнейшие определения древних
относительно формы атомов, их положения, направления их движения довольно
произвольны и внешни; при этом они находятся в прямом противоречии с
основным определением атомов. Атомами, принципом величайшей внешности и,
следовательно, величайшего отсутствия понятия болеет физика в учении о
молекулах, частицах, равно как и та наука о государстве, которая исходит из
единичной воли индивидуумов.

\subsection*{c) Многие одни "--- Отталкивание}
Одно и пустота составляют для-себя-бытие в его ближайшем наличном бытии.
Каждый из этих моментов имеет своим определением отрицание и вместе с тем
положен как некоторое наличное бытие. Взятые со стороны первого, одно и
пустота есть {\em соотношение} отрицания с отрицанием
как соотношение некоторого другого со своим другим; одно есть отрицание в
определении бытия, пустота "--- отрицание в определении небытия. Но одно есть
по существу лишь соотношение с собою, как соотносящее
{\em отрицание},~т.~е. оно само есть то, чем пустота
должна быть вне его. Но оба {\em положены} также и как
утвердительное {\em наличное бытие}, одно "--- как
для-себя-бытие как таковое, другое "--- как неопределенное наличное бытие
вообще, причем оба соотносятся друг с другом как с некоторым
{\em другим наличным бытием}. Для-себя-бытие одного
есть, однако, существенно идеальность наличного бытия и другого; оно
соотносится со своим другим не как с некоторым другим, а лишь как
{\em с собою}. Но так как для-себя-бытие фиксировано
как одно, как для-себя-{\em сущее}, как
{\em непосредственно} имеющееся налицо, то его
{\em отрицательное} соотношение с
{\em собою} есть вместе с тем соотношение с некоторым
{\em сущим}, а так как это соотношение также и
отрицательно, то то, с чем для-себя-бытие соотносится, остается
определенным как некоторое {\em наличное бытие} и
некоторое {\em другое}; как представляющее собою по
существу соотношение {\em с самим собою}, другое есть
не неопределенное отрицание как пустота, а есть равным образом
{\em одно}. Одно есть, следовательно,
{\em становление многими одними}.

Но, собственно говоря, это не становление, так как становление есть переход
{\em бытия} в {\em ничто};
напротив, {\em одно} становится лишь
{\em одним} же. Одно, соотнесенное, содержит в себе
отрицательное как соотношение и потому имеет это отрицательное
{\em в} нем самом. Вместо становления здесь,
следовательно, имеется, во-первых, собственное имманентное соотношение
одного; и, во-вторых, поскольку это соотношение есть отрицательное, а одно
есть вместе с тем сущее, постольку одно отталкивает само себя
{\em от себя}. Отрицательное соотношение одного с собою
есть, следовательно, {\em отталкивание}.

Это отталкивание как полагание {\em многих одних} через
само одно есть собственный выход одного вне себя, но выход, к таким лежащим
вне его, которые сами суть лишь одно. Это "--- отталкивание согласно
{\em понятию}, {\em в себе} сущее
отталкивание. Второе отталкивание отлично от этого и есть, прежде всего,
предносящееся представлению внешней рефлексии отталкивание не как
порождение многих одних, а лишь как взаимное неподпускание пред-положенных,
уже {\em имеющихся} одних. Следует затем посмотреть,
каким образом первое, {\em в себе} сущее отталкивание
определяет себя ко второму, внешнему.

Прежде всего следует установить, какими определениями обладают многие одни
как таковые. Становление многими или продуцированность многих
непосредственно исчезает как полагаемость; продуцированные суть одни не для
другого, а соотносятся бесконечно с самими собою. Одно отталкивает от себя
лишь само {\em себя}, оно, следовательно, не
становится, а {\em уже есть}. То, что мы представляем
себе как оттолкнутое, равным образом есть некоторое
{\em одно}, некоторое {\em сущее}.
Отталкивание и отталкиваемость принадлежат обоим одинаковым образом и не
составляют никакого различия между ними.

Одни суть, таким образом, {\em пред-положенные} в
отношении друг друга "--- {\em положенные} отталкиванием
одного от самого себя, {\em наперед} положенные как
{\em не} положенные; их положенность снята, они суть
{\em сущие} в отношении друг друга как соотносящиеся
лишь с собою.

Множественность представляется, стало быть, не некиим
{\em инобытием}, а некиим совершенно внешним одному
определением. Одно, отталкивая само себя, остается соотношением с самим
собою, как и то одно, которое принимается ближайшим образом за
отталкиваемое. Что одни суть {\em другие} в отношении
друг друга, что они объединены в определении множественности, не касается,
стало быть, одних. Если бы множественность была некоторым соотношением
самих одних друг с другом, то они взаимно ограничивали бы себя и имели бы в
самих себе утвердительно некоторое бытие-для-другого. Их соотношение "--- а
последнее они имеют благодаря их сущему {\em в себе}
единству, — как оно здесь {\em положено}, определено
как отсутствие всякого соотношения; оно есть опять-таки положенная ранее
{\em пустота}. Последняя есть их граница, но граница
внешняя им, в которой они не должны быть {\em друг для
друга}. Граница есть то, в чем ограничиваемые столь же
{\em суть}, сколь и {\em не суть};
но пустота определена как чистое небытие, и лишь это составляет их границу.

Отталкивание одного от самого себя есть раскрытие того, что одно есть в
себе, но бесконечность как {\em развернутая} есть здесь
{\em вышедшая вне себя бесконечность}; она вышла вне
себя вследствие непосредственности бесконечного, одного. Она есть столь же
некое простое соотношение одного с одним, сколь и, наоборот, абсолютное
отсутствие соотношений одного; она есть первое со стороны простого
утвердительного соотношения одного с собою; она есть последнее со стороны
того же соотношения как отрицательного. Или, иначе говоря, множественность
одного есть собственное полагание одного; одно есть не что иное, как
{\em отрицательное} соотношение одного с собою, и это
соотношение, стало быть, само одно, есть многие одни. Но вместе с тем
множественность безоговорочно внешня одному, ибо одно именно и есть снятие
инобытия, отталкивание есть его соотношение с собою и простое равенство с
самим собою. Множественность одних есть бесконечность как простодушно
производящее себя противоречие.

\subsection*{Примечание. Лейбницевская монада}

Мы упомянули выше о {\em лейбницевском идеализме}. Здесь
мы можем прибавить, что этот идеализм, исходя из мысли о
{\em представляющей монаде}, которую он определяет как
для-себя-сущую, дошел лишь до только что рассмотренного нами отталкивания,
и притом, лишь до {\em множественности} как таковой, в
которой каждый из одних есть лишь сам по себе, безразличен к наличному
бытию и для-себя-бытию других одних или, иначе говоря, других вообще нет
для одного. Монада есть сама по себе весь замкнутый мир; ни одна монада не
нуждается в других. Но это внутреннее многообразие, которым она обладает в
своем представлении, ничего не меняет в ее определении "--- быть для-себя.
Лейбницевский идеализм берет {\em множественность}
непосредственно как нечто {\em данное} и не постигает
ее как некое {\em отталкивание} монады; для него
поэтому множественность имеется лишь со стороны ее абстрактной внешности.
{\em Атомистика} не обладает понятием идеальности; она
понимает одно не как нечто такое, что имеет {\em в
самом себе} оба момента, момент для-себя-бытия и момент для-него-бытия,
понимает его, стало быть, не как идеализованное, а лишь как просто, сухо
для-себя-сущее. Но она идет дальше исключительно только безразличной
множественности; атомы вступают в дальнейшее определение в отношении друг
друга, хотя это происходит, собственно говоря, непоследовательно, между тем
как, напротив, в указанной безразличной независимости монад множественность
остается как неподвижное {\em основное определение},
так что их соотношение имеет место лишь в монаде монад или в
рассматривающем их философе.


\bigskip

\section*{C. Отталкивание и притяжение}
\subsection*{a) Исключение одного}
Многие одни суть сущие; их наличное бытие или соотношение друг с другом есть
не-соотношение, оно им внешне; это "--- абстрактная пустота. Но они сами суть
это отрицательное соотношение с собою
лишь[ссылка!] как соотношение с
{\em сущими} другими; это "--- выше вскрытое противоречие,
бесконечность, положенная в непосредственности бытия. Тем самым
отталкивание непосредственно {\em преднаходит} то, что
им отталкивается. Взятое в этом определении, оно есть
{\em исключение}; одно отталкивает только им
непорожденные, им неположенные многие одни. Это отталкивание как взаимное
или всестороннее, — относительно, ограничено бытием одних.

Множественность есть ближайшим образом неположенное инобытие; граница есть
лишь пустота, лишь то, в чем {\em нет} одних. Но они
{\em суть} также и в границе; они суть в пустоте или,
иначе говоря, их отталкивание есть их {\em общее
соотношение}.

Это взаимное отталкивание есть положенное {\em наличное
бытие} многих одних; оно не есть их для-себя-бытие, по которому они были бы
различены, как многое, лишь в некотором третьем, а их собственное,
сохраняющее их различие. — Они взаимно отрицают друг друга, полагают одно
другое как такое, которое есть лишь для {\em одного}.
Но они вместе с тем также и {\em отрицают}, что
{\em они суть лишь для-одного}; они
{\em отталкивают} эту свою
{\em идеальность} и {\em обладают
бытием}. — Таким образом, разлучены те моменты, которые в идеальности
безоговорочно соединены. Одно есть в своем для-себя-бытии также и
{\em для-одного}, но это одно, для которого оно есть,
есть само же оно; его различение от себя непосредственно снято. Но во
множественности различенное одно обладает бытием. Бытие-для-одного, как оно
определено в исключении, есть поэтому некоторое бытие-для-другого. Таким
образом, каждое из них отталкивается некоторым другим, снимается им и
превращается в такое одно, которое есть не для-себя, а для-одного, и притом
есть другое одно.

Для-себя-бытие многих одних оказывается согласно этому их самосохранением
благодаря опосредствованию их отталкивания в отношении друг друга, в
котором они взаимно снимают одно другое и полагают другие как некое голое
бытие-для-другого. Но вместе с тем оно (самосохранение) состоит в том,
чтобы отталкивать эту идеальность и полагать одни не сущими
для-некоторого-другого. Но это самосохранение одних через их отрицательное
соотношение друг с другом есть, наоборот их разложение.

Одни не только {\em суть}, но и сохраняют себя своим
взаимным исключением. Во-первых, то, благодаря чему они должны были бы
найти прочную опору их разнствования, защищающую их от того, чтобы
оказаться отрицаемыми, есть их {\em бытие} и притом их
{\em в-себе}{}-бытие, противостоящее их соотношению с
другим; это в-себе-бытие состоит в том, что они суть
{\em одни}. Но {\em все суть такое
одно}; вместо того чтобы иметь в своем в-себе-бытии твердую точку, на
которую опиралась бы их разница, они оказываются в нем
{\em одним и тем же}. Во-вторых, их наличное бытие и их
взаимоотношение,~т.~е. их {\em полагание самих себя как
одних}, есть взаимное отрицание; но это равным образом есть
{\em одно и то же} определение всех, которым они,
следовательно, полагают себя, наоборот, как тождественные, равно как и
благодаря тому, что они суть в себе одно и то же, их идеальность,
долженствующая быть положенной другими, есть {\em их
собственная} идеальность, которую они, стало быть, столь же мало
отталкивают. — Они суть, таким образом, по своему бытию и полаганию лишь
{\em одно} утвердительное единство.

Это рассмотрение одних, приводящее к заключению, что они по обоим своим
определениям "--- как поскольку они суть, так и поскольку они соотносятся друг
с другом "--- оказываются лишь одним и тем же и неразличимыми, есть лишь наше
сравнивание. — Но следует также посмотреть, что в их
{\em взаимоотношении положено} в них же самих. — Они
{\em суть}: эта предпосылка предполагается в указанном
взаимоотношении "--- и они суть лишь постольку, поскольку они взаимно отрицают
друг друга и вместе с тем не подпускают к самим себе этой своей
идеальности, своей отрицаемости,~т.~е. отрицают взаимное отрицание. Но они
суть лишь постольку, поскольку они отрицают: таким образом, когда
отрицается это их отрицание, отрицается также и их бытие. Правда, ввиду
того, что они {\em суть}, они этим отрицанием не
подверглись бы отрицанию, оно есть для них лишь нечто внешнее; это
отрицание другого отскакивает от них и попадает, коснувшись их, лишь в их
поверхность. Однако лишь благодаря отрицанию других одних они возвращаются
в самих себя; они имеют бытие лишь как это опосредствование: это их
возвращение есть их самосохранение и их для-себя-бытие. Так как их
отрицание не имеет никакого эффекта вследствие противодействия, которое
оказывают сущие как таковые или как отрицающие, то они не возвращаются в
себя, не сохраняют себя и не суть.

Выше мы уже выяснили, что одни суть одно и то же, каждое из них есть такое
же {\em одно}, как и другое. Это "--- не только наше
соотнесение, внешнее сведение вместе, а само отталкивание есть соотнесение;
одно, исключающее одни, соотносит само себя с ними, с одними,~т.~е. с самим
собою. Отрицательное отношение одних друг к другу есть, следовательно, лишь
некое {\em слияние с собою}. Это тождество, в которое
переходит их отталкивание, есть снятие той их разницы и внешности, которую
они как исключающие должны были, наоборот, отстоять друг против друга.

Это самополагание многих одних в единое одно
(dies sich in-Ein-Eines-setzen der vielen Eins)
есть {\em притяжение}.

\subsection*{Примечание. Положение о единстве одного и многого}

Самостоятельность, доведенная до того последнего заострения, которое мы
видим в для-себя-сущем одном, есть абстрактная, формальная
самостоятельность, сама себя разрушающая; это "--- величайшее, упорнейшее
заблуждение, принимающее себя за высшую истину. В своих более конкретных
формах она выступает как абстрактная свобода, как чистое «я», а затем,
далее, как нравственное зло. Это "--- свобода, впавшая в такую ошибку, что
полагает свою сущность в этой абстракции и ласкает себя мыслью, будто в
этом замыкании в себя (Bei-sich-Sein) она обретает себя в чистом виде.
Говоря определеннее, эта самостоятельность есть заблуждение, заключающееся
в том, что смотрят как на отрицательное на то и относятся как к
отрицательному к тому, что есть ее собственная сущность. Она есть, таким
образом, отрицательное отношение к самой себе, которое, желая обрести
собственное бытие, разрушает его, и это его деяние представляет собою лишь
проявление ничтожества этого деяния. Примирение заключается в признании,
что то, против чего направлено отрицательное отношение, есть, наоборот, его
сущность, заключается лишь {\em в отказе} от
отрицательности {\em своего} для-себя-бытия, вместо
того чтобы крепко держаться за это последнее.

Древнее изречение гласит, что {\em одно} есть
{\em многое} и что в особенности
{\em многое есть одно}. По поводу этого изречения мы
должны повторить сделанное выше замечание, что истина одного и многого,
выраженная в предложениях, выступает в неадэкватной форме, что эту истину
нужно понимать и выражать лишь как некое становление, как некий процесс,
отталкивание и притяжение, а не как бытие, взятое в предложении как
покоящееся единство. Выше мы упомянули и напомнили о диалектике
{\em Платона} в «Пармениде» касательно дедукции многого
из одного, а именно, из предложения, гласящего: одно есть. Внутренняя
диалектика понятия была нами указана; всего легче понимать диалектику
положения, гласящего, что {\em многое есть одно}, как
внешнюю рефлексию, и она имеет право быть здесь внешней, поскольку и
предмет, {\em многие}, есть то, что внешне друг другу.
Это сравнение многих между собою сразу дает тот результат, что одно всецело
определено лишь как другое; каждое есть одно, каждое есть одно из многих,
исключает другие, — так что они безоговорочно суть лишь одно и то же,
безоговорочно имеется налицо лишь {\em одно}
определение. Это "--- {\em факт}, и дело идет лишь о
понимании этого простого факта. Рассудок упрямо противится этому пониманию
лишь потому, что ему предносится, и притом правильно,
{\em также} и различие; но последнее так же не отпадает
вследствие сказанного факта, как и, обратно, этот факт не перестает
существовать, несмотря на различие. Можно было бы, следовательно, утешить
рассудок касательно его здравомысленного понимания факта различия, указав
ему, что и различие появится снова.

\subsection*{b) Единое одно притяжения}
Отталкивание есть саморасщепление одного ближайшим образом на многие,
отрицательное отношение которых бессильно, так как они предполагают друг
друга как сущие; оно есть лишь {\em долженствование}
идеальности; реализуется же последняя в притяжении. Отталкивание переходит
в притяжение, многие одни в единое одно. Эти два определения, отталкивание
и притяжение, ближайшим образом различаются, первое как реальность одних,
второе "--- как их положенная идеальность. Притяжение относится к отталкиванию
таким образом, что оно имеет последнее своей
{\em предпосылкой}. Отталкивание доставляет материю для
притяжения. Если бы не было никаких одних, то нечего было бы притягивать.
Представление о непрерывном притяжении, о непрерывном потреблении одних,
предполагает столь же непрерывное порождение одних; чувственное
представление о пространственном притяжении оставляет существовать поток
притягиваемых одних; вместо атомов, исчезающих в притягивающей точке,
выступает из пустоты другое множество атомов, и это, если угодно,
продолжается до бесконечности. Если бы притяжение было завершено,~т.~е.,
если бы мы представили себе, что многие приведены в точку единого одного,
то имелось бы лишь некое косное одно, уже не было бы более притяжения.
Налично сущая в притяжении идеальность заключает в себе еще также и
определение отрицания самой себя, те многие одни, соотношение с которыми
она составляет, и притяжение неотделимо от отталкивания.

Притяжение ближайшим образом равно присуще каждому из многих
{\em непосредственно} имеющихся одних; никакое из них
не имеет преимущества перед другим; в последнем случае имелось бы
равновесие в притяжении, собственно говоря, равновесие самих же притяжения
и отталкивания и косный покой без налично сущей идеальности. Но здесь не
может быть речи о преимуществе одного такого одного перед другим, что
предполагало бы некоторое определенное различие между ними, а, наоборот,
притяжение есть полагание имеющейся неразличности одних. Только само
притяжение есть впервые {\em полагание} некоего
отличного от других одного; они суть лишь непосредственные одни,
долженствующие сохранять себя через отталкивание; а через их положенное
отрицание возникает одно притяжения (das Eins der Attraktion), каковое одно
поэтому определено как опосредствованное, как
{\em одно, положенное как одно}. Первые одни как
непосредственные не возвращаются в своей идеальности назад в себя, а имеют
ее в некотором другом.

Но единое одно есть реализованная, положенная в одном идеальность; оно
притягивает через посредство отталкивания. Оно содержит это
опосредствование в самом себе {\em как свое
определение}. Оно, таким образом, не поглощает в себя притягиваемых одних
как в некоторую точку,~т.~е. оно не упраздняет их абстрактным образом. Так
как оно содержит в своем определении отталкивание, то последнее вместе с
тем сохраняет в нем одни как многие. Оно, так сказать, через свое
притяжение ставит нечто перед собою, приобретает некоторый объем или
наполнение. В нем, таким образом, есть вообще единство отталкивания и
притяжения.

\subsection*{c) Соотношение отталкивания и притяжения}
Различие между {\em одним} и
{\em многими} определилось в различие их
{\em соотношения} друг с другом, которое разложено на
два соотношения, на отталкивание и притяжение, каждое из коих сначала стоит
самостоятельно вне другого, но так, что они, однако, по существу связаны
вместе. Их еще неопределенное единство должно получить более определенные
очертания.

Отталкивание как основное определение одного выступает первым и
{\em непосредственным}, так же, как и его одни, хотя и
порожденные им, но вместе с тем положенные как непосредственные, и тем
самым оно выступает как безразличное к притяжению, которое привходит к
нему, как к такому пред-положенному, внешним образом. Напротив, притяжение
не предполагается отталкиванием, так что к полаганию его и его бытию первое
не должно быть причастно,~т.~е. отталкивание не есть уже в самом себе
отрицание самого себя, одни не суть уже в самих себе подвергшиеся
отрицанию. Таким образом, мы имеем отталкивание абстрактно, само по себе,
равно как и притяжение имеет по отношению к одним как
{\em сущим} аспект некоторого непосредственного
наличного бытия и привходит к ним спонтанейно (von sich aus), как некоторое
другое.

Если мы согласно этому возьмем голое отталкивание так просто, само по себе,
то оно будет рассеянием многих одних в неопределенную даль, лежащую вне
сферы самого отталкивания, ибо оно состоит в отрицании соотношения многих
одних друг с другом; отсутствие соотношения есть его, взятого абстрактно,
определение. Но отталкивание не есть только пустота; одни, как не имеющие
соотношений, не отталкивают, не исключают, что составляет их определение.
Отталкивание есть, по существу хотя и отрицательное, но все же
{\em соотношение}; взаимное неподпускание и избегание
не есть освобождение от того, что не подпускается и чего избегают;
исключающее еще находится {\em в связи с тем}, что из
него исключается. Но этот момент соотношения есть притяжение,
следовательно, притяжение в самом отталкивании. Оно есть отрицание того
абстрактного отталкивания, по которому одни были бы лишь соотносящимися с
собою, сущими, а не исключающими.

Но поскольку исходным пунктом служило отталкивание налично сущих одних и,
стало быть, притяжение также положено приходящим к нему внешним образом, то
при всей их нераздельности они все же еще удерживались одно вне другого как
разные определения. Теперь, однако, оказалось, что не только отталкивание
предполагается притяжением, но что имеет место также и обратное соотношение
отталкивания с притяжением, и первое равным образом имеет свою предпосылку
в последнем.

Согласно этому определению они нераздельны и вместе с тем каждое из них
определено по отношению к другому как долженствование и предел. Их
долженствование есть их абстрактная определенность как
{\em сущих в себе}, которая, однако, вместе с тем
безоговорочно выпирается за себя и соотносится с другой определенностью и
таким образом каждое из них имеет бытие через посредство своего
{\em другого} как другого; их самостоятельность состоит
в том, что они в этом опосредствовании положены друг для друга как некий
другой процесс определения, отталкивание как полагание многих, притяжение
как полагание одного, притяжение, вместе с тем как отрицание многих, а
отталкивание как отрицание их идеальности в одном "--- состоит в том, что
также и притяжение есть притяжение лишь
{\em посредством} отталкивания, а отталкивание есть
отталкивание лишь посредством притяжения. Но что в этом процессе
определения опосредствование с собою через {\em другое}
на самом деле, наоборот, отрицается, и каждое из этих определений есть
опосредствование себя с самим собою, это вытекает из более близкого их
рассмотрения и приводит их обратно к единству их понятия.

Во-первых, что каждое предполагает {\em само себя},
соотносится в своей предпосылке лишь с собою, это уже подразумевается в
наведении (in dem Verhalten) пока что еще относительных отталкивания и
притяжения.

Относительное отталкивание есть взаимное неподпускание
{\em имеющихся налицо} многих одних, которые, как
предполагается, преднаходят друг друга как непосредственные. Но что имеются
многие одни, в этом ведь и состоит самое отталкивание; предпосылка, которую
оно якобы имеет, есть лишь его собственное полагание. Далее, определение
{\em бытия}, которое якобы принадлежит одним сверх
того, что они суть положенные, — определение, благодаря которому они
оказались бы {\em предшествующими}, равным образом
принадлежит отталкиванию. Отталкивание есть то, через что одни проявляют и
сохраняют себя как одни, то, через что они как таковые
{\em имеют бытие}. Их бытием и служит само
отталкивание; последнее, таким образом, не есть некое относительное к
некоторому другому наличному бытию, а относится всецело лишь к самому себе.

Притяжение есть полагание одного как такового, реального одного, в отношении
которого многие в их наличном бытии определяются, как лишь идеализованные и
исчезающие. Таким образом, притяжение сразу же предполагает само себя, а
именно, предполагает себя в том определении других одних, согласно которому
они суть идеализованные, тогда как помимо этого эти другие одни должны были
бы быть для-себя-сущими, а {\em для других},
следовательно, также и для какого бы то ни было притягивающего
—~отталкивающими. В противовес этому определению отталкивания они получают
идеальность не через отношение к притяжению, а она уже предпослана, есть
{\em в себе} сущая идеальность одних, так как они как
одни "--- включая и то одно, которое представляют себе как притягивающее "--- не
отличны друг от друга, суть одно и то же.

Это самопредпосылание обоих определений, каждого из них самого по себе,
означает, далее, то, что каждое из них содержит в себе другое как момент.
Самопредпосылание вообще есть в то же время полагание себя как
{\em отрицательного} себя (das Negative seiner), —
отталкивание; а то, что здесь предпосылается, есть
{\em то же самое}, что и предпосылающее, — притяжение.
Что каждое из них есть {\em в себе} лишь момент,
означает, что каждое из них спонтанейно (aus sich selbst) переходит в
другое, отрицает себя в самом себе и полагает себя как другое самого себя.
Поскольку одно как таковое есть выхождение вне себя, поскольку оно само
состоит лишь в том, что оно полагает себя как другое, как множественное, а
множественное также состоит лишь в том, что оно сжимается в себя и полагает
себя как свое другое, как одно, и поскольку именно в этом они соотносятся
лишь с самими собою и каждое из них продолжает себя в своем другом,
постольку, следовательно, выхождение вне себя (отталкивание) и полагание
себя как одного (притяжение) уже в самих себе имеются нераздельными. Но в
относительных отталкивании и притяжении,~т.~е. в таком отталкивании и таком
притяжении, которое предполагает непосредственные,
{\em налично сущие} одни,
{\em положено}, что каждое из них есть это отрицание
себя в самом себе и тем самым также и продолжение себя в свое другое.
{\em Отталкивание} налично сущих одних есть
самосохранение одного путем взаимного неподпускания других, так что (1)
другие одни отрицаются {\em в нем} "--- это есть аспект
его наличного бытия или его бытия-для-другого; но этот аспект есть,
следовательно, притяжение как идеальность одних; и (2) одно есть
{\em в себе}, без соотношения с другими; но «в себе» не
только вообще давно уже перешло в для-себя-бытие, но и
{\em в себе}, по своему определению, одно есть
сказанное становление многими. — {\em Притяжение}
налично сущих одних есть их идеальность и полагание одного, в чем оно,
стало быть, как отрицание и продуцирование одного снимает само себя и как
полагание одного оказывается отрицанием самого себя в самом себе,
оказывается отталкиванием.

Этим развитие для-себя-бытия завершено и дошло до своего результата. Одно,
как соотносящееся {\em с самим собой бесконечным
образом},~т.~е., как положенное отрицание отрицания, есть опосредствование
в том смысле, что оно себя, как свое абсолютное (т.~е. абстрактное)
{\em инобытие} (т.~е.
{\em многие}), отталкивает от себя и, соотносясь с этим
своим небытием отрицательно, снимая его, именно в этом соотношении есть
лишь соотношение с самим собою; и одно есть лишь это становление, в котором
исчезло определение, что оно {\em начинается},~т.~е.
положено как непосредственное, сущее, и что оно равным образом и как
результат восстановило себя, сделавшись снова одним,~т.~е. таким же
{\em непосредственным}, исключающим одним; процесс,
который оно есть, повсюду полагает и содержит его в себе лишь как некоторое
снятое. Снятие, определившееся сначала лишь в относительное снятие, в
{\em соотношение} с другим налично сущим (каковое
соотношение, следовательно, само есть некоторое диферентное отталкивание и
притяжение), оказывается также и переходящим в бесконечное соотношение
опосредствования через отрицание внешних соотношений непосредственного и
налично сущего. При этом оно имеет своим результатом именно то становление,
которое ввиду неустойчивости его моментов есть опадание или, вернее,
слияние с собою, переход в простую непосредственность. Это бытие по тому
определению, которое оно теперь {\em получило},
{\em есть количество}.

Если обозреть вкратце моменты этого {\em перехода
качества в количество}, то окажется, что качественное имеет своим основным
определением бытие и непосредственность, в которой граница и определенность
так тождественны с бытием данного нечто, что вместе с их изменением
исчезает и само нечто; {\em положенное} таким образом,
оно определено как конечное. Вследствие непосредственности этого единства,
в котором {\em различие} исчезло, но в котором, как в
единстве {\em бытия} и {\em ничто},
оно в себе имеется, это различие как {\em инобытие}
вообще имеет место {\em вне} вышеупомянутого единства.
Это соотношение с другим противоречит непосредственности, в которой
качественная определенность есть соотношение с собою. Это инобытие
снимается в бесконечности для-себя-бытия, реализовавшего различие, которое
оно имеет в отрицании отрицания {\em у} и
{\em внутри} самого себя, сделав его одним и многим и
их соотношениями, и возведшего качественное в истинное, уже не
непосредственное, а положенное как согласующееся с собою, единство.

Это единство есть, стало быть ($\alpha $) {\em бытие}
лишь как {\em утвердительное},~т.~е., как
опосредствованная с собою через отрицание отрицания
{\em непосредственность}; бытие положено как единство,
{\em проходящее сквозь} свои определенности, границы
и~т.~д., которые положены в нем как снятые. ($\beta $)
{\em Наличное бытие}; оно есть по такому определению
отрицание или определенность как момент утвердительного бытия; однако, она
уже более не есть непосредственная, а есть рефлектированная в себя,
соотносящаяся не с другим, а с собою определенность, — безоговорочная
определенность, {\em в-себе}{}-определенность, — одно;
инобытие как таковое само есть для-себя-бытие. ($\gamma $)
{\em Для-себя-бытие} как то продолжающееся сквозь
определенность бытие, в котором одно и в-себе-определенность сами положены
как снятые. Одно вместе с тем определено как вышедшее за себя и как
{\em единство}; стало быть, одно,~т.~е. безоговорочно
определенная граница положена как граница, которая не есть граница, — как
граница, которая есть в бытии, но безразлична для него.

\subsection*{Примечание. Кантовское построение материи из
сил притяжения и отталкивания}

На притяжение и отталкивание, как известно, обыкновенно смотрят, как на
{\em силы}. Следует сравнить это их определение и
связанные с ним отношения с теми понятиями, которые у нас получились о них.
— В сказанном представлении они рассматриваются как самостоятельные, так
что они соотносятся друг с другом не по своей природе,~т.~е. каждое из них
не есть лишь переходящий в свою противоположность момент, а прочно остается
перед лицом другого тем же, что раньше. Их, далее, представляют себе, как
сходящиеся в некотором {\em третьем}, в
{\em материи}, но сходящиеся таким образом, что это их
схождение в одно (In-Eins-Werden) не считается их истиной, а каждое
признается некиим первым и само по себе сущим, материя же или ее
определения "--- положенными и произведенными ими. Когда говорят, что материя
{\em обладает внутри себя} силами, то под этим их
единством разумеют некоторую их связь, причем они вместе с тем
предполагаются как сущие внутри себя и свободные друг от друга.

Как известно, {\em Кант} конструировал
{\em материю из сил отталкивания и притяжения}, или, по
крайней мере, как он выражается, дал метафизические элементы этой
конструкции. — Не безынтересно будет рассмотреть ближе эту конструкцию. Это
{\em метафизическое} изложение предмета, который, как
казалось, не только сам, но и в своих определениях принадлежит лишь области
{\em опыта}, замечательно отчасти тем, что оно как
попытка понятия дало, по крайней мере, толчок новейшей философии природы,
философии, которая не делает основой науки природу как нечто чувственно
данное восприятию, а познает ее определения из абсолютного понятия; отчасти
же оно интересно также и потому, что часто еще и теперь не идут дальше
кантовской конструкции и считают ее философским началом и основой физики.

Такого рода существование, как чувственная материя, не есть правда, предмет
логики; она столь же мало является таковым, как и пространство и
пространственные определения. Но и в основе сил притяжения и отталкивания,
поскольку они понимаются как силы чувственной материи, лежат рассмотренные
здесь чистые определения одного и многих, равно как и их взаимоотношений,
которые я назвал отталкиванием и притяжением, потому что эти названия ближе
всего подходят.

Кантовский прием в дедукции материи из сказанных сил, который он называет
{\em конструкцией}, оказывается при более близком
рассмотрении не заслуживающим этого имени, если только не называть
конструкцией всякого рода рефлексию, хотя бы даже анализирующую, как и в
самом деле позднейшие натурфилософы называли
{\em конструкцией} также и самое плоское
рассуждательство и самую неосновательную смесь произвольного фантазирования
и лишенной мысли рефлексии "--- смесь, в которой в особенности пользовались,
протаскивая их повсюду, так называемыми факторами отталкивательной и
притягательной силы.

Прием Канта именно, в сущности говоря, {\em аналитичен},
а не конструктивен. Он уже {\em предполагает
представление материи} и затем спрашивает, какие требуются силы для того,
чтобы получить ее предполагаемые определения. Таким образом оказывается,
что он, с одной стороны, требует силы притяжения, так как при наличии
одного лишь отталкивания, без притяжения, не могло бы, собственно говоря,
быть никакой материи («Основные начала естествознания», стр. 53 и сл.).
Отталкивание же он, с другой стороны, также выводит из материи и указывает
в качестве его основания то обстоятельство, {\em что мы
представляем себе материю непроницаемой}, так как под таким именно
определением она являет себя {\em чувству осязания},
через которое, дескать, она нам открывается. Отталкивание потому нами сразу
же мыслится в {\em понятии} материи, что оно
(отталкивание) непосредственно дано вместе с ней, притяжение же, напротив,
мы прибавляем к ней посредством {\em умозаключений}. Но
и в основании этих умозаключений также лежит только что высказанное
соображение, что материя, которая обладала бы единственно лишь
отталкивательной силой, не исчерпывала бы того, что мы представляем себе
под материей. Совершенно очевидно, что здесь перед нами тот образ действия
рефлектирующего об опыте познания, который сначала
{\em воспринимает} в явлении те или другие определения,
кладет их затем в основание и принимает для так называемого
{\em объяснения} их соответствующие
{\em основные материи} или
{\em силы}, долженствующие произвести эти определения
явления.

Касательно указанного различия тех способов, какими познание находит в
материи силу отталкивания и силу притяжения, Кант замечает далее, что сила
притяжения все-таки тоже {\em принадлежит} к понятию
материи, {\em хотя она и не содержится в нем}. Кант
подчеркивает это последнее выражение. Но нельзя усмотреть, в чем тут
различие, ибо определение, принадлежащее к
{\em понятию} некоторой вещи, поистине
{\em необходимо должно содержаться в нем}.

Затруднение, заставляющее Канта прибегнуть к этой пустой уловке, состоит
здесь именно в том, что Кант с самого начала односторонне включает в
понятие материи единственно лишь определение
{\em непроницаемости}, которое мы согласно ему
{\em воспринимаем} посредством
{\em чувства осязания}, вследствие чего сила
отталкивания, как неподпускание некоторого другого к себе, дана-де
непосредственно. Но если далее говорится, что материя не может
{\em существовать} без притяжения, то в основании этого
утверждения лежит заимствованное из восприятия представление о материи;
определение притяжения должно, следовательно, равным образом встретиться
нам в восприятии. И мы действительно воспринимаем, что материя, кроме
своего для-себя-бытия, которое устраняет (aufhebt) бытие-для-другого
(оказывает сопротивление), обладает также и некоторым
{\em соотношением для-себя-сущих друг с другом},
пространственным {\em протяжением} и
{\em связностью} и в виде неподатливости (Starrheit),
твердости (Festigkeit) обладает очень прочной связностью. Объясняющая
физика требует для разрыва и~т.~д. тела наличия такой силы, которая
превосходила бы {\em притяжение} его частей друг к
другу. Из этого восприятия рефлексия может столь же непосредственно вывести
силу притяжения или принять ее как {\em данную}, как
она это сделала с силой отталкивания. И в самом деле, когда мы
рассматриваем те кантовские умозаключения, из которых согласно ему
выводится сила притяжения (доказательство теоремы, что возможность материи
требует силы притяжения как второй основной силы; там же), то мы
убеждаемся, что они не заключают в себе ничего другого, кроме того
соображения, что при одном только отталкивании материя не могла бы быть
{\em пространственной}. Так как материя предполагается
наполняющей пространство, то ей приписывается непрерывность, как основание
которой и принимается сила притяжения.

Хотя такая так называемая конструкция материи обладает в лучшем случае
аналитической заслугой, которая еще кроме того умаляется нечеткостью
изложения, мы все же должны признать весьма ценной основную мысль познать
материю из этих двух противоположных определений как из ее основных сил.
Кант старается главным образом об изгнании вульгарно-механических способов
представления, которые не идут дальше одного определения "--- непроницаемости,
{\em для-себя-сущей точечности}, и делают чем-то
{\em внешним} противоположное определение,
{\em соотношение} материи внутри себя или
{\em соотношение} друг с другом нескольких материй,
рассматриваемых в свою очередь как особенные одни, — об изгнании того
способа представления, который, как говорит Кант, не соглашается признать
никаких других движущих сил, кроме сил, движущих посредством давления и
толчка, следовательно, лишь посредством воздействия извне. Это носящее
{\em внешний} характер познание предполагает, что
движение как нечто внешнее для материи всегда уже
{\em имеется налицо}, и не помышляет о том, чтобы
понимать его как нечто внутреннее и постигать его в самой материи, которая
благодаря отсутствию такого понимания признается сама по себе неподвижной и
косной. Этой точке зрения предносится лишь обычная механика, а не
имманентное и свободное движение. — Хотя Кант устраняет сказанный внешний
характер, превращая притяжение ({\em соотношение}
материй друг с другом, поскольку эти материи принимаются отделенными друг
от друга, или {\em соотношение} материи вообще в ее
вне-себя-бытии) в {\em силу самой материи}, все же
принимаемые им две основные силы остаются, с другой стороны, внутри материи
внешними друг другу и, сами по себе, самостоятельными
{\em в отношении друг друга}.

Точно так же, как оказалось неосновательным то самостоятельное различие этих
двух сил, которое приписывается им с точки зрения указанного познания,
должно оказаться неосновательным и всякое другое различие, проводимое в
отношении их содержательного определения как нечто
{\em якобы неподвижное}, так как они, как они были
рассмотрены выше в их истине, суть лишь моменты, переходящие друг в друга.
— Теперь я рассмотрю эти дальнейшие различительные определения, как их
устанавливает Кант.

А именно, он определяет силу притяжения как
{\em проникающую} силу, благодаря которой одна материя
может {\em непосредственно} действовать на части другой
также и за пределами поверхности соприкосновения, отталкивательную же силу
он, напротив, определяет как {\em поверхностную} силу,
посредством которой материи могут действовать друг на друга только в общей
им поверхности соприкосновения. Довод, приводимый им в пользу того, что
отталкивание есть только поверхностная сила, гласит следующим образом:
«Каждая из {\em соприкасающихся} частей ограничивает
сферу действия другой, и отталкивательная сила не могла бы привести в
движение более отдаленную часть без посредства промежуточных частей;
проходящее поперек через них непосредственное действие одной материи на
другую посредством сил расширения (так называются здесь силы отталкивания)
невозможно» (см. там же, «Пояснения и добавления», стр. 67).

Мы должны сразу же напомнить о том, что, поскольку принимаются
{\em более близкие} или {\em более
отдаленные} части материи, постольку и {\em по
отношению к притяжению} равным образом возникает
{\em различие}: один атом, правда, действует на
{\em другой}, но {\em третий},
более отдаленный, между которым и первым, притягивающим, находится
{\em другой} атом, должен был бы сначала вступить в
сферу притяжения лежащего между ними, более близкого к нему атома, и первый
атом, следовательно, не мог бы оказывать на третий
{\em непосредственного} простого действия, из чего
вытекает, что действие силы притяжения есть такое же опосредствованное, как
и действие силы отталкивания. И далее: {\em истинное
проникание} силы притяжения должно было бы состоять только в том, что все
части материи {\em сами по себе} суть притягивающие, а
не в том, что известное их количество ведет себя пассивно и только один
атом активен. — Непосредственно же или, иначе говоря, по отношению к самой
силе отталкивания мы должны заметить, что в приведенной цитате говорится о
{\em соприкасающихся}, частях и, следовательно, о
{\em компактности} и
{\em непрерывности готовой} материи, не позволяющей
отталкиванию пройти через нее. Но эта компактность материи, в которой части
{\em соприкасаются} и уже не разделены более пустотой,
предполагает {\em устраненность} (Aufgehobensein)
{\em силы отталкивания}; соприкасающиеся части должны
быть признаны согласно господствующему здесь чувственному представлению об
отталкивании такими частями, которые не отталкивают друг друга. Из этого,
следовательно, вытекает совершенно тавтологически, что там, где мы
принимаем небытие отталкивания, отталкивание не может иметь места. Но из
этого ничего дальше не следует касательно определения силы отталкивания.
Если же мы еще подумаем о том, что соприкасающиеся части соприкасаются лишь
постольку, поскольку они еще держатся {\em вне друг
друга}, то мы убедимся, что сила отталкивания находится тем самым не только
на поверхности материи, но и внутри той сферы, которая якобы есть лишь
сфера притяжения.

Далее Кант принимает определение, что «через посредство силы притяжения
материя лишь {\em занимает некоторое пространство, не
наполняя его}» (там же); «так как материя не наполняет пространства
посредством силы притяжения, то последняя может действовать через
{\em пустое пространство}, ибо не имеется промежуточной
материи, которая ставила бы ей границы». — Это различие носит
приблизительно такой же характер, как вышеприведенное: там определение
принадлежит к понятию некоторой вещи, но не содержится в нем; здесь материя
лишь {\em занимает} некоторое пространство, но не
{\em наполняет} его. Там получается, что посредством
{\em отталкивания}, если мы остановимся на его первом
определении, нами одни отталкиваются и {\em соотносятся
друг с другом} лишь отрицательно, а именно {\em через
пустое пространство}. Здесь же получается, что как раз
{\em притягательная сила} сохраняет пространство
пустым; она {\em не наполняет} пространство посредством
своего соотнесения атомов,~т.~е. она удерживает атомы в
{\em отрицательном}
{\em соотношении}, друг с другом. — Как видим, здесь
Кант, приписывая силе притяжения как раз то, что он согласно первому
определению приписывал противоположной силе, бессознательно натыкается на
то, что лежит в природе вещей. В процессе установления различия этих двух
сил получилось, что одна сила перешла в другую. — Так посредством
отталкивания материя, согласно Канту, {\em наполняет},
напротив, некоторое пространство и, следовательно, посредством него
исчезает то пустое пространство, которое сила притяжения оставляет
существовать. И в самом деле, отталкивание, устраняя пустое пространство,
тем самым устраняет отрицательное соотношение атомов или одних,~т.~е. их
отталкивание,~т.~е. отталкивание определено как противоположность самого
себя.

К этому стиранию различий присоединяется еще и та путаница, что, как мы уже
заметили вначале, кантовское изображение противоположных сил аналитично, и
во всем этом изложении материя, которая еще должна быть выведена из ее
элементов, уже выступает как готовая и конституированная. В дефиниции
поверхностной и проникающей сил обе принимаются как движущие силы,
посредством которых {\em материи} могут действовать тем
или иным образом. — Они, следовательно, изображаются здесь не как силы,
посредством которых материя впервые получает существование, а как такие
силы, посредством которых она, уже готовая, лишь приводится в движение. Но
поскольку речь идет о силах, посредством которых различные материи
воздействуют друг на друга и движут друг друга, это есть нечто совершенно
другое, чем то определение и то соотношение, которое они должны были иметь
как моменты материи.

Такую же противоположность, как силы притяжения и отталкивания, представляют
собою в дальнейшем определении
{\em центростремительная} и
{\em центробежная} силы. Сначала кажется, что эти силы
являют существенное различие, так как в их сфере имеется неподвижное единое
одно, центр, по отношению к которому другие одни ведут себя как не
для-себя-сущие, и мы можем поэтому приводить в связь различие указанных сил
с этим предполагаемым различием между центральным одним и другими одними,
которые неподвижны по отношению к этому центральному одному. Но поскольку
этими силами пользуются для объяснения, для каковой цели принимают, как
принималось прежде относительно сил отталкивания и притяжения, что они
находятся в обратном количественном отношении, так что одна увеличивается с
уменьшением другой, постольку явление движения, для объяснения которого
{\em они} принимаются, и его неравенство должны еще
только оказаться их результатом. Однако достаточно только вникнуть в первое
попавшееся изображение какого-нибудь явления, например, неровной скорости,
которой обладает планета на ее пути вокруг ее центрального тела, стоит
только вникнуть в объяснение этого явления противоположностью этих сил,
чтобы сразу увидеть господствующую здесь путаницу и невозможность
разъединить их величины, так что всегда приходится принимать возрастающей
также и ту силу, которая в объяснении принимается убывающей, и обратно.
Чтобы сделать сказанное наглядным, потребовалось бы более пространное
изложение, чем то, которое мы можем дать здесь, но все необходимое будет
дано далее, когда дойдем до изложения {\em обратного
отношения}.

\bigskip



\part*{Второй отдел. Величина (количество).}

%         Первая глава. Количество.
Мы уже указали отличие количества от качества. Качество есть первая,
непосредственная определенность, количество же —~определенность, ставшая
безразличной для бытия, граница, которая вместе с тем и не есть граница,
для-себя-бытие, которое безоговорочно тождественно с бытием-для-другого,
"--- отталкивание многих одних, которое есть непосредственно
неотталкивание, непрерывность их.

Так как сущее-для-себя теперь положено таким образом, чтобы не исключать
другое, а наоборот, утвердительно продолжать себя в последнее, то,
поскольку {\em наличное бытие} снова выступает в этой
непрерывности и определенность этого наличного бытия
{\em вместе с тем} уже не стоит более в простом
соотношении, с собою, инобытие уже не есть более непосредственная
определенность налично сущего нечто, но положено так, что имеет себя как
отталкивающееся от себя, имеет соотношение с собою как определенность
скорее в некотором другом наличном бытии (в некотором сущем-для-себя); а
так как они {\em вместе с тем} даны
(sind) как безразличные рефлектированные в себя,
несоотносительные границы, то определенность есть вообще
{\em вне себя}, некое безоговорочно
{\em внешнее} себе, и нечто есть также внешнее; такая
граница, безразличие ее в самой себе и безразличие данного нечто к ней,
составляет {\em количественную} определенность этого
нечто.

Прежде всего надлежит отличать {\em чистое количество}
от него же как {\em определенного} количества, от
Quantum. Как чистое количество оно есть,
{\em во-первых}, возвратившееся в себя реальное
для-себя-бытие, не имеющее еще в себе никакой определенности; оно есть
сплошное (gediegene), продолжающее себя внутри
себя бесконечное единство.

Последнее, {\em во-вторых}, переходит в определенность,
полагаемую в нем как определенность, которая вместе с тем не есть таковая,
есть лишь внешняя определенность. Количество становится
{\em определенным количеством}. Определенное количество
есть безразличная,~т.~е. выходящая за себя, отрицающая самое себя
определенность. Как такое инобытие инобытия оно впадает в
{\em бесконечный} прогресс. Но бесконечное определенное
количество есть снятая безразличная определенность, оно есть восстановление
качества.

{\em В-третьих}, определенное количество в качественной
форме есть количественное {\em отношение}. Определенное
количество выходит за себя лишь вообще; в отношении же оно выходит за себя,
переходит в свое инобытие так, что последнее, в котором оно имеет свое
определение, вместе с тем положено, есть некоторое другое определенное
количество; тем самым его возвращенность в себя и соотношение с собою дано
(ist) как имеющееся в его инобытии.

В основании этого отношения еще лежит внешний характер определенного
количества; здесь относятся друг к другу (т.~е. имеют свое соотношение с
самими собою в таком вне-себя-бытии) {\em безразличные}
определенные количества. Отношение есть тем самым лишь формальное единство
качества и количества. Диалектика отношения состоит в его переходе в их
абсолютное единство, в {\em меру}.

\subsection*{Примечание}

В нечто его граница как качество есть по существу его определенность. Но
если мы под границей понимаем количественную границу, и, например, поле
изменяет эту свою границу, то оно остается полем как до, так и после этого.
Напротив, если изменяется его качественная граница, то это изменяется та
его определенность, через которую оно есть поле, и оно становится лугом,
лесом и~т.~д. — Краснота, будь она более интенсивной или более слабой, есть
всегда краснота; но, если она изменяет свое качество, она перестает быть
краснотой, она становится синевой и~т.~д. — Определение
{\em величины} как определенного количества, как оно
(определение) получилось выше, определение, состоящее в том, что в
основании лежит некоторое бытие как пребывающее,
{\em безразличное к определенности, которой оно
обладает}, получается на любом другом примере.

Под словом «величина» разумеется, как в данных нами примерах,
{\em определенное количество}, а не количество, и
главным образом вследствие этого нам приходится употреблять это
заимствованное из чужого языка
название[ссылка!].

Дефиниция {\em величины}, даваемая в математике, также
имеет в виду определенное количество. Обыкновенно определяют величину как
нечто, могущее {\em увеличиваться} или
{\em уменьшаться}. Но увеличивать значит сделать нечто
более {\em великим}, а уменьшать —~сделать нечто менее
{\em великим}. В этом имеется
{\em отличие} величины вообще от нее же самой, и
величиной было бы, таким образом, то, величина чего может изменяться.
Дефиниция оказывается неподходящей, поскольку в ней применяется то самое
определение, которое должно быть дефинировано. Поскольку в ней нельзя
употреблять это же самое определение, «более» или «менее» должны быть
разложены на некоторое прибавление как утверждение и притом, согласно
природе определенного количества, равным образом внешнее утверждение, и на
некоторое убавление как некоторое тоже внешнее отрицание. В такой
{\em внешний} характер как реальности, так и отрицания
определяет себя вообще природа {\em изменения} в
определенном количестве. Поэтому и в вышеуказанном несовершенном выражении
нельзя не усмотреть того главного момента, в котором все дело, а именно,
безразличия изменения, так что в самом его понятии содержится его
собственное «меньше» и «больше», его безразличие к самому себе.

\chapter*{Первая глава. Количество.}

\paragraph[А. \ Чистое количество]{А. \ Чистое количество}
Количество есть снятое для-себя-бытие; отталкивающее одно, относившееся к
исключенному одному лишь отрицательно, теперь, перешедши в
{\em соотношение} с последним, относится тождественно к
другому и, стало быть, потеряло свое определение; для-себя-бытие перешло в
притяжение. Абсолютная неподатливость отталкивающего
{\em одного} растаяла, перешла в это
{\em единство}, которое, однако, как содержащее в себе
это одно, определено вместе с тем через внутреннее отталкивание,
{\em есть единство с самим собой}, как
{\em единство вне-себя-бытия}. Притяжение есть, таким
образом, момент {\em непрерывности} в количестве.

{\em Непрерывность} есть, следовательно, простое,
саморавное соотношение с собой, непрерываемое никакой границей и никаким
исключением, но она есть {\em не непосредственное}
единство, а единство для-себя-сущих одних. В ней еще содержится
{\em внеположность множественности}, но содержится
вместе с тем, как нечто неразличенное,
{\em непрерываемое}. Множественность положена в
непрерывности так, как она есть в себе; многие суть одно, что и другое,
каждое равно другому, и множественность есть поэтому простое, лишенное
различий равенство. Непрерывность представляет собой этот момент
{\em равенства} внеположности
{\em самой себе}, самопродолжение различенных одних в
их отличные от них.

Непосредственно поэтому величина в непрерывности имеет момент
{\em дискретности} —~отталкивание в том виде, в каком
оно теперь является моментом в количестве. — Непрерывность есть
саморавенство, но саморавенство многого, которое, однако, не становится
исключающим; только отталкивание впервые расширяет саморавенство до
непрерывности. Дискретность поэтому есть с своей стороны сливающаяся
дискретность, в которой ее одни имеют своей связью не пустоту, не
отрицательное, а свою собственную непрерывность и не прерывают во многом
этого равенства с самими собою.

Количество есть единство этих моментов, непрерывности и дискретности, но оно
сначала есть это единство {\em в форме} одного из них,
{\em непрерывности}, как результат диалектики
для-себя-бытия, которое сжалось в форму самой себе равной
непосредственности. Количество как таковое есть этот простой результат,
поскольку он еще не развил и не положил в нем [в самом себе] своих
моментов. — Оно {\em содержит} их ближайшим образом,
будучи положено как для-себя-бытие, как это последнее есть поистине. Это
для-себя-бытие было по своему определению снимающим себя соотнесением с
самим собою, вековечным выхождением вне себя. Но оттолкнутое есть оно же
само; отталкивание есть поэтому то, что порождает продолжающееся течение
самого себя. Благодаря тождественности отталкиваемого это порождение
дискретного (dies Diszernieren) есть непрерываемая непрерывность, а
благодаря выхождению вне себя эта непрерывность, не будучи прерываемой,
есть вместе с тем множественность, которая столь же непосредственно
остается в своем равенстве с самой собою.

\subsection*{Примечание 1. Представление чистого количества}

Чистое количество еще не имеет границы или, иначе говоря, оно еще не есть
определенное количество, а поскольку оно становится определенным
количеством, граница также не служит его пределом; оно, наоборот, именно и
состоит в том, что граница не служит для него пределом, что оно имеет
для-себя-бытие внутри себя как некоторое снятое. То обстоятельство, что
дискретность есть в нем момент, может быть выражено так, что количество
повсюду и безоговорочно есть {\em реальная возможность}
одного, но что также и обратно, одно столь же безоговорочно дано (ist) как
непрерывное.

Для чуждого понятию {\em представления} непрерывность
легко превращается в {\em складывание}, а именно во
{\em внешнее} соотношение одних друг с другом, в
котором одно сохраняет свою абсолютную неподатливость и исключение других
одних. Но рассмотрение одного показало, что оно само по себе переходит в
притяжение, в свою идеальность и что поэтому непрерывность не внешня для
него, а принадлежит ему самому и имеет свое основание в его сущности. За
эту-то {\em внешность} непрерывности для одних и
цепляется атомистика, и отказаться от нее представлению очень трудно. —
Напротив, математика отвергает ту метафизику, которая полагала, что время
{\em состоит} из временных точек, пространство вообще
или ближайшим образом линия —~из пространственных точек, поверхность —~из
линий, все пространство —~из поверхностей; она не допускает таких
дискретных одних. Если она и определяет, например, величину поверхности,
как {\em сумму} бесконечно многих линий, то она видит в
этой дискретности лишь представление, принимаемое на один момент, и в
представлении о {\em бесконечном} множестве линий уже
заключается снятость их дискретности, так как пространство, которое они
должны составлять, является ведь ограниченным.

{\em Спиноза}, которому было преимущественно важно
выяснение понятия чистого количества, имеет в виду противоположность этого
понятия голому представлению, когда он высказывается о количестве следующим
образом:

{\em Quantitas duobus modis a
nobis concipitur, abstracte scilicet sive superficialiter, prout nempe
ipsam imaginamur, vel ut substantia, quod а solo intellectu
fit. Si itaque ad quantitatem attendimus, prout in imaginatione est, quod
saepe et facilius a nobis fit, reperietur finita, divisibilis et ex
partibus conflata; si autem ad ipsam, prout in intellectu est, attendimus
et eam, quatenus substantia est, concipimus, — quod difficillime fit, —
infinila, unica et indivisibilis reperietur. Quod omnibus, qui inter
imaginationem et intellectum distinguere sciverint, satis manifestum
erit.}

[«Количество представляется нами двояким образом: абстрактно или
поверхностно, а именно, как мы его воображаем, или же как субстанция, что
может быть сделано только интеллектом. Таким образом, если мы рассматриваем
количество, как оно существует в воображении, что бывает часто и гораздо
легче, то мы находим его {\em конечным, делимым и
состоящим из частей}; если же мы рассматриваем его, как оно существует в
интеллекте, и представляем его как субстанцию, что очень трудно, то мы
находим его {\em бесконечным, единым и неделимым}. Это
будет достаточно ясно для каждого, кто умеет различать между воображением и
интеллектом». — «Этика», ч. I, теорема 15-я, схолия, подчеркнуто Гегелем].

Если потребуют, чтобы мы дали более определенные примеры чистого количества,
то укажем, что таковыми служат пространство и время, а также материя
вообще, свет и~т.~д. и даже «я»; только под количеством, как мы уже
заметили выше, не следует понимать определенного количества. Пространство,
время и~т.~д. суть протяжения, множества, которые суть выхождение вне себя,
истечение, не переходящее, однако, в противоположность, в качество или в
одно, а представляющее собою, как выхождение вне себя, вековечное
{\em самопродуцирование} своего единства.

Пространство есть то абсолютное {\em вне-себя-бытие},
которое столь же безоговорочно непрерывно, есть инобытие и все снова и
снова инобытие, тождественное с собою; время есть некое абсолютное
{\em выхождение вне себя}, некое порождение одного,
момента времени, «{\em теперь}», каковое порождение
непосредственно есть уничтожение этого «теперь» и в свою очередь
непрерывное уничтожение этого прехождения, так что это самопорождение
небытия есть вместе с тем простое равенство и тождество с собою.

Что касается {\em материи} как количества, то в числе
{\em семи теорем}, сохранившихся от первой
диссертации [ссылка!] {\em Лейбница} (1-я страница первого тома его
сочинений), есть одна (а именно вторая), гласящая следующим образом: Non
omnino improbabile est, materiam et quantitatem esse realiter idem
(Не совсем невероятно, что материя и количество суть в действительности
одно и то же). — И в самом деле, эти понятия отличаются друг от друга лишь
тем, что количество есть чистое определение мысли, а материя есть это же
определение мысли во внешнем существовании. — «Я» (dem Ich) также присуще
определение чистого количества, поскольку «я» есть абсолютное становление
другим, некоторое бесконечное удаление или всестороннее отталкивание к
отрицательной свободе для-себя-бытия, однако такое отталкивание, которое
остается безоговорочно простой непрерывностью, — непрерывностью всеобщности
или у-себя-бытия, не прерываемой бесконечно многообразными границами,
содержанием ощущений, созерцаний и~т.~д. — Что касается тех, которые
восстают против того, чтобы понимать {\em множество как
простое единство}, и кроме того {\em понятия}, что
каждое из многих есть то же самое, что и другое, а именно, одно из многих
(поскольку здесь не идет речь о далее определенном многом, о зеленом,
красном и~т.~д., а о многом, рассматриваемом само по себе), требуют еще,
чтобы им дали {\em представление} об этом единстве, то
они найдут такого рода представления, сколько пожелают, в тех
непрерывностях, которые дают в простом созерцании дедуцированное понятие
количества, как имеющееся налицо.

\subsection*{Примечание 2. Кантовская антиномия неделимости и бесконечной
делимости времени, пространства, материи}

С природой количества, заключающейся в том, что оно есть указанное простое
единство дискретности и непрерывности, находится в связи спор или
{\em антиномия} касательно
{\em бесконечной делимости} пространства, времени,
материи и~т.~д.

Эта антиномия состоит исключительно только в том, что показывает
необходимость утверждать как дискретность, так и непрерывность.
Одностороннее утверждение дискретности приводит к признанию бесконечной или
абсолютной {\em разделенности} и, следовательно, к
признанию некоторого неделимого как первоначала; одностороннее утверждение
непрерывности приводит, напротив, к признанию бесконечной
{\em делимости}.

Кантовская критика чистого разума выставляет, как известно,
{\em четыре} (космологических)
{\em антиномии}, из которых
{\em вторая} касается той
{\em противоположности}, которую составляют
{\em моменты количества}.

Эти кантовские антиномии навсегда останутся важной частью критической
философии; они преимущественно и привели к ниспровержению предшествующей
метафизики и могут быть рассматриваемы как главный переход к новейшей
философии, так как они в особенности способствовали возникновению убеждения
в ничтожности категорий конечности со стороны
{\em содержания}, а это представляет собою более
правильный путь, чем формальный путь субъективного идеализма, согласно
которому их недостаток заключается лишь в том, что они субъективны, а не в
том, что они суть в самих себе. Но при всей своей великой заслуге
кантовское изложение антиномий все-таки весьма несовершенно; отчасти оно в
самом себе страдает связанностью и сбивчивостью, отчасти же оно неправильно
в отношении вывода, который предполагает, что познание не имеет никаких
других форм мышления, кроме конечных категорий. — В обоих отношениях эти
антиномии заслуживают более пристальной критики, которая ближе осветит их
точку зрения и метод, равно как и освободит основной пункт, в котором вся
суть, от той ненужной формы, в которую он втиснут.

Прежде всего замечу, что Кант примененным им принципом деления, который он
заимствовал из своей схемы категорий, хотел придать своим четырем
космологическим антиномиям видимость полноты. Однако более глубокое
вникновение в антиномическую или, вернее, в диалектическую природу разума
показывает нам, что вообще {\em всякое} понятие есть
единство противоположных моментов, которым можно было бы, следовательно,
придать форму антиномических утверждений. Становление, наличное бытие
и~т.~д. и всякое другое понятие могли бы, таким образом, доставить нам свои
особые антиномии, и, стало быть, можно выставить столько антиномий, сколько
получается понятий. — Античный скептицизм не пожалел труда и обнаружил это
противоречие или эту антиномию во всех понятиях, которые он нашел в науках.

Далее мы должны сказать, что Кант берет антиномию не в самих понятиях, а в
уже {\em конкретной форме} космологических определений.
Чтобы получить антиномию в чистом виде и трактовать ее в ее простом
понятии, следовало бы рассматривать определения мысли не в их применении к
представлению о мире, пространстве, времени, материи и~т.~д. и в смешении с
такими представлениями, а без этого конкретного материала, не имеющего в
этом отношении силы и значения, следовало бы рассматривать их в чистом
виде, сами по себе, так как единственно лишь эти определения мысли
составляют сущность и основание антиномий.

Кант дает следующее понимание антиномий: они «суть не софистические
ухищрения, а противоречия, на которые разум необходимо должен (по
кантовскому выражению) {\em наталкиваться}»; это важный
взгляд. «После того как разум усмотрел основание естественной видимости
антиномий, он, хотя уже не вводится ею в обман, все же продолжает
испытывать
иллюзию»[ссылка!]. — Критическое разрешение
антиномий при помощи так называемой трансцендентальной идеальности мира
восприятий приводит только к тому результату, что превращает так называемое
противоречие (Widerstreit) в нечто {\em субъективное},
в котором оно, конечно, все еще остается той же видимостью,~т.~е. остается
столь же неразрешенным, как и раньше. Их истинное разрешение может состоять
только в том, что два определения, будучи противоположными друг другу и
необходимо присущими одному и тому же понятию, не могут быть значимы в их
односторонности, каждое само по себе, а имеют свою истину лишь в их
снятости, в единстве их понятия.

При ближайшем рассмотрении оказывается, что кантовские антиномии не содержат
в себе ничего другого, кроме совершенно простого категорического
утверждения {\em каждого} из двух противоположных
моментов некоторого определения, взятого сам по себе, в его
{\em изолированности} от другого. Но при этом указанное
простое категорическое или, собственно говоря, ассерторическое утверждение
запрятано в сложной сети превратных, запутанных рассуждений, благодаря чему
должна получиться видимость доказательства и должен прикрываться, сделаться
незаметным чисто ассерторический характер утверждения; это обнаружится при
ближайшем рассмотрении этих рассуждений.

Имеющая сюда отношение антиномия касается так называемой
{\em бесконечной делимости материи} и основана на
противоположности моментов непрерывности и дискретности, содержащихся в
понятии количества.

{\em Тезис} этой антиномии гласит в изложении Канта
следующим образом:

«{\em Всякая сложная субстанция в мире состоит из
простых частей, и нигде не существует ничего другого, кроме простого или
составленного из него}».

Здесь простому, атому, противопоставляется
{\em сложное}, что по сравнению с непрерывностью или
сплошностью является очень отсталым определением. Субстрат, данный Кантом
этим абстракциям, а именно субстанции в мире, не означает здесь ничего
другого, кроме вещей, как они доступны чувственному восприятию, и не
оказывает никакого влияния на характер самой антиномии; можно было бы с тем
же успехом взять пространство или время. — Так как тезис говорит лишь о
{\em составности}, вместо того чтобы говорить о
{\em непрерывности}, то он, собственно говоря, есть тем
самым аналитическое или {\em тавтологическое}
предложение. Что сложное или составное есть само по себе не
{\em одно}, а лишь сочетанное внешним образом и что оно
{\em состоит из другого}, это является его
непосредственным определением. Но другое составного есть простое. Поэтому
является тавтологией сказать, что сложное или составное состоит из
простого. — Если уже задают вопрос, {\em из чего}
состоит некое нечто, то требуют, чтобы указали
{\em некое другое}, {\em сочетание}
которого образует это нечто. Если говорят, что чернила опять-таки состоят
из чернил, то это означает, что не понят смысл вопроса о составленности из
другого; этот вопрос остался без ответа, его лишь еще раз повторяют. Затем
возникает дальнейший вопрос, а именно, {\em состоит ли}
вообще то, о чем идет речь, {\em из чего-то другого}
или же нет? Но составное есть несомненно нечто такое, что должно быть
сочетанным и состоять из другого. — Если принимают то простое, которое есть
другое составного, лишь за {\em относительно простое},
которое само по себе в свою очередь есть составное, то вопрос остается и
после ответа, как до него. Представлению предносится лишь то или другое
составное, относительно которого можно указать, что то или другое нечто
есть {\em его} простое, но само по себе это последнее
есть опять-таки составное. Но здесь речь идет о
{\em составном как таковом}.

Что касается теперь кантовского {\em доказательства}
тезиса, то оно, как и все кантовские доказательства прочих антиномических
положений, берет {\em окольный путь} доказательства
{\em от противного}, который, как увидим, совершенно
излишен.

«Предположите» (начинает он), что составные субстанции «не состоят из
простых частей; в таком случае, если бы была
{\em устранена} мысленно
{\em всякая} составность, то не существовало бы никакой
сложной части, а так как (согласно только что сделанному нами допущению) не
существует никаких простых частей, то не осталось бы также и никакой
простой части, следовательно, не осталось бы абсолютно ничего и, значит, не
было бы дано существование какой бы то ни было субстанции».

Этот вывод совершенно правилен. Если нет ничего, кроме сложного, и мы
отмыслим все сложное, то ничего не остается, — с этим надо согласиться, но
можно было бы прекрасно обойтись без всего этого тавтологического
излишества и сразу начать доказательство с того, что следует за этим, а
именно:

«Либо невозможно устранить мысленно всякую сложность, либо после ее
устранения должно оставаться нечто, существующее без сложности,~т.~е. нечто
простое».

«Но в первом случае сложное не состояло бы в свою очередь из субстанций
({\em так как у последних сложность есть случайное
отношение субстанций}\footnote{\textcyrillic{К излишеству в самом способе
доказательства присоединяется здесь еще излишество слов: «Так как
}{\em \textcyrillic{у последних}}
(\textcyrillic{именно, у субстанций) сложность есть лишь случайное
отношение
}{\em \textcyrillic{субстанций}}».}{\em ,
без которого последние должны пребывать как сами по себе устойчивые
существа})».

Так как этот случай «противоречит предположению, то остается возможным лишь
второе, а именно, что субстанциально сложное в мире состоит из простых
частей».

В скобки как бы мимоходом заключено то основание, которое здесь представляет
собою главное и в сравнении с которым предшествующее совершенно излишне.
Дилемма состоит в следующем: либо сложное есть пребывающее, либо не оно
есть пребывающее, а простое. Если бы пребывающим было первое, именно,
сложное, то пребывающее не было бы субстанциями,
{\em ибо для субстанций сложность есть лишь случайное
отношение}. Но субстанции есть пребывающее; стало быть, то, что пребывает,
есть простое.

Ясно, что можно было бы без окольного пути доказательства от противного дать
в качестве доказательства вышеуказанное основание, присоединив его
непосредственно к тезису, и непосредственно вслед за тезисом, гласящим:
«сложная субстанция состоит из простых частей», привести» в качестве
доказательства вышеуказанное основание и продолжать:
{\em ибо} сложность есть лишь
{\em случайное} отношение субстанций, которое для них,
следовательно, внешне и не касается самих субстанций. — Если правильно, что
сложность есть нечто случайное, то сущностью, конечно, оказывается простое.
Но эта случайность, в которой вся суть, не доказывается Кантом, а прямо
принимается им —~и притом мимоходом, в скобках —~как нечто само собою
разумеющееся или побочное. Конечно, само собою понятно, что сложность есть
внешнее и случайное определение. Но если вместо непрерывности имеется в
виду лишь случайная совместность, то не стоило труда выставлять по этому
поводу антиномию или, правильнее сказать, нельзя было выставить антиномию.
Утверждение о простоте частей в таком случае, как сказано, лишь
тавтологично.

Мы видим, стало быть, что на окольном пути доказательства от противного в
доказательстве имеется то самое утверждение, которое должно получиться как
вывод из него. Можно поэтому формулировать доказательство короче следующим
образом:

Предположим, что субстанции не состоят из простых частей, а суть лишь
сложные. Но ведь можно устранить мысленно всякую сложность (ибо она есть
лишь случайное отношение); следовательно, после ее устранения не осталось
бы никаких субстанций, если бы они не состояли из простых частей. Но
субстанции должны у нас оказаться, так как мы предположили, что они
существуют; у нас не должно все исчезнуть, а кое-что должно остаться, ибо
мы предположили существование некоего такого пребывающего, которое мы
назвали субстанцией; это нечто, следовательно, необходимо должно быть
простым.

Чтобы покончить полностью с этим доказательством, мы должны рассмотреть еще
и заключение. Оно гласит следующим образом:

«Из этого непосредственно {\em следует}, что все
решительно вещи мира суть простые сущности, — {\em что
сложность есть лишь их внешнее состояние} и что разум должен мыслить
элементарные субстанции как простые существа».

Здесь мы видим, что внешний характер,~т.~е. случайность сложности приводится
как {\em следствие} после того, как ранее она была
введена в доказательство в скобках и применялась там в качестве довода.

Кант очень настаивает на том, что в противоречивых положениях антиномий он
не ищет фокусов, чтобы, так сказать (как обыкновенно выражаются), дать
адвокатское доказательство. Рассматриваемое доказательство приходится
обвинять не столько в фокусничестве, сколько в бесполезной вымученной
запутанности, служащей лишь к тому, чтобы достигнуть внешнего вида
доказательства и помешать читателю заметить во всей его прозрачности то
обстоятельство, что то, что должно появиться как следствие, составляет в
скобках краеугольный камень доказательства, — что вообще здесь нет
доказательства, а есть лишь предположение.

{\em Антитезис} гласит:

{\em Никакая вещь в мире не состоит из простых частей, и
в нем вообще не существует ничего простого}.

{\em Доказательство} антитезиса тоже ведется от
противного и в другом роде оно столь же неудовлетворительно, как и
предыдущее.

«Предположите, — читаем мы, — что сложная вещь как субстанция состоит из
простых частей. Так как всякое {\em внешнее отношение}
и, значит, также и всякая сложенность из субстанций возможна лишь в
{\em пространстве}, то пространство должно состоять из
стольких же частей, из скольких состоит занимающее его сложное. Но
пространство состоит не из простых частей, а из пространств. Следовательно,
каждая часть сложного должна занимать некоторое пространство».

«Но безоговорочно-первые части всякого сложного просты».

«Следовательно, простое занимает некоторое пространство».

«Но так как всякое реальное, занимающее некоторое пространство, заключает в
себе находящееся друг вне друга многообразие и, стало быть, сложно и притом
сложено из субстанций, то простое оказалось бы сложным субстанциальным, что
противоречиво».

Это доказательство можно назвать целым {\em гнездом}
(употребляя встречающееся в другом месте выражение Канта) ошибочных
способов рассуждения.

Прежде всего оборот доказательства от противного есть ни на чем не
основанная видимость. Ибо допущение, что {\em все
субстанциальное пространственно}, {\em пространство же
не состоит из простых} частей, есть прямое утверждение, которое Кант делает
непосредственным основанием того, что требуется доказать, и при наличности
которого все доказательство уже готово.

Затем это доказательство от противного начинается с предложения, что «всякая
сложенность из субстанций есть {\em внешнее}
отношение», но странным образом Кант сейчас же снова его забывает. А
именно, далее Кант ведет свое рассуждение так, что сложное возможно лишь
{\em в пространстве}, а пространство не состоит из
простых частей; следовательно, реальное, занимающее некоторое пространство,
сложно. Если только допущено, что сложность есть внешнее отношение, то сама
пространственность (так же, как и все прочее, что может быть выведено из
определения пространственности), единственно лишь в которой якобы возможна
сложность, есть именно поэтому для субстанций внешнее отношение, которое их
совершенно не касается и не затрагивает их природы. Именно на этом
основании, не следовало бы принимать, что, субстанции помещены в
пространстве.

Здесь, далее, предполагается, что пространство, в которое здесь помещены
субстанции, не состоит из простых частей; ибо оно есть некоторое
созерцание, а именно, согласно кантовскому определению, представление,
которое может быть дано только лишь одним единственным предметом, а не так
называемое дискурсивное понятие. — Как известно, из этого кантовского
различения созерцания и понятия возникло весьма неподобающее обращение с
созерцанием, и, чтобы избавить себя от труда, связанного с достижением
{\em в понятиях} (Begreifen), стали расширять ценность
и область созерцания так, чтобы оно совпадало со всяким познанием. Здесь
требуется только принять, что пространство, как и само созерцание, должно
быть вместе с тем {\em постигнуто в понятиях}, если,
именно, хотят вообще постигать в понятиях. Таким образом, возник бы вопрос,
не должны ли мы мыслить пространство согласно его понятию как состоящее из
простых частей, хотя как созерцание оно представляет собою простую
непрерывность, или, иначе говоря, пространство оказалось бы пораженным той
же антиномией, которая приписывалась только субстанции. И в самом деле,
если антиномия мыслится абстрактно, то она, как было указано, касается
количества вообще и, следовательно, также и пространства и времени.

Но так как в доказательстве принимается, что пространство не состоит из
простых частей, то это должно было бы служить основанием для того, чтобы не
помещать простого в этот элемент, не соответствующий определению простого.
— Но при этом получается также коллизия непрерывности пространства со
сложностью. Кант смешивает их друг с другом, подставляет первую вместо
второй (это приводит в умозаключении к {\em Quaternio
terminorum}). У Канта ясно высказанным определением пространства служит то,
что оно есть «{\em единое} и части его основаны лишь на
ограничениях, так что они не предшествуют единому всеобъемлющему
пространству, не суть как бы его {\em составные части},
из которых его можно было бы {\em сложить}» (Kr. d. r.
Vern изд. 2-е, стр. 39). Здесь непрерывность очень правильно и определенно
приписана пространству в {\em противоположность}
сложенности из составных частей. Напротив, в аргументации выходит, что
помещение субстанций в пространство влечет за собою некоторое
«{\em находящееся} друг вне друга многообразие» и
притом, «следовательно, некоторое сложное». А между тем, как было указано,
способ, каким многообразие оказывается находящимся в пространстве,
исключает, по определенному высказыванию Канта, сложность этого
многообразия и предшествующие единству пространства составные части.

В примечании к доказательству антитезиса нарочито приводится еще кроме того
другое основное представление критической философии, что мы имеем
{\em понятие} о телах лишь как о
{\em явлениях}, но что как таковые они необходимо
предполагают пространство как условие возможности всякого внешнего явления.
Следовательно, если под субстанциями разумеются лишь тела, как мы их видим,
осязаем, вкушаем и~т.~д., то, собственно говоря, о том, что они суть в их
понятии, здесь и не поднимается речь; дело идет только о чувственно
воспринимаемом. Таким образом, нужно было бы формулировать доказательство
антитезиса коротко, а именно следующим образом: весь опыт нашего видения,
осязания и~т.~д. показывает нам лишь составное; даже самые лучшие
микроскопы и тончайшие измерители не {\em натолкнули}
нас на что-либо простое. Стало быть, разум и не должен желать натолкнуться
на нечто простое.

Следовательно, если мы пристальнее присмотримся к противоположности этих
тезиса и антитезиса и освободим их доказательства от всякого бесполезного
излишества и запутанности, то доказательство антитезиса содержит в себе
—~тем, что оно помещает субстанции в пространство —~ассерторическое
допущение {\em непрерывности}, равно как и
доказательство тезиса —~тем, что оно допускает составность как способ
соотношений субстанций —~содержит в себе ассерторическое допущение
{\em случайности этого соотношения} и тем самым
допущение, что субстанции суть {\em абсолютные одни}.
Вся антиномия сводится, следовательно, к разъединению и прямому утверждению
двух моментов количества и притом утверждению их как безоговорочно
раздельных. Взятые со стороны одной только
{\em дискретности}, субстанция, материя, пространство,
время и~т.~д. безоговорочно разделены; их принципом служит одно. Взятое-же
со стороны {\em непрерывности}, это одно есть лишь
некое снятое; деление остается делимостью, остается
{\em возможность} делить как возможность, никогда не
доводящая в действительности до атома. Если же мы остановимся на том
определении, которое дано в том, что было сказано выше об этих
противоположностях, то мы убедимся, что в самой непрерывности заключается
момент атома, так как она безоговорочно есть возможность деления, а равно,
что та деленность, дискретность упраздняет также всякое различие одних, —
ибо каждое из простых одних есть то же самое, что и другое, —
следовательно, содержит в себе также их одинаковость и, стало быть, их
непрерывность. Так как каждая из двух противоположных сторон содержит в
самой себе свою другую и ни одна из них не может быть мыслима без другой,
то из этого следует, что ни одно из этих определений, взятое отдельно, не
истинно, а истинно лишь их единство. Это есть истинно диалектический способ
рассмотрения этих определений, равно как и истинный результат.

Бесконечно более остроумными и глубокими, чем рассмотренная кантовская
антиномия, являются диалектические примеры древней
{\em элейской школы}, в особенности примеры, касающиеся
движения, которые равным образом основаны на понятии количества и в нем
находят свое разрешение. Рассмотрение здесь еще и их сделало бы наше
изложение слишком пространным; они касаются понятий пространства и времени
и могут быть обсуждены при рассмотрении последних и в истории философии. —
Они делают величайшую честь разуму их изобретателей; они имеют своим
{\em результатом} чистое бытие Парменида, так как они
показывают разложение всякого определенного бытия в нем самом и суть,
следовательно, сами в себе {\em течение} Гераклита. Они
поэтому и достойны более основательного рассмотрения, чем обычное
заявление, что это только софизмы; каковое утверждение держится за
эмпирическое восприятие, по примеру столь ясного для здравого человеческого
рассудка прецедента Диогена, который, когда какой-то диалектик вскрывал
перед ним противоречие, содержащееся в движении, не счел нужным напрягать
далее свой разум, а немым хождением взад и вперед указал на чувственную
очевидность; такое утверждение и опровержение, разумеется, легче выдвинуть,
чем углубиться в мысль, внимательно вдуматься в те затруднения, к которым
приводит мысль, и притом мысль, не притянутая откуда-нибудь издалека, а
формирующаяся в самом обыденном сознании, и затем разрешить эти затруднения
с помощью самой же мысли.

То разрешение этих диалектических построений, которое дает
{\em Аристотель}, заслуживает великой похвалы и
содержится в его истинно спекулятивных понятиях о пространстве, времени и
движении. Он противополагает бесконечной делимости (которая, — так как ее
представляют себе, как будто она осуществляется, — тождественна с
бесконечной разделенностью, с атомами), на которой основаны самые
знаменитые из этих доказательств, непрерывность, свойственную одинаково как
времени, так и пространству, так что бесконечная,~т.~е. абстрактная
множественность оказывается содержащейся в непрерывности лишь
{\em в себе}, {\em в возможности}.
Действительным по отношению к абстрактной множественности, равно как и по
отношению к абстрактной непрерывности, служит их конкретное, сами время и
пространство, как в свою очередь по отношению к последним —~движение и
материя. Абстрактное существует (ist) лишь в себе или лишь в возможности;
оно существует лишь как момент некоторого реального.
{\em Бейль}, который в своем «Dictionnaire» (статья
«Зенон») находит данное Аристотелем разрешение зеноновской диалектики
pitoyable (жалким), не понимает, что значит, что материя делима до
бесконечности только в возможности; он возражает, что если материя делима
до бесконечности, то она {\em действительно} содержит в
себе бесконечное множество частей; это, следовательно, не бесконечное en
puissance (в возможности), а такое бесконечное, которое существует реально
и актуально. — В противоположность Бейлю следует, наоборот, сказать, что
уже сама {\em делимость} есть лишь возможность,
{\em а не существование частей}, и множественность
вообще положена в непрерывности лишь как момент, как снятое. — Остроумного
рассудка, в котором Аристотель, несомненно, также никем не превзойден,
недостаточно для того, чтобы понять и оценить его спекулятивные понятия,
точно так же, как грубого чувственного представления, о котором мы
рассказали выше, недостаточно для того, чтобы опровергнуть аргументацию
Зенона. Этот рассудок заблуждается, принимая за нечто истинное и
действительное такие сочиненные мыслью вещи, такие абстракции, как
бесконечное множество частей; указанное же чувственное сознание нельзя
заставить перейти от эмпирии к мыслям.

Кантовское разрешение антиномии также состоит лишь в том, что разум не
должен {\em залетать} за пределы
{\em чувственного восприятия}, а должен брать явления
такими, каковы они есть. Это разрешение оставляет в стороне самое
содержание антиномии; оно не достигает природы
{\em понятия} ее определений, каждое из которых, взятое
само по себе, изолированно, не имеет никакой силы (nichtig ist) и есть само
в себе лишь переход в свое другое, имеет своим единством количество и в
этом единстве —~свою истину.

\paragraph[В. \ Непрерывная и дискретная величина]{В. \ Непрерывная и
дискретная величина}
\ 1. Количество содержит в себе оба момента —~непрерывность и дискретность.
Оно должно быть положено в обоих моментах как в своих определениях. — Оно
уже с самого начала есть {\em их непосредственное}
единство,~т.~е. само оно ближайшим образом положено лишь в одном из своих
определений, в непрерывности, и есть, таким образом,
{\em непрерывная величина}.

Или, иначе говоря, непрерывность есть, правда, один из моментов количества,
которое завершено лишь в соединении с другим моментом, с дискретностью.
Однако количество есть конкретное единство лишь постольку, поскольку оно
есть единство {\em различенных} моментов. Последние
следует поэтому брать также и как различенные; мы должны, однако, не снова
разрешить их в притяжение и отталкивание, а брать их согласно их истине,
каждый в его единстве с другим,~т.~е. так, что каждый остается
{\em целым}. Непрерывность есть лишь связное,
компактное единство как единство дискретного;
{\em положенное} так, оно уже не есть только момент, а
все количество, {\em непрерывная величина}.

2. {\em Непосредственное} количество есть непрерывная
величина. Но количество не есть вообще некоторое непосредственное.
Непосредственность —~это та определенность, снятостью которой является само
количество. Последнее следует, стало быть, положить в имманентной ему
определенности, которой является одно. Количество есть
{\em дискретная величина}.

Дискретность есть подобно непрерывности момент количества, но она же сама
есть также и все количество, именно потому, что она есть момент в
последнем, в целом и, следовательно, как различное не выступает из этого
целого, из своего единства с другим моментом. — Количество есть бытие
вне-друг-друга в себе, а непрерывная величина есть это бытие-вне-друг-друга
как продолжающее себя без отрицания, как в самой себе равная связь.
Дискретная же величина есть эта внеположность как не непрерывная, как
прерываемая. Однако с этим множеством одних у нас не получается снова
множество атомов и пустота, вообще отталкивание. Так как дискретная
величина есть количество, то сама ее дискретность непрерывна. Эта
непрерывность в дискретном состоит в том, что одни суть равное друг другу
или, иначе говоря, в том, что они обладают одной и той же
{\em единицей}. Дискретная величина есть,
следовательно, внеположность многих одних, {\em как
равных}, не многие одни вообще, а положенные как
{\em многие некоторой единой единицы}.

\subsection*{Примечание. Обычное разлучение этих величин}

В обычных представлениях о непрерывной и дискретной величинах не принимают
во внимание того обстоятельства, что {\em каждая} из
этих величин имеет в себе оба момента, как непрерывность, так и
дискретность, и их отличие друг от друга составляет только то, какой из
двух моментов есть {\em положенная} определенность и
какой есть только в-себе-сущая определенность. Пространство, время, материя
и~т.~д. суть непрерывные величины, так как они суть отталкивания от самих
себя, изливающееся исхождение из себя, которое вместе с тем не есть переход
или отношение к некоторому качественно другому. Они имеют абсолютную
возможность того, чтобы одно повсюду было положено в них, положено не как
пустая возможность простого инобытия (как, например, говорят, что возможно,
чтобы вместо этого камня стояло бы дерево), а они содержат принцип одного в
самих себе; он есть одно из определений, из которых они конституированы.

Равным образом и обратно, в дискретной величине не следует упускать из вида
непрерывность; этим последним моментом, как показано, служит одно как
единица.

Непрерывная и дискретная величины могут быть рассматриваемы как
{\em виды} количества, но лишь постольку, поскольку
величина положена не под какой-нибудь внешней определенностью, а под
{\em определенностями ее собственных} моментов. Обычный
переход от рода к виду вводит в первый —~согласно некоторому
{\em внешнему} ему основанию деления, —
{\em внешние} определения. Непрерывная и дискретная
величины при этом еще не суть определенные величины; они суть лишь само
количество в каждой из его двух форм. Их называют величинами постольку,
поскольку они имеют вообще то общее с определенным количеством, что они
суть некоторая определенность в количестве.

\paragraph[С. \ Ограничение количества]{С. \ Ограничение количества}
Дискретная величина имеет, {\em во-первых}, принципом
одно и есть, {\em во-вторых}, множество одних;
{\em в-третьих}, она по существу непрерывна, она есть
одно, вместе с тем как снятое, как {\em единица}, есть
продолжение себя как такового в дискретности многих одних. Она поэтому
положена как {\em единая} величина, и ее определенность
есть одно, которое есть в этой положенности и наличном бытии
{\em исключающее} одно, граница в единице. Дискретная
величина как таковая, как предполагается, непосредственно не ограничена: но
как отличная от непрерывной величины она дана как некоторое такое наличное
бытие и нечто, определенность которого есть одно, а как определенность в
некотором наличном бытии есть также первое отрицание и граница.

Эта граница, помимо того, что она соотнесена с единицей и есть отрицание
{\em в последней}, {\em соотнесена}
как одно также и {\em с самой собой}; таким образом,
она есть объемлющая, охватывающая граница. Граница здесь сначала не
отличается от «нечто» ее наличного бытия, а как одно, она непосредственно
есть сам этот отрицательный пункт. Но то бытие, которое здесь ограничено,
дано по существу как непрерывность, в силу которой оно выходит за границу и
за это одно, и безразлично к ним. Реальное дискретное количество есть,
таким образом, {\em некоторое} количество или, иначе
говоря, определенное количество —~количество как некоторое наличное бытие и
нечто.

Так как то одно, которое есть граница, объемлет собою многие одни
дискретного количества, то она полагает их в такой же мере и как снятые в
нем; она есть граница в непрерывности вообще как таковой, и тем самым
различие между непрерывной и дискретной величинами здесь безразлично; или,
правильнее, она есть граница непрерывности как
{\em одной}, так и {\em другой};
{\em обе} переходят в ней к тому, чтобы быть
определенными количествами.

\bigskip


\chapter*{Вторая глава. Определённое качество.}
Определенное количество "--- {\em ближайшим образом}
количество с некоторой определенностью или границей вообще "--- есть
в своей совершенной определенности {\em число}. Определенное
количество диференцируется (unterscheidet sich),

{\em во-вторых}, прежде всего
{\em на экстенсивное} определенное количество, в
котором граница дана (ist) как ограничение налично сущего
{\em множества}, а затем, когда это наличное бытие
переходит в для-себя-бытие, на {\em интенсивное}
определенное количество, {\em градус}\footnote{Grad
по-немецки степень интенсивности, градус. В большинстве
случаев мы предпочли передавать «Grad» не через «степень», а через
«градус», чтобы избежать смешения с «Potenz» (математическая степень), хотя
«градус», может быть, несколько затушевывает качественный характер
интенсивной величины. "--- {\em Перев}.}, которое, как «для
себя» и в последнем как {\em безразличная граница},
столь же непосредственно оказывается {\em вне себя},
имеет свою определенность в некотором другом. Как это положенное
противоречие, состоящее в том, что оно, таким образом, просто определено
внутри себя и вместе с тем имеет свою определенность вне себя и отсылает за
ней вне себя, определенное количество,

{\em в-третьих}, как в самом себе внешне положенное,
переходит в {\em количественную бесконечность}.

\paragraph[А. \ Число]{А. \ Число}
Количество есть определенное количество или, иначе говоря, обладает границей
и как непрерывная и как дискретная величина. Различие этих видов пока что
не имеет здесь никакого значения.

Количество как снятое для-себя-бытие уже само по себе безразлично к своей
границе. Но тем самым ему также и не безразлично то обстоятельство, что оно
имеет границу или, другими словами, что оно есть некоторое определенное
количество; ибо оно содержит внутри себя одно, абсолютную определенность,
как свой собственный момент, который, следовательно, как положенный в его
(количества) непрерывности или единице, есть его граница, остающаяся,
однако, одним, которым она теперь вообще стала.

Это одно есть, стало быть, принцип определенного количества, но одно
{\em как количественное одно}. Благодаря этому оно,
во-первых, непрерывно, оно есть {\em единица};
{\em во-вторых}, оно дискретно, представляет собою
в-себе-сущее (как в непрерывной величине) или положенное (как в дискретной
величине) множество одних, которые одинаковы друг с другом, обладают
вышеуказанной непрерывностью, имеют одну и ту же единицу.
{\em В-третьих}, это одно есть также и отрицание многих
одних как простая граница, есть некое исключение из себя своего инобытия,
определение себя по отношению к {\em другим}
определенным количествам. Постольку одно есть ($\alpha $)
{\em соотносящаяся с собою} ($\beta $),
{\em объемлющая и} ($\gamma $)
{\em исключающая другое} граница.

Определенное количество, полностью положенное в этих определениях, есть
{\em число}. Полная положенность заключается в наличном
бытии границы как {\em множества} и, стало быть, в ее
отличности от единицы. Число представляется поэтому дискретной величиной,
но оно обладает также и непрерывностью в виде единицы. Оно поэтому и есть
определенное количество в совершенной
{\em определенности}, так как в числе граница выступает
в виде определенного {\em множества}, имеющего своим
принципом одно,~т.~е. нечто безоговорочно определенное. Непрерывность, в
каковой одно есть лишь {\em в себе}, как снятое
(положенное как единица), есть форма неопределенности.

Определенное количество, лишь как таковое, ограничено вообще; его граница
есть его абстрактная, простая определенность. Но поскольку оно есть число,
эта граница положена как {\em многообразная в себе
самой}. Число содержит в себе те многие одни, которые составляют его
наличное бытие, но содержит их не неопределенным образом, а определенность
границы имеет место именно в нем; граница исключает другое наличное
бытие,~т.~е. другие многие, и объемлемые ею одни суть некоторое
определенное множество,
{\em численность}\footnote{Anzahl; перевод через «численность» дан Лениным, хотя и
предположительно и как один из возможных переводов (см. «Ленинский сборник»
IX, стр. 82). Дебольский переводит —~«определенное число»; в
томе I Собрания сочинений Гегеля. — «определенное
множество». — {\em Перев}.}, в отношении которой как
дискретности, как она есть в числе, другим служит
{\em единица}, его непрерывность.
{\em Численность и единица} составляют
{\em моменты} числа.

Что касается численности, то следует еще рассмотреть ближе, каким образом
многие одни, из которых она {\em состоит}, заключены в
границе. Относительно численности правильно выражаются, говоря, что она
{\em состоит} из многих, ибо одни находятся в ней не
как снятые, а {\em суть} в ней, только как положенные
вместе с исключающей границей, к которой они относятся безразлично. Но
граница не относится безразлично к ним. При рассмотрении нами наличного
бытия отношение к нему границы оказалось ближайшим образом таким, что
наличное бытие как утвердительное оставалось по сю сторону своей границы, а
последняя, отрицание, находилась вне его, у его края; точно так же во
многих одних прерыв их и исключение других одних выступает как некоторое
определение, которое имеет место вне объемлемых одних. Но там получился
вывод, что граница пронизывает наличное бытие, простирается столь же
далеко, как последнее, и что нечто вследствие этого ограничено по своему
определению,~т.~е. конечно. — В области числовой количественности мы
представляем себе, например, сто так, что только сотое одно ограничивает
многие таким образом, что они составляют сто. С одной стороны, это
правильно; но, с другой стороны, среди ста одних никакое из них не обладает
преимуществом, так как они только одинаковы; каждое из них есть в такой же
мере сотое, как и другие; все они, следовательно, принадлежат к той
границе, благодаря которой данное число есть сто; для получения своей
определенности последнее не может обойтись ни без одного из них; прочие
одни, следовательно, не составляют в сравнении с сотым одним такого
наличного бытия, которое находилось бы вне границы или лишь внутри ее,
вообще было бы отлично от нее. Численность не есть поэтому некоторое
множество в {\em противоположность} объемлющему,
ограничивающему одному, а сама составляет это ограничивание, которое есть
некоторое определенное количество; многие образуют одно число,
{\em одну} двойку, {\em один}
десяток, {\em одну} сотню и~т.~д.

Итак, ограничивающее одно есть определенность в отношении другого, отличение
данного числа от других. Но это отличение не становится качественной
определенностью, а остается количественным, имеет место лишь в сравнивающей
{\em внешней} рефлексии. Число как одно остается
обращенным назад к себе и безразличным к другим. Это
{\em безразличие} числа к другим есть его существенное
определение; оно составляет его {\em определенность в
себе}, но вместе с тем и {\em его собственную
внешность}. — Число есть, таким образом,
{\em нумерическое} одно как абсолютно определенное,
которое вместе с тем обладает формой простой непосредственности и для
которого поэтому соотношение с другим является совершенно внешним. Как
такое одно, которое есть {\em число}, оно, далее, имеет
{\em определенность} (поскольку она есть
{\em соотношение с другим}) как свои моменты внутри
самого себя, в своем {\em различии единицы и
численности}, и численность сама есть множество
{\em одних},~т.~е. в нем самом имеется этот абсолютно
внешний характер. — Это противоречие числа или определенного количества
вообще внутри себя есть то качество определенного количества, в дальнейших
определениях которого (качества) это противоречие получает свое развитие.

\subsection*{Примечание 1. Арифметические действия. Кантовские априорные
синтетические суждения созерцания}

Пространственная и числовая величина обыкновенно рассматриваются как два
различных вида величин, причем понимают это различие таким образом, что
пространственная величина, взятая сама по себе, есть столь же определенная
величина, как и числовая величина. Их различие состоит согласно этому
способу рассмотрения лишь в определениях непрерывности и дискретности, как
определенное же количество они стоят на одной ступени. Геометрия имеет,
говоря вообще, своим предметом в виде пространственных величин непрерывную
величину, а арифметика в виде числовых величин —~дискретную. Но вместе с
этой неодинаковостью предмета они также не обладают одинаковым способом и
совершенством ограничения или определенности. Пространственная величина
обладает лишь ограничением вообще; поскольку она должна рассматриваться как
безоговорочно определенное количество, она нуждается в числе. Геометрия как
таковая не {\em измеряет} пространственных фигур, не
есть искусство измерения, она лишь {\em сравнивает} их.
В даваемых ею дефинициях определения также отчасти заимствуются ею из
{\em равенства} сторон, углов, из
{\em равного} расстояния. Так, например, круг,
основывающийся единственно только на {\em равенстве}
расстояния всех возможных в нем точек от одной центральной точки, не
нуждается для своего определения ни в каком числе. Эти определения,
основывающиеся на равенстве или неравенстве, суть подлинно геометрические.
Но их недостаточно, и для определения других фигур, например, треугольника,
четырехугольника требуется число, заключающее в своем принципе, в одном,
самостоятельную определяемость, а не определяемость с помощью чего-то
другого, стало быть, не определяемость через сравнение. Пространственная
величина имеет, правда, в точке определенность, соответствующую одному;
однако точка, поскольку она выходит вне себя, превращается в другое,
становится линией; так как она есть по существу лишь одно
{\em пространства}, то она в
{\em соотношении} становится некоторой такой
непрерывностью, в которой снята точечность, самостоятельная определяемость,
одно. Поскольку самостоятельная определяемость должна сохраниться во
вне-себя-бытии, приходится представлять себе линию как некоторое множество
одних, и она должна получить внутри себя {\em границу},
определение {\em многих} одних,~т.~е. мы должны брать
величину линии —~и точно так же и других пространственных определений —~как
число.

{\em Арифметика} рассматривает число и его фигуры, или,
вернее сказать, не рассматривает их, а оперирует с ними. Ибо число есть
безразличная, косная определенность; оно должно быть приведено в действие и
в соотношение {\em извне}. Способы этого соотнесения
суть виды {\em арифметических действий}. Они излагаются
в арифметике одно после другого, и ясно, что одно
{\em действие} зависит от другого. Однако в арифметике
не выделяется руководящая нить их последовательности. Но из самого
определения понятия числа легко получается тот систематический порядок, на
который справедливо притязает изложение этих элементов в учебниках. На эти
руководящие определения мы должны здесь вкратце обратить внимание читателя.

Число есть вообще вследствие своего принципа, одного, некое внешне
сочетанное, всецело аналитическая фигура, в которой нет никакой внутренней
связи. Так как оно, таким образом, есть лишь некое порожденное извне, то
всякое исчисление есть продуцирование чисел,
{\em считание} или, говоря более определенно,
{\em сосчитывание}. Разница в этом внешнем
продуцировании, совершающем постоянно лишь одно и то же, может заключаться
единственно только в различии по отношению друг к другу тех чисел, которые
должны быть сосчитываемы; такое различие само должно быть заимствовано из
чего-то иного и из внешнего определения.

Качественным различием, составляющим определенность числа, является то, с
которым мы познакомились, а именно, различие
{\em единицы} и {\em численности};
к этому различию сводится поэтому всякая определенность понятия, могущая
иметь место в арифметических действиях. Различие же, присущее числам как
определенным количествам, есть внешнее тождество и внешнее различие,
{\em равенство} и
{\em неравенство}, которые суть рефлективные моменты и
должны быть рассматриваемы среди определений сущности там, где трактуется о
различии.

Далее, нужно предпослать еще то замечание, что числа могут в общем быть
произведены двояким образом, либо через присовокупление (Zusammenfassen),
либо через разъединение уже присовокупленных; поскольку этот двоякий способ
произведения чисел имеет место при одинаковым образом определенном виде
считания, то совокуплению чисел (это можно назвать
{\em положительным} видом исчисления) соответствует
некоторое определенное разъединение их (это можно назвать
{\em отрицательным} видом исчисления), причем само
определение вида исчисления независимо от этой противоположности.

После этих замечаний переходим к указанию видов исчисления.

1. Первым порождением числа является совокупление многих как таковых,~т.~е.
многих, каждое из которых положено лишь как {\em одно},
— {\em нумерация}. Так как одни внешни друг другу, то
они представляются под каким-нибудь чувственным образом и операция,
посредством которой порождается число, есть сосчитывание по пальцам, по
точкам и~т.~п. Что такое четыре, пять и~т.~д., это может быть лишь
{\em показано}. Остановка в счете, будет ли совокуплено
столько-то одних или иное их количество, есть нечто случайное,
произвольное, так как граница внешня. — Различие численности и единицы,
появляющееся в дальнейшем развитии видов исчисления, служит основой
{\em системы} чисел —~двоичной, десятеричной и~т.~д.;
такая система покоится в общем на произвольном выборе той численности,
которая постоянно должна снова и снова быть взята как единица.

Возникшие посредством нумерации {\em числа} снова
подвергаются нумерации; и поскольку они положены так непосредственно, они
еще определены без всякого соотношения друг с другом, безразличны к
равенству и неравенству, их величины по отношению друг к другу случайны;
они поэтому вообще {\em неравны}; это
—~{\em сложение}. — Что 7 и 5 составляют 12, это узнают
таким образом, что к 7 принумеровывают на пальцах или как-нибудь иначе еще
5 одних, результат каковой операции сохраняют затем в памяти, помнят
{\em наизусть}, ибо здесь нет ничего внутреннего. И
точно так же узнают посредством сосчитывания на пальцах и~т.~д., что
$7 \times 5 = 35$; знают это тем путем, что к одной семерке
принумеровывается еще
одна семерка, повторяют это пять раз, и полученный результат также
запоминается наизусть. С трудом этого нумерирования, нахождения сумм,
произведений навсегда покончено готовой таблицей сложения или умножения,
которую нужно лишь заучить наизусть.

Кант рассматривает (во Введении к «Критике чистого разума», раздел V)
предложение: $7 + 5 = 12$ как синтетическое предложение. «Можно было бы, —
говорит он, — сначала, правда, подумать (конечно!), что это чисто
аналитическое предложение, вытекающее, согласно закону противоречия, из
{\em понятия суммы} пяти и семи». Понятие суммы ничего
более не означает, кроме того абстрактного определения, что эти два числа
{\em должны} быть совокуплены и притом как числа
внешним,~т.~е. чуждым понятию образом,~т.~е. означает, что начиная с 7
следует продолжать нумерацию до тех пор, пока не будут исчерпаны
долженствующие быть прибавленными одни, численность которых определена
числом 5; полученный результат носит уже заранее известное название
двенадцати. «Однако, — продолжает Кант, — при ближайшем рассмотрении мы
находим, что понятие суммы 7 и 5 ничего более не содержит в себе, кроме
{\em соединения} этих двух чисел в одно единственное,
чем вовсе еще не {\em мыслится},
{\em каково} это единственное число, соединяющее в себе
те два числа». «Сколько бы я ни расчленял свое понятие о таковой возможной
сумме, я все-таки не встречу в нем двенадцати». С
{\em мышлением} суммы, с расчленением понятия, переход
от указанной задачи к получающемуся результату в самом деле не имеет ничего
общего. «Нужно выйти за пределы этих {\em понятий},
прибегнуть к помощи созерцания, пяти пальцев и~т.~д. и, таким образом,
присоединить эти единицы {\em данных в созерцании} пяти
к {\em понятию} семи», — прибавляет он. Пять в самом
деле дано в созерцании,~т.~е. есть совершенно
{\em внешняя} сочетанность произвольно повторявшейся
мысли, одного; но 7 есть столь же мало понятие; здесь нет понятий, за
пределы которых нужно было бы выходить. Сумма 7 и 5 означает чуждое понятию
соединение этих двух чисел; это столь чуждое понятию нумерирование,
продолжающееся от 7 до тех пор, пока не будут исчерпаны пять единиц, можно
назвать сочетанием, синтезированием ровно с таким же правом, как и
нумерацию, начинающую с одного, — синтезированием, которое, однако, носит
совершенно аналитический характер, так как связь здесь всецело
искусственная, в ней нет ничего такого и в нее не привходит ничего такого,
что не наличествовало бы перед нами совершенно внешним образом. Требование
сложить 7 с 5 относится к требованию считать вообще как требование
продолжить прямую линию к требованию провести прямую линию.

Таким же бессодержательным, как выражение «синтезирование», является
определение, что это синтезирование совершается à priori. Правда, считание
не есть определение, принадлежащее области ощущений, которые согласно
кантовскому определению созерцания единственно только и остаются на долю à
posteriori, и считание есть несомненно операция, совершающаяся на почве
абстрактного созерцания,~т.~е. такого созерцания, которое определено
категорией одного и при котором абстрагируются как от всяких прочих
определений, принадлежащих области ощущения, так и от понятий. «À priori»
есть вообще нечто лишь смутное. Определение, принадлежащее области эмоций
—~влечение, склонность и~т.~д., в такой же мере имеет в себе момент
априорности, а пространство и время как существующие,~т.~е. временное и
пространственное, определены также и à posteriori.

В связи с этим мы можем прибавить, что в утверждении Канта о синтетическом
характере основоположений чистой геометрии также нет ничего основательного.
Признавая, что многие из них действительно аналитичны, он в доказательство
представления о синтетичности других приводит только ту аксиому, что прямая
линия есть кратчайшее расстояние между двумя точками. «А именно в моем
{\em понятии} о прямой не содержится никакая величина,
а содержится только качество; {\em понятие} о
кратчайшем расстоянии всецело, следовательно, привходит извне, и никаким
расчленением не может быть извлечено из {\em понятия
прямой линии}; здесь, следовательно, приходится брать себе в помощь
{\em созерцание} единственно лишь посредством которого
возможен синтез». — Но здесь дело вовсе и не идет о понятии прямого вообще,
а о прямой линии, последняя же есть уже нечто пространственное,
созерцаемое. Определение (или, если угодно, понятие) прямой линии ведь и
состоит ни в чем другом, как в том, что она есть
{\em безоговорочно} простая линия,~т.~е. в том, что в
своем выхождении вне себя (в так называемом движении точки) она
безоговорочно соотносится с собою, что в ее протяжении не положено никакой
разницы определения, никакого соотношения с некоторой другой точкой или
линией вне ее; она есть {\em безоговорочно простое
внутри себя направление}. Эта простота есть, разумеется, ее качество, и
если кажется, что трудно дать аналитическую дефиницию прямой линии, то это
происходит лишь из-за определения простоты или соотношения с самой собой и
только потому, что при операции определения рефлексия прежде всего имеет
преимущественно в виду некую множественность, операцию определения через
другое. Но само по себе нисколько не трудно понять это определение простоты
протяжения внутри себя, отсутствие в последнем определения через другое.
Дефиниция Эвклида не содержит в себе ничего другого, кроме этой простоты. —
Но переход этого качества в количественное определение (кратчайшего
расстояния), который якобы составляет синтез, исключительно и всецело
аналитичен. Линия как пространственная есть количество вообще; простейшим,
что можно сказать об определенном количестве, является
«{\em наименьшее}», а это последнее, высказанное о
линии, есть «{\em кратчайшее}». Геометрия может брать
эти определения как следствия из дефиниции; но
{\em Архимед} в своих книгах о шаре и цилиндре (см.
перев. {\em Гаубера}, стр. 4) поступил всего
целесообразнее, выставив указанное определение прямой линии как аксиому; он
делает это в таком же правильном смысле, в каком
{\em Эвклид} поставил в числе аксиом определение,
касающееся параллельных линий, так как развитие этого определения для того,
чтобы оно стало дефиницией, также потребовало бы не непосредственно
принадлежащих пространственности, а более абстрактных качественных
определений (подобно тому как только что в применении к прямой линии
потребовалось такое определение как простота), — одинаковости направления
и~т.~п. Эти древние сообщили также и своей науке пластический характер, их
изложение строго держалось своеобразия ее материи и поэтому исключало из
себя все, что было бы ему гетерогенно.

Понятие, которое Кант выставил в своем учении о синтетических суждениях à
priori, — понятие о {\em различном}, которое также и
{\em неотделимо} друг от друга, о
{\em тождественном}, которое в самом себе
{\em нераздельно} есть
{\em различие}, принадлежит к тому, что есть великого и
бессмертного в его философии. В созерцании это понятие, разумеется, также
имеется, так как это понятие есть само понятие, и всё есть в себе понятие;
но те определения, которые выделены в приведенных примерах, не выражают
его; число и считание есть, напротив, такое тождество и продуцирование
такого тождества, которое безоговорочно есть лишь внешнее тождество, лишь
поверхностный синтез, единство единиц, таких единиц, которые, напротив,
положены как в самих себе не тождественные друг с другом, а внешние, сами
по себе раздельные. В прямой линии в основании определения, что она есть
кратчайшее расстояние между двумя точками, должен лечь скорее лишь момент
абстрактного тождества, лишенного различия в нем самом.

Я возвращаюсь от этого отступления к самому сложению. Соответствующий ему
отрицательный вид исчисления, {\em вычитание}, есть в
свою очередь совершенно аналитическое отделение чисел, которые, как и в
сложении, определены лишь как вообще {\em неравные} в
отношении друг друга.

2. Ближайшим определением является {\em равенство} тех
чисел, над которыми должно быть произведено действие нумерации. Благодаря
этому равенству эти числа суть {\em единицы}, и в числе
появляется различие единицы и {\em численности}.
{\em Умножение} имеет задачей сосчитать воедино
численность таких единиц, которые сами суть тоже численности. При этом
безразлично, какое из двух чисел принимается за единицу и какое за
численность, безразлично, говорим ли мы четырежды три, где четыре есть
численность, а три —~единица, или, наоборот трижды четыре. — Мы уже указали
выше, что первоначальное нахождение произведения совершается посредством
простого процесса нумерации,~т.~е. сосчитывания на пальцах и~т.~д.;
получающаяся позднее способность {\em непосредственно}
указать произведение покоится на собрании таких произведений, на таблице
умножения и знании ее наизусть.

{\em Деление} есть отрицательный вид исчисления согласно
тому же определению различия. Здесь также безразлично, какой из двух
факторов, делитель ли или частное, мы примем за единицу и какой за
численность. Делитель принимается за единицу, а частное —~за численность,
когда задачей деления объявляется желание узнать,
{\em сколько раз} (численность) некоторое
{\em одно} число (единица) содержится в данном числе;
наоборот, делитель принимается за численность, а частное за единицу в том
случае, когда говорят, что требуется разделить некоторое число на данную
численность одинаковых частей и найти величину такой части (единицы).

3. Те два числа, которые определены одно относительно другого как единица и
численность, еще непосредственны относительно друг друга и потому вообще
{\em неравны}. Дальнейшим равенством служит равенство
самих единицы и численности; таким образом, поступательное движение к
равенству тех определений, которые заключаются в определении числа,
завершено. Считание согласно этому полному равенству есть
{\em возведение в степень} (отрицательным видом
исчисления служит здесь извлечение корня) и притом ближайшим образом
возвышение в {\em квадрат}; это —~полная определенность
нумерирования внутри самого себя, где (1) те многие числа, которые
слагаются, суть одни и те же, и (2) само их множество или численность
тождественно тому числу, которое берется многократно и служит единицей. Нет
никаких иных определений в понятии числа, которые могли бы представлять
собою некоторое различие, и не может также иметь место никакое дальнейшее
выравнивание того различия, которое заключается в числе. Возведение в
степени высшие, чем вторая, есть {\em формальное}
продолжение, которое отчасти —~при четных показателях —~есть лишь
{\em повторение} возведения в квадрат, и отчасти —~при
нечетных показателях —~есть появление снова прежнего неравенства; а именно,
при формальном равенстве (например, ближайшим образом в кубической степени)
нового множителя как с численностью, так и с единицей, он как единица
представляет собою нечто неравное по отношению к численности (по отношению
к второй степени, 3 —~по отношению к 3.3); еще большее неравенство имеется
при кубической степени четырех, где численность (3), показывающая, сколько
раз число, служащее единицей, должно быть помножено само на себя, отлична
от последнего. — Эти определения имеются сами по себе, как существенное
различие понятия, — численность и единица, и для того, чтобы выход из себя
вполне возвратился в себя, они должны быть выравнены. В только что
изложенном заключается, далее, основание, почему, с одной стороны, решение
уравнений высших степеней должно состоять в приведении их к квадратным
уравнениям и, почему, с другой стороны, уравнения нечетных степеней могут
быть определены лишь формально, и как раз, когда корни рациональны, они
могут быть найдены не иначе, как при помощи мнимых выражений,
представляющих собою противоположность того, что суть и выражают собою
корни. — Согласно сказанному единственно только арифметический квадрат
содержит в себе всецелую определенность, вследствие чего уравнения
дальнейших формальных степеней должны быть приведены к нему; точно так же,
как в геометрии прямоугольный треугольник содержит в себе всецелую
определенность внутри себя, выраженную в пифагоровой теореме, и поэтому для
полного определения всех прочих геометрических фигур они должны быть
приведены к нему.

В преподавании, подвигающемся вперед согласно логически построенному
суждению, изложение учения о степенях предшествует изложению учения о
пропорциях; последние, правда, примыкают к различию единицы и численности,
составляющему определение второго вида исчисления, однако они выходят за
пределы того одного, которое есть одно
{\em непосредственного} определенного количества, в
котором единица и численность суть лишь моменты; дальнейшее определение по
этим моментам[ссылка!] остается для него
самого также еще внешним. Число в отношении не есть уже
{\em непосредственное} определенное количество;
последнее имеет тогда свою определенность как опосредствование;
количественное[ссылка!] отношение мы рассмотрим
далее.

О вышеуказанном все дальнейшем и дальнейшем определении арифметических
действий можно сказать, что оно не есть философствование о них, не есть,
скажем, изложение их внутреннего значения, потому что оно на самом деле не
есть имманентное развитие понятия. Но философия должна уметь различать то,
что по своей природе есть внешний самому себе материал, должна знать, что в
таком материале поступательное движение понятия может происходить лишь
внешним образом и что моменты этого движения могут иметь бытие лишь в
своеобразной форме их внешности, какова здесь форма равенства и
неравенства. Различение сфер, к которым принадлежит та или другая
определенная форма понятия,~т.~е. сфер, в которых она имеется как
существование, служит существенным условием философствования о реальных
предметах, необходимым для того, чтобы мы идеями не нарушали своеобразия
внешнего и случайного и чтобы мы вместе с том не искажали этих идей и не
делали их формальными вследствие несоответственности материала. Но тот
внешний характер, который носит выступление моментов понятия в
вышеуказанном внешнем материале, в числе, есть здесь соответственная форма;
так как они представляют нам предмет в присущем ему смысле, а также ввиду
того, что они не требуют никакого спекулятивного подхода и потому
представляются легкими, они заслуживают применения в элементарных
учебниках.

\subsection*{Примечание 2. Употребление числовых определений для
выражения философских понятий}

Как известно, {\em Пифагор} изображал в
{\em числах разумные отношения} или
{\em философемы}, да и в новейшее время философия
применяла числа и формы их соотношений; как, например, степени и~т.~п.,
чтобы регулировать согласно им или выражать в них мысли. — В педагогическом
отношении число признавалось наиболее подходящим предметом внутреннего
созерцания, а занятие вычислением его отношений —~деятельностью духа, в
которой он делает наглядными свои наисобственнейшие отношения и вообще
основные отношения сущности. — В какой мере числу на самом деле принадлежит
эта высокая ценность, видно из его понятия, каким оно получилось выше.

Число обнаружилось для нас как абсолютная определенность количества, а его
стихия —~как ставшее безразличным различие; оно оказалось определенностью в
себе, которая вместе с тем положена лишь вполне внешне. Арифметика есть
аналитическая наука, так как все встречающиеся в ее предмете связи и
различия не зависят от него самого, а учинены ему совершенно извне. Она не
имеет конкретного предмета, который содержал бы в себе внутренние
отношения, первоначально скрытые для знания, не данные в непосредственном
представлении о нем, а долженствующие быть выявлены усилиями познания. Она
не только не содержит в себе понятия и, следовательно, задачи для
постигающего в понятиях (für das begreifende) мышления, но есть его
противоположность. Вследствие безразличия приведенного в связь к связи,
которой недостает необходимости, мышление оказывается здесь занятым
деятельностью, которая вместе с тем есть самое крайнее отчуждение от самого
себя, занятым насильственной деятельностью, заключающейся в том, что оно
{\em движется} в сфере
{\em безмыслия} и приводит в связь то, что не способно
носить характер необходимости. Предметом здесь служит абстрактная мысль о
{\em внешности} как таковой.

Как таковая {\em мысль} о носящем характер внешности,
число есть вместе с тем абстракция от чувственного многообразия; от
чувственного оно ничего другого не сохранило, кроме абстрактного
определения внешности; благодаря этому в числе чувственное наивозможно
ближе подведено к мысли. Число есть {\em чистая мысль}
о самоотчуждении мысли.

Поднимающийся над чувственным миром и познающий свою сущность дух, ища
стихии для своего чистого {\em представления}, для
{\em выражения своей сущности}, может поэтому раньше,
чем он постигнет, что сама мысль является этой стихией и обретет для ее
изображения чисто духовное выражение, вздумать избрать для этого
{\em число}, эту внутреннюю, абстрактную внешность.
Поэтому мы находим в истории науки, что уже рано применяли число для
выражения философем. Оно составляет последнюю ступень того несовершенства,
которое состоит в том, что всеобщее берется как обремененное чувственным.
Древние определенно сознавали, что число находится посередине между
чувственным и мыслью. Согласно Аристотелю («Метафизика», 1, 5)
{\em Платон} говорил, что помимо чувственного и идей,
посередине между ними, находятся математические определения вещей; от
чувственного они отличаются тем, что они невидимы (вечны) и неподвижны, а
от идей —~тем, что они суть некоторое множественное и сходное, между тем
как идея лишь безоговорочно тождественна с собою и едина внутри себя. —
Более подробное, основательно продуманное рассуждение об этом Модерата
Кадиксского приводится в «Malchi vita Pythagorae ed. Ritterhus», стр. 30 и
сл.; то обстоятельство, что пифагорейцам пришла в голову мысль остановиться
на числах, он приписывает тому, что они еще не были в состоянии
{\em ясно} постигнуть в
{\em разуме} основные идеи и первые начала, потому что
трудно продумать и выразить эти начала; числа хорошо служат для обозначения
при преподавании; пифагорейцы, между прочим, подражали в этом геометрам,
которые, не умея выражать телесное в мысли, применяют фигуры и говорят, что
{\em это} —~треугольник, причем они требуют, чтобы не
принимали за треугольник лежащий перед глазами чертеж, а лишь представляли
себе с помощью последнего мысль о треугольнике. Так, например, пифагорейцы
выразили как {\em одно} мысль о единстве,
тождественности и равенстве, а также и основание гармонии, связи и
сохранения всего, основание тождественного с самим собою и~т.~д. — Излишне
заметить, что пифагорейцы перешли от выражения в числах также и к выражению
в мыслях, к определенно названным категориям равного и неравного, границы и
бесконечности; уже касательно вышеуказанного выражения в числах сообщается
(там же, в примечаниях к стр. 31 цитированного издания, взятых из «Жизни
Пифагора» у Фотия, стр. 722), что пифагорейцы различали между монадой и
единицей; монаду они принимали за мысль, а единицу за число; и точно так же
и число два они принимали за арифметическое выражение, а диаду (ибо таково,
видимо, то название, которое оно у них носит) —~за мысль о неопределенном.
— Эти древние, во-первых, очень ясно усматривали неудовлетворительность
числовой формы для выражения определений мысли, и столь же правильно они,
далее, требовали настоящего выражения мысли вместо того первого выражения,
являвшегося только крайним выходом; насколько опередили они в своих
размышлениях тех, которые в наше время снова считают чем-то похвальным и
даже основательным и глубоким замену определений мысли самими числами и
такими числовыми определениями, как, например, степенями, а затем
—~бесконечно большим, бесконечно малым, единицей, деленной на
бесконечность, и прочими подобного рода
определениями[ссылка!], которые даже и сами по
себе часто представляют собою превратный математический формализм, —
считают основательным и глубоким возвращение к вышеупомянутому беспомощному
детству.

Что касается вышеприведенного выражения, что число занимает промежуточное
положение между {\em чувственным} и мыслью, имея вместе
с тем то общее с первым, что оно есть по своей природе (an ihr)
{\em многое}, внеположное, то следует заметить, что
само это многое, принятое в мысль чувственное, есть принадлежащая ей
(мысли) категория внешнего в самом себе. Дальнейшие, конкретные, истинные
{\em мысли}, представляющие собою наиболее живое,
наиболее подвижное, только {\em соотнесением и
занимающееся}, превращаются в мертвенные, неподвижные определения, когда их
перемещают в эту стихию вне-себя-бытия. Чем богаче становятся мысли
определенностью и тем самым также и соотношением, тем более запутанным, с
одной стороны, а с другой стороны, тем более произвольным и лишенным смысла
становится их изложение в таких формах, как числа. Единица, два, три,
четыре, генада или монада, диада, триада, тетрактис, еще близки к
совершенно {\em простым},
{\em абстрактным} понятиям; но когда числа должны
переходить к изображению конкретных отношений, тогда оказывается тщетным
стремление сохранить связь между ними и понятием.

А когда для характеристики движения понятия, благодаря каковому движению оно
только и есть понятие, обозначают определения мысли через одно, два, три,
четыре, то этим предъявляется к мышлению самое жестокое требование.
Мышление движется тогда в стихии своей противоположности, отсутствия
соотношений. Его дело становится тогда работой безумия. Постигнуть,
например, что одно есть три, а три —~одно, потому так трудно, что одно есть
нечто лишенное соотношений и, следовательно, не являет в самом себе того
определения, посредством которого оно переходило бы в свою
противоположность, а, напротив, состоит именно в полном исключении такого
рода соотношения и отказе от него. Рассудок, наоборот, пользуется этим
против спекулятивной истины (например, против истины, заключающейся в
учении, называемом учением о триединстве) и
{\em перечисляет} те определения последней, которые
составляют одно единство, чтобы выставить ее как явную бессмыслицу,~т.~е.
он сам впадает в бессмыслицу, превращая в лишенное соотношений то, что
всецело есть соотношение. Слово «триединство» употребляется, конечно, не в
расчете на то, что рассудок будет рассматривать
{\em одно} и число как
{\em существенную} определенность содержания. Это слово
выражает собою презрение к рассудку, который в своем тщеславии, однако,
упорно держится одного и числа как таковых и выставляет это тщеславие как
оружие против разума.

Принимать числа, геометрические фигуры просто за
{\em символы}, как это часто проделывали с кругом,
треугольником и~т.~д. (круг, например, принимался за символ вечности,
треугольник —~за символ триединства), есть, с одной стороны, нечто
совершенно невинное; но нелепо, с другой стороны, предполагать, что этим
выражают нечто большее, чем то, что {\em мысль}
способна {\em постигнуть и выразить}. Если в таких
символах, как и в других, создаваемых {\em фантазией} в
народной мифологии и вообще в поэзии, в сравнении с которыми чуждые
фантазии геометрические фигуры сверх того скудны, — если в самом деле и в
первых и в последних заключается глубокая мудрость, глубокое
{\em значение}, но это как раз исключительно задача
мышления сделать явной мудрость, которая в них лишь
{\em сокрыта} (darin liegt) и притом сокрыта не только
в них, но и в природе и духе. В символах истина в силу чувственного
элемента еще {\em помутнена} и
{\em закутана}; она вполне открывается сознанию только
в форме мысли; их {\em значением} служит лишь сама
мысль.

Но заимствование математических категорий с целью что-нибудь определить
относительно метода или содержания философской науки уже потому оказывается
по существу чем-то превратным, что, поскольку математические формулы
обозначают мысли и различия понятия, это их значение, наоборот, должно быть
сначала указано, определено и оправдано в философии. В своих конкретных
науках последняя должна почерпать логическое из логики, а не из математики.
Обращение для установления логического в философии к тем формам
(Gestaltungen), которые это логическое принимает в других науках и из
которых многие суть только предчувствия, а другие —~даже искажения этого
логического, может быть лишь крайним средством, к которому прибегает
философское бессилие. Голое применение таких заимствованных формул есть
сверх того внешний способ действия; самому применению должно было бы
предшествовать осознание как их ценности, так и их значения; но такое
осознание дается лишь мыслительным рассмотрением, а не авторитетом, который
эти формулы получили в математике. Таким их осознанием, является сама
логика, и это осознание совлекает их частную форму, делает ее излишней и
никчемной, исправляет ее, и исключительно лишь оно сообщает им оправдание,
смысл и ценность.

Какое значение имеет употребление числа и счета, поскольку оно должно
составлять главную {\em педагогическую} основу, это из
предшествующего само собою ясно. Число есть нечувственный предмет, и
занятие им и его сочетаниями —~нечувственное занятие; дух, следовательно,
этим приучается к рефлексии в себя и к внутренней абстрактной работе, что
имеет большое, но одностороннее значение. Ибо, с другой стороны, так как в
основании числа лежит лишь внешнее, чуждое мысли различие, то указанная
работа становится безмысленной, механической. Требуемое ею напряженное
усилие состоит преимущественно в том, чтобы удерживать чуждое понятию, и
комбинировать его, не прибегая к понятию. Содержанием здесь служит пустое
одно; богатое содержание (der gediegene Inhalt) нравственной и духовной
жизни и индивидуальных ее образований, которое, как благороднейшая пища,
должно служить великим средствам воспитания юношеского духа, вытесняется
бессодержательной единицей. Результатом этих упражнений, когда их делают
главным делом и преимущественным занятием, может быть только то, что дух по
форме и содержанию опустошается и притупляется. Так как счет есть столь
внешнее и, стало быть, механическое занятие, то оказалось возможным
изобрести {\em машины}, совершеннейшим образом
выполняющие арифметические действия. Если бы о природе счета было известно
хотя бы только одно это обстоятельство, то одним этим был бы решен вопрос,
какова ценность зряшной мысли сделать счет главным средством воспитания
духа и этим подвергать его пытке, заставляя его усовершенствовать себя до
того, чтобы стать машиной.

\paragraph[В. \ Экстенсивное и интенсивное определенное количество]{В.
\ Экстенсивное и интенсивное определенное количество}
\subparagraph[а) \ Различие между ними]{а) \ Различие между ними}
1. Определенное количество, как оказалось в предшествующем, имеет свою
определенность как границу в численности. Оно есть некое в себе дискретное,
некое множественное, не имеющее такого бытия, которое было бы отлично от
его границы и имело бы ее вне себя. Определенное количество, взятое таким
образом со своей границей, которая есть некое многообразное в себе самой,
есть {\em экстенсивная величина}.

Следует отличать {\em экстенсивную} величину от
{\em непрерывной}. Первой непосредственно
противоположна не {\em дискретная},
{\em а интенсивная} величина. Экстенсивная и
интенсивная величины суть определенности самой количественной
{\em границы}, определенное же количество тождественно
со своей границей. Непрерывная и дискретная величины суть, напротив,
определения {\em величины в себе},~т.~е. количества как
такового, поскольку мы, имея дело с определенным количеством, отвлекаемся
от границы. — Экстенсивная величина имеет момент непрерывности в самой себе
и в своей границе, так как ее множественное есть вообще непрерывное;
постольку граница как отрицание выступает {\em в этом
равенстве} многих как ограничение единства. Непрерывная величина есть
продолжающее себя количество безотносительно к какой бы то ни было границе,
и, поскольку мы ее представляем себе с таковой границей, последняя есть
некое ограничение вообще, {\em без того чтобы в нем
была положена дискретность}. Определенное количество, взятое лишь как
непрерывная величина, еще не определено истинно для себя, так как в ней
отсутствует одно, от которого зависит то обстоятельство, что имеется
определенность для себя, а также отсутствует и число. И точно так же
дискретная величина есть непосредственно лишь различенное многое вообще,
которое, поскольку оно как таковое должно было бы иметь границу, было бы
только некоторым множеством (eine Menge),~т.~е. некоторым неопределенно
ограниченным; чтобы оно получило определенность определенного количества,
для этого требуется совокупление многих воедино, благодаря чему они
полагаются тождественными с границей. Каждая, и непрерывная и дискретная,
величина, как {\em определенное количество} вообще,
положила в себе (an ihr) лишь одну из тех двух сторон, которыми оно вполне
определено и благодаря которым оно становится
{\em числом}. Последнее есть непосредственно
{\em экстенсивное} определенное количество,
{\em простая} определенность, которая есть по существу
численность, однако численность одной и той же
{\em единицы}; оно отлично от числа только тем, что
определенность в последнем явно положена как множественность.

2. Однако определение величины посредством числа не нуждается в отличии от
какой-либо другой величины, так что выходило бы, что для определенности
этой величины требуются она сама и некоторая другая величина; она в этом не
нуждается потому, что определенность величины есть вообще
для-себя-определенная, безразличная, просто с собою соотнесенная граница, а
в числе она положена, как заключенная в для-себя-сущее одно, и имеет
внешность, соотношение с другим, {\em внутри самой
себя}. Далее, само это присущее границе многое, как и многое вообще, есть
не неравное внутри себя, а непрерывное. Каждое из многих есть то же самое,
что другое; поэтому оно, как многое, сущее друг вне друга или дискретное,
не образует определенности как таковой. Это многое, стало быть, сливается
само по себе в свою непрерывность и становится простым единством. —
Численность есть лишь момент числа, но {\em не как
множество нумерических одних} оно {\em составляет}
определенность числа, а эти одни как безразличные, внешние себе, сняты в
возвращенности числа в себя. Внешность, составлявшая характер одних во
множестве, исчезает в одном как соотношении числа с самим собою.

Граница определенного количества, которое как экстенсивное имело свою
налично сущую определенность в виде внешней самой себе численности,
переходит, следовательно, в {\em простую
определенность}. В этом простом определении границы оно есть
{\em интенсивная величина}; и граница или
определенность, которая тождественна с определенным количеством, теперь так
и положена как простое; это —~{\em градус}.

Градус есть, следовательно, определенная величина, определенное количество,
но не вместе с тем множество (Menge) или много [одних]
{\em внутри самого себя}; он есть только некая
{\em многость} (Mehrheit), причем
{\em многость} есть многое, сжатое (zusammengenommen) в
{\em простое} определение, наличное бытие, перешедшее
обратно в для-себя-бытие. Его определенность должна быть, правда, выражена
некоторым {\em числом} как полной определенностью
определенного количества, но она дана не как
{\em численность}, а просто есть лишь один градус.
Когда говорят о десяти, двадцати градусах, определенное количество, имеющее
столько градусов, есть десятый, двадцатый градус, а не численность и сумма
этих градусов, — в таком случае оно было бы экстенсивным количеством, — а
оно есть лишь один градус, десятый, двадцатый градус. Он содержит в себе
определенность, заключающуюся в численности «десять», «двадцать», но
содержит их не как многие, а есть число как
{\em снятая} численность, как простая
{\em определенность}.

3. В числе определенное количество положено в своей полной определенности; а
как интенсивное определенное количество (которое представляет собою
для-себя-бытие числа) определенное количество положено таким, каково оно
согласно своему понятию или в себе. А именно, та форма соотношения с собою,
которую оно имеет в градусе, есть вместе с тем его
{\em внебытие относительно себя}. Число как
экстенсивное определенное количество есть нумерическая множественность и
имеет таким образом внешность внутри себя; эта последняя, как многое
вообще, сжимается в неразличимость и снимает себя в том обстоятельстве, что
число есть одно (in dem Eins der Zahl), в соотношении числа с самим собою.
Но определенное количество имеет свою определенность в виде численности;
оно, как было указано выше, содержит ее в себе, хотя она уже больше не
положена в нем. Таким образом, {\em градус}, который,
как простой внутри самого себя, уже больше {\em не
имеет} этого {\em внешнего инобытия внутри себя}, имеет
его {\em вне себя} и соотносится с ним как со своей
определенностью. Внешняя ему множественность составляет определенность той
простой границы, которой он сам по себе является. Что численность, которая
в экстенсивном определенном количестве должна была находиться внутри числа,
сняла себя в градусе, это определяется, следовательно, далее, так, что она
положена вне его. Поскольку число положено как одно, как рефлектированное в
себя соотношение с собою самим, оно исключает из себя безразличие и внешний
характер численности и есть {\em соотношение с собою
как соотношение посредством самого себя с некиим внешним}.

В этом определенное количество получает соответственную своему понятию
реальность. {\em Безразличие} определенности составляет
его качество,~т.~е. определенность, которая в самой себе дана как внешняя
себе определенность. — Согласно этому градус есть простая определенность
величины {\em среди} некоторого
{\em множества} таких интенсивностей, которые суть
разные и каждая из которых есть лишь простое соотношение с самим собою, но
которые вместе с тем находятся в существенном взаимосоотношении с другими,
так что каждая имеет свою определенность в этой непрерывности с другими.
Это соотношение градусов посредством самих себя со своим другим делает
поднятие и опускание по шкале градусов непрерывным поступательным
движением, течением, представляющим собою непрерывающееся, неделимое
изменение. Каждое из многих, различаемых в этом процессе, не отделено от
других, а имеет свою определенность только в них. Как соотносящееся с собою
определение величины каждый из градусов безразличен к другим; но он в такой
же мере и соотнесен в себе с этой внешностью, он есть то, что он есть,
только посредством нее; его соотношение с собою есть одновременно
небезразличное соотношение с внешним, имеет в этом последнем соотношении
свое качество.

\subparagraph[b) \ Тождество экстенсивной и интенсивной величины]{b)
\ Тождество экстенсивной и интенсивной величины}
Градус не есть внутри себя некоторое внешнее себе. Он, однако, не есть
{\em неопределенное} одно, этот принцип числа вообще,
который не есть численность, разве только отрицательно, поскольку он не
есть численность. Интенсивная величина есть прежде всего некоторое простое,
{\em одно из многих}; имеются многие градусы; они,
однако, не {\em определены} ни как простое одно, ни как
многие, а определены лишь в {\em соотношении этого
вне-себя-бытия} или, иначе говоря, в тождестве одного и множественного.
Если, таким образом, многие, как таковые, и находятся вне простого градуса,
то в его соотношении с ними и состоит его определенность. Он, таким
образом, содержит в себе численность. Точно так же как двадцать в качестве
экстенсивной величины содержит в себе двадцать одних как дискретных, так и
определенный градус содержит их в себе как непрерывность, которую просто
образует собою эта определенная множественность. Он есть
{\em двадцатый} градус, и он есть двадцатый градус лишь
посредством этой численности «двадцать», которая как таковая находится вне
его.

Определенность интенсивной величины должна быть поэтому рассмотрена с двух
сторон. Эта величина определена через {\em другие}
интенсивные определенные количества и связана непрерывно со своим
инобытием, так что в этом соотношении с последним и состоит ее
определенность. И вот, поскольку она, {\em во-первых},
есть {\em простая} определенность, она определена в
{\em противопоставлении} другим градусам; она их
исключает из себя и имеет свою определенность в этом исключении. Но она,
{\em во-вторых}, определена в самой себе; она
определена в численности, как в {\em своей}
численности, а не в ней как исключенной или, иначе говоря, не в численности
других градусов. Двадцатый градус содержит двадцать градусов в самом себе;
он не только определен как отличный от девятнадцатого, двадцать первого
и~т.~д., а его определенностью служит {\em его}
численность. Но поскольку эта численность есть
{\em его} численность и определенность дана
существенным образом как численность, он есть экстенсивное определенное
количество.

Экстенсивная и интенсивная величины суть, следовательно, одна и та же
определенность определенного количества: они отличаются между собою только
тем, что одна имеет численность как внутри нее, а другая —~как вне нее.
Экстенсивная величина переходит в интенсивную, так как ее многое само по
себе сжимается в единицу, вне которой выступает многое. Но и, наоборот, это
простое имеет свою определенность только в численности и притом как в
{\em своей} численности; как безразличное к иначе
определенным интенсивностям, оно имеет внешний характер численности в самом
себе; таким образом, интенсивная величина есть по существу также и
экстенсивная величина.

Вместе с этим тождеством появляется {\em качественное
нечто}, ибо это тождество есть единица, соотносящаяся с собою посредством
{\em отрицания своих различий}, а эти различия
составляют налично сущую определенность величины. Это отрицательное
тождество есть, следовательно, {\em нечто} и притом
нечто, безразличное к своей количественной определенности.
{\em Нечто} есть некое определенное количество; но
теперь качественное наличное бытие, как оно есть в себе,
{\em положено} как безразличное к этому обстоятельству.
Можно было раньше говорить об определенном количестве, о числе, как
таковом, и~т.~д. без некоторого нечто, которое было бы его субстратом. Но
теперь нечто как {\em налично-сущее для себя} выступает
против этих своих определений, будучи
{\em опосредствовано} с собою через отрицание
последних, и ввиду того, что оно обладает некоторым определенным
количеством, оно выступает как нечто, которое имеет и экстенсивное и
интенсивное определенное количество. Его {\em единая}
определенность, которую оно имеет как определенное количество, положена в
различенных моментах {\em единицы} и
{\em численности}; она одна и та же не только
{\em в себе}, а полагание ее в этих различиях как
экстенсивного и интенсивного количества есть возвращение в это единство,
которое как отрицательное есть положенное безразличном к ним нечто.

\subsection*{Примечание 1. Примеры этого тождества}

В обыденном представлении {\em экстенсивные} и
{\em интенсивные определенные количества} различаются
как {\em виды величин} таким образом, как будто есть
одни предметы, обладающие только интенсивной величиной, и другие,
обладающие только экстенсивной величиной. К этому прибавилось, далее,
выдвинутое известным философским естествознанием представление, которое
превращало множественное, {\em экстенсивное}, например,
в основном определении материи, что она наполняет пространство, равно как и
в других понятиях, — в {\em некое интенсивное} в том
смысле, что интенсивное как {\em динамическое} есть
истинное определение, и, например, плотность или, иначе говоря,
специфическое наполнение пространства следует понимать по существу согласно
этому естествознанию не как известное {\em множество} и
{\em численность} материальных частей в определенном
количестве пространства, а как известную {\em степень}
(Grad) наполняющей пространство силы материи.

При этом следует различать двоякого рода определения. В том, что получило
название преобразования механического способа рассмотрения в динамический,
выступает понятие {\em существующих друг вне друга
самостоятельных частей}, которые лишь внешне соединены в некое целое, и
другое, отличное от первого, понятие {\em силы}. То,
что в наполнении пространства рассматривается, с одной стороны, лишь как
некоторое множество внешних друг другу атомов, рассматривается, с другой
стороны, как проявление лежащей в основании простой силы. — Но этим
отношениям целого и частей, силы и ее проявления, которые противополагаются
тут друг другу, здесь еще не место, и они будут рассмотрены далее. Однако
уже здесь можно указать на то, что хотя отношение
{\em силы} и ее проявления, соответствующее понятию
интенсивного, и есть ближайшим образом более истинное понятие по сравнению
с отношением целого и частей, все же сила еще не становится вследствие
этого менее односторонней, чем интенсивное, и
{\em проявление} как внешность экстенсивного
{\em неотделимо} также и от силы, так что в обеих
формах, и в экстенсивном и интенсивном, имеется
{\em одно и то же содержание}.

Другой выступающей здесь определенностью является
{\em количественная} определенность как таковая,
которая снимается как экстенсивное определенное количество и превращается в
градус как в якобы истинное определение; но мы уже показали, что последнее
определение содержит в себе также и первое, так что одна форма существенна
для другой и, следовательно, всякое существование (Dasein) представляет
собою как экстенсивное, так и интенсивное определенное количество.

Примером выставленного нами положения служит поэтому все на свете, поскольку
оно выступает в некотором определении величины. Даже
{\em число} необходимо имеет непосредственно в нем эту
двоякую форму: оно есть некоторая численность и постольку оно есть
экстенсивная величина; но оно есть также одно —~десяток, сотня, постольку
оно начинает переходить в интенсивную величину, так как в этой единице
множественное сливается в простое. Одно есть {\em в
себе} экстенсивная величина, его можно представить себе как любую
численность частей. Так, например, {\em десятое},
{\em сотое} есть это простое интенсивное, имеющее свою
определенность в находящемся вне его многом,~т.~е. в экстенсивном. Число
есть десять, сто и вместе с тем в системе чисел —~десятое, сотое; и то и
другое есть одна и та же определенность.

Одно в круге называется {\em градусом}, потому что часть
{\em круга} имеет по существу свою определенность в
находящемся вне ее многом, определена как одно из замкнутой численности
таких одних. Градус круга, взятый как простая пространственная величина,
есть лишь обычное число; рассматриваемый же как градус, он есть интенсивная
величина, имеющая смысл лишь как определенная численностью градусов, на
которые разделен круг, подобно тому, как число вообще имеет смысл только в
ряде чисел.

Величина более конкретного предмета проявляет свою двойственность (то
обстоятельство, что она есть как экстенсивная, так и интенсивная величина)
в двояком определении его существования (Dasein): в одном из этих
определений предмет выступает как некое {\em внешнее},
в другом —~как некое {\em внутреннее}. Так, например,
{\em масса} как вес есть
{\em экстенсивная величина}, поскольку она составляет
некоторую численность фунтов, центнеров и~т.~д., и она же есть
{\em интенсивная величина}, поскольку она оказывает
известное давление. Величина давления есть нечто простое, некоторая
степень, имеющая свою определенность в шкале степеней давления. Как
оказывающая давление, масса выступает в качестве внутри-себя-бытия,
субъекта, которому присуще различие интенсивной величины. — И обратно, то,
что оказывает эту {\em степень} давления, способно
сдвинуть с места известную {\em численность} фунтов
и~т.~д. и этим измеряет свою величину.

Или, скажем, {\em теплота} имеет некоторый
{\em градус}: градус теплоты, будь он 10-й, 20-й
и~т.~д., есть некоторое простое ощущение, нечто субъективное. Но этот
градус существует также и как {\em экстенсивная}
величина, как расширение некоторой жидкости, например, ртути в термометре,
воздуха или глины и~т.~д. Более высокая степень температуры выражается как
более длинный ртутный столбик или как более узкий глиняный цилиндр; она
нагревает бóльшее пространство, таким же образом, как меньшая степень
температуры нагревает лишь меньшее пространство.

Более высокий {\em тон} как более
{\em интенсивный} есть вместе с тем
{\em бóльшее число} колебаний; или другой пример: более
громкий тон, которому приписывается более высокая
{\em степень}, слышен в более обширном пространстве. —
Более интенсивной {\em краской} можно окрасить
одинаковым образом бóльшую поверхность, чем более слабой краской; или еще
один пример: {\em более светлое}, представляющее собою
другой вид интенсивности, видно далее, чем менее светлое и~т.~д.

Равным образом и в области {\em духовного высокая
интенсивность} характера, таланта, гения имеет соответственно
{\em широкий} охват, соответственно
{\em широкое} действие и
{\em многосторонние} соприкосновения.
{\em Наиболее глубокое} понятие обладает
{\em наиболее общим} применением и значением.

\subsection*{Примечание 2. Кантово применение определения степени
к бытию души}

{\em Кант} сделал своеобразное употребление из
применения определенности интенсивного определенного количества к
метафизическому определению {\em души}. В критике
метафизических положений о душе, которые он называет паралогизмами чистого
разума, он занимается, между прочим, рассмотрением умозаключения от
простоты души к ее неуничтожаемости. Он выдвигает против этого
умозаключения то возражение («Критика чистого разума», изд. 2-е, стр. 414),
что «даже если мы и признаем, что душа обладает такой простой природой, так
как она, именно, не содержит в себе никакого внеположного многообразия и,
следовательно, никакой {\em экстенсивной} величины, все
же ей, как и {\em какому бы то ни было другому
существующему}, нельзя отказать в обладании
{\em интенсивной} величиной,~т.~е.
{\em степенью} реальности в отношении всех ее
способностей и даже вообще в отношении всего того, что составляет
существование, а эта степень может {\em уменьшаться},
проходя через все {\em бесконечно многие меньшие}
степени, и мнимая субстанция может быть, таким образом, превращена в ничто,
если и не путем деления, то путем постепенного ослабления (remissio) ее
сил; ибо даже {\em сознание} всегда обладает некоторой
{\em степенью}, которая все еще может уменьшаться, и,
следовательно, может уменьшаться также и способность сознавать себя и таким
же образом могут уменьшаться и все прочие способности». — Душа
рассматривается в рациональной психологии, которая была абстрактной
метафизикой, не как дух, а как некое лишь непосредственно
{\em сущее}, как {\em душа-вещь}.
Таким образом Кант имеет право применять к ней категорию определенного
количества, «как к какому бы то ни было другому существующему», а поскольку
это существующее определено как простое, Кант имеет право применять к нему
категорию интенсивного определенного количества. Духу, правда, присуще
{\em бытие}, но совершенно другой интенсивности, чем
интенсивность интенсивного определенного количества, а именно, ему присуща
такая интенсивность, в которой форма лишь непосредственного бытия и все его
категории даны как снятые. Нужно было допустить устранение не только
категории экстенсивного определенного количества, а и вообще категории
определенного количества. Следует, однако, сделать еще и дальнейший шаг и
познать, каким образом в вечной природе духа находятся и из нее проистекают
наличное бытие, сознание, конечность без того, чтобы он вследствие этого
становился вещью.

\subparagraph[с) \ Изменение определенного количества]{с) \ Изменение
определенного количества}
Различие между экстенсивными и интенсивными определенными количествами
безразлично для определенности определенного количества, как таковой. Но
вообще определенное количество есть определенность, положенная как снятая,
есть безразличная граница, такая определенность, которая в такой же мере
есть также и отрицание самой себя. В экстенсивной величине это различие
развито, интенсивная же величина есть {\em наличное
бытие} этой внешности, которую представляет собой внутри себя определенное
количество. [В интенсивном определенном количестве] это различие положено
как его собственное противоречие внутри самого себя, заключающееся в том,
что оно есть такого рода простая, {\em соотносящаяся с
собою} определенность, которая представляет собою отрицание самой себя,
имеет свою определенность не в ней, а в некотором другом определенном
количестве.

Определенное количество, следовательно, по самому своему качеству, положено
в абсолютной непрерывности со своей внешностью, со своим инобытием. Оно
поэтому не только {\em может} выходить за пределы
всякой определенности величины, последняя не только
{\em может} изменяться, но прямо
{\em положено}, что она
{\em необходимо должна} изменяться. Определение
величины продолжает себя, непрерывно переходя в свое инобытие таким
образом, что оно имеет свое бытие только в этой непрерывности с некоторым
другим; оно есть не {\em сущая}, а
{\em становящаяся} граница.

Одно бесконечно или, иначе говоря, оно есть соотносящееся с собою отрицание
и поэтому отталкивание себя от самого себя. Определенное количество равным
образом бесконечно, оно положено как соотносящаяся с собою отрицательность;
оно отталкивает себя от себя самого. Но оно есть некоторое
{\em определенное} одно; такое одно, которое перешло в
наличное бытие и границу, есть, следовательно, отталкивание определенности
от себя самой, порождение не равного себе самому, каково отталкивание
одного, а своего инобытия; в нем самом теперь положено, что оно
{\em выводит за себя} и становится другим. Оно состоит
в том, чтобы увеличиваться или уменьшаться; оно есть внешность
определенности в нем самом.

Определенное количество, стало быть, выводит само себя за себя; это другое,
которым оно становится, само есть ближайшим образом некоторое определенное
количество; но оно в такой же мере дано также и как некоторая не сущая, а
выводящая сама себя за себя граница. Снова возникшая в этом выхождении
граница есть, следовательно, безоговорочно лишь такая граница, которая
снова снимает себя и выводит к дальнейшей границе,
{\em и так далее до бесконечности}.

\paragraph[С. \ Количественная бесконечность]{С. \ Количественная
бесконечность}
\subparagraph[а) \ Ее понятие]{а) \ Ее понятие}
Определенное количество изменяется и становится другим определенным
количеством. Дальнейшее определение этого изменения, а именно, что оно
продолжается {\em до бесконечности}, заключается в том,
что определенное количество поставлено, как в самом себе противоречащее
себе. — Определенное количество становится некиим
{\em другим}; но оно
{\em продолжается} в своем инобытии; другое есть,
следовательно, также некоторое определенное количество. Но последнее есть
другое не только {\em некоторого} определенного
количества, но и {\em самого} определенного количества
как такового, его отрицание как ограниченного, следовательно, есть его
неограниченность, {\em бесконечность}. Определенное
количество есть некоторое {\em долженствование}. Оно
означает, что оно {\em определено само по себе}, и эта
самостоятельная определенность (Für-sich-bestimmtsein) есть скорее
{\em определенность в некотором другом}, и, наоборот,
оно есть снятая определенность в некотором другом, есть
{\em безразличная} самодовлеющая устойчивость (Bestehen
für sich).

Конечность и бесконечность вследствие этого сразу же получают каждая в самой
себе двоякое и притом противоположное значение. Определенное количество
{\em конечно}, во-первых, как ограниченное вообще и,
во-вторых, как то, что отсылает за пределы самого себя, как определенность
в некотором другом. {\em Бесконечность} же его есть,
во-первых его неограниченность и, во-вторых, его возвращенность в себя,
безразличное для-себя-бытие. Если мы сразу же сравним между собою эти
моменты, то окажется, что определение конечности определенного количества,
отсылание за себя к некоторому другому, в котором заключается его
определение, есть в такой же мере определение бесконечного; отрицание
границы есть тот же выход за определенность, так что определенное
количество имеет в этом отрицании, в бесконечном, свою последнюю
определенность. Другим моментом бесконечности служит безразличное к границе
для-себя-бытие; само же определенное количество есть ограниченное таким
образом, что оно есть само по себе безразличное к своей границе и, значит,
к другим определенным количествам и к своему выходу за самого себя.
Конечность и (долженствующая быть отдельной от нее, дурная) бесконечность в
лице определенного количества уже имеют в самих себе каждая момент другой.

Качественное и количественное бесконечное отличаются друг от друга тем, что
в первом противоположность между конечным и бесконечным качественна, и
переход конечного в бесконечное или, иначе говоря, их взаимоотношение
имеется лишь во «{\em в себе}», в их понятии.
Качественная определенность дана как непосредственная и соотносится по
существу с инобытием как с другим для нее бытием; она не
{\em положена} так, чтобы иметь свое отрицание, свое
другое {\em в себе самой}. Напротив, величина есть, как
таковая, {\em снятая} определенность; она
{\em положена} так, чтобы быть неравной с собою и
безразличной к самой себе, быть поэтому тем, что изменяется. Качественные
конечное и бесконечное поэтому абсолютно,~т.~е. абстрактно, противостоят
друг другу; их единством служит лежащее в основании
{\em внутреннее} соотношение. Конечное поэтому
продолжается в свое другое только {\em в себе}, а не
{\em в нем}. Напротив, количественное бесконечное
{\em соотносится в нем самом} со своим бесконечным, в
котором оно имеет свою абсолютную определенность. Это их соотношение
представляет собою ближайшим образом
{\em количественно-бесконечный прогресс}.

\subparagraph[b) \ Количественный бесконечный прогресс]{b) \ Количественный
бесконечный прогресс}
Бесконечный прогресс есть вообще выражение противоречия, в данном случае
—~выражение того противоречия, которое заключается в количественно-конечном
или, иными словами, в определенном количестве вообще. Он есть то
взаимоопределение конечного и бесконечного, которое мы рассмотрели выше в
качественной сфере, с тем различием, что, как мы только что указали, в
количественном граница в самой себе выводит себя в свое потустороннее и
продолжается в нем, и тем самым, наоборот, и количественно бесконечное
положено имеющим в себе самом определенное количество; ибо определенное
количество есть в своем вне-себя-бытии вместе с тем оно же само, его
внешность есть его определение.

{\em Бесконечный прогресс} есть лишь
{\em выражение} этого противоречия, а
{\em не} его {\em разрешение}; но
вследствие непрерывного продолжения одного определения в его другом он дает
кажущееся разрешение в виде соединения обоих определений. В том виде, как
он первоначально положен, он есть {\em заданность}
бесконечного, а не его достижение, есть вековечное
{\em порождение} его, причем он не выходит за само
определенное количество, и бесконечное не становится некиим положительным и
наличным. В понятии определенного количества подразумевают, что у
последнего есть некое {\em потустороннее} ему. Это
потустороннее есть, {\em во-первых}, абстрактный момент
{\em небытия} определенного количества; последнее
разлагается в самом себе; таким образом оно соотносится со своим
{\em потусторонним} как со своей бесконечностью; это
—~соотношение по {\em качественному} моменту
противоположности. Но, {\em во-вторых}, определенное
количество находится в непрерывной связи с этим потусторонним; определенное
количество именно и состоит в том, что оно есть другое самого себя, внешнее
самому себе; стало быть, это внешнее равным образом есть не некое другое,
чем определенное количество; {\em потустороннее} или
бесконечное, следовательно, само есть {\em некоторое
определенное количество}. Потустороннее, таким образом, возвращено назад из
его бегства, и бесконечное оказывается достигнутым. Но так как это ставшее
теперь посюсторонним потустороннее есть опять-таки некоторое определенное
количество, то здесь в свою очередь положена лишь новая граница; последняя
как определенное количество снова убежала от себя самой, вышла как таковое
вне себя и оттолкнулась от самой себя в свое небытие, в свое потустороннее,
которое столь же перманентно становится определенным количеством, сколь и
последнее отталкивается от самого себя и становится потусторонним.

Непрерывное продолжение определенного количества в свое другое производит
соединение обоих в выражениях: {\em бесконечно большое}
или {\em бесконечно малое}. Так как в обоих еще имеется
определение определенного количества, то они остаются изменчивыми и, стало
быть, не достигается та абсолютная определенность, которая была бы
некоторым для-себя-бытием. Это {\em вне-себя-бытие}
определения положено в двояком бесконечном, противополагающемся по
направлениям {\em увеличения} и
{\em уменьшения}, — в бесконечно-большом и
бесконечно-малом. В каждом из них, взятом само по себе, определенное
количество {\em сохраняется} в вековечной
противоположности к своему потустороннему. Как бы мы ни увеличивали
какую-нибудь величину, она по сравнению с бесконечным сжимается до
полнейшей незначительности. Поскольку она соотносится с бесконечным как со
своим небытием, то противоположность {\em качественна};
расширившееся определенное количество поэтому ничего не отвоевало от
бесконечного; последнее, как и раньше, есть его небытие. Или, иначе говоря,
увеличение определенного количества не есть
{\em приближение} к бесконечному; ибо различие между
определенным количеством и его бесконечностью имеет по существу также и тот
{\em момент}, что оно не есть количественное различие.
Это —~лишь более суженное выражение противоречия; бесконечно-большое должно
быть некоторым {\em большим},~т.~е. некоторым
определенным количеством, и {\em бесконечным},~т.~е.
таким, которое не есть определенное количество. — И точно так же бесконечно
малое есть как малое некоторое определенное количество и остается поэтому
абсолютно,~т.~е. качественно, слишком большим для бесконечного и
противоположно последнему. В обоих сохраняется противоречие бесконечного
прогресса, который якобы нашел, в них свой последний этап (sein Ziel).

Эту бесконечность, которую упорно определяют как потустороннее конечного,
следует назвать {\em дурной количественной
бесконечностью}. Она, подобно качественной дурной бесконечности, есть
вековечный переход туда и обратно от одного члена остающегося противоречия
к другому, от границы к ее небытию и от последнего снова обратно к такой же
самой границе. В количественном прогрессе то, к чему совершается
поступательный переход, есть, правда, не некоторое абстрактно другое
вообще, а некоторое определенное количество, положенное как разное; однако
оно остается одинаковым образом противоположным своему отрицанию. Поэтому
этот прогресс равным образом есть не поступательное шествие и продвижение
дальше, а как раз повторение одного и того же, полагание, устранение, и
снова полагание и снова устранение. Это —~бессилие отрицания, к которому
через самое устранение снова возвращается, как продолжающееся, то, что им
было устранено. Здесь два определения так связаны между собою, что они
безоговорочно бегут друг от друга; и, убегая друг от друга, они не могут
разлучиться, а остаются связанными в их взаимном убегании.

\subsection*{Примечание 1. Высокое мнение о бесконечном прогрессе}

Дурная бесконечность преимущественно в форме
{\em количественного бесконечного прогресса} —~этого
постоянного перехода за границу, который есть бессилие устранить ее и
вековечное возвращение в нее, — обыкновенно принимается за нечто
возвышенное и некоторого рода служение богу, равно как и в философии такой
прогресс также рассматривался как последнее слово (ein Letztes). Этот
прогресс не раз служил поводом для тирад, которыми восхищались как
возвышенными произведениями. Но на самом деле эта
{\em новейшая} возвышенность возвеличивает не самый
{\em предмет}, который, напротив, оказывается
убегающим, а лишь {\em субъекта}, поглощающего в себя
столь большие количества. Скудость этого остающегося субъективным
восхождения по количественной лестнице сама себя обличает в признании, что
оно представляет собой напрасную трату труда, так как не в состоянии
приблизить к бесконечной цели, для достижения которой нужно, разумеется,
взяться за дело совершенно иначе.

В приводимых нами далее такого рода тирадах выражено вместе с тем и то, во
что переходит и чем заканчивается такого рода взлет.
{\em Кант}, например, приводит как нечто возвышенное
следующее («Критика практического разума», заключение).

«Когда субъект мысленно поднимается выше того места, которое он занимает в
чувственном мире, и расширяет связь до бесконечно больших размеров, — связь
звезд и звезд, и еще звезд, миров и миров, и еще миров, систем и систем, и
еще систем, да сверх того расширяет эту связь во времени, рассматривая
безграничные времена их периодического движения, его начало и дальнейшее
продолжение, то представление не выдерживает этого поступательного движения
в неизмеримую даль, где за {\em самым отдаленным миром
все еще} есть {\em более отдаленный}, где прошлое, как
бы далеко назад мы ни проследили его, все еще имеет
{\em более отдаленное} прошлое, а будущее, как бы
далеко мы его ни проследили вперед, {\em все еще} имеет
впереди себя другое будущее; {\em мысль не выдерживает}
этого представления о неизмеримом, подобно тому, как кончается
{\em падением} или
{\em головокружением} сон, в котором снится человеку,
что он совершает длинный путь, идет все дальше и дальше, необозримо дальше,
и не видать конца».

Это изложение помимо того, что оно дает сжатое и вместе с том богатое
изображение содержания того взлета, который вызывается количественным
бесконечным прогрессом, заслуживает похвалы особенно за ту правдивость, с
которой оно указывает, чем кончается этот взлет: мысль не выдерживает этого
представления, и оно кончается падением и головокружением. Заставляет же
мысль изнемочь, вызывает ее падение и головокружение не что иное как
{\em скука} от повторения, при котором граница исчезает
и снова появляется и снова исчезает, и так всегда одно ради другого и одно
в другом, в потустороннем посюстороннее, в посюстороннем потустороннее,
перманентно возникает и исчезает, вызывая лишь чувство
{\em бессилия} этого бесконечного или этого
долженствования, которое хочет и не может справиться с конечным.

Галлерово, так называемое Кантом {\em страшное описание
вечности} обыкновенно также вызывает особенное восхищение, но часто как раз
не за то, в чем состоит подлинная его заслуга.
Галлер говорит:

«Ich häufe ungeheure Zahlen,

Gebürge Millionen auf,

Ich setze Zeit auf Zeit und Welt auf Welt zu Hauf,

Und wenn ich von der grausen Höh'

Mit Schwindeln wieder nach dir seh',

Ist alle Macht der Zahl, vermehrt zu tausendmalen,

Noch nicht ein Teil von dir.

Ich zieh' sie ab und du liegst ganz vor mir».

(«Нагромождаю тьму чисел, мильоны гор, нагромождаю времена над временами,
миры над мирами. И когда я со страшной высоты снова взираю с
головокружением на тебя, то вся сила чисел, умноженная тысячекратно, еще не
составляет и части тебя. {\em Я их вычитаю, а ты "--- весь
предо мною}»).

Если этому нагромождению чисел и миров придается значение
{\em описания вечности}, то при этом упускается из
внимания, что сам поэт объявляет это так называемое «страшное» выхождение
чем-то тщетным и пустым и что он кончает тем, что лишь
{\em путем отказа} от этого бессодержательного
бесконечного прогресса {\em предстает перед ним} и
становится {\em наличным} само истинное бесконечное.

Среди {\em астрономов} были такие, которые очень охотно
похвалялись возвышенностью их науки, усматривая эту возвышенность в том,
что астрономия имеет дело с таким {\em неизмеримым}
множеством звезд, с такими {\em неизмеримыми}
пространствами и временами, в которых расстояния и периоды, уже и сами по
себе столь огромные, служат единицами и которые, сколь бы многократно их ни
брали, все же снова оказываются малыми до незначительности. Пустое
удивление, которому они при этом предаются, плоские надежды, что в
загробной жизни они будут перекочевывать с одной звезды на другую и,
странствуя так по неизмеримому пространству, будут приобретать все новые и
новые сведения {\em того же сорта}, — эти свои пустое
удивление и плоские надежды они выдавали за основную черту превосходства их
науки. А между тем она достойна изумления не вследствие такой
количественной бесконечности, а, напротив, вследствие
{\em тех отношений меры и законов}, которые разум
познает в этих предметах и которые суть разумное бесконечное в
противоположность указанной неразумной бесконечности.

Бесконечности, относящейся к внешнему чувственному созерцанию,
{\em Кант} противопоставляет другую бесконечность,
состоящую в том, что

«Индивидуум обращается назад к своему незримому «я» и противопоставляет
абсолютную свободу своей воли как некоторое чистое «я» всем ужасам судьбы и
тирании; для него исчезают все окружающие его вещи, начиная с его ближайшей
обстановки, и рассыпается в прах также и то, что представляется прочным,
миры за мирами, и он, одинокий, познает {\em себя как
равного самому себе}».

«Я» в этом одиночестве с самим собою есть, правда, достигнутое
потустороннее; оно пришло к самому себе, находится
{\em у себя, по сю сторону}; в чистом самосознании
абсолютная отрицательность приведена к утверждению и наличию, которое в
вышеуказанном поступательном движении по чувственному определенному
количеству лишь убегает. Но это чистое «я», фиксируя себя в своей
абстрактности и бессодержательности, имеет перед собою противоположное ему
существование вообще, полноту природного и духовного универсума как некое
потустороннее. Получается то же самое противоречие, которое лежит в
основании бесконечного прогресса, а именно такое возвращение к себе,
которое вместе с тем непосредственно есть бытие-вне-себя, соотношение со
своим другим как со своим небытием; это соотношение остается некоторым
{\em страстным стремлением}, потому что «я» фиксировало
для себя, с одной стороны, свою бессодержательную и лишенную опоры пустоту,
а с другой стороны, — как свое потустороннее, — в отрицании все же
остающуюся наличной полноту.

К своему изложению, этих двух возвышенностей Кант прибавляет замечание, что
«удивление (по отношению к первой, внешней) и уважение (по отношению ко
второй, внутренней возвышенности), хотя и
{\em побуждают к исследованию}, не могут, однако,
заменить его и вознаградить за его
{\em отсутствие}»[ссылка!]. — Он, следовательно,
объявляет эти взлеты не удовлетворяющими разума, который не может
остановиться на них и связанных с ними чувствах и признавать потустороннее
и пустое последним словом.

Но в качестве последнего слова брали бесконечный прогресс преимущественно в
его применении к {\em морали}. Только что указанная
вторая противоположность между конечным и бесконечным, как
противоположность между многообразным миром и поднявшимся к своей свободе
«я», носит ближайшим образом качественный характер. Самоопределение «я»
стремится вместе с тем к тому, чтобы определить природу и освободить себя
от нее; таким образом, оно само через себя соотносится со своим другим,
которое как внешнее наличное бытие есть некое множественное и тоже
количественное. Соотношение с некоторым количественным само становится
количественным; отрицательное соотношение «я» с этим количественным, власть
«я» над «не-я», над чувственностью и внешней природой, изображается поэтому
так, что моральность может и должна все более и более
{\em возрастать}, а власть чувственности все более и
более {\em уменьшаться}. Но полное соответствие воли с
моральным законом переносится в идущий до бесконечности прогресс,~т.~е.
изображается как {\em абсолютно недостижимое}
потустороннее, и именно то обстоятельство, что оно недостижимо, и должно
быть якорем спасения и истинным утешением; ибо моральность должна быть
борьбою, а последняя существует только при несоответствии воли с законом, и
этот закон, следовательно, есть для нее безоговорочно потустороннее.

В этом противоположении «я» и «не-я» или чистая воля и моральный закон, с
одной стороны, и природа и чувственность воли —~с другой стороны,
предполагаются совершенно самостоятельными и безразличными друг к другу.
Чистая воля имеет свой своеобразный закон, находящийся в существенном
соотношении с чувственностью, а природа и чувственность, со своей стороны,
имеют законы, о которых нельзя сказать ни того, что они заимствованы из
области воли и соответствуют ей, ни даже того, что они, хотя и отличаются
от нее, все же заключают в себе некоторое существенное соотношение с нею.
Эти законы определены вообще сами по себе, они суть готовые и замкнутые
внутри себя. Но вместе с тем оба они суть моменты
{\em одной и той же простой сущности}, «я»; воля
определена как нечто отрицательное по отношению к природе, так что она
существует лишь постольку, поскольку существует такое от него отличное,
которое снимается ею, но которое в этом процессе снимания соприкасается с
нею и даже воздействует на нее. Природе и ей же в качестве чувственности
человека —~природе, как представляющей собою самостоятельную систему
законов, — ограничение некиим другим безразлично; она сохраняется в этой
ограничиваемости, вступает самостоятельно в соотношение с волей и столь же
ограничивает волю, руководящуюся моральным законом (воление морального
закона), сколь и ограничивается ею. — В том же самом акте, в котором воля
определяет себя и снимает инобытие некоей природы, это инобытие положено
как существующее, продолжающее существовать в своей снимаемости, и значит,
оказывается не снятым. Заключающееся в этом противоречие не находит своего
разрешения в бесконечном прогрессе, а, напротив, изображается и
утверждается как неразрешенное и неразрешимое; борьба моральности и
чувственности изображается как сущее в себе и для себя, абсолютное
отношение.

Бессилие справиться с качественной противоположностью между конечностью и
бесконечностью и постигнуть идею истинной воли, субстанциальную свободу,
ищет прибежища в {\em величине}, чтобы воспользоваться
ею как посредницей, так как она есть снятое качественное, ставшее
безразличным различие. Однако, так как в основании лежат попрежнему оба
члена противоположности как качественно разные, то благодаря тому, что они
ведут себя в своем взаимоотношении как определенные количества, каждое из
них, наоборот, сразу же положено безразличным к этому изменению. Природа
определяется через «я», чувственность через веление добра; изменение,
произведенное этим волением в чувственности, есть лишь количественное
различие, такое изменение, которое оставляет ее тем, что она есть.

В более абстрактном изложении кантовской философии или, по крайней мере, ее
принципов, именно, в фихтевом наукоучении бесконечный прогресс составляет
таким же образом основу и последнее слово. За первым основоположением этого
изложения, «я»~=~«я», следует второе независимое от первого
основоположение, именно, {\em противоположение} «не-я»;
и сразу же принимается, что {\em соотношение} между
этими двумя основоположениями есть также и
{\em количественное} различие: «не-я»
{\em частью} определяется «я» и
{\em частью} не определяется им. «Не-я», таким образом,
продолжается в свое небытие так, что оно в своем небытии остается
противоположным как некое неснятое. Поэтому, после того как заключающиеся
здесь противоречия были развиты Фихте в его системе, конечным выводом
оказалось то же самое отношение, которое служило отправным пунктом: «не-я»
остается бесконечным
толчком[ссылка!], безоговорочно другим; последним
взаимоотношением «не-я» и «я» служит бесконечный прогресс,
{\em страстное влечение и стремление} —~то же самое
противоречие, с которого начали.

Так как количественность есть определенность, положенная как снятая, то
думали, что для единства абсолютного, для единой субстанциальности
получается большой выигрыш или, вернее, выигрывается все, если понизят в
ранге противоположность вообще, сведя ее к исключительно лишь
количественному различию. {\em Всякая противоположность
только количественна} —~таково было в продолжение некоторого времени
основное положение новейшей
философии[ссылка!]; противоположные
определения имеют одну и ту же сущность, одно и то же содержание, они суть
реальные стороны противоположности, поскольку каждая из них имеет в себе
оба определения противоположности, оба фактора, но только на одной стороне
{\em преобладает} один фактор, а на другой —~другой, в
одной стороне один из факторов, некая материя или деятельность, дан в
{\em большем количестве} или в
{\em более сильной степени}, чем другой. — Поскольку
здесь предполагаются разные вещества или деятельности, количественное
различие, наоборот, подтверждает и завершает их внешний характер и
безразличие друг к другу и к их единству. Различие в
{\em абсолютном} единстве, утверждают, есть только
количественное; между тем, хотя количественное есть снятая непосредственная
определенность, оно, однако, есть лишь несовершенное, лишь
{\em первое} отрицание, а не бесконечное отрицание, не
отрицание отрицания. — Так как бытие и мышление представляются здесь в виде
количественных определений абсолютной субстанции, то и они как определенные
количества становятся вполне внешними друг для друга и лишенными связи,
подобно тому как в низшей сфере такими внешними являются углерод, азот
и~т.~д. Только нечто третье, а именно внешняя рефлексия, отвлекается от их
различия и познает их {\em внутреннее, сущее} лишь
{\em в себе}, а не также и {\em для
себя} единство. Стало быть, на самом деле это единство представляют себе
лишь как первое {\em непосредственное} единство или,
иначе говоря, только как {\em бытие}, которое в своем
количественном различии {\em остается} равным самому
себе, но не {\em полагает} себя равным через себя само;
оно, следовательно, не постигнуто как отрицание отрицания, как бесконечное
единство. Только в качественной противоположности рождается положенная
бесконечность, для-себя-бытие, и само количественное определение переходит,
как это сейчас будет выяснено ближе, в качественное.

\bigskip

\subsection*{Примечание 2. Кантовская антиномия ограниченности и
неограниченности мира во времени и в пространстве}

Мы уже упомянули выше, что {\em кантовские антиномии}
суть изложения противоположности конечного и бесконечного в
{\em более конкретном} виде, в применении к более
специальным субстратам представления. Рассмотренная там антиномия касалась
противоположности между качественной конечностью и бесконечностью. В другой
антиномии, а именно в {\em первой} из четырех
космологических антиномий, более рассматривается количественная граница в
ее противоречиях. Я поэтому подвергну здесь исследованию эту антиномию.

Она касается вопроса о том, {\em ограничен ли или не
ограничен мир во времени и пространстве}. — Можно было бы с одинаковым
успехом рассматривать эту противоположность также и в отношении самих
времени и пространства, ибо признаем ли мы, что время и пространство суть
отношения самих вещей, или признаем, что они суть лишь формы созерцания,
это ничего не меняет по отношению к антиномичности приписываемых им
определений ограниченности или неограниченности.

Более подробный разбор этой антиномии покажет равным образом, что оба
положения, а равно и доказательства их, которые, как и выше рассмотренные,
ведутся от противного, сводятся не к чему иному, как к двум следующим
простым, противоположным утверждениям: {\em граница
существует}, и: {\em должно перейти за границу}.

Тезис гласит:

«{\em Мир имеет начало во времени, и по пространству он
также заключен в границы}».

{\em Одна часть} доказательства, та, которая касается
{\em времени}, принимает противное.

«Допустим, что мир не имеет начала во времени; тогда оказывается, что
{\em до всякого данного момента времени} прошла
вечность и, стало быть, {\em протек} бесконечный ряд
следующих друг за другом состояний вещей в мире. Но бесконечность ряда в
том именно и состоит, что он никогда не может быть
{\em завершен} посредством последовательного синтеза.
Стало быть, протекший бесконечный мировой ряд невозможен и, значит, начало
мира есть необходимое условие его существования, что и требовалось
доказать».

{\em Другая часть} доказательства, касающаяся
{\em пространства}, сводится к времени. Соединение
частей бесконечного в пространстве мира потребовало бы бесконечного
времени, которое должно было бы рассматриваться как протекшее, поскольку мы
должны рассматривать мир в пространстве не как некое становящееся, а как
некое завершенное данное. Но относительно времени показано в первой части
доказательства, что невозможно принимать бесконечное протекшее время.

Но сразу же видно, что не было никакой нужды вести доказательство от
противного или даже вообще вести доказательство, так как в самом же
кантовском доказательстве лежит в основании то, что должно было быть
доказано. А именно, в нем принимается некоторый или любой
{\em данный момент времени}, до которого прошла
вечность (вечность имеет здесь лишь маловажный смысл некоторого
дурно-бесконечного времени). Но {\em данный момент
времени} означает не что иное как некоторую определенную
{\em границу} во времени. В доказательстве,
следовательно, {\em предполагается} граница времени как
действительная. Но это и есть именно {\em то}, что
должно было {\em быть доказано}. Ибо тезис состоит в
том, что мир имеет начало во времени.

Здесь имеется только то различие, что {\em допущенная}
граница времени есть некоторое «теперь», как конец протекшего до этого
времени, а та граница, наличие которой требуется доказать, есть «теперь»
как начало некоторого будущего. Но это различие несущественно. «Теперь»
принимается как момент, в который {\em протек}
бесконечный ряд следующих друг за другом состояний вещей в мире,
следовательно, как конец, как {\em качественная}
граница. Если бы это «теперь» рассматривалось лишь как количественная
граница, которая имеет текучий характер и за которую не только должно
перейти, но которая только и состоит в том, что она переходит за себя, то
оказалось бы, что бесконечный временной ряд в ней не
{\em протек}, а продолжает течь, и рассуждение
доказательства отпало бы. Напротив, в кантовском доказательстве момент
времени принимается как качественная граница для прошлого, но при этом он
есть вместе с тем {\em начало} для будущего, — ибо
{\em сам по себе} каждый момент времени есть
соотношение прошлого и будущего, — он равным образом есть
{\em абсолютное},~т.~е. абстрактное
{\em начало} для будущего,~т.~е. то, что должно было
быть доказано. Дело нисколько не меняется от того, что до его (этого
момента времени) будущего и до этого начала последнего имеется некоторое
прошлое; так как этот момент времени есть качественная граница —~а
необходимость принимать его за качественную границу содержится в
определении {\em завершенного}, протекшего,
{\em и, следовательно, не продолжающегося}, — то время
в нем {\em прервано} и это прошлое оказывается не
находящимся в соотношении с тем временем, которое могло быть названо
будущим лишь в отношении к этому прошедшему и которое поэтому без такого
соотношения есть лишь время вообще, имеющее абсолютное начало. Но если бы
оно —~как это в самом деле и есть —~через «теперь», через данный момент
времени находилось в соотношении с прошедшим, если бы оно, следовательно,
было определено как будущее, то, с другой стороны, и этот момент времени
также не был бы границей, бесконечный временной ряд продолжался бы в том,
что называлось будущим, и не был бы, как это принимает доказательство,
{\em завершен}.

На самом деле время есть чистое количество; употребляемый в доказательстве
«{\em момент времени}», в котором время якобы
прерывается, есть, напротив, лишь {\em снимающее само}
{\em себя} для-себя-бытие данного «теперь».
Доказательство делает лишь одно: утверждаемую тезисом абсолютную границу
времени оно заставляет нас представить себе как некий
{\em данный момент времени} и прямо принимает, что он
завершен,~т.~е. что он есть абстрактная точка; это —~популярное
определение, которое чувственное представление с легкостью готово принять
за {\em границу}, в результате чего в доказательстве
признается как допущение то, что до этого было выставлено как то, что
требуется доказать.

{\em Антитезис} гласит:

«{\em Мир не имеет начала и границ в пространстве; он
бесконечен как в отношении времени, так и в отношении пространства}».

{\em Доказательство} антитезиса равным образом исходит
из допущения противного.

«Допустим, что мир имеет начало. Так как начало есть такое существование,
которому предшествует время, когда, еще не было данной вещи, то началу мира
должно было предшествовать время, когда еще не было мира,~т.~е. пустое
время. Но в пустом времени невозможно
{\em возникновение} какой бы то ни было вещи, так как
никакая часть такого времени не заключает в себе преимущественно перед
другой частью какого-либо {\em отличительного условия}
существования, а не несуществования. Следовательно, в мире некоторые ряды
вещей могут иметь начало, но сам мир не может иметь начала и в отношении
протекшего времени бесконечен».

Это доказательство от противного содержит в себе, как и другие, прямое и
недоказанное утверждение того, что оно должно было доказать. А именно, оно
принимает сначала некое потустороннее по отношению к мировому существованию
пустое время; но затем {\em продолжает далее} также и
{\em мировое существование} так, что оно, выходя
{\em за себя вовне}, входит {\em в
это пустое время}, упраздняет этим последнее и, следовательно,
{\em продолжает существование до бесконечности}. Мир
есть некоторое существование; доказательство предполагает, что это
существование {\em возникает} и что возникновение имеет
{\em предшествующее} ему во времени
{\em условие}. Но сам
{\em антитезис в том} именно и
{\em состоит}, что нет никакого безусловного
существования, никакой абсолютной границы, и это мировое существование
всегда требует некоторого {\em предшествующего
условия}. Стало быть, подлежащее доказательству находится в доказательстве
как допущение. — Далее, доказательство ищет затем
{\em условия} в пустом времени, а это означает, что оно
принимает, что условие носит временной характер и, следовательно, есть
наличное бытие, некое ограниченное. Стало быть, делается вообще допущение,
что мир как наличное бытие предполагает некоторое другое обусловленное
наличное бытие во времени и~т.~д. до бесконечности.

Доказательство бесконечности мира в {\em пространстве}
таково же. В виде доказательства от противного принимается пространственная
конечность мира; «последний, следовательно, находился бы в пустом
неограниченном пространстве и имел бы некоторое
{\em отношение} к нему; но такое отношение мира к тому,
что {\em не есть какой бы то ни было предмет}, есть
ничто».

Здесь также в доказательстве прямо берется как предпосылка то, что требуется
доказать. Здесь прямо принимается, что ограниченный пространственный мир
находится в пустом пространстве и имеет к нему некоторое
{\em отношение},~т.~е. что необходимо
{\em выходить} вовне, за него, — с одной стороны, в
пустоту, в потустороннее миру и {\em небытие} этого
мира, но, с другой стороны, этот мир находится в
{\em отношении} с этим потусторонним,~т.~е., входя в
него, {\em продолжается} в нем, и мы, следовательно,
должны представлять себе это потустороннее наполненным мировым
существованием. Бесконечность мира в пространстве, которую провозглашает
антитезис, есть не что иное, как, с одной стороны, пустое пространство и, с
другой стороны, {\em отношение} мира к нему,~т.~е.
продолжение в нем мира, наполнение его. Это противоречие, предположение,
что пространство одновременно и пусто и наполнено, есть бесконечный
прогресс существования в пространстве. Это самое противоречие, отношение
мира к пустому пространству, прямо кладется в основание доказательством.

Поэтому тезис и антитезис и доказательства их представляют собою не что
иное, как противоположные утверждения, что имеется некоторая
{\em граница} и что она вместе с тем есть лишь
{\em упраздненная} граница; что граница имеет некое
потустороннее, с которым она, однако, находится в
{\em соотношении} и куда мы должны выходить за нее, но
где снова возникает такая граница, которая не есть граница.

{\em Разрешение} этих антиномий, как и предыдущих,
трансцендентально,~т.~е. оно состоит в утверждении, что пространство и
время как формы созерцания идеальны в том смысле, что мир
{\em в самом себе} не находится в противоречии с собою,
не есть некое упраздняющее себя, а лишь {\em сознание}
в своем созерцании и в соотношении созерцания с рассудком и разумом есть
противоречащая самой себе сущность. Это уже слишком большая нежность по
отношению к миру —~удалить из него противоречие, перенести, напротив, это
противоречие в дух, в разум и оставить его там неразрешенным. Дух в самом
деле столь силен, что может переносить противоречие, но он же и умеет
разрешать его. Это однако вовсе не значит, что так называемый мир (как бы
его ни именовали —~объективным ли, реальным миром или согласно
трансцендентальному идеализму —~субъективным созерцанием и определяемой
категориями рассудка чувственностью) свободен хоть где-нибудь от
противоречия, но он не в силах выносить его и потому предоставлен
возникновению и прехождению.

\subparagraph[с) \ Бесконечность определенного количества]{с)
\ Бесконечность определенного количества}
I. {\em Бесконечное определенное количество} как
{\em бесконечно большое} или
{\em бесконечно малое} есть само в себе бесконечный
прогресс; оно есть определенное количество как некоторое большое или малое
и вместе с тем —~небытие определенного количества. Бесконечно большое и
бесконечно малое суть поэтому образы представления, которые при ближайшем
рассмотрении оказываются жалкими туманами и тенями. Но в бесконечном
прогрессе это противоречие имеется в выявленном виде, и тем самым имеется в
выявленном виде то, что составляет природу определенного количества,
которое достигло своей реальности как интенсивная величина и теперь
{\em положено} в своем
{\em наличном бытии} таким, каково оно есть в своем
{\em понятии}. Это тождество мы теперь и должны
рассмотреть.

Определенное количество как градус просто, соотнесено с собою и определено
как нечто находящееся в самом себе. Так как через эту простоту инобытие и
определенность упразднены в нем, то последняя внешня ему; оно имеет свою
определенность вне себя. Это его вне-себя-бытие есть прежде всего
{\em абстрактное небытие} определенного количества
вообще, дурная бесконечность. Но это небытие есть, далее, также некоторая
величина; определенное количество продолжается в своем небытии, ибо оно
имеет свою определенность как раз во внешнем ему; это его внешнее есть
поэтому само тоже определенное количество; таким образом, указанное его
небытие, бесконечность, ограничивается,~т.~е. это потустороннее снимается,
оно само определено как определенное количество, которое, следовательно, в
своем отрицании остается у самого себя.

Но это есть как раз то, что определенное количество как таковое есть
{\em в себе}. Ибо оно есть {\em оно
же само} через свое вне-себя-бытие; внешность составляет то, через что оно
есть определенное количество, находится у себя. Следовательно, в
бесконечном прогрессе {\em понятие} определенного
количества {\em положено}.

Если мы его (бесконечный прогресс) возьмем сначала в его абстрактных
определениях, как они предлежат нам, то мы увидим, что
{\em в нем имеется снятие определенного количества, но
имеется также и снятие его потустороннего, имеется, следовательно, как
отрицание определенного количества, так и отрицание этого отрицания}. Его
истиной служит их единство, в котором они суть, однако, как моменты. — Оно
есть разрешение противоречия, выражением которого он служит, и ближайшим
смыслом бесконечного прогресса является поэтому
{\em восстановление понятия величины}, заключающегося в
том, что она есть безразличная или внешняя граница. Когда говорят о
бесконечном прогрессе как таковом, то обращают внимание обыкновенно только
на то, что каждое определенное количество, как бы оно ни было велико или
мало, может исчезать, что должно быть возможным выходить за него, но не на
то, что само это его снятие, потустороннее, дурно-бесконечное, также
исчезает.

Уже {\em первое} снятие, то отрицание качества вообще,
через которое полагается определенное количество, есть
{\em в себе} снятие отрицания, — определенное
количество есть снятая качественная граница и, следовательно, снятое
отрицание, — но вместе с тем оно таково лишь {\em в
себе}; положено же оно как некое наличное бытие, а затем его отрицание
фиксировано как бесконечное, как потустороннее определенному количеству,
которое стоит как некое посюстороннее, как некое
{\em непосредственное}; таким образом, бесконечное
определено лишь как {\em первое} отрицание, и таковым
оно выступает в бесконечном прогрессе. Но мы уже показали, что в последнем
имеется нечто большее, имеется отрицание отрицания, или то, что бесконечное
есть поистине. Ранее мы это рассматривали так, что тем самым восстановлено
{\em понятие} определенного количества; это
восстановление означает ближайшим образом, что его наличное бытие получило
свое более детальное определение, а именно, возникло
{\em определенное согласно своему понятию определенное
количество}, что отлично от {\em непосредственного
определенного количества}; {\em внешность} есть теперь
противоположность самой себе, положена как момент самой
{\em величины}, — возникло определенное количество,
взятое так, что оно посредством своего небытия, бесконечности, имеет свою
{\em определенность} в другом определенном
количестве,~т.~е. есть {\em качественно} то, что оно
есть. Однако это сравнение {\em понятия} определенного
количества с его наличным бытием есть больше дело нашего размышления, есть
отношение, которое здесь еще не дано. Пока, что имеется только то
определение, что определенное количество возвращается к
{\em качеству}, теперь уже определено качественно. Ибо
его своеобразие, его качество состоит во внешности, безразличии
определенности, и оно теперь положено как то, что в своем внешнем есть,
наоборот, оно же само, соотносится в нем с самим собою, определено в
простом единстве с собою,~т.~е. {\em качественно}. —
Это качественное определено еще ближе, а именно, как для-себя-бытие, ибо
соотношение с самим собою, к которому оно пришло, произошло из
опосредствования, из отрицания отрицания. Определенное количество уже
больше не имеет бесконечности, для-себя-определенности вне себя, а имеет ее
в самом себе.

Бесконечное, имеющее в бесконечном прогрессе лишь пустое значение небытия,
недостигнутого, но искомого потустороннего, есть на самом деле не что иное,
как {\em качество}. Определенное количество как
безразличная граница выходит за себя до бесконечности; оно этим стремится
не к чему другому как к для-себя-определенности, к качественному моменту,
который, однако, таким образом, есть лишь некоторое долженствование. Его
безразличие к границе, следовательно, отсутствие у него для-себя-сущей
определенности, и его выхождение за себя есть то, что делает определенное
количество определенным количеством; это его выхождение должно
подвергнуться отрицанию и найти себе в бесконечном свою абсолютную
определенность.

Выражая это в совершенно общем виде, скажем: определенное количество есть
снятое качество; но определенное количество бесконечно, выходит за себя,
оно есть отрицание себя; это его выхождение есть, следовательно,
{\em в себе} отрицание того отрицания, которому
подвергнуто качество, восстановление последнего; и положено именно то, что
внешность, выступавшая как потустороннее, определена как
{\em собственный момент} определенного количества.

Определенное количество этим положено как оттолкнутое от себя, благодаря
чему, следовательно, имеются два определенных количества, которые, однако,
сняты, суть лишь как моменты {\em единого единства}, и
это единство есть определенность определенного количества. — Последнее,
{\em соотнесенное}, таким образом, в своей внешности
{\em с собою} как безразличная граница, и,
следовательно, положенное качественно, есть
{\em количественное отношение}. — В отношении
определенное количество внешне себе, отлично от самого себя; эта его
внешность есть соотношение одного определенного количества с другим
определенным количеством, каждое из которых значимо лишь в этом своем
соотношении со своим другим; и это соотношение составляет определенность
определенного количества, представляющего собою такое единство.
Определенное количество имеет в нем не безразличное, а качественное
определение, в этой своей внешности возвратилось в себя, есть в ней то, что
оно есть.

\subsection*{Примечание 1. Определенность понятия математического
бесконечного}

{\em Математическое бесконечное} интересно, с одной
стороны, произведенным им расширением математики и теми великими
результатами, которые были достигнуты благодаря введению его в последнюю,
но, с другой стороны, оно достойно внимания вследствие того, что этой науке
еще не удалось оправдать посредством понятия (понятие мы здесь берем в
собственном его смысле) его применение. Предложенные оправдания основаны, в
конечном счете, на {\em правильности результатов},
получающихся при помощи этого определения, правильности,
{\em доказанной из других оснований}, но не на ясности
предмета и операции, посредством, которой получаются эти результаты, и даже
больше того: приводимые оправдания содержат признание того, что сама эта
операция неправильна.

Это уже само по себе есть нечто неудовлетворительное; такой образ действия
ненаучен. Но он влечет за собою еще и ту невыгоду, что математика, не зная
природы этого своего орудия вследствие того, что не справилась с его
метафизикой и критикой, не могла также определить, объем его применения и
обеспечить себя от злоупотребления им.

В философском же отношении математическое бесконечное важно потому, что на
самом деле в его основании лежит понятие истинного бесконечного и оно стоит
куда выше, чем обычно так называемое
{\em метафизическое бесконечное}, исходя из которого
против него выдвигаются возражения. От этих возражений наука математика
часто умеет спасаться лишь тем, что она отвергает компетенцию метафизики,
утверждая, что ей нет дела до этой науки, что ей нечего заботиться о
понятиях последней, если только она действует последовательно на своей
собственной почве. Она-де должна рассматривать не то, что истинно в себе, а
то, что истинно в ее области.

При всех своих возражениях против математического бесконечного метафизика не
может отрицать или опровергнуть блестящих результатов, которые дало его
применение, а математика не умеет выяснить метафизику своего собственного
понятия и поэтому не в состоянии также и дать вывод тех приемов, которые
делает необходимым применение бесконечного.

Если бы над математикой тяготело единственно только затруднение, причиняемое
{\em понятием} вообще, то она могла бы без околичностей
оставить его в стороне, поскольку именно понятие есть нечто большее, чем
только указание существенных определенностей,~т.~е. рассудочных определений
какой-нибудь вещи, а в недостаточной отчетливости этих определенностей
математику никак нельзя упрекнуть; она могла бы оставить в стороне это
затруднение, ибо она не есть такого рода наука, которая должна иметь дело с
понятиями своих предметов и порождать свое содержание посредством развития
понятия, хотя бы только путем (рассудочных) рассуждений. Но при методе
применения ею своего бесконечного она встречает
{\em главное противоречие} в самом том
{\em своеобразном методе}, на котором она вообще
основана как наука. Ибо исчисление бесконечного дозволяет и требует таких
приемов, которые она должна отвергать при действиях над конечными
величинами, и вместе с тем она обращается со своими бесконечными
величинами, как с конечными определенными количествами и хочет применять к
первым те же самые приемы, которые имеют место при действиях над
последними. Основной чертой развития этой науки является то, что она
применяла к {\em трансцендентным} определениям и
действиям над ними форму обычного исчисления.

При всей этой противоречивости своих операций математика показывает, что
результаты, которые она получает посредством их, вполне совпадают с теми,
которые она получает посредством математического метода в собственном
смысле, посредством геометрического и аналитического методов. Однако
{\em частью} это касается не всех результатов, и целью
введения исчисления бесконечно-малых является не только сокращение обычного
пути, а получение таких результатов, которых последний не может дать.
{\em Частью} же, с другой стороны, следует сказать, что
{\em успех} сам по себе не оправдывает
{\em характера пути} (die Manier des Wegs). А этот
характер исчисления бесконечных оказывается пораженным видимостью
{\em неточности}, которую он сам себе придает, когда
конечные величины увеличиваются на бесконечно малую величину, и эта
последняя в дальнейших действиях частью сохраняется, но некоторою частью ее
также и пренебрегают. Этот прием заключает в себе ту странность, что,
несмотря на признаваемую неточность, получается результат, который не
только {\em довольно точен} и так
{\em близок} к истинному результату, что
{\em можно не обращать внимания} на разницу, но и
{\em совершенно точен}. В самом же
{\em действии}, предшествующем результату,
{\em нельзя обойтись без представления}, что некоторые
величины не равны нулю, но так {\em незначительны}, что
их можно оставить без внимания. Однако в том, что понимают под
математической определенностью, совершенно отпадает всякое различие между
большей или меньшей точностью, точно так же, как в философии не может итти
речь о большей или меньшей вероятности, а единственно только о истине. Если
метод и употребление бесконечных и оправдывается успехом, то все-таки вовсе
не излишне, несмотря на это, требовать их оправдания; такое требование
представляется не столь излишним, как, например, представляется излишним
требовать доказательства права пользоваться собственным
носом[ссылка!]. Ибо в математическом познании,
как представляющем собою научное познание, имеет существенную важность
доказательство, а в отношении получаемых результатов тоже оказывается, что
строго математический метод не для всех их доставляет доказательство от
успеха, которое, однако, и помимо этого является лишь внешним
доказательством.

Стоит труда рассмотреть ближе математическое понятие бесконечного и те
наиболее замечательные попытки, которые ставят себе целью оправдать
пользование им и устранить затруднение, тяготеющее над методом.
Рассмотрение этих оправданий и определений математического бесконечного,
которые я изложу в этом примечании более пространно, бросит вместе с тем
наиболее яркий свет и на самую природу истинного понятия и покажет, что оно
предносилось уму авторов этих попыток и лежало в основании последних.

Обычное определение математического бесконечного гласит, что оно есть
{\em величина, больше которой}, — если она определена
как бесконечно большая, или {\em меньше которой}, если
она определена как бесконечно малая, — {\em уже нет}
или —~в другой формулировке —~как величина, которая в первом случае больше,
а во втором меньше какой угодно другой величины. — В этой дефиниции,
конечно, не выражено истинное понятие, а, наоборот, как мы уже заметили,
здесь выражено лишь то же самое противоречие, которое содержится в
бесконечном прогрессе. Но посмотрим, что содержится в ней
{\em самой по себе}. Величина определяется в математике
как то, что может быть увеличиваемо или уменьшаемо, следовательно, вообще,
как безразличная граница. И вот, так как бесконечно-большое или
бесконечно-малое есть нечто такое, что уже больше не может быть
увеличиваемо или уменьшаемо, то оно на самом деле уже больше
{\em не есть определенное количество} как таковое.

Этот вывод необходим и непосредственен. Но именно это соображение,
показывающее, что определенное количество, — а я называю в этом примечании
определенным количеством вообще то, что оно есть, а именно конечное
количество, — снято, обыкновенно как раз и не приходит на ум, а между тем
оно-то и составляет затруднение для обычного понимания, так как требуется,
чтобы определенное количество, когда оно бесконечно, мыслилось как некое
снятое, как такое нечто, которое не есть определенное количество, но
{\em количественная определенность которого тем не
менее сохраняется}.

Если обратимся к тому, как относится к этому определению
{\em Кант}\footnote{В примечании к тезису
первой космологической антиномии в «Критике чистого разума».}, то увидим,
что он его находит несогласующимся с тем, что понимают
{\em под бесконечным целым}. «Согласно обычному понятию
та величина бесконечна, больше которой (т.~е. больше содержащегося в ней
{\em множества} данных единиц) не может быть никакая
другая величина; но никакое множество не есть наибольшее, так как всегда
возможно прибавить к нему одну или несколько единиц. — Относительно же
бесконечного целого мы не представляем себе, как оно
{\em велико}, и, следовательно, его понятие не есть
понятие некоторого {\em максимума} (или минимума), а мы
мыслим через это понятие лишь его {\em отношение} к
произвольно взятой {\em единице}, относительно которой
оно больше, чем всякое число. Смотря по тому, примем ли мы эту единицу
большей или меньшей, бесконечное будет большим или меньшим; но
бесконечность, так как она состоит только в
{\em отношении} к этой данной единице, остается всегда
одной и той же, хотя, разумеется, абсолютная величина целого этим вовсе не
будет узнана».

Кант порицает рассматривание бесконечного целого как некоторого максимума,
как некоторого {\em завершенного} множества данных
единиц. Максимум или минимум как таковой все еще представляется некоторым
определенным количеством, множеством. Такое представление не может
отклонить указанный Кантом вывод, который приводит к большему или меньшему
бесконечному. Вообще, когда бесконечное представляют себе как определенное
количество, для него сохраняет значение различие большего или меньшего. Но
эта критика не касается понятия истинного математического бесконечного,
бесконечной разности, ибо последняя уже больше не есть конечное
определенное количество.

Напротив, понятие бесконечности, даваемое Кантом, понятие, которое он
называет истинно трансцендентальным, гласит, что «последовательный
{\em синтез} единицы в измерении определенного
количества {\em никогда} не может быть
{\em завершен}». В этом понятии предполагается, как
данное, определенное количество вообще; требуется, чтобы оно было
превращено посредством синтеза {\em единицы} в
некоторую численность, в долженствующее быть указанным определенное
количество, но, по утверждению Канта, невозможно когда-либо закончить этот
синтез. Здесь очевидно выражено не что иное как бесконечный прогресс,
который только представляют себе
{\em трансцендентально},~т.~е., собственно говоря,
субъективно и психологически. Само по себе, дескать, определенное
количество завершено, но трансцендентальным образом,~т.~е. в
{\em субъекте}, сообщающем ему
{\em отношение} к некоторой единице, возникает лишь
такое определение определенного количества, которое не завершено и
безоговорочно обременено потусторонним. Следовательно, здесь вообще
застревают в противоречии, которое содержится в величине, но распределяют
это противоречие между объектом и субъектом, так что на долю первого
выпадает ограниченность, а на долю второго —~выхождение за каждую
представляемую им себе определенность, выхождение в дурную бесконечность.

Мы, напротив, уже сказали выше, что определение математического бесконечного
и притом так, как его употребляют в высшем анализе, соответствует понятию
истинного бесконечного; теперь мы предпримем сопоставление этих двух
определений в более развернутом виде. — Что касается, прежде всего, истинно
бесконечного определенного количества, то оно определилось как
{\em в самом себе} бесконечное; оно таково, поскольку,
как мы выяснили, и конечное определенное количество или определенное
количество вообще, и его потустороннее или дурное бесконечное
{\em одинаково} сняты. Снятое определенное количество
возвратилось тем самым к простоте и к соотношению с собою самим, но не
только так, как экстенсивное определенное количество, когда оно перешло в
интенсивное определенное количество, имеющее свою определенность в
некотором внешнем многообразии лишь {\em в себе},
причем оно, однако, по предположению безразлично к этому многообразию и
отлично от него. Бесконечное определенное количество содержит, напротив,
во-первых, внешность и, во-вторых, ее отрицание в нем самом. Таким образом,
оно уже больше не есть некоторое конечное определенное количество, не есть
некоторая определенность величины, имеющая
{\em наличное бытие как определенное количество}, а оно
просто, и поэтому имеет бытие лишь как {\em момент};
оно есть определенность величины в {\em качественной}
форме; его бесконечность состоит в том, что оно дано как некоторая
{\em качественная определенность}. Таким образом, оно
как момент находится в существенном единстве со своим другим, имеет бытие,
лишь как определенное этим своим другим,~т.~е. оно обладает значением лишь
в связи с некиим, находящимся к нему {\em в отношении}.
{\em Вне этого отношения} оно
{\em нуль}, между тем как раз определенное количество
как таковое, согласно предположению, безразлично к
{\em отношению} и тем не менее является в нем некоторым
{\em непосредственным} покоящимся определением.
{\em В отношении} оно, как представляющее собою лишь
момент, не есть некое стоящее особняком (für sich) безразличное; в
бесконечности как {\em для-себя-бытии}, оно, будучи
вместе с тем некоторой количественной определенностью, имеет бытие лишь как
некоторое «{\em для одного}».

Понятие бесконечного, как оно изложено здесь абстрактно, окажется лежащим в
основании математического бесконечного, и оно само сделается яснее, когда
мы рассмотрим различные ступени выражения определенного количества как
{\em момента отношения}, начиная с низшей ступени, на
которой оно еще есть вместе с тем определенное количество как таковое, и
кончая высшей, где оно получает значение и выражение бесконечной величины в
собственном смысле.

Итак, возьмем сначала определенное количество в том
{\em отношении}, в котором оно есть
{\em дробное число}. Такая дробь, например,  $\frac 2
7$  не есть такое определенное количество, как 1, 2, 3 и~т.~д.; она есть,
правда, обыкновенное конечное число, однако не непосредственное, подобно
целым числам, а, как дробь, определенное посредственно
{\em двумя другими числами}, которые суть в отношении
друг друга численность и единица, причем и единица также есть некоторая
численность. Но взятые абстрагированно от этого их ближайшего определения в
отношении друг друга и рассматриваемые лишь со стороны того, что в том
качественном соотношении, в котором они здесь находятся, происходит с ними,
как с определенными количествами 2 и 7 помимо этого соотношения суть
безразличные определенные количества; но так как они здесь выступают как
{\em моменты} друг друга и, стало быть, некоторого
третьего (того определенного количества, которое называется показателем),
то они имеют значение не как 2 и 7, а лишь со стороны их определенности в
{\em отношении друг друга}. Вместо них можно поэтому
поставить также 4 и 14 или 6 и 21 и~т.~д. до бесконечности. Тем самым они,
следовательно, начинают получать качественный характер. Если бы они имели
значение просто как определенные количества, то пришлось бы признать, что 2
и 7 суть одно —~лишь 2, а другое —~лишь 7; 4, 14, 6, 21 и~т.~д. суть
безоговорочно нечто другое, чем эти числа и, поскольку они суть лишь
непосредственные определенные количества, они не могут быть подставлены
одни вместо других. Но поскольку 2 и 7 имеют значение не со стороны той
определенности, что они суть такие определенные количества, постольку их
безразличная граница снята; они, стало быть, с этой стороны заключают в
себе момент бесконечности, ибо они не только как раз уже больше не суть то,
что они суть, а еще кроме того сохраняется их количественная
определенность, но как в себе сущая качественная определенность, — а
именно, согласно тому, что они значат в отношении. Вместо них может быть
поставлено бесконечное множество других чисел, так что величина дроби не
изменяется вследствие той определенности, которую имеет отношение.

Но выражение, которое бесконечность находит в изображении ее числовой
дробью, еще несовершенно потому, что оба члена дроби, 2 и 7, могут быть
изъяты из отношения, и тогда они суть обыкновенные безразличные
определенные количества; их соотношение, то обстоятельство, что они суть
члены отношения и моменты, есть для них нечто внешнее и безразличное. И
точно так же само их {\em соотношение} есть
обыкновенное определенное количество, показатель отношения.

{\em Буквам}, которыми оперируют в общей
арифметике,~т.~е. ближайшей всеобщности, в которую возводятся числа, не
присуще свойство обладать определенной числовой величиной; они суть лишь
всеобщие знаки и неопределенные возможности любой определенной величины.
Дробь  $\frac a b$  представляется поэтому более подходящим выражением
бесконечного, так как {\em a} и
{\em b}, изъятые из их взаимоотношения, остаются
неопределенными и не обладают особой им принадлежащей величиной, даже
будучи отделены друг от друга. — Однако, хотя эти буквы положены как
неопределенные величины, их смысл все же состоит в том, что они суть
какое-либо конечное определенное количество. Так как они суть хотя и
всеобщее представление, но лишь об {\em определенном
числе}, то для них равным образом безразлично то обстоятельство, что они
находятся в отношении, и вне последнего они сохраняют то же самое значение.

Если присмотримся еще ближе к тому, что представляет собою отношение, то мы
увидим, что ему присущи оба определения: оно,
{\em во-первых}, есть некоторое определенное
количество, но последнее есть, {\em во-вторых}, не
некоторое непосредственное, а определенное количество, содержащее в себе
качественную противоположность; оно вместе с тем остается в отношении тем
данным, безразличным определенным количеством благодаря тому, что оно
возвращается из своего инобытия, из противоположности, в себя, и,
следовательно, есть также некоторое бесконечное. Эти два определения,
развитые в их отличии друг от друга, представляются в следующей
общеизвестной форме.

Дробь  $\frac 2 7$  может быть выражена как 0,285714..., \  $\frac 1{1-a}$ ,
— как  $1 + a + a^2 + a^3$  и~т.~д. Таким образом, она имеет бытие как
{\em некоторый бесконечный ряд}; сама дробь называется
суммой или {\em конечным выражением} этого ряда. Если
сравним между собою эти два выражения, то окажется, что одно, бесконечный
ряд, уже представляет ее не как отношение, а с той стороны, что она есть
некоторое определенное количество как {\em множество}
таких количеств, которые присоединяются одно к другому, — как некоторая
численность. — Что величины, долженствующие ее составить как некоторую
численность, сами в свою очередь состоят из десятичных дробей, сами,
следовательно, состоят из отношений, это здесь не имеет значения; ибо это
обстоятельство касается особого рода {\em единицы} этих
величин, а не их, поскольку они конституируют
{\em численность}; ведь и состоящее из нескольких цифр
целое число десятеричной системы также считается по существу одной
{\em численностью} и не обращается внимания на то, что
она состоит из {\em произведений} некоторых чисел на
число десять и его степени. Равным образом здесь не имеет значения то
обстоятельство, что имеются другие дроби, нежели взятая в виде примера
дробь  $\frac 2 7$ , которые, будучи обращены в десятичные дроби, не дают
бесконечного ряда; однако каждая из них может быть изображена как таковой
ряд в числовой системе другой единицы.

Так как в бесконечном ряде, который должен представлять собою дробь как
некоторую численность, исчезает тот аспект, что она есть отношение, то
исчезает также и тот аспект, что она, как мы показали выше, имеет
бесконечность в {\em ней} (в дроби). Но эта
бесконечность вошла другим образом, а именно, сам ряд бесконечен.

Какого рода эта бесконечность ряда, явствует само собою; это —~дурная
бесконечность прогресса. Ряд содержит в себе и представляет собою то
противоречие, что нечто, являющееся отношением и имеющее внутри себя
{\em качественную} природу, изображается как лишенное
отношений, как исключительно только {\em определенное
количество}, как численность. Следствием этого противоречия оказывается то,
что в численности, выражаемой в ряде, всегда чего-то недостает, так что
всегда нужно выходить за пределы того, что положено, чтобы достигнуть
требуемой определенности. Закон этого поступательного движения известен; он
заключается в определении определенного количества, содержащегося в дроби,
и в природе формы, в которой это определение должно быть выражено. Можно,
правда, посредством продолжения ряда сделать численность столь точной,
сколь это {\em нужно}. Однако изображение численности
посредством ряда всегда остается лишь
{\em долженствованием}; оно обременено некоторым
{\em потусторонним}, которое не может быть устранено,
так как выражение в виде {\em численности} того, что
основано на {\em качественной} определенности,
представляет собою {\em постоянное противоречие}.

В этом бесконечном ряде действительно имеется та
{\em неточность}, которая в истинном математическом
бесконечном встречается лишь как видимость. Не следует смешивать эти два
вида математического бесконечного, точно так же, как не следует смешивать
вышеуказанные два вида философского бесконечного. Первоначально применяли
для изображения истинного математического бесконечного
{\em форму ряда}, и в новейшее время она опять была
вызвана к жизни. Но она для него не необходима. Напротив, как сделается
ясным в дальнейшем, бесконечное бесконечного ряда существенно отлично от
этого истинного бесконечного. Он, напротив, уступает в этом отношении даже
выражению бесконечного, даваемому дробью.

А именно, {\em бесконечный ряд} содержит в себе дурную
бесконечность, так как то, что должно быть выражено рядом, остается
{\em долженствованием}, а то, что он выражает,
обременено неисчезающим потусторонним и {\em отлично}
от того, что должно быть выражено. Он бесконечен не из-за тех своих членов,
которые положены, а потому, что они неполны, так как другое, которое по
существу принадлежит к ним, находится по ту сторону их; то, что в нем есть,
хотя бы положенных членов было сколь угодно много, есть лишь конечное в
собственном смысле этого слова, положено как конечное,~т.~е. как нечто
такое, {\em что не есть то, чем оно должно быть}.
Напротив, то, что называется {\em конечным выражением}
или {\em суммой} такого ряда, безупречно; оно содержит
в себе полностью то значение, которого ряд только ищет; убегавшее
потустороннее снова возвращено назад; то, что этот ряд есть, и то, чем он
должен быть, уже не разделено, а есть одно и то же.

Различие между ними, скажем сразу, заключается ближе в том, что в
бесконечном ряде {\em отрицательное} находится
{\em вне} тех его членов, которые имеются налицо, так
как они признаются лишь частями {\em численности}.
Напротив, в конечном выражении, которое есть отношение,
{\em отрицательное} находится внутри него как
определяемость членов отношения {\em друг другом},
которая есть возвращение в себя, соотносящееся с собою единство как
отрицание отрицания ({\em оба} члена отношения имеют
бытие лишь как {\em моменты}), и, следовательно,
{\em имеет внутри себя} определение бесконечности. —
Таким образом, обыкновенно {\em так называемая сумма}, 
$\frac 2 7$  или  $\frac 1{1-a}$ , есть на самом деле
{\em отношение}, и это так называемое
{\em конечное выражение} есть истинно
{\em бесконечное выражение}. Напротив, бесконечный
{\em ряд} есть на самом деле
{\em сумма}; его цель состоит в том, чтобы представить
то, что в себе есть отношение, в форме некоторой суммы, и имеющиеся налицо
члены ряда имеют бытие как члены не некоторого отношения, а агрегата. Он,
далее, есть скорее {\em конечное выражение}, ибо он
есть несовершенный агрегат и остается чем-то существенно недостаточным. По
тому, что в нем имеется, он есть некоторое определенное количество, но
вместе с тем меньшее того определенного количества, которым он должен быть;
а затем, и то, чего ему недостает, также есть некоторое определенное
количество; эта недостающая часть и есть на самом деле то, что называется в
ряде бесконечным только с той формальной стороны, что она есть некоторое
недостающее, некоторое небытие; по своему же содержанию она есть конечное
определенное количество. Только то, что налично в ряде, вместе с тем, чего
ему недостает, составляет то, что представляет собою дробь, то определенное
количество, которым он также {\em должен} быть, но
которым он не в состоянии быть. — Слово «бесконечное» также и в сочетании
«бесконечный ряд» обыкновенно кажется мнению чем-то высоким и
величественным; это —~вид суеверия, суеверие рассудка. Мы видели, что оно,
наоборот, сводится к определению {\em недостаточности}.

Можно еще заметить, что существование таких бесконечных рядов, которые не
суммируются, есть в отношении формы ряда вообще обстоятельство внешнее и
случайное. Эти ряды содержат в себе высший вид бесконечности, чем
суммирующиеся ряды, а именно, несоизмеримость или, иначе говоря,
невозможность представить содержащееся в них количественное отношение как
некоторое определенное количество, хотя бы в виде дроби. Но свойственная им
{\em форма ряда} как таковая содержит в себе то же
самое определение дурной бесконечности, какое присуще суммируемому ряду.

Только что указанная на примере дроби и ее ряда превратность выражения имеет
также место, когда {\em математическое} бесконечное
—~не только что названное, а истинное —~называют
{\em относительным} бесконечным, а, напротив, обычное
{\em метафизическое}, под которым разумеют абстрактное,
дурное бесконечное, {\em абсолютным}. На самом же деле,
наоборот, это метафизическое бесконечное лишь относительно, потому что
отрицание, которое оно выражает, противоположно границе лишь в том смысле,
что последняя остается {\em существовать} вне него и не
снимается им; напротив, математическое бесконечное поистине сняло конечную
границу внутри себя, так как ее потусторонность соединена с нею.

Преимущественно в том смысле, в котором мы показали, что так называемая
сумма или конечное выражение бесконечного ряда должно быть, наоборот,
рассматриваемо как бесконечное выражение, {\em Спиноза}
выставляет и поясняет примерами понятие истинной бесконечности в
противоположность дурной. Его понятие будет лучше всего освещено, если я
рассмотрю сказанное им об этом предмете непосредственно вслед за только что
изложенными соображениями.

Он сначала определяет {\em бесконечное} как
{\em абсолютное утверждение} существования какой-нибудь
природы, а конечное, напротив, как
{\em определенность}, как
{\em отрицание}. Абсолютное утверждение некоторого
существования следует именно понимать как его соотношение с
{\em самим собою}, означающее, что оно есть не
благодаря тому, что другое есть; конечное же есть, напротив, отрицание,
прекращение как {\em соотношение} с некоторым
{\em другим}, начинающимся {\em вне
его}. Абсолютное утверждение некоторого существования, правда, не
исчерпывает понятия бесконечности; это понятие означает, что бесконечность
есть утверждение не как непосредственное, а лишь как восстановленное через
рефлексию другого в само себя, или, иначе говоря, как отрицание
отрицательного. Но у Спинозы субстанция и ее абсолютное единство имеют
форму неподвижного,~т.~е. не опосредствующего себя с самим собою единства,
— форму некоторой оцепенелости, в которой еще не находится понятие
отрицательного единства самости, субъективность.

Математическим примером, которым он поясняет истинное бесконечное (письмо
XXIX), служит пространство между двумя неравными кругами, один из которых
находится внутри другого, не касаясь его, и которые не концентричны. Он,
повидимому, придавал столь большое значение этой фигуре и тому понятию, в
качестве примера
которого[ссылка!] он ее применяет, что сделал ее
эпиграфом своей
«Этики»[ссылка!], — «Математики», говорит он:
«умозаключают, что неравенства, возможные в таком пространстве, бесконечны
не от бесконечного {\em множества} частей, ибо
{\em величина} этого пространства является
{\em определенной} и
{\em ограниченной} и я могу предположить такое
пространство большим или меньшим, а они делают этот вывод на том основании,
что {\em природа этой вещи} превосходит всякую
определенность»[ссылка!]. — Как видим, Спиноза
отвергает то представление о бесконечном, согласно которому представляют
себе его как множество или как незавершенный ряд, и напоминает, что в
пространстве, приводимом им как пример, бесконечное не находится по ту
сторону, а налично и полно; это пространство есть нечто ограниченное, но
бесконечное именно потому, «что природа вещи превосходит всякую
определенность», так как содержащееся в нем определение величины вместе с
тем не может быть представлено как некоторое определенное количество или,
употребляя вышеприведенное выражение Канта,
{\em синтезирование} не может быть закончено, доведено
до некоторого дискретного —~определенного количества. — Каким образом
противоположность между {\em непрерывным} и
{\em дискретным} определенным количеством приводит к
бесконечному, — это мы разъясним в одном из следующих примечаний. —
Бесконечное некоторого ряда Спиноза называет
{\em бесконечным воображения}, бесконечное же, как
соотношение с собою самим, он называет {\em бесконечным
мышления} или infinitum actu (актуально бесконечным). Оно именно actu,
{\em действительно} бесконечно, так как оно завершено
внутри себя и налично. Так например, ряд
$0,285714...$ или $1+a+a^2+a^3...$
есть лишь бесконечное воображение или мнения, ибо он не обладает
действительностью, ему безоговорочно чего-то недостает. Напротив,
$\frac 2 7$ или $\frac 1{1-a}$ есть в действительности не только то,
что ряд представляет собою в своих наличных членах, но вдобавок к этому
еще и то, чего ему недостает, чем он только {\em должен быть}.
$\frac 2 7$ или $\frac 1{1-a}$
есть такая же конечная величина, как заключенное между двумя
кругами пространство и его неравенства в примере Спинозы, и, подобно этому
пространству, она может быть сделана большей или меньшей. Но отсюда не
получается несообразность большего или меньшего бесконечного, так как это
определенное количество целого не касается отношения его моментов,
{\em природы вещи}, т.~е. качественного определения
величины; то, что в бесконечном ряде {\em имеется
налицо}, есть также некоторое конечное определенное количество, но кроме
того еще нечто недостаточное. — Напротив,
{\em воображение} не идет дальше определенного
количества как такового и не принимает во внимание качественного
соотношения, составляющего основание имеющейся несоизмеримости.

Несоизмеримость, имеющая место в примере, приводимом Спинозой, заключает в
себе вообще криволинейные функции и приводит к тому бесконечному, которое
ввела математика при действиях с этими функциями и вообще при действиях
{\em с функциями переменных величин}; последнее есть
именно то {\em истинно математическое}, качественное
бесконечное, которое мыслил также и Спиноза. Это определение мы должны
здесь рассмотреть ближе.

Что касается, прежде всего, признаваемой столь важной категории
{\em переменности}, под которую подводятся соотносимые
в этих функциях величины, то они ближайшим образом переменны не в том
смысле, в котором в дроби  $\frac 2 7$  переменны оба числа 2 и 7,
поскольку вместо них можно поставить также 4 и 14, 6 и 21 и~т.~д. до
бесконечности без изменения значения дроби. В этом смысле можно еще с
большим правом поставить в дроби  $\frac a b$  вместо
{\em a} и {\em b} любые числа без
изменения того, что должно выражать собою  $\frac a b$ . Лишь в том смысле,
что также и вместо {\em x} и
{\em y} в какой-либо функции можно поставить
бесконечное,~т.~е. неисчерпаемое {\em множество} чисел,
{\em a} и {\em b} суть такие же
переменные величины, как и {\em x} и
{\em y}. Поэтому выражение
«{\em переменные величины}» страдает неясностью и
неудачно выбрано для определений величин, интересность которых и способы
действий над которыми коренятся {\em в чем-то
совершенно другом}, чем только в их переменности.

Чтобы сделать ясным, в чем заключается истинное определение тех моментов
какой-нибудь функции, которыми занимается высший анализ, мы должны снова
вкратце обозреть указанные выше ступени. В дробях  $\frac 2 7$  или  $\frac
a b$  числа 2 и 7, каждое само по себе, суть определенные количества и
соотношение для них несущественно; {\em a} и
{\em b} также должны быть представителями таких
определенных количеств, которые остаются тем, что они суть, также и вне
отношения. Далее,  $\frac 2 7$  и  $\frac a b$  суть также некоторые
постоянные определенные количества, некоторые частные; отношение составляет
некоторую численность, единицей которой служит знаменатель, а численностью
этих единиц —~числитель или обратно. Если бы мы подставили вместо 2 и 7 —~4
и 14 и~т.~д., то отношение осталось бы тем же самым также и как
определенное количество. Но это существенно изменяется, например, в функции
 $\frac{y^2} x=p$ ; здесь, правда, {\em x} и
{\em y} имеют значение определенных количеств; но
определенное частное имеют не {\em x} и
{\em y}, а лишь {\em x} и
{\em y\textsuperscript{2}}. Благодаря этому указанные
{\em члены} отношения {\em x} и
{\em y} не только не суть,
{\em во-первых}, такие-то определенные количества, но
и, {\em во-вторых} их
{\em отношение} не есть некоторое постоянное
определенное количество (а также и не {\em имеется в
виду} таковое, как это, например, имеет место при
{\em a} и {\em b}), не есть
постоянное частное, а это частное {\em как определенное
количество} совершенно {\em переменно}. Но это зависит
только от того, что {\em x} находится в отношении не к
{\em y}, а к {\em квадрату}
{\em y}. Отношение некоторой величины к
{\em степени} есть не
{\em определенное количество}, а по существу
{\em качественное} отношение.
{\em Степенное отношение} есть
{\em то обстоятельство}, которое должно рассматриваться
как {\em основное определение}. — В функции же прямой
линии  $y=\mathit{ax}$  выражение  $\frac y x=a$  есть обыкновенная дробь и
частное; эта функция есть поэтому лишь {\em формально}
функция переменных величин или, иначе говоря, {\em x} и
{\em y} представляют собою здесь то же самое, что
{\em a} и {\em b} в\textit{ }
$\frac a b$ , они не имеют того определения, под которым их рассматривает
диференциальное и интегральное исчисление. — Вследствие
{\em особенной} природы переменных величин в этом
способе рассмотрения было бы целесообразно ввести для них как особое
название, так и {\em особые обозначения}, отличные от
обычных названия и обозначений {\em неизвестных
величин} в каждом конечном, определенном ли или неопределенном уравнении, —
это было бы указанием их существенного отличия от таких просто неизвестных
величин, которые в себе суть вполне определенные количества или
определенная совокупность определенных количеств. — И в самом деле, лишь
отсутствие сознания своеобразия того, что составляет интерес высшего
анализа и чем вызваны потребность в диференциальном исчислении и
изобретение его, привело к включению функций первой степени, каково
уравнение прямой линии, в состав этого особого исчисления; доля вины за
такой формализм ложится также и на то недоразумение, по которому полагают,
что возможно выполнить само по себе правильное требование
{\em обобщения} какого-нибудь метода тем, что
опускается та {\em специфическая} определенность, на
которой основана потребность в этом методе, так что считается, что дело
идет в рассматриваемой нами области только о
{\em переменных величинах вообще}. Значительная доля
формализма в рассмотрении, равно как и трактовке этих предметов, несомненно
не имела бы места, если бы поняли, что диференциальное исчисление касается
не переменных величин как таковых, а {\em степенных
определений}.

Но имеется еще дальнейшая ступень, на которой выступает в своем своеобразии
математическое бесконечное. В уравнении, в котором
{\em x} и {\em y} положены
ближайшим образом как определенные некоторым степенным отношением,
{\em x} и {\em y} как таковые
должны еще означать некоторые определенные количества; и вот это значение
совершенно утрачивается в так называемых
{\em бесконечно малых разностях},
{\em dx}, {\em dy} уже не суть
определенные количества и не должны обозначать таковых, а имеют значение
лишь в своем соотношении, {\em имеют смысл лишь как
моменты}. Они уже больше не суть {\em нечто}, если
принимать нечто за определенное количество, не суть конечные разности; но
они также и {\em не} суть
{\em ничто}, не суть лишенный определения нуль. Вне
своего отношения они —~чистые нули, но их следует брать только как моменты
отношения, как {\em определения} диференциального
коэфициента  $\frac{\mathit{dx}}{\mathit{dy}}$ .

В этом понятии бесконечного определенное количество подлинно завершено в
некоторое качественное наличное бытие; оно положено как действительно
бесконечное; оно снято не только как то или иное определенное количество, а
как определенное количество вообще. Но при этом
{\em сохраняется количественная определенность} как
{\em элемент} определенных количеств, как принцип или,
как также выражались, она сохраняется {\em в своем
первом понятии}.

Против этого понятия и направлено все то нападение, которому подверглось
основное определение математики этого бесконечного, — диференциального и
интегрального исчисления. Неправильные представления самих математиков
вызвали непризнание этого понятия; но преимущественно вина за эти нападки
ложится на неспособность оправдать этот предмет как
{\em понятие}. Но понятия, как было указано выше,
математика не может здесь обойти, ибо как математика бесконечного она не
ограничивается рассмотрением {\em конечной}
определенности своих предметов, — как например, в чистой математике
пространство и число и их определения рассматриваются и соотносятся друг с
другом лишь со стороны их конечности, — а она приводит заимствованное
оттуда и рассматриваемое ею определение в
{\em тождество с его противоположностью}, превращая,
например, кривую линию в прямую, круг в многоугольник и~т.~д. Поэтому
действия, к которым она позволяет себе прибегать в диференциальном и
интегральном исчислении, находятся в полном противоречии с природой
исключительно только конечных определений и их соотношений и, стало быть,
могли бы найти свое оправдание только в {\em понятии}.

Если математика бесконечного настаивала на том, что эти количественные
определения суть исчезающие величины,~т.~е. такие величины, которые уже
больше не суть какие-либо определенные количества, но не суть также и
ничто, а еще представляют собою известную
{\em определенность относительно другого}, то
нападавшим на нее казалось, что ничего нет яснее того, что не может быть
такого, как они выражались, {\em среднего состояния}
между бытием и ничто. — Каково значение этого возражения и так называемого
среднего состояния, это уже было указано выше при рассмотрении категории
становления, примечание 4. Конечно, единство бытия и ничто не есть
{\em состояние}; состояние было бы таким определением
бытия и ничто, в которое эти моменты, так сказать, попали только случайно,
как бы впав в болезнь или подвергшись внешнему воздействию со стороны
ошибочного мышления, между тем как эта средина и это единство, исчезание,
которое есть также и становление, напротив, единственно и есть их
{\em истина}.

То, что бесконечно, говорили далее, не {\em подлежит
сравнению} как большее или меньшее; поэтому, не может быть отношения
бесконечного к бесконечному, по порядкам или достоинствам бесконечного, а
между тем мы встречаем таковые различия бесконечных разностей в науке,
трактующей о них. — В основании этого уже упомянутого выше возражения все
еще лежит то представление, что здесь идет речь об
{\em определенных количествах}, сравниваемых как
определенные количества, и что определения, которые уже не суть
определенные количества, не имеют больше никакого отношения друг к другу. В
действительности же дело обстоит наоборот: то, что
{\em только} находится в отношении, не есть
определенное количество. Определенное количество есть такое определение,
которое вне своего отношения должно иметь совершенно безразличное к другим
наличное бытие, определение, которому должно быть безразлично его отличие
от некоего другого, между тем как качественное есть, напротив, лишь то, что
оно есть в своем различии от другого. Поэтому указанные бесконечные
величины не только сравнимы, но имеют бытие лишь как моменты сравнения,
отношения.

Я приведу важнейшие определения, которые были даны в математике относительно
этого бесконечного; из них сделается ясным, что в их основании лежит такая
мысль о предмете, которая согласуется с развитым здесь понятием, но что
создатели этой отрасли математики не обосновали этой мысли как понятие, и в
применениях они вынуждены были прибегать к обходным средствам,
противоречащим их лучшему делу.

Эта мысль не может быть определена правильнее, чем то сделал
{\em Ньютон}. Я оставлю здесь в стороне определения,
принадлежащие к представлению движения и скорости (от которых он главным
образом и заимствовал название {\em флюксий}), так как
в них мысль выступает не с надлежащею абстрактностью, а конкретно, смешана
с формами, лежащими вне существа дела. Эти флюксии объясняются Ньютоном в
том смысле
(Princ. mathem. phil. nat., lib. I, Lemma XI, Schol.),
что он понимает под ними не
{\em неделимые} "--- форма, которою пользовались более
ранние математики, Кавальери и другие, и которая содержит в себе понятие
{\em само по себе} определенного количества, "--- а
{\em исчезающие делимые}. Он объясняет далее, что он
понимает под ними не суммы и отношения определенных частей, а
{\em пределы} (limites) {\em сумм}
и {\em отношений}. Против этого выдвигают, дескать, то
возражение, что у исчезающих величин не может быть никакого
{\em последнего отношения}, так как прежде, чем они
исчезли, оно не последнее, а когда они исчезли, нет никакого отношения. Но
под отношением исчезающих величин, указывает Ньютон, следует понимать не то
отношение, которое имеет место {\em до} или
{\em после} их исчезновения, а то отношение,
{\em вместе с которым} они исчезают (quacum
evanescunt). Точно так же {\em первое} отношение
возникающих величин есть то отношение, {\em вместе с
которым} они возникают.

В соответствии с состоянием научного метода того времени давалось лишь
объяснение, что под таким-то выражением следует понимать то-то. Но
заявление, что под таким-то выражением следует понимать то-то, есть,
собственно говоря, лишь субъективное предложение или же историческое
требование, причем не показывают, что такое понятие само по себе необходимо
и обладает внутренней истинностью. Но вышеизложенное показывает, что
выставленное Ньютоном понятие соответствует тому, как в предшествующем
изложении получилась бесконечная величина из рефлексии определенного
количества внутрь себя. Под флюксиями Ньютон понимает величины в их
исчезновении,~т.~е. величины, которые уже больше не суть определенные
количества; он, далее, понимает под ними не отношения определенных частей,
а {\em пределы отношения}. Стало быть, исчезают
согласно этому пониманию как определенные количества сами по себе, члены
отношения, так и самое отношение, поскольку оно было определенным
количеством, предел отношения величин есть то, в чем оно есть и не есть;
это означает, точнее, что оно есть то, в чем определенное количество
исчезло, и тем самым сохранились лишь отношение как качественно
количественное отношение, и его члены —~тоже как качественно количественные
моменты. — Ньютон к этому прибавляет, что из того обстоятельства, что
существуют последние отношения исчезающих величин, не следует заключать,
что существуют последние, величины «{\em неделимые}».
Это было бы опять-таки скачком от абстрактного отношения к таким его
членам, которые должны были бы сами по себе, вне своего соотношения, иметь
известное значение, как неделимые, как нечто, что было бы одним,
безотносительным.

Чтобы предостеречь против этого недоразумения, он, кроме того, напоминает,
что {\em последние отношения} суть не отношения
{\em последних величин}, а только пределы, к которым
{\em отношения} безгранично убывающих величин
приближаются больше, чем всякая {\em данная},~т.~е.
конечная разность, но которых они не преступают, чтобы стать ничем. — Под
{\em последними величинами} можно было бы именно
понимать, как мы уже сказали, неделимые или одни. Но из определения
последнего отношения устранено представление как о безразличном
безотносительном одном, так и о конечном определенном количестве. — Но не
нужно было бы ни {\em безграничного убывания}, которое
Ньютон приписывает определенному количеству и которое лишь служит
выражением бесконечного прогресса, ни определения делимости, которое здесь
уже больше не имеет никакого непосредственного значения, если бы требуемое
определение было развито далее в понятие некоторого такого определения
величины, которое есть исключительно лишь момент отношения.

Касательно {\em сохранения отношения} в
{\em исчезающих определенных количествах} мы встречаем
у других авторов (например, у {\em Карно},
Réflexions sur la métaphysique du Calcul infinitésimal)
выражение, что {\em в силу закона непрерывности}
исчезающие величины прежде, чем исчезнуть, продолжают
сохранять то отношение, из которого они происходят. — Это представление
{\em выражает} собою истинную природу дела, поскольку
здесь разумеется не та непрерывность определенного количества, которую оно
являет нам в бесконечном прогрессе, непрерывность, заключающаяся в том, что
определенное количество так продолжается в своем исчезновении, что
{\em по ту сторону} его снова возникает лишь некоторое
конечное определенное количество, некоторый {\em новый
член ряда}. Однако {\em непрерывное} движение вперед
всегда представляют себе так, что проходятся значения, которые еще суть
конечные определенные количества. Напротив, в том переходе, который
совершается в истинное бесконечное, {\em непрерывным}
оказывается отношение; оно настолько {\em непрерывно} и
сохраняется, что переход исключительно только и состоит в том, что он
выделяет отношение в чистом виде и заставляет исчезнуть безотносительное
определение,~т.~е. то обстоятельство, что определенное количество,
являющееся членом отношения, еще есть определенное количество также и
тогда, когда оно положено вне этого соотношения. — Это очищение
количественного отношения есть постольку не что иное, как то, что имеет
место, когда некоторое эмпирическое {\em существование}
(Dasein) постигается через {\em понятие} (begriffen
wird). Эмпирическое существование благодаря этому поднимается выше самого
себя таким образом, что его понятие содержит те же определения, которые
содержит оно само, но охваченные в их существенности и вдвинутые в
{\em единство} понятия, в котором они потеряли свое
безразличное, чуждое понятию существование (Bestehen).

Столь же интересна и другая форма ньютоновой трактовки интересующих нас
величин, а именно, рассмотрение их как
{\em производящих величин} или
{\em начал}. {\em Производная}
величина (genita) —~это произведение или частное, корни, прямоугольники,
квадраты, а также стороны прямоугольников, квадратов, — вообще,
{\em конечная величина}. — «Рассматривая ее как
переменную, как возрастающую или убывающую в постоянном движении и течении,
я понимаю под названием {\em моментов} ее
{\em моментальные приращения} или
{\em убывания}. Но не следует принимать эти моменты за
частицы, имеющие определенную величину (particulae finitae). Такие частицы
суть не самые {\em моменты}, а величины,
{\em произведенные} из моментов; под последними же
следует понимать находящиеся в становлении
{\em принципы} или {\em начала}
конечных величин». — Ньютон отличает здесь определенное количество от него
же самого, рассматривает его двояко: так, как оно есть продукт или налично
сущее, и так, как оно есть в своем {\em становлении}, в
своем {\em начале} и
{\em принципе}, то есть как оно есть в своем
{\em понятии} или —~здесь это равнозначно —~в своем
качественном определении; в последнем количественные различия, бесконечные
приращения или убывания суть лишь моменты; только уже ставшее есть нечто
перешедшее в безразличие наличного бытия и во внешность, — определенное
количество. — Но если философия истинного понятия и должна признать эти
приведенные касательно приращений или убываний определения бесконечного, то
мы должны вместе с тем сразу же заметить, что самые формы приращения
и~т.~д. имеют место {\em внутри} категории
непосредственного определенного количества и вышеуказанного непрерывного
движения вперед, и что представления о
{\em приращении}, {\em приросте},
увеличении {\em x} на {\em dx} или
{\em i} и~т.~д. должны рассматриваться скорее как
имеющиеся в этих методах основные недостатки, как постоянное препятствие к
выделению в чистом виде определения качественного момента количества из
представления об обычном определенном количестве.

По сравнению с указанными определениями является очень отсталым
{\em предоставление о бесконечно-малых величинах},
содержащееся также и в самих представлениях о приращении или убывании.
Согласно представлению о бесконечно-малых величинах они носят такой
характер, что следует {\em пренебрегать} не только ими
самими по отношению к конечным величинам, но также их высшими порядками по
отношению к низшим, а равно произведениями нескольких таких величин по
отношению к одной. — У {\em Лейбница} особенно ярко
выступает это требование о таком {\em пренебрежении},
применению какового давали место также и предыдущие изобретатели методов,
касающихся этих величин. Именно это обстоятельство сообщает указанному
исчислению при всем выигрыше в удобстве видимость неточности и явной
неправильности хода его действий. — {\em Вольф}
стремился сделать это пренебрежение величинами понятными по обычному своему
способу делать популярными излагаемые им вопросы,~т.~е. путем нарушения
чистоты понятия и подстановки на его место неправильных чувственных
представлений. А именно, он сравнивает пренебрежение бесконечно малыми
разностями высших порядков относительно низших с образом действия геометра,
измерение которым высоты горы нисколько не делается менее точным, если
ветер снесет песчинку с ее вершины, или с пренебрежением высотой домов и
башен при вычислении лунных затмений (Element. Mathes. univ.,
Tom~I, El. Analys. math., P.~II, С~I, см.~Schol.).

Если снисходительная справедливость (die Billigkeit) здравого человеческого
рассудка и допускает такую неточность, то все геометры, напротив, отвергали
такого рода представление. Сама собою напрашивается мысль, что в
математической науке не идет речь о такой эмпирической точности и что
математическое измерение путем ли вычислений или путем геометрических
построений и доказательств совершенно отлично от землемерия, от измерения
данных в опыте линий, фигур и~т.~п. Да и помимо того, как уже было указано
выше, аналитики, сравнивая между собою результаты, получаемые строго
геометрическим путем, с результатами, получаемыми посредством метода
бесконечно малых разностей, доказывают, что они тождественны и что большая
или меньшая точность здесь вовсе не имеет места. А ведь само собою понятно,
что абсолютно точный результат не мог бы получиться из неточного хода
действия. Однако, с другой стороны, несмотря на протесты против этого
способа оправдания, никак нельзя обойтись без
{\em самого этого приема} "--- без пренебрежения величиной
на основании ее незначительности. И в этом состоит трудность, заставляющая
аналитиков стараться сделать понятным и устранить заключающуюся здесь
бессмыслицу.

По этому вопросу следует главным образом привести мнение
{\em Эйлера}. Полагая в основание общее определение
Ньютона, он настаивает на том, что диференциальное исчисление рассматривает
{\em отношения приращений} некоторой величины, причем,
однако, {\em бесконечно малая разность} как таковая
должна быть рассматриваема совершенно {\em как нуль}
(Institut Calc. different., р.~I, с.~III).
"--- Как это следует понимать, видно из
вышеизложенного; бесконечно малая разность есть нуль лишь по количеству, а
не качественный нуль; а как нуль по количеству, она есть лишь чистый момент
отношения. Она не есть различие {\em на некоторую
величину}. Но именно потому, с одной стороны, вообще ошибочно называть
моменты, именуемые бесконечно малыми величинами, также и приращениями или
убываниями и {\em разностями}. В основании этого
определения лежит предположение, что к первоначально имеющейся конечной
величине нечто {\em прибавляется} или нечто от нее
{\em отнимается}, что совершается некоторое вычитание
или сложение, некоторое {\em арифметическое},
{\em внешнее} действие. Но что касается перехода от
функции переменной величины к ее диференциалу, то по нему видно, что он
носит совершенно другой характер, а именно, как мы уже разъяснили, он
должен рассматриваться как сведение конечной функции к качественному
отношению ее количественных определений. — С другой стороны, сразу
бросается в глаза, что когда говорят, что приращения суть сами по себе нули
и что рассматриваются лишь их отношения, то это само по себе ошибочно, ибо
нуль уже не имеет вообще никакой определенности. Это представление, стало
быть, хотя и доходит до отрицания количества и определенно высказывает это
отрицание, не схватывает вместе с тем последнего в его положительном
значении качественных определений количества, которые, если пожелаем
вырвать их из отношения и брать их как определенные количества, окажутся
лишь нулями. — {\em Лагранж} (Théorie des fonct. analyt. Introd.)
замечает о представлении {\em пределов} или {\em последних
отношений}, что, хотя и можно очень хорошо представить себе отношение двух
величин, покуда они остаются конечными, это отношение не дает рассудку
ясного и определенного понятия, как только его члены становятся
одновременно нулями. — И в самом деле, рассудок должен пойти далее той
чисто отрицательной стороны, что члены отношения суть как определенные
количества нули, и понять их положительно как качественные моменты. — А то,
что {\em Эйлер} (в указанном месте § 84 и сл.)
прибавляет далее касательно данного им определения, чтобы показать, что две
так называемые бесконечно малые величины, которые якобы суть не что иное,
как нули, тем не менее находятся в отношении друг к другу, и потому для их
обозначения употребляется не знак нуля, а другие знаки, — не может быть
признано удовлетворительным. Он хочет это обосновать различием между
арифметическим и геометрическим отношениями; в первом мы обращаем внимание
на разность, во втором —~на частное, и, хотя арифметическое отношение между
любыми двумя нулями всегда одинаково, это не значит, что можно сказать то
же самое о геометрическом отношении; если $2 : 1 = 0 : 0$,
то по свойству пропорции, так как первый член вдвое больше второго, третий
член тоже должен быть вдвое больше четвертого; поэтому на основании этой
пропорции отношение $0 : 0$ должно быть взято, как отношение
$2 : 1$. "--- Также и по обычной арифметике $n \times 0 = 0$; следовательно,
$n : 1 = 0 : 0$. "--- Однако именно потому, что $2 : 1$ или $n : 1$
есть отношение определенных количеств, ему не соответствует ни отношение,
ни обозначение $0 : 0$.

Я воздерживаюсь от дальнейшего увеличения числа приведенных взглядов, так
как рассмотренные уже достаточно показали, что в них, правда, скрыто
содержится истинное понятие бесконечного, но что оно, однако, не выделено и
не сформулировано во всей его определенности. Поэтому, когда высказывающие
эти взгляды переходят к самому действию, то на нем не может сказаться
истинное определение понятия, а, напротив, возвращается снова конечная
определенность количества, и действие не может обойтись без представления о
лишь {\em относительно малом}. Исчисление делает
необходимым подвергать так называемые бесконечные величины обычным
арифметическим действиям сложения и~т.~д., основанным на природе конечных
величин, и тем самым хотя бы на мгновение признавать эти бесконечные
величины конечными и трактовать их как таковые. Исчисление должно было бы
обосновать правомерность того, что оно, с одной стороны, тянет эти величины
вниз, вовлекает их в эту сферу и трактует их как приращения или разности, а
с другой стороны, пренебрегает ими как определенными количествами после
того, как оно только что применяло к ним формы и законы конечных величин.

Я приведу еще самое существенное о попытках геометров устранить эти
затруднения.

Более старые аналитики меньше затрудняли себя такими сомнениями; но старания
более новых аналитиков были направлены преимущественно к тому, чтобы
возвратить исчисление бесконечно малых к очевидности
{\em собственно геометрического метода} и с помощью
этого метода достигнуть в математике {\em строгости
доказательств древних} (выражения {\em Лагранжа}).
Однако, так как принцип анализа бесконечного по своей природе выше, чем
принцип математики конечных величин, то анализ бесконечного сам собою сразу
же должен был отказаться от того рода
{\em очевидности}, подобно тому, как философия также не
может притязать на ту отчетливость, которой обладают науки о чувственном,
например, естественная история, или подобно тому, как еда и питье считаются
более понятными вещами, чем мышление и постижение посредством понятия
(Begreifen). Поэтому нам придется говорить лишь о стараниях достигнуть
строгости доказательств древних.

Некоторые математики пытались обойтись совершенно без понятия бесконечного и
дать без него то, что казалось связанным с его употреблением. —
{\em Лагранж}, например, рассказывает о методе,
изобретенном {\em Ланденом}, и говорит о нем, что он
является чисто аналитическим и не употребляет бесконечно малых разностей, а
сначала вводит {\em различные значения} переменных
величин и в дальнейшем {\em приравнивает} их между
собою. Лагранж, впрочем, заявляет, что в этом методе утрачиваются
свойственные диференциальному исчислению преимущества, а именно простота
метода и легкость действия. — Это —~прием, в котором есть нечто
соответственно тому, из которого исходит {\em Декартов}
метод касательных, о котором нам придется ниже еще говорить подробнее.
Здесь можем заметить, что в общем виде сразу ясно, что этот прием,
заключающийся в том, чтобы придавать переменным величинам различные
значения и затем приравнивать их между собою, принадлежит вообще к другому
кругу математической трактовки, чем сам метод диференциального исчисления,
и им не выделяется подлежащее далее более пристальному рассмотрению
своеобразие того простого отношения, к которому сводится действительное,
конкретное определение этого исчисления, а именно —~отношения производной
функции к первоначальной.

Более ранние из новых математиков, как например,
{\em Ферма}, {\em Барроу} и др.,
которые впервые пользуются бесконечно малыми в том применении, которое
позднее привело к разработке диференциального и интегрального исчисления, а
затем также {\em Лейбниц} и последующие математики,
равно как и {\em Эйлер}, всегда откровенно
высказывались, что считают дозволительным отбрасывать произведения
бесконечно малых разностей так же, как и их высшие степени только на том
основании, что они {\em относительно}, по сравнению с
низшими порядками, {\em исчезают}. Исключительно на
этом соображении покоится у них {\em основная теорема},
а именно, определение того, что такое диференциал произведения или степени,
{\em ибо }{\em к этому сводится все
теоретическое учение}. Остальное есть отчасти механизм действий, отчасти же
приложение, которое, однако, как мы покажем далее, на самом деле
представляет более высокий или, лучше сказать, единственный интерес. —
Относительно же того вопроса, который мы рассматриваем теперь, следует
здесь привести лишь то элементарное соображение, что на основании того же
рассуждения о {\em незначительности} принимается как
основная теорема о кривых, что элементы кривых, а именно
{\em приращения} абсциссы и ординаты имеют между собою
то же {\em отношение}, как
{\em подкасательная} и
{\em ордината}. С целью получить подобные треугольники
дуга, составляющая наряду с двумя приращениями третью сторону того
треугольника, который справедливо назывался когда-то
{\em характеристическим} треугольником, рассматривается
как прямая линия, как часть касательной, и потому одно из приращений —~как
доходящее до касательной. Эти допущения поднимают, с одной стороны,
вышеуказанные определения выше природы конечных величин; но, с другой
стороны, здесь применяется к моментам, называемым теперь бесконечными,
такой прием, который значим лишь относительно конечных величин и при
котором мы не имеем права чем-либо пренебрегать на основании его
незначительности. Затруднение, тяготеющее над методом, остается при таком
образе действия во всей своей силе.

Здесь мы должны указать на замечательный прием {\em Ньютона}
(Princ. Mathem. phil. nat., lib.~II, Lemma~II, после propos.~VII)
"--- на изобретенный им остроумный кунштюк для устранения арифметически
неправильного отбрасывания произведений бесконечно малых разностей или
высших порядков этих последних при нахождении диференциалов. Он находит
диференциал произведения, "--- из которого легко затем вывести диференциалы
частного, степени и~т.~п. "--- следующим образом. Произведение, если уменьшить
$x$ и $y$, каждый порознь
{\em на половину} его бесконечной разности, переходит в
 $\mathit{xy}-\frac{\mathit{xdy}} 2-\frac{\mathit{ydx}}
2+\frac{\mathit{dxdy}} 4$ , а если увеличить {\em x} и
{\em y} ровно настолько же, то произведение переходит в
 $\mathit{xy}+\frac{\mathit{xdy}} 2+\frac{\mathit{ydx}}
2+\frac{\mathit{dxdy}} 4$ . Если от этого второго произведения отнять
первое, то получается разность  $\mathit{ydx}+\mathit{xdy}$ , которая есть
{\em избыток приращения на целые}
{\em dx} и {\em dy}, так как на это
приращение отличаются оба произведения; следовательно, это и есть
диференциал \  $\mathit{xy}$ . — Как видим, при этом приеме сам собою
отпадает член, представлявший главное затруднение, произведение двух
бесконечных разностей  $\mathit{dxdy}$ . Но, несмотря на имя
{\em Ньютона}, следует сказать, что это, хотя и весьма
элементарное, действие неправильно; неправильно, что 
$\left(x+\frac{\mathit{dx}} 2\right)\left(y+\frac{\mathit{dy}}
2\right)-\left(x-\frac{\mathit{dx}} 2\right)\left(y-\frac{\mathit{dy}}
2\right)=\left(x+\mathit{dx}\right)\left(y+\mathit{dy}\right)-\mathit{xy}$
.

Только потребность обосновать ввиду его важности исчисление флюксий могла
заставить такого математика, как Ньютон, обмануть себя подобным способом
доказательства.

Другие формы, которыми пользуется Ньютон при выводе диференциала, связаны с
конкретными, относящимися к движению значениями элементов и их степеней. —
При употреблении {\em формы ряда}, которое вообще
характерно для его метода, слишком напрашивается сказать, что мы всегда
имеем возможность путем прибавления дальнейших членов взять величину
{\em с той степенью точности},
{\em которая нам нужна}, и что отброшенные величины
{\em относительно незначительны}, что вообще результат
есть лишь {\em приближение}; и он здесь также
удовлетворился этим основанием, подобно тому, как он в своем методе решения
уравнений высших степеней путем приближения отбрасывает высшие степени,
получающиеся при подстановке в данное уравнение каждого найденного еще
неточного значения, на том же грубом основании, что они малы; см.
{\em Lagrange}, Equations Numériques, р. 125.

{\em Ошибка}, в которую впал
{\em Ньютон}, разрешая задачу путем отбрасывания
существенных высших степеней, ошибка, которая дала повод противникам
торжествовать победу своего метода над его методом и истинный источник
которой обнаружил {\em Лагранж} в своем новейшем ее рассмотрении
(Théorie des fonct. analyt., 3-me~р., ch.~IV),
доказывает, что употребление этого орудия еще страдало
{\em формализмом} и {\em неуверенностью}. {\em Лагранж}
показывает, что {\em Ньютон} впал в свою ошибку
вследствие того, что он пренебрегал членом ряда, содержащим ту степень,
которая была важна для данной задачи. {\em Ньютон}
придерживался формального, поверхностного принципа отбрасывания членов
ввиду их относительной малости. — А именно, известно, что в
{\em механике} членам ряда, в который разлагается
функция какого-нибудь движения, придается
{\em определенное значение}, так что первый член или
первая функция относится к моменту скорости, вторая —~к силе ускорения, а
третья —~к сопротивлению сил. Поэтому члены ряда должны рассматриваться
здесь не только как {\em части} некоторой суммы, но как
{\em качественные моменты некоторого целостного
понятия}. Благодаря этому {\em отбрасывание} остальных
членов, принадлежащих дурно бесконечному ряду, имеет
{\em смысл}, совершенно
{\em отличный} от отбрасывания их на основании их
относительной малости\footnote{
Обе точки зрения весьма просто
сопоставлены у {\em \bfseries Лагранжа} при
применении теории функций в механике, в главе о прямолинейном движении
(Théorie des fonct. 3-me~р., ch.~I, art.~IV). Если
рассматривать пройденное пространство как функцию протекшего времени, то
получается уравнение $x=\mathit{ft}$, которое, разложенное
как $f(t+\vartheta )$, даёт
$\mathit{ft} + \vartheta f't + \frac{\vartheta ^{'2}} 2 f''t + $
и~т.~д. Следовательно, пространство, пройденное в данное время,
изображается формулой
$ = \vartheta f't+\frac{\vartheta ^2}
2 f''t + \frac{\vartheta ^3}{2 \cdot 3} f'''t + $
и~т.~д. Движение, посредством которого проходится это
пространство, говорят нам, {\em составлено}, {\em следовательно} (т.~е.
вследствие того, что аналитическое разложение в ряд дает много и притом
бесконечно много членов) "--- из различных частичных движений,
соответствующие времени пространства которых суть
$ \vartheta f't+\frac{\vartheta ^2}
2 f''t + \frac{\vartheta ^3}{2 \cdot 3} f'''t + $
и~т.~д. Первое частичное
движение есть в известном нам движении формально-равномерное движение со
скоростью $f't$, второе равномерно ускоренное, зависящее
от силы ускорения, пропорциональной $f^{\text{[2033?]}}t$ .
«А так как прочие члены не {\em относятся ни к какому простому
известному движению, то нет надобности принимать} их в
отдельности {\em во внимание}, и мы покажем, что
{\em от них можно абстрагироваться} при определении движения в
начале момента времени». Это и показывается, но, конечно, только путём
{\em сравнения} вышеуказанного ряда, члены которого {\em все} должны были
служить для определения {\em величины}
пространства, пройденного в данное время, с данным в § 3 для
падения тел уравнением $x=\mathit{at}+\mathit{bt}^2$ , в
котором имеются только эти два члена. Но это уравнение само получило этот
вид лишь благодаря предположению {\em объяснения}, {\em даваемого} членам,
возникающим {\em посредством аналитического разложения в ряд},
это предположение заключается в том, что равномерно ускоренное движение
{\em составлено} из формально равномерного движения, совершающегося
с достигнутой в предыдущую часть времени скоростью, и некоторого прибавка
($a$ в уравнении $s=at^2$), т.~е. эмпирического коэфициента,
приписываемого силе тяжести, а ведь это есть такое различение, которое
отнюдь не имеет существования или основания в природе вещей, но есть лишь
ошибочно получившее характер физического положения выражение того, что
получается при принятии некоторой определенной аналитической трактовки.}.
Разрешение проблемы, данное Ньютоном, оказалось ошибочным не потому, что в
нем не принимаются во внимание члены ряда лишь как
{\em части некоторой суммы}, а потому, что не
принимается во внимание {\em член, содержащий то
качественное определение}, в котором было все дело.

В этом примере качественный {\em смысл} есть то, от чего
ставится в зависимость прием. В связи с этим мы можем тотчас же выставить
общее утверждение, что все затруднение касательно самого принципа было бы
устранено, если бы вместо формализма, состоящего в том, что определение
{\em диференциала} усматривают лишь в дающей ему это
{\em имя} задаче, т.~е. в
{\em различии} вообще некоторой функции от ее
{\em изменения} после того, как ее переменная величина
получила некоторое {\em приращение}, — если бы вместо
этого формализма было указано качественное значение принципа и действие
было бы поставлено в зависимость от этого качественного значений. В этом
смысле диференциал от  $x^n$  оказывается вполне исчерпанным первым членом
ряда, получающегося путем разложения выражения  $(x+\mathit{dx})^n$ . Что
прочие члены не принимаются во внимание, проистекает, таким образом, не из
их относительной малости; здесь не предполагается никакой такой неточности,
погрешности или ошибки, которая бы {\em выравнивалась}
и {\em исправлялась} другой ошибкой, — взгляд, исходя
преимущественно из которого, {\em Карно} оправдывает
обычный метод исчисления бесконечно-малых. Так как дело идет
{\em не} о некоторой {\em сумме}, а
о некотором {\em отношении}, то диференциал оказывается
вполне найденным {\em посредством первого члена}; там
же, где есть нужда в дальнейших членах, в диференциалах высших порядков, их
нахождение состоит не в продолжении ряда, как
{\em суммы}, а в повторении одного и того же
{\em отношения}, которое единственно имеют в виду и
которое, стало быть, {\em завершено} уже в
{\em первом члене}. Потребность в
{\em форме} некоторого {\em ряда},
в суммировании этого ряда и все, что связано с этим, должны в таком случае
быть совершенно отделены от указанного {\em интереса к
отношению}.

Разъяснения, даваемые {\em Карно} относительно метода
бесконечных величин, представляют собою наиболее очищенное и ясное
изложение того, что нам встретилось в вышеуказанных представлениях. Но при
переходе к самим действиям у него более или менее появляются обычные
представления о бесконечной {\em малости} отбрасываемых
членов {\em по сравнению} с другими. Он оправдывает
метод скорее тем, что {\em результаты} оказываются
правильными, и {\em полезностью} введения
{\em неполных} уравнений, как он их называет (т.~е.
таких уравнений, в которых совершается такое арифметически неправильное
отбрасывание), для упрощения и сокращения исчисления, — чем самой природой
вещи.

{\em Лагранж}, как известно, снова возвратился к
первоначальному методу Ньютона, к методу рядов, дабы быть свободным от
трудностей, которые влечет за собою представление о бесконечно-малом, равно
как и метод первых и последних отношений и пределов. Относительно его
исчисления функций, прочие преимущества которого в отношении точности,
абстрактности и всеобщности достаточно известны, мы должны отметить как
касающееся занимающего нас вопроса лишь, то, что оно покоится на той
основной теореме, что разность, не превращаясь в нуль,
{\em может быть принята столь малой, чтобы каждый член
ряда превосходил по своей величине сумму всех следующих за ним членов}. —
При этом методе также начинают с категорий
{\em приращения} и {\em разности}
(по сравнению с первоначальной функцией) той функции, переменная величина
которой получает {\em приращение}, что и вызывает
появление скучного ряда; равно как в дальнейшем члены ряда, которые должны
быть отброшены, принимаются в соображение лишь с той стороны, что они
составляют некоторую {\em сумму}, и основанием, почему
они отбрасываются, полагается относительность их
{\em определенного количества}. Отбрасывание,
следовательно, и здесь не сводится в общем виде к той точке зрения, которая
отчасти встречается в некоторых приложениях, в которых, как мы упомянули
раньше, члены ряда должны иметь определенное
{\em качественное значение} и оставляются без внимания
не потому, что они незначительны по величине, а потому, что они
незначительны по качеству; отчасти же само отбрасывание отпадает в той
существенной точке зрения, которая определенно выступает относительно так
называемых диференциальных коэфициентов лишь в так называемом
{\em приложении} диференциального исчисления у
{\em Лагранжа}, что мы разъясним подробнее в следующем
примечании.

{\em Качественный характер вообще}, свойственный (как мы
здесь доказали, трактуя о той форме величины, о которой идет речь) тому,
что при этом называется бесконечно малым, обнаруживается непосредственнее
всего в той категории {\em предела отношения}, которая
приведена выше и проведение которой в диференциальном исчислении
рассматривалось как некоторый особого рода метод. Из соображений в суждении
Лагранжа об этом методе, что ему недостает легкости применения и что
выражение «{\em предел}» не дает определенной идеи, мы
остановимся на втором и рассмотрим ближе аналитическое значение этого
метода. В представлении о пределе именно и содержится вышеуказанная
истинная категория {\em качественного} определения
отношения между переменными величинами; ибо те их формы, которые появляются
в нем, {\em dx} и {\em dy}, должны
быть взяты здесь просто лишь как моменты выражения 
$\frac{\mathit{dy}}{\mathit{dx}}$  и само 
$\frac{\mathit{dx}}{\mathit{dy}}$  должно рассматриваться как единый
неделимый знак. Что при этом для механизма исчисления, особенно в его
приложении, утрачивается преимущество, которое он извлекает из того
обстоятельства, что члены диференциального коэфициента отделяются друг от
друга, — это следует здесь оставить в стороне. Этот предел должен быть
теперь {\em пределом} некоторой данной функции; он
должен указать известное значение в связи с нею, определяемое способом
вывода. Но с голой категорией предела мы не подвинулись бы дальше, чем с
тем, о чем дело шло в этом примечании, имеющем целью показать, что
бесконечно-малое, выступающее в диференциальном исчислении как
{\em dx} и {\em dy}, имеет не
только отрицательный, пустой смысл некоторой
{\em не}{}-конечной,
{\em не}{}-данной величины, как это имеет место,
например, в тех случаях, когда говорится: «бесконечное множество», «и~т.~д.
до бесконечности» и~т.~п., а определенный смысл качественной определенности
количественного, момента отношения как такового. Однако эта категория,
взятая в таком смысле, еще не имеет отношения к тому, что есть некоторая
данная функция, еще не влияет сама по себе на трактовку этой функции и не
приводит к такому употреблению указанного определения, которое должно было
бы иметь место в последней; таким образом, и представление предела, если
этому представлению не дозволяют итти дальше такой доказанной относительно
него определенности, также ни к чему не привело бы. Но выражение «предел»
уже само по себе подразумевает, что он есть предел
{\em чего-то},~т.~е. выражает известное значение,
определяемое функцией переменной величины; и мы должны посмотреть, каков
характер этого конкретного оперирования им.

Он должен быть пределом {\em отношения} друг к другу тех
двух {\em приращений}, на которые по сделанному
допущению {\em увеличиваются} две переменные величины,
соединенные в одном уравнении, из коих одна рассматривается как функция
другой; приращение берется здесь вообще неопределенным, и постольку о
бесконечно-малом нет еще и речи. Но прежде всего путь, которым отыскивается
этот предел, приводит к тем же непоследовательностям, которые имеются в
других методах. Этот путь именно таков. Если  $y=\mathit{fx}$ , то при
переходе  $y$ {\em  }в  $y+k$  \ \ \ \ \  $\mathit{fx}$
 должна переходить в  $\mathit{fx}+\mathit{ph}+qh^2+rh^3$  и~т.~д.
Следовательно,  $k=\mathit{ph}+qh^2$  и~т.~д., и  $\frac k
h=p+\mathit{qh}+rh^2$  и~т.~д. Если теперь {\em k} и
{\em h} исчезают, то исчезает и второй член ряда кроме
{\em p}, каковое {\em p} и
оказывается пределом отношения этих двух приращений. Отсюда видно, что
{\em h} как определенное количество полагается
{\em =~0}, но что вследствие
этого  $\frac k h$  еще не обращается вместе с тем в  $\frac 0 0$ , а
остается некоторым отношением. И вот представление
{\em предела} должно доставить ту выгоду, что оно
устранит заключающуюся в этом непоследовательность;
{\em p} должно вместе с тем быть не действительным
отношением, которое было бы  $=\frac 0 0$ , а лишь тем определенным
значением, к которому отношение может {\em приближаться
бесконечно},~т.~е. так, чтобы {\em разность могла стать
меньше всякой данной разности}. Более определенный смысл
{\em приближения} касательно того, что собственно
должно сближаться между собою, будет рассмотрен ниже. — Но что
количественное различие, определяемое не только как
{\em могущее}, но и как
{\em долженствующее быть} менее всякой данной величины,
уже больше не есть количественное различие, это само собою ясно; это так же
очевидно, как только что-нибудь может быть очевидным в математике; но этим
мы не пошли дальше  $\frac{\mathit{dy}}{\mathit{dx}}=\frac 0 0$ . Напротив,
если  $\frac{\mathit{dy}}{\mathit{dx}}=p$ ,~т.~е. принимается за некоторое
определенное количественное отношение, как это и есть на самом деле, то,
наоборот, получается затруднение для предположения, что  $h=0$ ,
предположения, единственно путем которого и получается  $\frac k h=p$ .
Если же согласиться, что $\frac k h=0$  —~и в самом деле, раз  $h=0$ , то
само собою {\em k} также делается  $=0$ , ибо
приращение {\em k} к {\em у} имеет
место лишь при условии существования приращения
{\em h}, — то надо было бы спросить, что представляет
собою {\em p}, которое есть некоторое совершенно
определенное количественное значение. На этот вопрос сразу же получается
простой, сухой ответ, гласящий, что оно есть коэфициент, и нам указывают,
путем какого вывода он возникает, — известным определенным образом
выведенная первая производная функция некоторой первоначальной функции.
Если удовольствоваться этим ответом, как и в самом деле
{\em Лагранж по существу дела} удовольствовался им, то
общая теория науки диференциального исчисления и непосредственно сама та
одна форма, которая называется {\em теорией пределов},
освободилась бы от приращений, а затем и от их бесконечной или какой угодно
малости, от трудности, состоящей в том, что кроме первого члена или,
вернее, лишь коэфициента первого члена, все остальные члены ряда, которые
неминуемо появляются благодаря введению этих приращений, снова устраняются;
да помимо этого она очистилась бы также и от всего связанного с этим
дальнейшего, от формальных категорий прежде всего бесконечного,
бесконечного приближения, а затем и от дальнейших здесь столь же пустых
категорий непрерывной величины\footnote{Категория
{\em непрерывной} или {\em текучей величины} появляется вместе с рассмотрением
{\em внешнего и эмпирического} изменения величин,
приведенных некоторым уравнением в такую
связь, что одна есть функция другой; но так как научным предметом
диференциального исчисления служит известное (обыкновенно выражаемое через
диференциальный коэфициент) {\em отношение}, каковая
определенность может быть названа также и {\em законом}, то для
этой специфической определенности простая непрерывность есть отчасти
чужеродный аспект, отчасти же во всяком случае абстрактная, а здесь— пустая
категория, так как ею ничего не выражается о законе непрерывности. —
В какие формальные дефиниции при этом кроме того впадают,
показывает остроумное общее изложение моим уважаемым коллегой проф.
{\em Дирксеном} основных
определений, употребляемых для вывода диференциального исчисления,
изложение, которое он дает в связи с критикой некоторых новых сочинений по
этой науке, помещенной в Jahrb. {\em f}.
wissensch. Kritik, 1827, Nr.~153 и сл. Там на стр.~1251
дается даже такая дефиниция: «Непрерывная величина, континуум, есть всякая
величина, которая мыслится нами находящейся в таком состоянии становления,
при котором последнее совершается не
{\em скачкообразно}, а путем {\em непрерываемого движения
вперед}». Но ведь это тавтология, повторение того, что есть
и самое {\em definitum}.} и всех еще других, которые
считается нужным ввести, как например,
{\em стремление},
{\em становление}, {\em повод к
изменению}. Но в таком случае требовалось бы показать, какое еще
{\em значение} и
{\em ценность}, т.~е. какую
{\em связь} и какое
{\em употребление} для дальнейших математических целей
имеет {\em p} помимо того, для теории совершенно
достаточного сухого определения, что оно есть не что иное, как полученная
путем разложения бинома производная функция; об этом будет сказано во
{\em втором примечании}. — Здесь же мы ближайшим
образом дадим разбор той путаницы, которую вышеприведенное столь обычное в
изложениях употребление представления о
{\em приближении} внесло в понимание собственной,
качественной определенности того отношения, в котором было ближайшим
образом все дело.

Мы показали, что так называемые бесконечно малые разности выражают собою
исчезание членов отношения как определенных количеств и что то, что после
этого остается, есть их количественное отношение, исключительно лишь
поскольку оно определено качественным образом; качественное отношение здесь
настолько не теряется, что оно скорее есть именно то, что получается
благодаря превращению конечных величин в бесконечные. В этом, как мы
видели, состоит вся суть дела. "--- Так например, {\em в
последнем отношении} исчезает определенные количества абсциссы и ординаты.
Но члены этого отношения остаются по существу один "--- элементом ординаты,
а другой "--- элементом абсциссы. Так как здесь применяют обычный способ
представления, состоящий в том, что одна ордината
{\em бесконечно приближается} к другой, то одна
ордината, раньше отличная от другой ординаты, переходит в последнюю, а
раньше различная абсцисса переходит в другую абсциссу; но ордината по
существу не переходит в абсциссу и абсцисса не переходит в ординату.
Оставаясь и далее в рамках этого примера переменных величин, следует
сказать, что элемент ординаты должен быть понимаем не как
{\em отличие одной ординаты от другой ординаты}, а как
отличие или {\em качественное} определение величины
относительно {\em элемента абсциссы};
{\em принцип одной переменной величины и принцип
другой} находятся во взаимном отношении между собой. Различие, не будучи
уже больше различием конечных величин, перестало быть многообразным внутри
самого себя, оно сжалось в простую интенсивность, в определенность одного
качественного момента отношения относительно другого.

Но эта суть дела затемняется тем обстоятельством, что то, что мы только что
назвали элементом, например, ординаты, понимается затем как
{\em разность} или
{\em приращение}, в том смысле, что оно будто бы есть
лишь различие между определенным количеством одной ординаты и определенным
количеством другой. {\em Предел} здесь, следовательно,
не имеет смысла отношения; он считается лишь тем последним значением, к
которому другая величина того же рода постоянно приближается таким образом,
что она может сколь угодно мало отличаться от него и что последнее
{\em отношение} есть отношение
{\em равенства}. Таким образом, бесконечно малая
разность оказывается как бы неустойчивостью различия (das Schweben eines
Unterschieds) одного определенного количества от другого и ее качественная
природа, по которой $dx$ есть по существу
определение отношения не к $x$, а к
$dy$, отступает в представлении на задний план. В
диференциальном исчислении заставляют
$dx^2$ исчезнуть относительно
$dx$, но еще больше исчезает $dx$ относительно $x$, а
это поистине означает: {\em $dx$ находится в отношении
лишь к $dy$}. ---~В таких изложениях геометры стараются преимущественно
о том, чтобы сделать {\em понятным приближение} некоторой
величины к ее пределу, и держаться того аспекта различия одного
определенного количества от другого, в котором оно не есть различие и,
однако, все еще есть различие. Но помимо всего прочего приближение есть
само по себе ничего не говорящая и ничего не делающая понятным категория;
уже $dx$ оставил приближение позади себя, он ни
близок ни более близок, и бесконечная близость сама есть лишь отрицание
близости и приближения.

Стало быть, поскольку вышло так, что приращения или бесконечно-малые
разности рассматриваются лишь со стороны определенного количества, которое
в них исчезает, и лишь как его предел, их понимают при этом как
{\em безотносительные} моменты. Из этого вытекало бы не
выдерживающее критики представление, будто в последнем отношении
дозволительно приравнивать между собою, например, абсциссу с ординатой, или
же синус, косинус, тангенс, sinus versus и что угодно еще. ---~Может
казаться, что такое представление получает силу в том случае, когда дуга
рассматривается как касательная; ибо и дуга, конечно, тоже
{\em несоизмерима с прямой линией} и ее элемент имеет
прежде всего другое {\em качество}, чем элемент прямой
линии. Может показаться еще более бессмысленным и недозволительным, чем
смешение абсциссы, ординаты, sinus versus, косинуса и~т.~д. принимать
круглые квадраты, принимать часть дуги, хотя бы и бесконечно малую, за
кусочек касательной и, следовательно, трактовать ее как прямую линию.
---~Однако такую трактовку следует по существу отличать от вызвавшего
порицание смешения; она имеет свое оправдание в том, что в том треугольнике,
который имеет своими сторонами элемент некоторой дуги и элемент ее абсциссы
и ординаты, {\em отношение остается тем же самым}, как
если бы элемент дуги был элементом прямой линии, касательной;
{\em углы}, составляющие
{\em существенное отношение}, т.~е. то отношение,
которое сохраняется в этих элементах, когда мы абстрагируемся от присущих
им конечных величин, суть те же самые. — Можно выразиться об этом и таким
образом, что прямые линии как бесконечно малые стали кривыми линиями, и
отношение между ними при их бесконечности стало отношением между кривыми.
Так как согласно дефиниции прямой линии она есть
{\em кратчайшее} расстояние между двумя точками, то ее
отличие от кривой линии основано на определении
{\em множества}, на {\em меньшем}
множестве различимого в этом расстоянии, что, стало быть, есть
{\em количественное} определение. Но это определение в
ней исчезает, когда мы принимаем ее за интенсивную величину, за бесконечный
момент, за элемент; а вместе с тем исчезает и ее отличие от кривой линии,
основанное исключительно только на различии определенного количества.
---~Следовательно, как бесконечные,
прямая линия и дуга не сохраняют никакого
количественного отношения друг к другу и тем самым на основании принятой
дефиниции не имеют больше также и никакого качественного отличия друг от
друга, а первая переходит во вторую.

Родственным и, тем не менее, отличным от приравнивания разнородных
определений оказывается само по себе неопределенное и совершенно
безразличное допущение, что {\em бесконечно малые
части} одного и того же целого {\em равны} между собою.
Однако примененное к разнородному внутри себя предмету, т.~е. к такому
предмету, который обременен существенною неравномерностью количественных
определений, это допущение порождает содержащееся в теореме высшей механики
своеобразно превратное утверждение, гласящее, что в
{\em равные} и притом бесконечно малые промежутки
времени проходятся бесконечно малые части кривой в
{\em равномерном} движении, причем утверждение это
касается такого движения, в котором в равные
{\em конечные}, т.~е. существующие части времени,
проходятся {\em конечные}, т.~е. существующие
{\em неравные} части кривой, т.~е., стало быть,
касается движения, которое как существующее неравномерно и признается
таковым. Эта теорема есть словесное выражение того, что должен означать
собою аналитический член, получающийся в приведенном выше разложении
формулы неравномерного, но, впрочем, соответствующего некоторому закону
движения. Более ранние математики старались выразить результаты вновь
изобретенного исчисления бесконечно-малых, которое и без того всегда имело
дело с конкретными предметами, в словах и предложениях и представить их в
геометрических обозначениях, главным образом для того, чтобы применять их
для вывода теорем по обычному способу доказательства. Члены математической
формулы, на которые анализ разлагал {\em величину}
предмета, например, движения, получали, таким образом,
{\em предметное} значение, например, значение скорости,
ускоряющей силы и~т.~п. Они должны были согласно такому значению доставлять
правильные положения, физические законы, и сообразно их аналитической
связи, должны были определяться также и их объективные связи и отношения,
как например, должно было именно определяться, что в равномерно ускоренном
движении существует особая пропорциональная временам скорость, к которой
кроме того всегда присоединяется приращение, сообщаемое силой тяжести.
Такие предложения выставляются в новой, получившей аналитическую форму
механике исключительно как результаты исчисления, причем она не заботится о
том, имеют ли они сами по себе самостоятельный
{\em реальный} смысл,~т.~е. такой смысл, которому
соответствует некоторое существование, не заботится также и о том, чтобы
это доказать. Трудность сделать понятной связь таких определений, когда их
берут в определенно реальном смысле, например, объяснить переход от просто
равномерной скорости к равномерному ускорению, считается совершенно
устраненной аналитическим рассмотрением, в котором сказанная связь есть
простое следствие отныне прочного авторитета действий исчисления.
Нахождение единственно только путем вычисления законов,
{\em выходящих за пределы опыта},~т.~е. таких
предложений о существовании, которые сами не имеют существования, выдается
за торжество науки. Но в первое, еще наивное время исчисления
бесконечно-малых математики всячески старались указать и обосновать
самостоятельный реальный смысл этих представленных в геометрических
построениях определений и положений и применять их в таком смысле для
доказательства главных положений, о которых шла речь (ср.
{\em Ньютоново} доказательство основного положения его
теории тяготения в Princ. mathemat. philosophiae naturalis, Hb.~I,
sect.~II, prop.~I, с астрономией {\em Шуберта} (изд. 1-е, т. III, § 20),
где он вынужден признать, что дело обстоит не
{\em совсем так},~т.~е. что в пункте, составляющем
самый нерв доказательства, дело обстоит не так, как это принимает Ньютон).

Нельзя отрицать, что в этой области многое, преимущественно при помощи
тумана, напущенного бесконечно малыми, было допущено в качестве
доказательства ни на каком другом основании, как только потому, что то, что
получалось, всегда было заранее известно, и доказательство, построенное
таким образом, что получался уже известный вывод, давало по крайней мере
{\em видимость некоторого остова доказательства},
видимость, которую все же предпочитали простой вере или опытному знанию. Но
я не колеблясь решаюсь сказать, что рассматриваю эту манеру только как
простое фокусничество и шарлатанничание доказательствами, и причисляю к
такого рода фокусничанию даже ньютоновы доказательства и, в особенности, те
из них, которые принадлежат к только что приведенным, за которые
превозносили Ньютона до небес и ставили выше
{\em Кеплера}, утверждая, что первый доказал
математически то, что второй нашел {\em лишь опытным
путем}.

Пустой остов таких доказательств был воздвигнут с целью доказать физические
законы. Но математика вообще не может доказать количественных определений
физики, поскольку они суть законы, имеющие своим основанием
{\em качественную природу} моментов; математика не
может этого сделать по той простой причине, что она не есть философия,
{\em не} исходит {\em из понятия},
и поэтому качественное, поскольку оно не почерпается лемматически из опыта,
лежит вне ее сферы. Отстаивание {\em чести} математики,
настаивание на том, что все встречающиеся в ней положения должны быть
{\em строго доказаны}, заставляло ее часто забывать
свои границы. Так, например, казалось противным ее достоинству просто
признать {\em опыт} источником и единственным
доказательством встречающихся в ней {\em опытных
}{\em положений}. Позднее было достигнуто более
определенное сознание этой истины; но до тех пор, пока сознание не уяснит
себе различие между тем, что может быть доказано, и тем, что может быть
лишь заимствовано из другого источника, равно как и различие между тем, что
представляет собою лишь член аналитического разложения, и тем, что
представляет собою физическое существование, до тех пор научность не сможет
достигнуть строгой и чистой позиции. — А что касается указанного остова
ньютоновых доказательств, то его без сомнения еще настигнет такой же
справедливый суд, который настиг другое необоснованное искусственное
построение Ньютона, состоявшее из {\em оптических
экспериментов} и связанных с ними {\em умозаключений}.
Прикладная математика еще полна такого рода варевом из опыта и рефлексии.
Но подобно тому, как уже с довольно давних пор стали
{\em фактически} игнорировать в науке одну часть
ньютоновской оптики за другой, причем, однако, совершают ту
непоследовательность, что продолжают держаться, хотя и в противоречии с
этим, прочих частей ее, точно так же является
{\em фактом}, что часть упомянутых обманчивых
доказательств уже сама собою пришла в забвение или заменена другими
доказательствами.

\subsection*{Примечание 2. Цель диференциального исчисления, выведенная
из его приложения}

В предшествующем примечании мы рассмотрели отчасти определенность понятия
{\em бесконечно малого}, применяемого в диференциальном
исчислении, отчасти же основу его введения в последнее. И то и другое суть
абстрактные и потому сами по себе также и легкие определения. Так
называемое {\em приложение} представляет больше
трудностей, равно как и более интересную сторону; элементы этой конкретной
стороны составят предмет настоящего примечания. — Весь метод
диференциального исчисления полностью дан в положении, что 
$\mathit{dx}^n=\mathit{dx}^{n-1}\mathit{dx}$  или 
$\frac{f\left(x+i\right)-\mathit{fx}} i=P$ ,~т.~е. равняется
{\em коэфициенту} первого члена двучлена 
$(x+\mathit{dx})^n$  или  $(x\text{~}+\text{~}i)^n$
[ссылка!], разложенного по степеням {\em dx} или {\em i}. Дальше нечему
учиться новому; вывод ближайших форм, диференциала произведения,
показательной функции и~т.~д. получается из этой формулы механически; в
короткое время, в каких-нибудь полчаса —~с нахождением диференциалов дано
также и обратное, нахождение первоначальной функции на основании
диференциалов, интегрирование —~можно овладеть всей теорией. Задерживает на
ней дальше лишь старание усмотреть, сделать для себя понятным, каким
образом после того, как одна {\em сторона} задачи,
{\em нахождение этого коэфициента}, решена так легко
аналитическим,~т.~е. совершенно арифметическим способом, посредством
разложения функции переменной величины, получившей через приращение форму
двучлена, оказывается правильной также и {\em другая
сторона}, а именно, отбрасывание всех членов возникающего ряда, кроме
первого. Если бы оказалось, что единственно только этот коэфициент и нужен,
то с его нахождением было бы покончено, как мы сказали, менее чем в полчаса
со всем, что касается теории, и отбрасывание прочих членов ряда
представляло бы так мало затруднений, что скорее, наоборот, о них, как о
членах ряда (как второй, третьей и~т.~д. производной функции, их
определение равным образом уже закончено с определением первого члена),
вовсе и не было бы речи, так как в них совершенно нет надобности.

Можно здесь предпослать то замечание, что по методу диференциального
исчисления сразу видно, что он изобретен и установлен не как нечто
самодовлеющее; он не только не обоснован сам по себе, как особый способ
аналитического действия, но насильственность, заключающаяся в том, что
прямо отбрасываются члены, получающиеся посредством разложения функции,
несмотря на то, что {\em все} это разложение признается
{\em полностью} относящимся к
{\em делу} —~ибо дело именно и усматривается в
{\em различии} разложенной функции переменной величины
(после того, как ей придана форма двучлена) от первоначальной функции, —
скорее совершенно противоречит всем математическим принципам. Как
потребность в таком образе действий, так и отсутствие внутреннего его
оправдания сразу же указывают на то, что его источник и основание находятся
где-то вне его. Это не единственный случай в науке, когда то, что в
качестве элементарного ставится вначале и из чего, как предполагается,
должны быть выведены положения данной науки, оказывается неочевидным и
имеющим, наоборот, свой повод и обоснование в последующем. История
возникновения диференциального исчисления показывает, что оно получило свое
начало преимущественно в различных так называемых методах касательных,
{\em которые представляли собою как бы кунштюки};
характер действия после того, как он был распространен также и на другие
предметы, был осознан позднее и получил выражение в абстрактных формулах,
которые теперь старались также возвести в ранг
{\em принципов}.

Мы показали выше, что определенность понятия так называемых
{\em бесконечно-малых} есть
{\em качественная} определенность таких количеств,
которые ближайшим образом, как определенные количества, положены
находящимися в отношении друг к другу, а затем в связи с этим следовало
эмпирическое исследование, ставившее себе целью обнаружить эту
определенность понятия в тех имеющихся описаниях или дефинициях бесконечно
малого, которые берут его как бесконечно малую разность и тому подобное. —
Мы это сделали лишь для того, чтобы достигнуть абстрактной определенности
понятия как таковой. Дальнейший вопрос состоит в том, какой характер носит
переход от нее к математической форме и ее приложению. Для этой цели нужно
сначала еще далее развить теоретическую сторону, определенность понятия,
которая окажется в себе самой не совсем бесплодной; затем следует
рассмотреть отношение ее к приложению и доказать относительно их обоих,
насколько это здесь уместно, что получающиеся общие выводы вместе с тем
соответствуют тому, что является существенным в диференциальном исчислении,
и тому способу, каким оно достигает своей цели.

Прежде всего следует напомнить, что мы уже разъяснили мимоходом ту форму,
которую рассматриваемая нами теперь определенность понятия имеет в области
математики. Мы сначала обнаружили качественную определенность
количественного в количественном {\em отношении}
вообще; но помимо этого уже при выводе различных так называемых видов счета
(см. относящееся к этому примечание) мы, забегая вперед, указали, что
именно в {\em степенном отношении}, которое нам
предстоит рассмотреть ближе в своем месте, число через приравнение моментов
его понятия, единицы и численности положено, как возвратившееся к себе
самому, и тем самым получает в себе самом момент бесконечности,
для-себя-бытия,~т.~е. определяемости самим собою. Ясно выраженная
качественная определенность величин принадлежит поэтому, как равным образом
было уже упомянуто выше, по существу степенным определениям, а так как
специфическая черта диференциального исчисления заключается в том, что оно
оперирует качественными формами величин, то свойственным ему математическим
предметом необходимо должно быть рассмотрение форм степеней, и все задачи и
их решения, для которых применяется диференциальное исчисление, показывают,
что интерес сосредоточивается в них единственно лишь на разработке
степенных определений как таковых.

Как ни важна эта основа и хотя она сразу же выдвигает на первый план нечто
определенное вместо чисто формальных категорий переменных, непрерывных или
бесконечных величин и~т.~п. или функций вообще, она все же еще слишком
обща; ведь с тем же самым имеют дело и другие действия; уже возвышение в
степень и извлечение корня, а затем действия над показательными функциями и
логарифмами, ряды, уравнения высших степеней интересуются и занимаются
исключительно отношениями, основанными на степенях. Нет сомнения, что все
они в своей совокупности составляют систему учения о степенях; но ответ на
вопрос, какие именно из этих отношений, в которые могут быть поставлены
степенные определения, суть те, которые составляют собственный предмет и
интерес диференциального исчисления, должен быть почерпнут из него
самого,~т.~е. из его так называемых {\em приложений}.
Последние и составляют на самом деле самую суть, действительный способ
действия в математическом разрешении известного круга проблем; этот способ
действия существовал раньше теории или общей части, и приложением оно было
названо позднее лишь по отношению к созданной впоследствии теории, которая
ставила себе целью отчасти установить общий метод этого способа действия,
отчасти же дать ему принципы,~т.~е. оправдание. Какими тщетными были, для
господствовавшего до сих пор понимания этого способа действия, старания
найти принципы, которые действительно разрешили бы выступающее здесь
противоречие, а не извиняли бы или не прикрывали бы его ссылками на
незначительность того, что согласно математическим правилам необходимо, но
здесь должно быть отбрасываемо, или, что сводится к тому же, ссылками на
возможность бесконечного или какого угодно приближения и~т.~п., — это мы
показали в предшествующем примечании. Если бы всеобщее этого способа
действия было абстрагировано из той действительной части математики,
которая именуется диференциальным исчислением, иным образом, чем это
происходило до сих пор, то эти принципы и труд, затраченный над их
установлением, оказались бы столь же излишни, сколь они, взятые сами по
себе, оказываются чем-то неправильным и остающимся противоречивым.

Если будем доискиваться этого своеобразия путем простого обозрения того, что
имеется в этой части математики, то мы найдем в качестве ее предмета

$\alpha $) уравнения, в которых какое угодно число величин (мы можем здесь
остановиться вообще на {\em двух}) связано в одно
определенное целое, так что эти величины,
{\em во-первых}, имеют свою определенность в
{\em эмпирических величинах}, как твердых пределах, а
затем, в определенной связи как с последними, так и между собою, как это
вообще имеет место в уравнениях; но так как здесь имеется лишь одно
уравнение для обеих величин (в том случае, если величин более двух, то и
число уравнений соответственно увеличивается, но всегда число уравнений
будет меньше числа величин), то это —~уравнения
{\em неопределенные}.
{\em Во-вторых}, они связаны так, что одна из тех черт,
которые характерны для того способа, каким эти величины имеют здесь свою
определенность, заключается в том, что они (по крайней мере одна из них)
даны в уравнении в {\em степени высшей}, чем первая
степень.

Относительно этого мы должны сделать несколько замечаний. Укажем, во-первых,
что величины, взятые со стороны первого из вышеизложенных определений,
всецело носят характер лишь таких {\em переменных}
величин, какие встречаются в задачах
{\em неопределенного} анализа. Они неопределенны, но
так, что если одна получает откуда-нибудь извне некоторое совершенно
определенное значение,~т.~е. некоторое числовое значение, то и другая также
становится определенной, — одна есть {\em функция}
другой; категории переменных величин, функций и тому подобное имеют
поэтому, как уже сказано выше, для освещения той специфической
определенности величин, о которой здесь идет речь, лишь
{\em формальное} значение, так как они отличаются такой
общностью, в которой еще не содержится то специфическое, на которое
направлен весь интерес диференциального исчисления, и это специфическое не
может быть выведено из них при посредстве анализа; они суть взятые сами по
себе, простые, незначительные, легкие определения, которые мы делаем
трудными лишь тогда, когда вкладываем в них то, чего в них нет, для того,
чтобы затем получить возможность вывести его из них, а именно, когда мы
приписываем им специфическое определение диференциального исчисления. — Что
же касается, далее, так называемой {\em константы}, то
о ней можно заметить, что она есть ближайшим образом некоторая безразличная
эмпирическая величина, имеющая для переменных величин определяющее значение
лишь по своему эмпирическому определенному количеству, как предел их
максимума и минимума; но способ соединения такого рода констант с
переменными величинами сам есть один из моментов для природы той частной
функции, которую образуют эти величины. Но и наоборот, сами константы тоже
суть функции. Поскольку, например, прямая линия имеет значение
{\em параметра} параболы, это ее значение состоит в
том, что она есть функция  $\frac{y^2} x$  ; точно так же, как в разложении
двучлена вообще та константа, которая есть коэфициент первого члена ряда,
есть сумма корней, коэфициент второго члена —~сумма их произведений по два
и~т.~д., стало быть, эти константы суть здесь вообще функции корней. Там,
где в интегральном исчислении константа определяется из данной формулы, она
постольку трактуется как ее функция. Эти коэфициенты будут рассмотрены нами
далее и в другом определении как функции, конкретное значение которых
составляет их главный интерес.

Но то своеобразие, которым рассмотрение переменных величин в диференциальном
исчислении отличается от их характера в неопределенных задачах, мы должны
видеть в том, что по крайней мере одна из этих величин или даже все они
имеют степень выше первой, причем опять-таки безразлично, все ли они имеют
одну и ту же высшую степень или они имеют неодинаковую степень;
специфическая неопределенность, которой они здесь отличаются, зависит
исключительно от того, что они суть функции друг друга именно в таком
{\em степенном отношении}. Благодаря этому изменение
переменных величин детерминировано {\em качественно} и,
стало быть, оно {\em непрерывно}, и эта непрерывность,
которая сама по себе есть опять-таки лишь формальная категория некоторого
{\em тождества} вообще, некоторой сохраняющейся в
изменении, остающейся саморавною определенности, имеет здесь свой
детерминированный смысл, и притом единственно только в степенном отношении,
которое не имеет своим показателем никакого определенного количества и
составляет {\em не-количественную}, пребывающую
определенность отношения переменных величин. Поэтому следует возразить
против формализма другого рода, что первая степень есть степень лишь в
отношении к высшим степеням; сам же по себе взятый
{\em x} есть лишь какое-нибудь неопределенное
определенное количество. Поэтому не имеет смысла диференцировать
{\em само по себе} уравнения  $y=\mathit{ax}+b$ ,
уравнение прямой линии, или  $s=\mathit{ct}$ , уравнение просто равномерной
скорости. Если из  $y=\mathit{ax}$  или также из  $y=\mathit{ax}+b$ 
получается  $a=\frac{\mathit{dy}}{\mathit{dx}}$  , или из  $s=\mathit{ct}$ 
получается  $\frac{\mathit{ds}}{\mathit{dt}}=c$  то в такой же мере
определением тангенса является  $a=\frac y x$ или определением просто
равномерной скорости  $\frac s t=c$ . Последняя выражается через 
$\frac{\mathit{dy}}{\mathit{dx}}$  в связи с тем, что выдается за
разложение [в ряд] формулы равномерно ускоренного движения. Но что в
системе такого движения встречается момент простой, просто
равномерной,~т.~е. не определенной высшею степенью одного из моментов
движения, скорости, — это само есть, как замечено выше, бессодержательное,
основанное единственно только на рутине метода допущение. Так как метод
исходит из представления о получаемом переменной величиной приращении, то,
конечно, приращение может получить и такая переменная величина, которая
есть лишь функция первой степени; если же после этого, чтобы найти
диференциал, мы берем отличие возникшего таким образом второго уравнения от
данного, то сразу же обнаруживается пустота действия в том, что, как мы уже
заметили, уравнение до и после этого действия остается для так называемых
приращений тем же, что и для самих переменных величин.

$\beta $). Сказанным определяется природа уравнения, над которым нужно будет
производить действия, и теперь следует указать,
{\em каков} тот {\em интерес}, на
удовлетворение которого направлено {\em произведение
этих действий}. Это рассмотрение может нам дать лишь уже знакомые
результаты, результаты такого рода, какие по форме имеются в особенности в
понимании этого предмета {\em Лагранжем}; но я придал
изложению совершенно элементарный характер, чтобы устранить примешавшиеся
сюда чужеродные определения. — Основой для действий над уравнением
указанного вида оказывается то, что степень {\em внутри
ее самой} понимается как некоторое отношение, как
{\em система определений отношения}. Степень, указали
мы выше, есть число, поскольку оно пришло к тому, что его изменения
{\em определены им же самим}, его моменты, единица и
численность, тождественны,— вполне, как мы выяснили ранее, ближайшим
образом в квадрате, более формально (что не составляет здесь разницы) в
высших степенях. Степень (ввиду того что она как
{\em число} —~хотя бы мы и предпочитали выражение
«величина», как более общее, она {\em в себе} всегда
есть число —~есть некоторое {\em множество}, могущее
быть изображенным также и как {\em сумма}) может
ближайшим образом быть разложена внутри себя самой на любое множество
чисел, которые не имеют никакого другого определения как относительно друг
друга, так и относительно их суммы, кроме того, что они все вместе равны
последней. Но степень может быть также разложена на
{\em сумму} таких различий, которые определены
{\em формой степени}. Если степень принимается за
сумму, то в виде суммы рассматривается также и ее основное число, корень, и
оно может быть разложено любым образом, каковое разнообразие разложений
есть однако нечто безразличное, эмпирически количественное. Сумма, каковою
должен быть корень, сведенная к ее простой определенности,~т.~е. к ее
истинной всеобщности, есть {\em двучлен}; всякое
дальнейшее увеличение числа членов есть простое
{\em повторение} того же определения и потому нечто
пустое\footnote{Лишь формализмом той {\em всеобщности}, на
которую необходимо притязает анализ, объясняется то, что вместо того, чтобы
для разложения степени в ряд брать двучлен  $ (a + b)^n $ ,
берут многочлен $ (a + b + c + d \dots)^n $ , как
это делается также и во многих других случаях; эту форму следует считать,
так сказать, кокетничанием видимостью всеобщности; двучленом исчерпывается
{\em суть дела}; посредством его разложения в ряд мы находим
{\em закон}, а истинной всеобщностью и является как раз
{\em закон}, а не то внешнее, лишь пустое повторение закона, которое это
$ a + b + c + d \dots $ единственно только и порождает.}. Единственно
важным является здесь, стало быть, та {\em качественная
определенность} членов, которая получается посредством
{\em возвышения в степень} принимаемого за сумму корня,
каковая определенность заключается единственно только в том изменении,
которым является возвышение в степень. Эти члены суть, следовательно,
всецело {\em функции возвышения в степень и} [самой]
{\em степени}. Это изображение числа как
{\em суммы} некоторого
{\em множества} таких членов, которые суть функции
возвышения в степень, а затем интерес нахождения
{\em формы} таких функций и, далее, этой
{\em суммы} из множества таких членов, поскольку это
нахождение должно зависеть только от сказанной формы, — все это составляет,
как известно, особое учение о {\em рядах}. Но при этом
мы должны существенно различать еще дальнейший интерес, а именно,
{\em отношение самой лежащей в основании величины}, —
определенность которой, поскольку она есть некоторый комплекс,~т.~е. в
данном случае уравнение, {\em заключает в себе}
некоторую степень, — {\em к функциям ее возвышения в
степень}. Это отношение, совершенно абстрагированное от вышеназванного
интереса нахождения {\em суммы}, окажется тем углом
зрения, который вытекает из действительной науки, как единственный,
имеющийся в виду диференциальным исчислением.

Однако сначала нужно прибавить к сказанному еще одно определение или, лучше
сказать, устранить из сказанного одно заключающееся в нем определение. А
именно, мы сказали, что переменная величина, в определение которой входит
степень, рассматривается {\em внутри ее самой} как
сумма и притом как система членов, поскольку последние суть функции
возвышения в степень, вследствие чего также и корень рассматривается как
сумма, и рассматривается так в своей простой определенной форме как
двучлен;  $x^n = (y + z)^n = (y^n + \mathit{ny}^{n-1}z + {\dots})$ . Это изображение
исходило, в целях разложения степени в ряд,~т.~е. в целях получения функций
возвышения в степень, из {\em суммы} как таковой; но
здесь {\em дело не идет} ни о
{\em сумме} как таковой, ни о происходящем из нее
{\em ряде}, а от суммы должно брать только
{\em соотношение}.
{\em Соотношение} величин как таковое есть то, что, с
одной стороны, остается после того, как отвлекаются от plus некоторой суммы
как таковой, и что, с другой стороны, требуется для нахождения функций,
получающихся в результате разложения в ряд данной степени. Но такое
соотношение уже определено тем, что здесь предмет есть уравнение, что 
$y^m = ax^n$  уже также есть комплекс нескольких (переменных) величин,
содержащий в себе их степенное определение. В этом комплексе каждая из этих
величин безоговорочно положена как находящаяся в
{\em соотношении} с другой со значением, можно было бы
сказать, некоторого plus в ней самой, — положена как функция прочих
величин; их характер функций друг друга сообщает им это определение plus'а,
но тем же самым —~определение чего-то совершенно
{\em неопределенного}, а не приращения, инкремента
и~т.~п. Мы, однако, могли бы также и оставить в стороне эту абстрактную
точку зрения; можно совершенно просто остановиться на том, что после того,
как переменные величины даны в уравнении как функции друг друга, так что
эта определенность заключает в себе отношение степеней, теперь сравниваются
между собою также и функции {\em возвышения в степень}
каждой из них, — каковые вторые функции определены далее не чем иным, как
самим возвышением в степень. Можно {\em сначала}
выдавать за {\em произвол} или
{\em возможность} сведение степенного уравнения
переменных величин к отношению функций, получающихся в результате их
разложения в ряд; лишь дальнейшая {\em цель}, польза,
употребление должны указать {\em пригодность} такого
его преобразования; эта перестановка и вызвана единственно только ее
полезностью. Если выше мы исходили из изображения этих стеленных
определений на примере некоторой такой величины, которая как
{\em сумма} принимается за
{\em различенную внутри себя}, то это служило отчасти
лишь для того, чтобы указать, какого вида эти функции, отчасти же в этом
заключается способ их нахождения.

Мы, таким образом, имеем перед собой обычное аналитическое разложение в ряд,
понимаемое для целей диференциального исчисления так, что переменной
величине дается приращение {\em dx},
{\em i}, а затем степень двучлена раскладывается в
соответствующий ряд. Но так называемое приращение должно быть не
определенным количеством, а лишь {\em формой}, все
значение которой сводится к тому, чтобы быть вспомогательным средством.
Стремятся же в этом случае, по признанию, определеннее всего выраженному
{\em Эйлером} и {\em Лагранжем}, а
затем подразумеваемому вышеупомянутым представлением о пределе, лишь к
получающимся при этом степенным определениям переменных величин, к так
называемым {\em коэфициентам} (эти коэфициенты суть,
правда, коэфициенты приращения и его степеней, которые определяют порядок
ряда и которым принадлежат различные коэфициенты). При этом можно сделать
еще и то замечание, что так как приращение, не имеющее определенного
количества, принимается лишь для целей разложения в ряд, то было бы всего
уместнее обозначить его единицей (цифрой 1), потому что приращение всегда
встречается в разложении только как множитель, а множитель «единица» как
раз и достигает той цели, чтобы приращение не вносило никакой
количественной определенности и никакого количественного изменения.
Напротив, {\em dx}, сопровождаемый ложным
представлением о некоторой количественной разности, и другие знаки, как
например, {\em i}, обремененные бесполезною здесь
видимостью всеобщности, всегда выглядят, как некоторое
{\em определенное количество} и
{\em его степени}, и притязают, что они суть нечто
такое, каковое притязание заставляет затем трудиться над тем, чтобы,
несмотря на это, {\em избавиться} от них,
{\em отбросить} их. Для сохранения формы ряда,
развернутого по степеням, можно было бы с таким же удобством присоединять
обозначения показателей как indices (индексы) и к единице. Но и помимо
этого необходимо абстрагироваться от ряда и от определения коэфициентов по
месту, которое они занимают в ряде, так как отношение между всеми ими одно
и то же; вторая функция выводится из первой точно так же, как первая из
первоначальной, и для той, которая по счету является второй, первая
производная функция есть опять-таки первоначальная. По существу же интерес
направлен не на ряд, а единственно только на получающееся в результате
развертывания ряда степенное определение в его отношении к
{\em для него непосредственной} величине. Стало быть,
вместо того, чтобы считать это определение
{\em коэфициентом первого} члена развертывающегося
ряда, было бы предпочтительнее (так как каждый член есть
{\em первый} относительно следующих за ним членов ряда,
а такая степень в качестве степени приращения, как и сам ряд, не имеет сюда
отношения) употреблять простое выражение
«{\em производная степенная функция}», или, как мы
сказали выше, «{\em функция возвышения величины в
степень}», причем предполагается известным, каким образом получение
производной функции берется как заключенное
{\em внутри} некоторой степени развертывание.

Но если в этой части анализа собственно-математическое начало есть не что
иное, как нахождение функции, определенной через развертывание степени, то
является дальнейший вопрос, что следует предпринять с полученным таким
образом отношением, в чем его {\em приложение} и
{\em употребление}, или на самом дело вопрос, для какой
{\em цели} ищут таких функций. Диференциальное
исчисление вызвало к себе большой интерес именно тем, что оно находило
такие отношения {\em в конкретных предметах}, которые
могут быть сведены к этим абстрактным аналитическим отношениям.

Но относительно приложимости само собой получается, прежде всего, следующий
вывод, который еще до того, как сделаем заключение из случаев приложения,
вытекает из самой природы вещей в силу обнаруженного выше характера
моментов степени. Раскладывание степенных величин, посредством которого
получаются функции их возвышения в степень, если абстрагироваться от более
детальное определения, характеризуется ближайшим образом вообще тем, что
величина {\em понижается} на одну степень, получает
ближайшую низшую степень. Такие действия, следовательно, делаются
{\em приложимыми} в таких
{\em предметах}, в которых также имеется такое различие
степенных определений. Если будем иметь в виду
{\em пространственную определенность}, то мы найдем,
что она содержит те три измерения, которые мы, чтобы отличить их от
абстрактных различий высоты, длины и ширины, можем обозначить как
{\em конкретные} измерения, а именно, линию,
поверхность и целостное пространство; а поскольку они берутся в их
простейших формах и в отношении к самоопределению и, стало быть, к
аналитическим измерениям, то мы получаем прямую линию, плоскостную
поверхность (и ее же как квадрат) и куб. Прямая линия имеет некоторое
эмпирическое определенное количество, но с плоскостью появляется
качественное, степенное определение; более детальные модификации, например
то обстоятельство, что это происходит уже и с плоскими кривыми, мы можем
оставить без рассмотрения, поскольку здесь дело идет прежде всего лишь о
различии в общем виде. Тем самым возникает также потребность
{\em переходить от высшего степенного определения к
низшему и наоборот}, поскольку, например, линейные определения должны быть
выведены из данных уравнений поверхности и~т.~п. или наоборот. — Далее,
{\em движение}, в каковом должно рассматривать
отношение величин пройденного пространства и соответствующего протекшего
времени, обнаруживается в различных определениях просто равномерного,
равномерно ускоренного, попеременно равномерно ускоренного и равномерно
замедленного, — возвращающегося в себя движения; так как эти различные виды
движения выражаются в отношениях величин их моментов, пространства и
времени, то для них получаются уравнения, содержащие различные степенные
определения, а поскольку может явиться потребность определить некоторый вид
движения или же пространственные величины, с которыми связан некоторый вид
движения, посредством другого вида движения, это действие равным образом
приводит к переходу от одной степенной функции к другой, высшей или низшей.
— Примеров этих двух предметов достаточно для той цели, для которой они
приведены.

Видимость случайности, представляемая диференциальным исчислением в его
приложениях, упростилась бы уже одним сознанием природы тех областей, в
которых может иметь место приложение, и своеобразной потребности и условий
этого приложения. Но в пределах самих этих областей важно далее знать,
между какими {\em частями} предметов математической
задачи имеет место тот род отношения, который своеобразно полагается
диференциальным исчислением. Мы должны сразу же заметить предварительно,
что при этом нужно принимать во внимание двоякого рода отношения. Действие
понижения степени некоторого {\em уравнения},
рассматриваемое со стороны производных функций его переменных величин, дает
результат, который {\em в самом себе} поистине уже есть
не уравнение, а некоторое {\em отношение}. Это
отношение есть предмет {\em собственно
диференциального} {\em исчисления}. Но именно поэтому,
во-вторых, здесь имеется также отношение самого более высокого степенного
определения (первоначального уравнения) к низшему (производной функции).
Это второе отношение мы должны оставить пока в стороне; впоследствии оно
окажется. своеобразным предметом {\em интегрального
исчисления}.

Рассмотрим сначала первое отношение и возьмем для —~долженствующего быть
заимствованным из области так называемого приложения —~определения того
момента, в котором заключается интерес действия, простейший пример кривых,
определяемых уравнением второй степени. Как известно, уравнением
{\em непосредственно} дано в некотором степенном
определении отношение координат. Следствиями основного определения являются
определения других связанных с координатами прямых линий: касательной,
подкасательной, нормальной и~т.~п. Но уравнения между этими линиями и
координатами суть {\em линейные} уравнения; те целые,
как части которых определены эти линии, суть прямоугольные треугольники,
составленные {\em прямыми} линиями. Переход от
основного уравнения, содержащего степенное определение, к этим линейным
уравнениям содержит в себе вышеуказанный переход от первоначальной
функции,~т.~е. от той функции, которая представляет собою некоторое
{\em уравнение}, к производной функции, которая есть
некоторое {\em отношение} и притом отношение между
известными, содержащимися в кривой, линиями. Связь между
{\em отношением} этих линий и
{\em уравнением} кривой и есть то, что требуется найти.

Небезынтересно привести здесь ту историческую справку, что первые
открыватели умели указать найденное ими решение лишь совершенно
эмпирическим образом, не будучи в состоянии объяснить само действие,
оставшееся совершенно внешним. Я ограничиваюсь указанием на
{\em Барроу}, учителя Ньютона. В своих Lect. opt. et
geom., в которых он решает задачи высшей геометрии по методу неделимых,
отличающемуся ближайшим образом от особенностей диференциального
исчисления, он сообщает, «так как его друзья этого настойчиво просят»
(Lect. X), также и свой метод определения касательных. Нужно прочесть у
него самого, как он решает эту задачу, чтобы составить надлежащее
представление о том, как его указания относительно этого метода носят
характер указания о совершенно {\em внешнем правиле}, в
том же стиле, как излагалось когда-то в учебниках арифметики тройное
правило или, еще лучше, так называемая проба арифметических действий
девяткою[ссылка!]. Он чертит те
маленькие линии, которые впоследствии были названы
{\em приращениями в характеристическом треугольнике}
кривой линии, и затем в виде простого {\em правила}
предписывает {\em отбросить} как
{\em излишние} те члены, которые в ходе развертывания
уравнения выступают как степени или произведения этих приращений (etenim
isti termini {\em nihilum} valebunt)[ссылка!], а также и те члены,
которые содержат величины, определяемые лишь из первоначального уравнения
(позднейшее вычитание первоначального уравнения из него же с приращениями),
и, наконец, {\em подставить вместо приращения ординаты
самую ординату и вместо приращения абсциссы —~подкасательную}. Нельзя, если
дозволительно так выразиться, изложить способ более школьно-педантически;
последняя подстановка представляет собою сделанное в обычном
диференциальном методе {\em основой} определения
касательной {\em допущение пропорциональности}
приращений ординаты и абсциссы ординате и подкасательной; в правиле Барроу
это допущение выступает во всей своей наивной наготе. Был найден простой
способ определения подкасательной; способы
{\em Роберваля} и {\em Ферма}
сводятся к чему-то сходному —~метод нахождения наибольших и наименьших
значений, из которого исходил последний, покоится на тех же основах и том
же приеме. Математической страстью того времени было нахождение так
называемых {\em методов},~т.~е. этого рода правил, и
притом делать из них секрет, что было не только легко, но в известном
отношении даже нужно, и нужно именно потому, что было легко, а именно
потому, что изобретатели находили лишь эмпирически внешнее правило, а не
метод,~т.~е. не нечто, выведенное из признанных начал. Такие так называемые
методы {\em Лейбниц}
{\em воспринял} от своего времени;
{\em Ньютон} также {\em воспринял}
их от своего времени, а непосредственно —~от своего учителя; обобщением их
формы и их применимости они проложили новые пути в науках, но, занимаясь
этим делом, они чувствовали вместе с тем потребность освободить прием от
характера чисто внешних правил и старались дать ему требуемое оправдание.

Анализируя метод ближе, мы увидим, что истинный ход действия в нем таков.
{\em Во-первых}, степенные определения (разумеется,
переменных величин), содержащиеся в уравнении, понижаются, приводятся к их
первым функциям. Но этим {\em меняется значение} членов
уравнения. Поэтому уже нет более уравнения, а возникло лишь
{\em отношение} между первой функцией одной переменной
величины и первой функцией другой переменной. Вместо  $\mathit{px}=y^2$  мы
имеем  $p\text{~}:\text{~}2y$  или вместо  $2\mathit{ax}-x^2=y^2$ мы имеем 
$a\text{–}x\text{~}:\text{~}y$ , что позднее стали обыкновенно обозначать
как отношение  $\frac{\mathit{dy}}{\mathit{dx}}$ . Уравнение есть уравнение
кривой, а это отношение, совершенно зависящее от него, выведенное (выше
—~согласно голому {\em правилу}) из него, есть,
напротив, некоторое линейное отношение, которому пропорциональны известные
линии;  $p\text{~}:\text{~}2y$  \ или  $a\text{–}x\text{~}:\text{~}y$  сами
суть отношения прямых линий данной кривой, а именно отношения координат и
параметра; но {\em этим мы еще ничего
}{\em не узнали}. Мы желаем знать о
{\em других} встречающихся в кривой линиях, что
{\em им присуще указанное отношение}, желаем найти
равенство двух отношений. — Следовательно, является вопрос,
{\em во-вторых}, какие прямые линии, определяемые
природой кривой, находятся в таком отношении? —~Но это то, что
{\em уже ранее} было
{\em известно}, а именно, что такое полученное
указанным путем отношение есть отношение ординаты к подкасательной. Это
нашли остроумным геометрическим способом древние; новые же изобретатели
открыли только эмпирический прием, как придать уравнению кривой такой вид,
чтобы получилось то первое отношение, о котором
{\em уже было известно}, что оно равно отношению,
содержащему в себе ту линию (здесь —~подкасательную), которая подлежит
определению. Частью это придание уравнению желаемого вида было задумано и
проведено методически —~диференцирование, — частью же были изобретены
воображаемые приращения координат и воображаемый, образованный из этих
приращений и такого же приращения касательной характеристический
треугольник, дабы пропорциональность отношения, найденного путем понижения
степени уравнения, с отношением ординаты и подкасательной была представлена
не как нечто эмпирическое, взятое лишь из давно знакомого, а как нечто
доказанное. Однако это давно знакомое оказывается вообще (а самым
неоспоримым образом в вышеуказанной форме правил) единственным побуждением
к допущению —~и соответственно, единственным оправданием для допущения
{\em характеристического треугольника и указанной
пропорциональности}.

{\em Лагранж} отбросил эту симуляцию и вступил на
подлинно научный путь; его методу мы обязаны тем, что усмотрели, в чем
дело, так как он состоит в том, чтобы отделить друг от друга те два
перехода, которые следует сделать для решения задачи, и рассматривать и
доказывать каждую из этих сторон отдельно. Одна часть этого решения —~мы
при более близком указании хода действия продолжаем пользоваться как
примером элементарной задачей нахождения подкасательной —~теоретическая или
общая часть, а именно, нахождение {\em первой функции}
из данного уравнения кривой, регулируется особо; эта часть дает некоторое
{\em линейное отношение}, следовательно, отношение
прямых линий, встречающихся в системе определения кривой. Другая часть
решения состоит в нахождении тех линий в кривой, которые находятся в
указанном отношении. Это теперь осуществляется прямым путём
(Théorie des Fonct. Anal., р.~II, chap.~II), т.~е. не
прибегая к характеристическому треугольнику, а именно, не делая допущения о
бесконечно малых дугах, ординатах и абсциссах и не давая им определений
{\em dy} и {\em dx},~т.~е. членов
указанного отношения, и не устанавливая вместе с тем непосредственно
значения равенства этого отношения с самими ординатой и подкасательной.
Линия (равно как и точка) имеет свое определение лишь постольку, поскольку
она составляет сторону некоторого треугольника, и определение точки имеется
лишь в треугольнике. Это, скажем мимоходом, есть основное положение
аналитической геометрии, которое приводит к координатам, или, что то же
самое, в механике к параллелограмму сил, именно поэтому совершенно не
нуждающемуся в многочисленных стараниях доказать его. — Подкасательная
теперь принимается за сторону треугольника, другими сторонами которого
являются ордината и соответствующая ей касательная. Последняя как прямая
линия имеет своим уравнением  $p=\mathit{aq}$  (прибавление
{\em +b} бесполезно для определения и делается лишь
ради излюбленной всеобщности); определение отношения  $\frac p q$  есть
{\em a}, коэфициент величины
{\em q}, который есть соответственная первая функция
уравнения, но который должен вообще рассматриваться лишь как  $a=\frac p q$
,~т.~е., как сказано, как существенное определение прямой линии,
составляющей касательную к данной кривой. Далее, поскольку берется первая
функция уравнения кривой, она есть также
{\em определение некоторой прямой линии}; далее, так
как {\em p}, одна координата первой прямой линии, и
{\em y}, ордината кривой, — берутся как тождественные,
так как, стало быть, принимаются, что точка, в которой указанная
принимаемая как касательная первая прямая линия соприкасается с кривой,
есть вместе с тем начальная точка прямой линии, определяемой первой
функцией кривой, то все дело в том, чтобы показать, что эта вторая прямая
линия совпадает с первой,~т.~е. есть касательная, или, выражаясь
алгебраически, показать, что так как $y=\mathit{fx}$  и  $p=\mathit{Fq}$ ,
а теперь принимается, что  $y=p$ , и, стало быть  $\mathit{fx}=\mathit{Fq}$
, то и  $f'x$  тоже  $=F'q$ . Что употребляемая как касательная прямая и та
прямая линия, которая определена из уравнения его первой функцией,
совпадают, что эта последняя есть, стало быть, касательная, это
показывается с помощью {\em приращения}
{\em i} абсциссы и определяемого через разложение
функции приращения ординаты. Здесь, следовательно, также появляется
пресловутое приращение; однако следует различать способ, каким оно вводится
для только что указанной цели, и разложение функции по этому приращению от
вышеупомянутого употребления приращения для нахождения диференциального
уравнения и для характеристического треугольника. Употребление, сделанное
здесь, правомерно и необходимо; оно входит в круг геометрии, так как
геометрическое определение касательной как таковой требует, чтобы между нею
и кривой, с которой она имеет одну общую точку, не могло быть другой прямой
линии, также проходящей через эту точку. Ибо с принятием этого определения
качество касательной или не-касательной сводится к
{\em различию по величине}, и касательной оказывается
та линия, на которую приходится исключительно с точки зрения того
определения, которое здесь важно, {\em наибольшая
малость}. Эта, на первый взгляд, лишь относительная малость не содержит в
себе ничего эмпирического,~т.~е. ничего зависящего от определенного
количества как такового; она положена качественно природой формулы, если
различие того момента, от которого находится в зависимости долженствующая
быть сравниваемой величина, есть различие степени; так как последнее
сводится к {\em i} и
{\em i\textsuperscript{2}} и так как
{\em i}, которое ведь в конце концов должно означать
некоторое число, следует представлять затем как дробь, то
{\em i\textsuperscript{2}}
{\em само по себе} меньше, чем
{\em i}, так что даже представление, что можно
приписывать {\em i} {\em любую
величину}, здесь излишне и даже неуместно. Именно поэтому доказательство
большей малости не имеет ничего общего с бесконечно малым, и последнее
следовательно отнюдь не должно появляться здесь.

Хотя бы только за его красоту и за ныне скорее забытую, но вполне
заслуженную славу, которой он пользовался, я хочу здесь еще сказать о
{\em декартовом} методе касательных; он, впрочем, имеет
также отношение к природе уравнений, о которой мы должны будем затем
сделать еще дальнейшее замечание. Декарт излагает этот самостоятельный
метод, в котором требуемое линейное определение также находится из той же
производной функции, в своей и в других отношениях оказавшейся столь
плодотворной геометрии (Oeuvres compl. ed. Cousin, tom.~V, liv.~II, p.~357
ss.), уча в ней о великой основе природы уравнений и их геометрического
построения, а тем самым об очень расширенном анализе, о распространении его
на геометрию вообще. Проблема получает у него форму задачи —~провести
прямые линии перпендикулярно к любому месту кривой, чем определяется
подкасательная, и~т.~д. Мы вполне понимаем то чувство удовлетворения по
поводу своего открытия, касавшегося предмета всеобщего научного интереса
того времени и являвшегося всецело геометрическим, тем самым поднимавшегося
так высоко над вышеупомянутыми методами голых правил, которые давались его
соперникам, — то чувство, которое он выразил там в следующих словах:
«J’ose dire, que c’est ceci le problème le plus utile et le plus général,
non seulement que je sache, mais même que j’aie jamais désiré de savoir
en géometrie».
(«Я осмеливаюсь сказать, что это "--- самая полезная и самая всеобщая
геометрическая задача не только из всех тех, которые я знаю, но также и из
всех тех, которые я когда-либо желал знать в геометрии»). — Для решения
этой задачи он кладет в основание аналитическое уравнение прямоугольного
треугольника, образуемого ординатой той точки кривой, к которой должна быть
перпендикулярной требуемая в задаче прямая линия, затем ею же самой,
нормальной, и, в-третьих, поднормальною,~т.~е. той частью оси, которая
отрезывается ординатою и нормальною. Из известного уравнения кривой в
уравнение означенного треугольника подставляется затем значение ординаты
или абсциссы; таким образом получается уравнение второй степени (и Декарт
показывает, как и те кривые, уравнения которых содержат высшие степени,
также сводятся к уравнению второй степени), в котором встречается лишь одна
из переменных величин и притом в квадрате и в первой степени, — квадратное
уравнение, которое сначала выступает как так называемое нечистое уравнение.
Затем Декарт соображает, что если мы представим себе рассматриваемую точку
кривой точкой пересечения последней и круга, то этот круг пересечет кривую
еще в другой точке и тогда получается для двух тем самым возникающих и
неодинаковых $x$ два уравнения с одинаковыми
константами и одинаковой формы или же одно уравнение с неодинаковыми
значениями $x$. Но уравнение делается одним уравнением лишь для
{\em одного} треугольника, в котором гипотенуза
перпендикулярна к кривой,~т.~е. оказывается нормальной, что представляют
себе таким образом, что заставляют совпасть обе точки пересечения кривой
кругом, и, следовательно, последний становится касающимся кривой. Но тем
самым отпадает также и то обстоятельство, что корни
$x$ или {\em y} квадратного уравнения неодинаковы. В квадратном
же уравнении с двумя одинаковыми корнями коэфициент члена, содержащего
неизвестные в первой степени, вдвое больше лишь
{\em одного} корня; это дает нам уравнение, посредством
которого мы находим искомые определения. Этот ход решения должен быть
признан гениальным приемом истинно аналитической головы, с которым не может
сравниться принимаемая всецело ассерторически пропорциональность
подкасательной и ординаты якобы бесконечно малым (так называемым)
приращениям абсциссы и ординаты.

Полученное этим путем конечное уравнение, в котором коэфициент второго члена
квадратного уравнения равен удвоенному корню или неизвестному, есть то же
самое уравнение, которое находят посредством приема, применяемого
диференциальным исчислением. Уравнение  $x^2-\mathit{ax}-b=0$  после его
диференцирования дает новое уравнение  $2x-a=0$ ; а уравнение 
$x^3-\mathit{px}-q=0$  дает  $3x^2-p=0$ . Но при этом напрашивается
замечание, что отнюдь не само собою разумеется, что такое производное
уравнение также и правильно. При уравнении с двумя переменными величинами,
которые от того, что они переменные, все-таки не теряют характера
неизвестных величин, получается, как мы указали выше, лишь некоторое
{\em отношение}, по тому указанному простому основанию,
что замещение самих степеней функциями возвышения в степень изменяет
значение обоих членов уравнения, и само по себе еще неизвестно, имеет ли
еще место между ними уравнение при таком измененном значении. Уравнение 
$\frac{\mathit{dy}}{\mathit{dx}}=P$  ничего другого вовсе и не выражает,
кроме того, что {\em P} есть некоторое
{\em отношение}, и не надо приписывать 
$\frac{\mathit{dy}}{\mathit{dx}}$ никакого другого реального смысла. Но об
этом отношении  $=P$  также еще неизвестно, какому другому отношению оно
равно; лишь такое уравнение, {\em пропорциональность},
впервые сообщает ему численное значение и смысл. — Точно так же как (что
было указано выше) то значение, которое называли приложением, берется
извне, эмпирически, так и в тех полученных путем диференцирования
уравнениях, о которых идет речь, для того, чтобы знать, верны ли еще
полученные уравнения, должно быть известно из какого-то другого источника,
имеют ли они одинаковые корни. Но на это обстоятельство в учебниках не
дается определенных и ясных указаний; оно устраняется тем, что уравнение с
одним неизвестным ({\em x}), приведенное к нулю, тотчас
же приравнивается к другому неизвестному ({\em y}),
откуда затем при диференцирования получается, конечно, 
$\frac{\mathit{dy}}{\mathit{dx}}$ , которое есть только некоторое
отношение. Исчисление функций, конечно, должно иметь дело с функциями
возвышения в степень, а диференциальное исчисленное с диференциалами, но из
этого само по себе отнюдь еще не следует, что величины, диференциалы или
функции возвышения в степень которых мы берем, сами также должны быть
{\em лишь} функциями {\em других}
величин. И кроме того в теоретической части, там, где даются указания, как
должны быть выведены диференциалы, еще нет и мысли о том, что величины,
оперировать с которыми согласно такому способу их вывода она учит, сами
должны быть функциями других величин.

Относительно отбрасывания констант при диференцировании можно еще
{\em обратить внимание читателя} на то, что это
отбрасывание имеет здесь тот смысл, что константа оказывается безразличной
для определения корней в случае их равенства, каковое определение
исчерпывается коэфициентом второго члена уравнения. Так, в приведенном
примере Декарта константа есть квадрат самого корня, следовательно,
последний может быть определен как из константы, так и из коэфициентов,
поскольку вообще как она, так и коэфициенты суть функции корней уравнения.
В обычном изложении опущение так называемых констант (связанных с прочими
членами лишь посредством знаков + и –) достигается простым механизмом
приема, состоящего в том, что для нахождения диференциала сложного
выражения приращение сообщается лишь переменным величинам и сформированное
благодаря этому выражение вычитается из первоначального. Смысл констант и
их отбрасывания, вопрос о том, в какой мере они сами суть функции и нужны
ли они или не нужны со стороны этого определения, не подвергается
обсуждению.

С отбрасыванием констант находится в связи одно замечание, которое можно
сделать относительно {\em названий} диференцирования и
интегрирования, замечание, сходное с тем, которое мы сделали раньше
относительно наименований «конечное» и «бесконечное
выражение» [ссылка!], а именно, что в их
определении содержится скорее противоположное тому, что выражается этими
названиями. Диференцирование означает полагание разностей; но
диференцирование, наоборот, уменьшает число измерений уравнения и в
результате отбрасывания константы устраняется один из моментов
определенности; как мы уже заметили, корни переменной величины
приравниваются, {\em их разность},
{\em следовательно},
{\em устраняется}. Напротив, при интегрировании следует
снова присоединить константу; уравнение благодаря этому несомненно
интегрируется, но в том смысле, что ранее устраненная
{\em разность} корней
{\em восстанавливается}, положенное равным снова
диференцируется. — Обычный способ выражения способствует тому, чтобы
оставить в тени существенную природу предмета и все сводить к подчиненной и
даже чуждой главной стороне дела точке зрения отчасти бесконечно-малой
разности, приращения и~т.~п., отчасти же голой разности вообще между данной
и производной функцией, не обозначая их специфического,~т.~е. качественного
различия.

Другую главную область, к которой прилагается диференциальное исчисление,
представляет {\em механика}; попутно мы отчасти уже
касались смысла различных степенных функций, получающихся при элементарных
уравнениях ее предмета, {\em движения}; здесь я буду
говорить о них непосредственно. Уравнение, а именно математическое
выражение просто равномерного движения  $c=\frac s t$  или  $s=\mathit{ct}$
, в котором пройденные пространства пропорциональны протекшим временам по
некоторой эмпирической единице {\em c}, величине
скорости, не имеет смысла диференцировать; коэфициент с уже совершенно
определен и известен, и здесь не может иметь места никакое дальнейшее
развертывание степени, никакое дальнейшее разложение в ряд. — Как
анализируется  $s=at^2$ , уравнение движения падения тел, об этом мы уже
вкратце сказали выше; первый член анализа 
$\frac{\mathit{ds}}{\mathit{dt}}=2\mathit{at}$  выражается словесно и,
следовательно, понимается, как существующий реально таким образом, что он
есть член некоторой {\em суммы} (каковое представление
мы уже давно устранили), одна часть движения и притом та часть его, которая
приписывается силе инерции,~т.~е., просто-равномерной скорости таким
образом, что в {\em бесконечно-малых} частях времени
движение принимается за {\em равномерное}, а в
{\em конечных} частях времени,~т.~е. в существующих на
самом деле, — за неравномерное. Разумеется,  $\mathit{fs}=2\mathit{at}$  и
значение {\em a} и {\em t}, взятых
сами по себе, известно, равно как известно и то, что этим самым дано
определение скорости равномерного движения: так как  $a=\frac s{t^2}$ , то
вообще  $2\mathit{at}=\frac{2s} t$ ; но этим мы нисколько не подвинулись
вперед в нашем знании; лишь ложное предположение, будто  $2\mathit{at}$ 
есть часть движения как некоторой {\em суммы}, дает
ложную видимость физического предложения. Самый множитель,
{\em a}, эмпирическая единица —~некоторое определенное
количество, как таковое —~приписывается тяготению; если здесь применяют
категорию силы тяготения, то нужно сказать, что, наоборот, как раз целое 
$s=at^2$  есть действие или, лучше сказать, закон тяготения. — То же самое
верно и относительно выведенного из 
$\frac{\mathit{ds}}{\mathit{dt}}=2\mathit{at}$  положения, гласящего, что
если бы прекратилось действие силы тяжести, то тело со скоростью,
приобретенной им в {\em конце} своего падения, прошло
бы во время, равное времени его падения, пространство вдвое большее
пройденного. — В этом положении заключается также и сама по себе превратная
метафизика: {\em конец} падения или
{\em конец} той части времени, в которое падало тело,
всегда сам еще есть некоторая часть времени; если бы он
{\em не был} таковой частью, то наступил бы
{\em покой} и, следовательно, не было бы никакой
скорости; скорость может быть установлена лишь по пространству, пройденному
в некоторую часть времени, а не в конце ее. Если же кроме того и в других
физических областях, где вовсе нет никакого движения, как например
относительно поведения света (помимо того, что называют его
распространением в пространстве) и относительно определений величин в
цветах, применяют диференциальное исчисление и первая производная функция
некоторой квадратной функции здесь также именуется скоростью, то на это
следует смотреть, как на еще более несостоятельный формализм выдумывания
существования. —

Движение, изображаемое уравнением  $s=at^2$ , говорит Лагранж, мы находим в
опыте падения тел; простейшим следующим за ним было бы движение, уравнением
которого является  $s=ct^3$ , но такого движения не оказывается в природе;
мы не знали бы, что может означать собою коэфициент
{\em c}. Если это верно, то, напротив, существует
движение, уравнением которого является  $s^3=at^2$  —~кеплеровский закон
движения тел солнечной системы. И разрешение вопроса о том, что здесь
должна означать первая производная функция  $\frac{2\mathit{at}}{3s^2}$ 
и~т.~д., а также дальнейшая непосредственная разработка этого уравнения
путем диференцирования, развитие законов и определений указанного
абсолютного движения, отправляясь от {\em этой исходной
точки зрения}, должно бы, конечно, представить собою интересную задачу, в
решении которой анализ явил бы себя в достойнейшем блеске.

Таким образом само по себе взятое приложение диференциального исчисления к
элементарным уравнениям движения не представляет
{\em реального} интереса; формальный же интерес
проистекает из общего механизма исчисления. Но иное значение получает
разложение движения в отношении определения его траектории; если последняя
есть кривая и ее уравнение содержит высшие степени, то требуются переходы
от прямолинейных функций возвышения в степень к самим степеням, а так как
первые должны быть выведены из первоначального уравнения движения,
содержащего фактор времени, с элиминированием времени, то этот фактор
вместе с тем должен быть низведен к тем низшим функциям развертывания, из
которых могут быть получены означенные уравнения линейных определений. Эта
сторона приводит к рассмотрению интереса другой части диференциального
исчисления.

Сказанное доселе имело своей целью выделить и установить простое
специфическое определение диференциального исчисления и показать наличие
этого определения на некоторых элементарных примерах. Это определение, как
оказалось, состоит в том, что из уравнения степенных функций находят
коэфициент члена разложения, так называемую первую производную функцию, и
что обнаруживают наличие того {\em отношения}, которое
она собою представляет, в моментах конкретного предмета, посредством
какового, полученного таким образом уравнения между обоими отношениями
определяются сами эти моменты. Мы должны вкратце рассмотреть также и
принцип {\em интегрального исчисления} и установить,
что получается из его приложения для его специфического конкретного
определения. Понимание этого исчисления было нами упрощено и определено
более правильно уже благодаря одному тому, что мы его больше не принимаем
за {\em метод суммирования}, как его назвали в
противоположность диференцированию (в котором приращение считается
существенным ингредиентом), вследствие чего интегрирование представлялось
находящимся в существенной связи с формой ряда. — Что касается задачи этого
исчисления, то таковой, во-первых, так же как и в диференциальном
исчислении, является теоретическая или, скорее, формальная задача, но, как
известно, обратная задаче диференцирования. Здесь исходят из функции,
рассматриваемой как {\em производная}, как коэфициент
ближайшего члена, получающегося в результате разложения в ряд некоторого,
пока еще неизвестного уравнения, а из этой производной должна быть найдена
первоначальная степенная функция; та функция, которая в естественном
порядке развертывания должна быть рассматриваема как первоначальная, здесь
выводится, а рассматривавшаяся ранее как производная есть здесь данная или
вообще начальная. Но формальная сторона этого действия представляется уже
выполненной диференциальным исчислением, так как в последнем
устанавливается вообще переход и отношение первоначальной функции к
функции, получающейся в результате разложения в ряд. Если при этом отчасти
уже для того, чтобы взяться за ту функцию, из которой следует исходить,
отчасти же для того, чтобы осуществить переход от нее к первоначальной
функции, оказывается необходимым во многих случаях прибегнуть к
{\em форме ряда}, то следует прежде всего твердо
помнить, что эта форма как таковая не имеет непосредственно ничего общего с
собственным принципом интегрирования.

Но другой стороной задачи этого исчисления является с точки зрения
формальной операции его приложение. А последнее само представляет собой
{\em задачу} узнать, какое предметное
{\em значение} (в вышеуказанном смысле) имеет та
первоначальная функция, которую мы находим по данной функции, принимаемой
за первую [производную]. Может казаться, что с этим учением, взятым само по
себе, также покончено уже в диференциальном исчислении. Однако здесь
появляется дальнейшее обстоятельство, вследствие которого дело оказывается
не так просто. А именно, так как в этом исчислении оказывается, что
благодаря первой производной функции уравнения кривой получилось некоторое
линейное отношение, то тем самым мы также знаем, что интегрирование этого
отношения дает уравнение кривой в виде отношения абсциссы и ординаты; или,
если бы было дано уравнение для площади кривой, то диференциальное
исчисление должно было бы предварительно научить нас относительно значения
первой производной функции такого уравнения, что эта функция представляет
ординату как функцию абсциссы, стало быть, представляет уравнение кривой.

Но главное дело здесь в том, какой из моментов определения предмета
{\em дан} в самом уравнении, ибо лишь от данного может
отправляться аналитическая трактовка, чтобы переходить от него к прочим
определениям предмета. Дано, например, не уравнение поверхности, образуемой
кривою, и не уравнение возникающего посредством ее вращения тела, а также и
не уравнение некоторой дуги этой кривой, а лишь отношение абсциссы и
ординаты в уравнении самой кривой. Переходы от указанных определений к
самому этому уравнению не могут уже поэтому быть предметом самого
диференциального исчисления; нахождение таких отношений есть дело
интегрального исчисления.

Но, далее, было уже показано, что диференцирование уравнения с несколькими
переменными величинами дает степенной член разложения (die
Entwicklungspotenz) [ссылка!] или диференциальный
коэфициент не как уравнение, а только как отношение; задача состоит затем в
том, чтобы в моментах предмета указать для этого
{\em отношения}, которое есть
{\em производная} функция, другое равное ему. Напротив,
предметом интегрального исчисления является само
{\em отношение первоначальной к производной}, в этом
случае данной функции, и задача состоит в том, чтобы указать значение
искомой первоначальной функции в предмете данной первой производной функции
или, вернее, так как это {\em значение}, например,
площадь, ограничиваемая кривой или подлежащая ректифицированию,
представляемая в виде прямой кривая и~т.~д., уже высказано как
{\em задача}, то требуется показать, что такое
определение может быть найдено посредством некоторой первоначальной
функции, и вместе с тем показать, каков тот
{\em момент} предмета, который
{\em для этой цели} должен быть принят
{\em за исходную} функцию, каковою в данном случае
служит производная функция.

Обычный метод, пользующийся представлением бесконечно малой разности,
слишком облегчает себе задачу. Для квадратуры кривых линий он принимает
бесконечно малый треугольник, произведение ординаты на элемент (т.~е. на
бесконечно малую часть) абсциссы, за трапецию, имеющую одной своей стороной
бесконечно-малую дугу, противоположную сказанной бесконечно-малой части
абсциссы. Произведение это и интегрируется в том смысле, что интеграл дает
сумму бесконечно многих трапеций, ту плоскость, которую требуется
определить,~т.~е. {\em конечную} величину сказанного
элемента плоскости. И точно так же обычный метод образует из
бесконечно-малой дуги и соответствующих ей ординаты и абсциссы
прямоугольный треугольник, в котором квадрат этой дуги считается равным
сумме квадратов обоих других бесконечно малых, интегрирование которых и
дает конечную дугу.

Этот прием имеет своей предпосылкой то общее открытие, которое лежит в
основании этой области анализа и которое здесь выступает в виде положения о
том, что квадратура кривой, выпрямленная дуга и~т.~д. находится к известной
(данной уравнением кривой) функции {\em в отношении так
называемой первоначальной функции к производной}. Здесь дело идет о том,
чтобы в случае, если известная часть какого-нибудь математического предмета
(например, некоторой кривой) принимается за производную функцию, узнать,
какая другая его часть выражается соответствующей первоначальной функцией.
Мы знаем, что если данная уравнением кривой функция
{\em ординаты} принимается за производную функцию, то
соответствующая ей первоначальная функция есть выражение величины
отрезанной этой ординатой и кривой {\em плоскости}, что
если как производная функция рассматривается
{\em известное определение касательной}, то ее
первоначальная функция выражает величину соответствующей этому определению
{\em дуги} и~т.~д. Однако заботу о том, чтобы узнать и
доказать, что эти отношения —~отношение первоначальной функции к
производной в отношение величин двух частей или двух обстоятельств
математического предмета —~образуют пропорцию, — заботу об этом снимает с
себя метод, пользующийся бесконечно-малым и механически оперирующий им.
Своеобразной заслугой является уже то остроумие, с которым на основании
результатов, известных уже заранее из других источников, этот метод
открывает, что известные и именно такие-то стороны математического предмета
находятся между собою в отношении первоначальной функции к производной.

Из этих двух функций производная или, как она была определена выше, функция
возвышения в степень, есть здесь, в интегральном исчислении,
{\em данная} по отношению к первоначальной функции,
которая еще должна быть найдена из нее путем интегрирования. Однако первая
дана не непосредственно, а равно не дано уже само по себе, какая часть или
какое определение математического предмета должно быть рассматриваемо как
производная функция, дабы через приведение этого определения к
первоначальной функции найти другую часть или другое определение предмета,
то определение, величину которого требуется установить. Обычный метод,
сразу же представляющий, как мы сказали, известные части предмета как
бесконечно-малые в форме производных функций, находимых из первоначально
данного уравнения предмета вообще посредством диференцирования (как,
например, для выпрямления кривой бесконечно-малые абсциссы и ординаты),
принимает за таковые те части или определения, которые можно привести в
такую связь с предметом задачи (в нашем примере с дугой), также
представляемым, как бесконечно-малый, которая установлена элементарной
математикой, благодаря чему, если известны означенные части, то
определяется также и та часть, величину которой требуется найти; так,
например, для выпрямления кривой указанные три бесконечно-малых приводятся
в связь уравнения прямоугольного треугольника, для ее квадратуры ордината и
бесконечно-малая абсцисса приводятся в связь некоторого произведения,
причем площадь принимается вообще за арифметическое произведение линий.
Переход от этих так называемых элементов площади, дуги и~т.~д. к величине
самих площадей, дуги и~т.~д. считается тогда лишь восхождением от
бесконечного выражения к конечному или к {\em сумме}
бесконечно многих элементов, из которых, согласно предположению, состоит
искомая величина.

Можно, поэтому, сказать лишь поверхностно, что интегральное исчисление есть
только обратная, но вообще более трудная проблема диференциального
исчисления. Дело обстоит, напротив, скорее так, что
{\em реальный} интерес интегрального исчисления
направлен исключительно на взаимное отношение первоначальной и производной
функции в конкретных предметах.

{\em Лагранж} и в этой части исчисления столь же мало
соглашался отделаться от трудности, которую представляли эти проблемы,
рассмотренным гладким способом путем принятия вышеуказанных прямых
допущений. Для разъяснения сущности дела будет полезно привести здесь также
и некоторые детали его приема на немногих примерах. Этот прием ставит себе
как раз задачей отдельно {\em доказать}, что между
частными определениями некоторого математического целого, например
некоторой кривой, имеет место отношение первоначальной функции к
производной. Но в силу природы самого отношения, приводящего в связь в
некотором математическом предмете кривые с прямыми линиями, линейные
измерения и функции с поверхностно-плоскостными измерениями и их функцией
и~т.~д., приводящего, следовательно, в связь
{\em качественно разное}, это не может быть выполнено в
указанной области прямым путем, и определение, таким образом, можно
понимать лишь как середину между некоторым
{\em большим} и некоторым
{\em меньшим}. Благодаря этому, правда, само собою
снова появляется форма {\em приращения} с
{\em плюсом} и {\em минусом}, и
бодрое «developpons» («развернем в ряд») снова очутилось на своем месте; но
мы уже говорили выше о том, что здесь приращения имеют лишь арифметическое
конечное значение. Из развертывания того условия, что подлежащая
определению величина больше некоторого легко определяемого предела и меньше
другого предела, выводится затем, например, что функция ординаты есть
первая производная функция к функции площади.

Выпрямление прямых по способу, показанному
{\em Лагранжем}, который при этом исходит из
архимедовского принципа, интересно тем, что оно проливает свет на
{\em перевод} архимедовского метода на язык принципа
нового анализа, а это позволяет бросить взгляд во внутренний строй и в
истинный смысл действия, механически производимого другим путем. Способ
действия при этом по необходимости аналогичен вышеуказанному способу.
Архимедовский принцип, согласно которому дуга кривой больше соответствующей
ей хорды и меньше суммы двух касательных, проведенных в конечных точках
дуги, поскольку эти касательные заключены между этими точками и точкой их
пересечения, не дает прямого уравнения. Переводом этого архимедовского
основного определения на язык новой аналитической формы служит изобретение
такого выражения, которое, взятое само по себе, есть простое основное
уравнение, между тем как указанная форма лишь выставляет
{\em требование} двигаться, совершать переходы до
бесконечности между некоторым слишком большим и некоторым слишком малым,
которые каждый раз получают определенную величину, причем в результате
такого постоянного движения всегда получаются опять-таки лишь новые слишком
большие и слишком малые, но во все более и более тесных пределах.
Посредством формализма бесконечно-малых сразу же создается уравнение 
$dz^2=dx^2+dy^2$ . Исходя из указанной основы,
{\em лагранжево} изложение доказывает, напротив, что
величина дуги есть первоначальная функция к некоторой производной функции,
характеризующий член которой сам есть функция отношения производной функции
к первоначальной функции ординаты.

Так как в способе {\em Архимеда}, точно так же, как и
позднее в исследовании {\em Кеплером} стереометрических
предметов, встречается представление о бесконечно-малом, то это
обстоятельство слишком часто приводилось в качестве авторитета в пользу
того употребления, которое делают из этого представления в диференциальном
исчислении, причем не выделялись черты своеобразия и отличия.
Бесконечно-малое означает прежде всего отрицание определенного количества
как такового,~т.~е. так называемого {\em конечного}
выражения или той завершенной определенности, которой обладает определенное
количество как таковое. И точно так же в последующих знаменитых методах
{\em Валериуса}, {\em Кавальери} и
др., основанных на рассмотрении {\em отношений}
геометрических предметов, основным определением является положение о том,
что {\em определенное количество}, как определенное
количество таких определений, которые ближайшим образом рассматриваются
лишь в отношении, оставляется для этой цели в стороне, и эти определения
должны быть принимаемы сообразно с этим за {\em не
имеющие величины} (Nicht-Grosses). Но отчасти этим не познано и не выделено
то {\em утвердительное} вообще, которое лежит за
исключительно отрицательным определением и которое выше оказалось, говоря
абстрактно, {\em качественной} определенностью
величины, состоящей, говоря более определенно, в степенном отношении;
отчасти же, поскольку само это отношение в свою очередь включает в себя
множество ближе определенных отношений, как например, отношение между
некоторой степенью и функцией, получающейся в результате ее разложения в
ряд, они должны были бы быть в свою очередь обоснованы всеобщим и
отрицательным определением того же бесконечно-малого и выведены из него. В
только что приведенном изложении {\em Лагранжа} найдено
то определенное утвердительное, которое заключается в архимедовом способе
развертывания задачи, и тем самым приему, обремененному неограниченным
выхождением, дана его настоящая граница. Величие нового изобретения,
взятого само по себе, и его способность разрешать до того времени
неприступные задачи, а те задачи, которые и ранее были разрешимы, разрешать
более простым способом, — это величие следует видеть исключительно в
открытии отношения первоначальной функции к так называемой производной
функции и тех частей математического целого, которые находятся в таком
отношении.

Данное нами изложение взглядов можно считать достаточным для нашей цели,
заключающейся в том, чтобы подчеркнуть своеобразие того отношения величин,
которое служит предметом рассматриваемого здесь особого вида исчисления.
Излагая эти взгляды, мы могли ограничиться простыми задачами и способом их
решения; и ни цели, которая исключительно имелась здесь в виду (а именно:
установить определенность понятия рассматриваемых определений), ни силам
автора не соответствовало бы обозреть весь объем так называемого приложения
диференциального и интегрального исчисления и завершить индукцию, гласящую,
что найденный принцип лежит в основании этих видов исчисления, сведением
всех их задач и решений последних к этому принципу. Но изложенное
достаточно показало, что, как каждый особый вид исчисления имеет своим
предметом особую определенность или особое отношение величины и такое
отношение конституирует сложение, умножение, возвышение в степень и
извлечение корня, счет посредством логарифмов, рядов и~т.~д., — точно так
же обстоит дело и с диференциальным и интегральным исчислением; для того
отношения, которое присуще этому исчислению, наиболее подходящим названием
было бы отношение степенной функции к функции ее развертывания или
возвышения в степень, так как это название всего ближе к пониманию сущности
дела. Лишь так, как в этом исчислении вообще применяются равным образом и
действия, основанные на других отношениях величин, например сложение
и~т.~д., в нем применяются также и отношения логарифмов, круга и рядов, в
особенности для того, чтобы сделать более удобными выражения, нужные для
требуемых действий вывода первоначальных функций из функций развертывания.
С формой ряда диференциальное и интегральное исчисление имеет, правда, тот
ближайший общий интерес, что оба они стремятся определить те функции
развертывания, которые в рядах называются коэфициентами членов; но в то
время как интерес этого исчисления простирается лишь на отношение
первоначальной функции к ближайшему коэфициенту ее развертывания, ряд
стремится представить некоторую {\em сумму} в виде
множества членов, расположенного по степеням, снабженным этими
коэфициентами. Бесконечное, имеющее место в бесконечном ряде,
неопределенное выражение отрицания определенного количества вообще, не
имеет ничего общего с утвердительным определением, заключающимся в
бесконечном этого исчисления. И точно так же бесконечно-малое как
{\em приращение}, посредством которого развертывание
принимает форму ряда, есть лишь внешнее средство для развертывания, и его
так называемая бесконечность не имеет никакого другого значения, кроме
значения такого средства; ряд, так как он на самом деле не есть то, что
требуется, приводит к некоторой {\em избыточности},
вновь отбросить которую стоит лишнего труда. Этой необходимостью лишнего
труда страдает также и метод {\em Лагранжа}, который
вновь прибег преимущественно к форме ряда, хотя благодаря именно этому
методу в том, что называют {\em приложением}, выступает
истинное своеобразие высшего анализа, так как, не втискивая
{\em в предметы} форм {\em dx},
{\em dy} и~т.~д., метод Лагранжа прямо указывает ту
часть этих предметов, которой свойственна определенность производной
функции (функции развертывания), и этим обнаруживает, что форма ряда вовсе
не есть то, о чем здесь идет речь\footnote{В вышеприведенной
критике (Jahrb. für wissensch. Krit., Bd.~II, 1827, Nr.~155, 6~и~сл.)
помещены интересные высказывания основательного ученого специалиста
г.~Шпера, приведенные из его «Principien des Fluentenkalkuls», Braunschweig,
1826, касающиеся именно одного из обстоятельств, существенно способствующих
внесению в диференциальное исчисление темноты и ненаучности, и
согласующиеся со сказанным нами относительно того, как обстоит вообще дело
с {\em\bfseries теорией} этого исчисления.
«Чисто {\em\bfseries арифметических исследований},
"--- говорится там, "--- которые, правда, из всех
подобных больше всего имеют отношение к диференциальному исчислению, не
отделили от собственно диференциального исчисления, и даже принимали, как
например, {\em\bfseries Лагранж}, эти
исследования {\em\bfseries за самую суть},
между тем как на последнюю смотрели лишь как на их
{\em\bfseries приложения}. Эти
арифметические исследования обнимают собою правила диференцирования, вывод
теоремы Тейлора и~т.~д. и даже различные методы интегрирования.
{\em\bfseries Дело же обстоит как раз наоборот}: эти {\em\bfseries приложения}
суть именно то, что составляет {\em\bfseries предмет собственно}
диференциального исчисления, все же те арифметические
рассуждения (Entwicklungen) и действия оно
{\em\bfseries предполагает} известными из анализа». ---~Мы показали, как у
{\em\bfseries Лагранжа} отделение так
называемого приложения от приема общей части, исходящего из рядов, служит
именно к тому, чтобы сделать явственным
{\em\bfseries своеобразное дело}
диференциального исчисления, взятого само по себе. Но ввиду
интересного усмотрения автора, что именно так называемые
{\em\bfseries приложения} и составляют
{\em\bfseries предмет собственно}
диференциального исчисления, нужно удивляться, каким образом
он впадает в (приведенную там же) формальную метафизику
{\em\bfseries непрерывной} величины,
{\em\bfseries становления},
{\em\bfseries течения} и~т.~д., и еще хочет
даже умножить этот баласт; эти определения
{\em\bfseries формальны} потому, что они
суть лишь общие категории, не указывающие именно
{\em\bfseries специфической стороны дела},
которую следовало познать и абстрагировать из конкретных
учений, из приложений.}.

\bigskip


\chapter*{Третья глава. Количественное Отношение.}
\subsection[Третий отделМера]{\textstyleHeadiiFirstLine{Третий
отдел}\newline
Мера}
\hypertarget{Toc478978670}{}В мере соединены абстрактно выраженные качество
и количество. {\em Бытие} как таковое есть
непосредственное равенство определенности с самою собою. Эта
непосредственность определенности сняла себя. Количество есть бытие,
возвратившееся в себя таким образом, что оно теперь есть простое равенство
с собою как безразличие к определенности. Но это безразличие есть лишь
внешность, характеризующаяся тем, что количество имеет определенность не в
себе самом, а в другом. Следующее за ним теперь третье есть соотносящаяся с
самою собою внешность; как соотношение с собою, оно вместе с тем есть
{\em снятая} внешность и имеет в ней самой отличие от
себя, которое как внешность есть {\em количественный},
а как вобранная обратно в себя —~{\em качественный}
момент.

Так как {\em модальность} приводится в числе категорий
трансцендентального идеализма после количества и качества, причем между
последними и ею вставляется отношение, то можно упомянуть о ней здесь. Эта
категория имеет там то значение, что она есть отношение
{\em предмета к мышлению}. Согласно смыслу учения
указанного идеализма мышление вообще существенно внешне вещи-в-себе.
Поскольку прочие категории имеют лишь то трансцендентальное определение,
что принадлежат {\em сознанию}, но как то,
{\em что в нем объективно}, постольку модальность, как
категория отношения к субъекту, содержит в себе в относительном смысле
определение {\em рефлексии} в себя;~т.~е. присущая
прочим категориям объективность недостает категориям модальности;
последние, по выражению {\em Канта}, нисколько не
умножают понятия как определение объекта, а лишь выражают отношение к
способности познания (Kr. d. rein. Vernunft, изд. 2-е, стр. 99, 266). —
Категории, которые Кант объединяет под названием модальности, —
возможность, действительность и необходимость —~встретятся нам в дальнейшем
в своем месте. Бесконечно важную форму тройственности, — хотя она у Канта
появляется пока что лишь как формальный луч света (formeller Lichtfunken),
— он применил не к родам своих категорий (количество, качество и~т.~д.), а
также не к ним применил и название «категории», но лишь к их видам; поэтому
он не мог найти третьей категории к качеству и количеству.

У {\em Спинозы модус} есть
также третье, следующее за субстанцией и атрибутом; он его объявляет
{\em состояниями} субстанции или тем, что находится в
другом, через которое оно и постигается. Это третье есть согласно этому
понятию лишь внешность как таковая, и мы уже указали в другом месте, что у
Спинозы неподвижной субстанциальности недостает возвращения в себя самоё.

Сделанное нами здесь замечание в более общем виде распространяется на все те
пантеистические системы, которые были до некоторой степени разработаны
мыслью. Бытие, единое, субстанция, бесконечное, сущность —~есть первое; по
отношению к этой абстракции второе, всякая определенность, может быть
вообще столь же абстрактно охарактеризовано как лишь конечное, лишь
акциденциальное, преходящее, внесущественное и несущественное и~т.~д., как
это обычно на первых порах происходит в совершенно формальном мышлении. Но
мысль о связи этого второго с первым напрашивается так настойчиво, что
приходится понимать вместе с тем это второе, как находящееся в единстве с
первым; так, например, у Спинозы {\em атрибут} есть вся
субстанция, но субстанция, как ее постигает рассудок, который сам есть
некоторое ограничение или модус; модус же,~т.~е. несубстанциальное вообще,
которое может быть постигаемо лишь из некоторого другого, составляет, таким
образом, другую, противоположную крайность к субстанции, третье вообще.
{\em Индийский} пантеизм в своей чрезвычайной
фантастике, взятый абстрактно, также получил разработку, тянущуюся через
эту безмерную фантастику как умеряющая нить и придающую ей некоторый
интерес Брама, единое абстрактного мышления, переходит через получение
облика Вишны, в особенности в форме Кришны, в третье, в Сиву. Определением
этого третьего служит модус, изменение, возникновение и прехождение, вообще
область внешнего. Если эта индийская троица соблазнила кое-кого сравнивать
ее с христианской, то мы должны сказать, что хотя в них можно распознать
общий им элемент определения понятия, однако вместе с тем существенно важно
более определенно осознать различие между ними; это различие не только
бесконечно, но истинная бесконечность и составляет самое это различие.
Третий принцип индийского пантеизма есть по своему определению распад
субстанциального единства, переход его в свою противоположность, а
{\em не возвращение его} к себе, — есть скорее
бездуховное, чем дух. В истинной же троичности имеется не только единство,
но и единение, умозаключение доведено в ней до
{\em содержательного} и
{\em действительного} единства, которое в своем
совершенно конкретном определении есть {\em дух}.
Вышеуказанный принцип модуса и изменения, правда, не исключает вообще
единства. Так, в спинозизме именно модус как таковой есть неистинное, и
лишь субстанция есть истинное, все должно быть сведено к этой последней, и
это сведение оказывается погружением всяческого содержания в пустоту, в
лишь формальное бессодержательное единство. Точно так же и Сива есть в свою
очередь великое целое, не отличное от Брамы, сам Брама,~т.~е. различие и
определенность лишь снова исчезают, но не сохраняются, не снимаются, и
единство не сводится к конкретному единству, раздвоение не приводится
обратно к применению. Высшая цель для человека, ввергнутого в сферу
возникновения и прохождения, вообще в область модальности, есть погружение
в бессознательность, единство с Брамой, уничтожение; то же самое
представляет собою буддистская нирвана, ниббана и~т.~п.

Если модус есть вообще абстрактная внешность, безразличие как к
качественным, так и к количественным определениям, и внешнее,
несущественное считается не имеющим важности в сфере сущности, то, с другой
стороны, касательно многого признается, что все зависит от
{\em вида и способа}; этим сам модус объявляется
существенно принадлежащим к субстанциальной стороне вещи, а это весьма
неопределенное отношение означает по меньшей мере то, что это внешнее не
есть столь абстрактно внешнее.

Здесь модус имеет определенный смысл {\em меры}.
Спинозовский модус, как и индусский принцип изменения, есть безмерное.
Греческое еще неопределенное сознание того, что
{\em все имеет меру}, так что даже Парменид ввел после
абстрактного бытия {\em необходимость}, как
{\em всем вещам поставленную древнюю границу}, это
сознание заключает в себе начаток гораздо более высокого понятия, чем
субстанция и различие от нее модуса.

Более развитая, более рефлектированная мера есть необходимость; судьба,
{\em немезида}, сводится в общем к определенности меры
именно в том смысле, что то, что {\em дерзновенно
превозносится}, что делает себя слишком высоким, слишком великим,
приводится ею к другой крайности, унижается, доводится до ничтожности и тем
самым восстанавливается средняя мера, посредственность. — «Абсолютное, бог
есть {\em мера} всех вещей», — это положение не более
пантеистично, чем дефиниция: «абсолютное, бог есть
{\em бытие}», но первое бесконечно более истинно. —
Мера есть, правда, внешний вид и способ, некоторое «больше» или «меньше»,
но она вместе с тем также и рефлектирована в себя, есть не только
безразличная и внешняя, но и в-себе-сущая определенность; она, таким
образом, есть {\em конкретная истина бытия}; народы
поэтому почитали в мере нечто неприкосновенное, святое.

В мере уже подготовлена идея {\em сущности}, а именно в
ней подготовлено тождество с самим собой в непосредственной определенности,
так что сказанная непосредственность понижается через это тождество с собою
до некоторого опосредствованного, равно как тождество с собою также
опосредствовано лишь через эту внешность, но есть опосредствование
{\em с собою}; это —~рефлексия, определения которой
{\em суть}, но даны (sind) в этом бытии безоговорочно
лишь как моменты ее отрицательного единства. В мере качественное
количественно; определенность или различие дано (ist) в ней как
безразличное; тем самым оно есть такое различие, которое не есть различие;
оно снято; эта количественность как возвращение в себя, в котором она дана
(ist) как качественное составляет в-себе-и-для-себя-бытие, которое есть
{\em сущность}. Но мера есть сущность пока что лишь
{\em в себе} или, иначе говоря, в понятии; это
{\em понятие} меры еще не
{\em положено}. Мера, еще как таковая, сама есть
{\em сущее} единство качественного и количественного;
ее моменты суть, как некоторое наличное бытие, некоторое качество и
определенные количества этого качества, которые (качество и количество)
пока что лишь в себе неотделимы, но еще не имеют значения этого
рефлектированного определения. Развитие меры заключает в себе различение
этих моментов, но вместе с тем и их {\em соотнесение},
так что то тождество, которое они суть {\em в себе},
становится их взаимным отношением друг к другу,~т.~е. становится
{\em положенным}. Смыслом (die Bedeutung) этого
развития является реализация меры, в которой она полагает себя в отношении
к себе самой и тем самым полагает себя вместе с тем как момент; через это
опосредствование она определяется как снятая; ее непосредственность, как и
непосредственность ее моментов, исчезает; они оказываются
рефлектированными; таким образом, выступив как то, что она есть по своему
понятию, она перешла в {\em сущность}.

Мера есть прежде всего {\em непосредственное} единство
количественного и качественного, так что

{\em во-первых}, имеется {\em одно
определенное количество}, которое имеет качественное значение и выступает
{\em как мера}. Ее дальнейшее определение заключается в
том, что {\em в ней}, во {\em в
себе} определенном, выступает различие ее моментов, качественной и
количественной определенности. Эти моменты сами определяются далее в целые
меры, которые постольку имеют бытие как
{\em самостоятельные}; поскольку они по существу
соотносятся друг с другом, мера становится

{\em во-вторых}, {\em отношением}
специфических определенных количеств {\em как
самостоятельных мер}. Но их самостоятельность вместе с тем покоится по
существу на количественном отношении и различии по величине. Таким образом,
их самостоятельность становится переходом друг в друга. Мера тем самым идет
ко дну, погружается (geht zu Grunde) в {\em безмерном}.
— Но это потустороннее меры есть ее отрицательность лишь в себе самой;
поэтому

{\em в-третьих}, положена неразличенность (индиференция)
определений меры и, как реальная, мера с содержащейся в этой
неразличенности отрицательностью положена как
{\em обратное отношение мер}, которые как
самостоятельные качества существенно покоятся лишь на своем количестве и на
своем отрицательном соотношении друг с другом, и тем самым оказывается, что
они суть лишь моменты их истинно самосостоятельного единства, которое есть
их рефлексия в себя и полагание последней, —
{\em сущность}.

Развитие меры, как мы его попытались изложить в последующем, есть одна из
труднейших материй; начинаясь с непосредственной, внешней меры, оно должно
было бы, с одной стороны, перейти далее к абстрактному дальнейшему
определению количественного ({\em к математике
природы}), а, с другой стороны, вскрыть связь этого определения меры с
{\em качествами} вещей природы —~по крайней мере в
общем виде, ибо определенное доказательство проистекающей из понятия
конкретного предмета {\em связи} качественного и
количественного есть дело особых наук о конкретном (примеры таких
доказательств, касающиеся закона падения тел и закона свободного движения
небесных тел, смотри в Энциклопедии философских наук, изд. 3-е, § 267 и 270
и примечания к ним). При этом уместно заметить вообще, что различные формы,
в которых реализуется мера, принадлежат также различным
{\em сферам природной реальности}. Полное, абстрактное
безразличие развитой меры,~т.~е. ее {\em законов},
может иметь место только в сфере {\em механизма}, в
котором конкретно телесное есть лишь сама являющаяся абстрактной материя;
качественные различия материи имеют по существу своей определенностью
количественное; {\em пространство} и
{\em время} суть сами чистые внешности, а
{\em множество} (die Menge) материй, массы,
интенсивность {\em веса} точно так же суть внешние
определения, имеющие свою своеобразную определенность в количественном.
Напротив, такая определенность величины абстрактно-материального уже в
области {\em физики}, а еще больше в
{\em органической} природе, нарушается множественностью
и, значит, конфликтом качеств. Но здесь не только появляется конфликт между
качествами как таковыми, а мера подчиняется здесь более высоким отношениям,
и {\em имманентное развитие} меры сводится скорее к
простой форме непосредственной меры. Члены животного организма имеют меру,
которая как некоторое простое определенное количество находится в отношении
к другим определенным количествам других членов; пропорции человеческого
тела суть прочные отношения таких определенных количеств; естествознанию
еще предстоит задача проникнуть в связь таких величин с органическими
функциями, от которых они целиком зависят. Но ближайшим примером понижения
некоторой имманентной меры на степень исключительно внешним образом
детерминированной величины служит {\em движение}. В
небесных телах оно есть свободное, определяемое лишь понятием движение,
величины которого тем самым также находятся в зависимости только от понятия
(см. выше), но органическими существами оно понижается до
{\em произвольного} или механически-правильного,~т.~е.
вообще до абстрактного, формального движения.

Но еще в меньшей степени находит себе место своеобразное, свободное развитие
меры в царстве духа. Легко, например, усмотреть, что такой республиканский
государственный строй, как например афинский или строй аристократический,
смешанный с демократией, может иметь место лишь при известной величине
государства; что в развитом гражданском обществе количества индивидов,
занятых в различных промыслах, находятся между собою в известном отношении;
но это не дает ни законов мер, ни особых форм этого отношения. В области
духовного как такового мы встречаем различия
{\em интенсивности} характера,
{\em силы} воображения, чувств, представлений и~т.~п.;
но за пределы этой неопределенной характеристики
«{\em силы}» или «{\em слабости}»
определение не выходит. Какими тусклыми и совершенно пустыми оказываются
так называемые законы, устанавливаемые касательно отношения силы и слабости
ощущений, представлений и~т.~д., мы убеждаемся, обратившись к руководствам
по психологии, старающимся найти такого рода законы.



\part*{Третий отдел. Мера.}

\chapter*{Первая глава. Специфическое количество.}
\input{speckolich.tex}

\chapter*{Вторая глава. Реальная мера.}
\input{realnmera.tex}

\chapter*{Третья глава. Становление сущности.}
\input{stanovsuw.tex}

\backmatter

\clearpage
\tableofcontents*
\clearpage

\end{document}
