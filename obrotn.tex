\paragraph[В. \ Обратное отношение]{В. \ Обратное отношение}
\hypertarget{Toc478978667}{}1. Отношение, как оно получилось теперь, есть
{\em снятое} прямое отношение; оно было
{\em непосредственным} и, стало быть, еще не истинно
определенным; теперь же определенность привзошла к нему так, что показатель
считается произведением, единством единицы и численности. Со стороны его
непосредственности его можно было (как было показано выше) принимать
безразлично и за единицу и за численность, вследствие чего он и был лишь
определенным количествам вообще и, стало быть, преимущественно
численностью; одна сторона была единицей, и ее следовало принимать за одно,
а другая сторона была ее неизменной численностью, которая вместе с тем была
и показателем; качество последнего состояло, следовательно, лишь в том, что
это определенное количество принималось за неизменное или, вернее,
неизменное понималось лишь в смысле определенного количества.

В обратном же отношении показатель как определенное количество равным
образом есть некое непосредственное и нечто, принимаемое за неизменное. Но
это определенное количество не есть {\em неизменная
численность }по отношению к другому члену
{\em отношения}, принимаемому за
{\em единицу}; это в предшествующем неизменное
отношение теперь скорее, наоборот, положено как изменчивое; когда в
качестве одного из членов обратного отношения берут какое-нибудь другое
определенное количество, то другой член отношения уже более не остается
{\em той же самой численностью} единиц первого члена. В
прямом отношении эта единица есть лишь общее обоих членов; она как таковая
продолжается в другой член, в численность; сама численность, взятая особо,
или, иначе говоря, показатель, безразлична к единице.

Но при той определенности отношения, какую мы имеем теперь, численность как
таковая изменяется по отношению к единице, относительно которой она
составляет другую сторону отношения; если мы принимаем за
{\em единицу} какое-нибудь другое определенное
количество, то численность становится другой. Поэтому, хотя показатель
также и здесь есть лишь некоторое непосредственное, лишь произвольно
принимаемое за неизменное определенное количество, но он не сохраняется как
таковое в стороне отношения, и эта сторона, а тем самым и прямое отношение
сторон изменчиво. Поэтому в рассматриваемом теперь отношении показатель,
как определяющее определенное количество, положен отрицательным по
отношению к себе как к определенному количеству отношения, положен тем
самым как качественный, как граница, так что, следовательно, качественное
выступает особо, отличным от количественного. — В прямом отношении
{\em изменение} обоих членов есть лишь
{\em одно} изменение определенного количества, каковым
принимается единица, представляющая собою общее обеих сторон отношения, и,
следовательно, во сколько раз одна сторона увеличивается или уменьшается,
во столько же раз увеличивается или уменьшается также и другая; само
отношение безразлично к этому изменению; последнее внешне ему. В обратном
же отношении изменение, хотя оно по безразличному количественному моменту
также произвольно, удерживается {\em внутри отношения},
и также и это произвольное количественное выхождение подвергается
ограничению отрицательной определенностью показателя как некоторой
границей.

2. Следует рассмотреть эту качественную природу обратного отношения еще
ближе, а именно в ее реализации, и разъяснить содержащуюся в ней
переплетенность утвердительного с отрицательным. — Определенное количество
положено здесь как то, что качественно определяет определенное
количество,~т.~е. само себя, как представляющее себя в нем [в самом себе]
своей границей. Тем самым оно есть, {\em во-первых},
некоторая непосредственная величина как {\em простая}
определенность, {\em целое} как
{\em сущее}, утвердительное определенное количество.
Но, {\em во-вторых}, эта непосредственная
{\em определенность} есть вместе с тем
{\em граница}; для этого оно различено на два
определенных количества, которые ближайшим образом суть другие относительно
друг друга; но как их качественная определенность, и притом эта
определенность как полная, оно есть единство единицы и численности,
произведение, множителями которого они служат. Таким образом, показатель их
отношения, с одной стороны, тождественен в них с собою и есть то их
утвердительное, благодаря чему они суть определенные количества; с другой
стороны, он, как положенное в них отрицание, есть в них то
{\em единство}, вследствие которого каждое,
представляющее собою ближайшим образом некоторое непосредственное,
ограниченное определенное количество вообще, вместе с тем есть такое
ограниченное, что оно только {\em в себе тождественно}
со своим другим. {\em В-третьих}, он как простая
определенность есть отрицательное единство этого своего различения на два
определенных количества и граница их взаимного ограничения.

Согласно этим определениям оба момента
{\em ограничивают} друг друга внутри показателя, и один
момент есть отрицательное другого, так как показатель есть их определенное
единство; один момент становится во столько раз меньше, во сколько другой
становится больше; каждый имеет свою величину постольку, поскольку он
заключает в себе величину другого, именно ту величину, которая недостает
другому. Каждая величина продолжает себя, таким образом,
{\em отрицательно} в другую; сколько численности есть в
ней, столько она устраняет в другой как численности, и она есть то, что она
есть, только через отрицание или границу, которая полагается в ней другою.
Каждая, таким образом, {\em содержит} в себе также и
другую и измеряется ею, ибо каждая должна быть только тем определенным
количеством, которым не является другая; для значения каждой из них
величина другой необходима и, стало быть, от нее неотделима.

Эта непрерывность каждой в другой составляет момент единства, благодаря
которому они находятся в отношении —~момент
{\em единой} определенности, простой границы, которая
есть показатель. Это единство, целое, образует
{\em в-себе-бытие} каждой из сторон отношения, от
какового в-себе-бытия отлична ее {\emналичная}
величина, по которой каждая сторона есть лишь постольку, поскольку она
отнимает у другой часть их общего в-себе-бытия —~целого. Но она может
отнять у другой лишь столько, сколько нужно для того, чтобы сделать себя
равной этому в-себе-бытию. Она имеет свой максимум в показателе, который по
указанному второму определению есть граница их взаимного ограничения. А так
как каждая есть момент отношения лишь постольку, поскольку она ограничивает
другую и, стало быть, ограничивается другою, то, делаясь равною своему
в-себе-бытию, она утрачивает это свое определение; при этом не только
другая величина становится
нулем~\textstyleEndnodeLink{(\ref{bkm:Ref474666305}}\label{bkm:bm56}\textstyleEndnodeLink{)},
но и она сама исчезает, так как она, согласно предположению, есть не голое
определенное количество, а она должна быть тем, что она, как таковое, есть
лишь как такого рода момент отношения. Таким образом, каждая сторона
отношения есть противоречие между определением ее как ее
в-себе-бытия,~т.~е. единства того целого, которым служит показатель, и
определением ее как момента отношения; это противоречие есть
{\em бесконечность}, снова появившаяся в новой,
своеобразной форме.

Показатель есть {\em граница} членов своего отношения,
внутри которой они обратно друг другу увеличиваются и уменьшаются, причем
они не могут стать равными показателю по той утвердительной определенности,
которая свойственна ему как определенному количеству. Таким образом, как
граница их взаимного ограничения, он есть ({\em $\alpha
$}) их {\em потустороннее}, к которому они могут
{\em бесконечно} приближаться, но которого они не могут
достигнуть. Эта бесконечность, с которой они к нему приближаются, есть
дурная бесконечность бесконечного прогресса; она сама конечна, имеет свой
предел в своей противоположности, в конечности каждого члена и самого
показателя, и есть поэтому лишь {\em приближение}. Но
({\em $\beta $}) дурная бесконечность вместе с тем
здесь {\em положена} как то, что она есть
{\em поистине}, а именно, лишь как
{\em отрицательный момент} вообще, по которому
показатель есть относительно различенных определенных количеств отношения
{\em простая граница} как в-себе-бытие, с которым
соотносят их конечность как безоговорочно изменчивое, но которое как их
отрицание остается безоговорочно отличным от них. Это бесконечное, к
которому они могут лишь приближаться, в таком случае наличествует также и
как {\em утвердительное посюстороннее}; это —~простое
определенное количество показателя. В показателе достигнута та
потусторонность, которой обременены стороны отношения; он есть
{\em в себе} единство обеих или тем самым он есть в
себе другая сторона каждой из них; ибо каждая имеет лишь столько величины,
сколько ее не имеет другая, вся ее определенность лежит, таким образом, в
другой, и это ее в-себе-бытие есть как утвердительная бесконечность просто
показатель.

3. Но тем самым получился переход обратного отношения в некоторое другое
определение, чем то, которым оно первоначально обладало. Последнее состояло
в том, что некоторое определенное количество как непосредственное вместе с
тем имеет то соотношение с другим, что становится тем больше, чем последнее
становится меньше, есть то, что оно есть, лишь через отрицательное
отношение к другому; и равным образом состояло в том, что некоторая третья
величина есть общий предел этого их увеличения. Это изменение, в
противоположность к качественному как {\em твердой},
{\em неизменной} границе, составляет здесь их
своеобразие; они имеют определение {\em переменных}
величин, для которых то неизменное есть некоторое бесконечное
потустороннее.

Но определения, которые обнаружились перед нами и которые мы должны свести
воедино, заключаются не только в том, что это бесконечное потустороннее
есть вместе с тем некоторое имеющееся налицо и какое-нибудь конечное
определенное количество, а и в том, что его неизменность —~вследствие
которой оно есть такое бесконечное потустороннее по отношению к
количественному и которая есть качественная сторона бытия лишь как
абстрактное соотношение с самою собою —~развилась в опосредствование себя с
самим собою в своем другом, в конечных членах отношения. Всеобщий момент
этих определений заключается в том, что вообще целое как показатель есть
граница взаимного ограничения обоих членов, что, стало быть, положено
{\em отрицание отрицания}, а тем самым бесконечность,
{\em утвердительное} отношение к самому себе. Более
определенный момент заключается в том, что {\em в себе}
показатель как произведение уже есть единство единицы и численности, а
каждый из обоих членов отношения есть лишь один из этих двух моментов,
благодаря чему показатель, следовательно, включает их в себя и
{\em в себе} соотносится в них с самим собою. Но в
обратном отношении различие развилось в характеризующую количественное
бытие {\em внешность} и качественное дано не только как
неизменное, а также не только как лишь непосредственно включающее в себя
моменты, а как смыкающееся во-{\em вне-себя-сущем
инобытии с собою}. Это определение и выделяется как результат в
обнаружившихся доселе моментах. А именно, показатель оказывается
в-себе-бытием, моменты которого реализованы в определенных количествах и в
их изменчивости вообще. Безразличие их величин в их изменении
представляется в виде бесконечного прогресса; в основании этого лежит то,
что в их безразличии их определенность как раз и состоит в том, чтобы иметь
свою величину в величине другого и, стало быть,
({\em $\alpha $}) по утвердительной стороне их
определенного количества быть {\em в себе} полным
показателем. И точно так же они имеют ({\em ß}) своим
отрицательным моментом, своим взаимным ограничиванием величину показателя;
их граница есть его граница. То обстоятельство, что они уже больше не имеют
никакой другой имманентной границы,
никакой~\textstyleEndnodeLink{(\ref{bkm:Ref474666541}}\label{bkm:bm57}\textstyleEndnodeLink{)}
твердой непосредственности, положено в бесконечном прогрессе их наличного
бытия и их ограничения, в отрицании всякой особенной величины. Это
отрицание есть согласно этому {\em отрицание} того
вне-себя-бытия показателя, которое изображено в них, и он,~т.~е. тот,
который сам вместе с тем представляет собой некоторое определенное
количество вообще и также и развернут в определенные количества, тем самым
положен, как сохраняющийся, сливающийся с собою в отрицании их
безразличного существования, положен, таким образом, как определяющий это
выхождение за себя.

Отношение определилось таким образом в {\em степенное
отношение}.
