\subsection[Третий отделМера]{\textstyleHeadiiFirstLine{Третий
отдел}\newline
Мера}
\hypertarget{Toc478978670}{}В мере соединены абстрактно выраженные качество
и количество. {\em Бытие} как таковое есть
непосредственное равенство определенности с самою собою. Эта
непосредственность определенности сняла себя. Количество есть бытие,
возвратившееся в себя таким образом, что оно теперь есть простое равенство
с собою как безразличие к определенности. Но это безразличие есть лишь
внешность, характеризующаяся тем, что количество имеет определенность не в
себе самом, а в другом. Следующее за ним теперь третье есть соотносящаяся с
самою собою внешность; как соотношение с собою, оно вместе с тем есть
{\em снятая} внешность и имеет в ней самой отличие от
себя, которое как внешность есть {\em количественный},
а как вобранная обратно в себя —~{\em качественный}
момент.

Так как {\em модальность} приводится в числе категорий
трансцендентального идеализма после количества и качества, причем между
последними и ею вставляется отношение, то можно упомянуть о ней здесь. Эта
категория имеет там то значение, что она есть отношение
{\em предмета к мышлению}. Согласно смыслу учения
указанного идеализма мышление вообще существенно внешне вещи-в-себе.
Поскольку прочие категории имеют лишь то трансцендентальное определение,
что принадлежат {\em сознанию}, но как то,
{\em что в нем объективно}, постольку модальность, как
категория отношения к субъекту, содержит в себе в относительном смысле
определение {\em рефлексии} в себя;~т.~е. присущая
прочим категориям объективность недостает категориям модальности;
последние, по выражению {\em Канта}, нисколько не
умножают понятия как определение объекта, а лишь выражают отношение к
способности познания (Kr. d. rein. Vernunft, изд. 2-е, стр. 99, 266). —
Категории, которые Кант объединяет под названием модальности, —
возможность, действительность и необходимость —~встретятся нам в дальнейшем
в своем месте. Бесконечно важную форму тройственности, — хотя она у Канта
появляется пока что лишь как формальный луч света (formeller Lichtfunken),
— он применил не к родам своих категорий (количество, качество и~т.~д.), а
также не к ним применил и название «категории», но лишь к их видам; поэтому
он не мог найти третьей категории к качеству и количеству.

У {\em Спинозы модус} есть
также третье, следующее за субстанцией и атрибутом; он его объявляет
{\em состояниями} субстанции или тем, что находится в
другом, через которое оно и постигается. Это третье есть согласно этому
понятию лишь внешность как таковая, и мы уже указали в другом месте, что у
Спинозы неподвижной субстанциальности недостает возвращения в себя самоё.

Сделанное нами здесь замечание в более общем виде распространяется на все те
пантеистические системы, которые были до некоторой степени разработаны
мыслью. Бытие, единое, субстанция, бесконечное, сущность —~есть первое; по
отношению к этой абстракции второе, всякая определенность, может быть
вообще столь же абстрактно охарактеризовано как лишь конечное, лишь
акциденциальное, преходящее, внесущественное и несущественное и~т.~д., как
это обычно на первых порах происходит в совершенно формальном мышлении. Но
мысль о связи этого второго с первым напрашивается так настойчиво, что
приходится понимать вместе с тем это второе, как находящееся в единстве с
первым; так, например, у Спинозы {\em атрибут} есть вся
субстанция, но субстанция, как ее постигает рассудок, который сам есть
некоторое ограничение или модус; модус же,~т.~е. несубстанциальное вообще,
которое может быть постигаемо лишь из некоторого другого, составляет, таким
образом, другую, противоположную крайность к субстанции, третье вообще.
{\em Индийский} пантеизм в своей чрезвычайной
фантастике, взятый абстрактно, также получил разработку, тянущуюся через
эту безмерную фантастику как умеряющая нить и придающую ей некоторый
интерес Брама, единое абстрактного мышления, переходит через получение
облика Вишны, в особенности в форме Кришны, в третье, в Сиву. Определением
этого третьего служит модус, изменение, возникновение и прехождение, вообще
область внешнего. Если эта индийская троица соблазнила кое-кого сравнивать
ее с христианской, то мы должны сказать, что хотя в них можно распознать
общий им элемент определения понятия, однако вместе с тем существенно важно
более определенно осознать различие между ними; это различие не только
бесконечно, но истинная бесконечность и составляет самое это различие.
Третий принцип индийского пантеизма есть по своему определению распад
субстанциального единства, переход его в свою противоположность, а
{\em не возвращение его} к себе, — есть скорее
бездуховное, чем дух. В истинной же троичности имеется не только единство,
но и единение, умозаключение доведено в ней до
{\em содержательного} и
{\em действительного} единства, которое в своем
совершенно конкретном определении есть {\em дух}.
Вышеуказанный принцип модуса и изменения, правда, не исключает вообще
единства. Так, в спинозизме именно модус как таковой есть неистинное, и
лишь субстанция есть истинное, все должно быть сведено к этой последней, и
это сведение оказывается погружением всяческого содержания в пустоту, в
лишь формальное бессодержательное единство. Точно так же и Сива есть в свою
очередь великое целое, не отличное от Брамы, сам Брама,~т.~е. различие и
определенность лишь снова исчезают, но не сохраняются, не снимаются, и
единство не сводится к конкретному единству, раздвоение не приводится
обратно к применению. Высшая цель для человека, ввергнутого в сферу
возникновения и прохождения, вообще в область модальности, есть погружение
в бессознательность, единство с Брамой, уничтожение; то же самое
представляет собою буддистская нирвана, ниббана и~т.~п.

Если модус есть вообще абстрактная внешность, безразличие как к
качественным, так и к количественным определениям, и внешнее,
несущественное считается не имеющим важности в сфере сущности, то, с другой
стороны, касательно многого признается, что все зависит от
{\em вида и способа}; этим сам модус объявляется
существенно принадлежащим к субстанциальной стороне вещи, а это весьма
неопределенное отношение означает по меньшей мере то, что это внешнее не
есть столь абстрактно внешнее.

Здесь модус имеет определенный смысл {\em меры}.
Спинозовский модус, как и индусский принцип изменения, есть безмерное.
Греческое еще неопределенное сознание того, что
{\em все имеет меру}, так что даже Парменид ввел после
абстрактного бытия {\em необходимость}, как
{\em всем вещам поставленную древнюю границу}, это
сознание заключает в себе начаток гораздо более высокого понятия, чем
субстанция и различие от нее модуса.

Более развитая, более рефлектированная мера есть необходимость; судьба,
{\em немезида}, сводится в общем к определенности меры
именно в том смысле, что то, что {\em дерзновенно
превозносится}, что делает себя слишком высоким, слишком великим,
приводится ею к другой крайности, унижается, доводится до ничтожности и тем
самым восстанавливается средняя мера, посредственность. — «Абсолютное, бог
есть {\em мера} всех вещей», — это положение не более
пантеистично, чем дефиниция: «абсолютное, бог есть
{\em бытие}», но первое бесконечно более истинно. —
Мера есть, правда, внешний вид и способ, некоторое «больше» или «меньше»,
но она вместе с тем также и рефлектирована в себя, есть не только
безразличная и внешняя, но и в-себе-сущая определенность; она, таким
образом, есть {\em конкретная истина бытия}; народы
поэтому почитали в мере нечто неприкосновенное, святое.

В мере уже подготовлена идея {\em сущности}, а именно в
ней подготовлено тождество с самим собой в непосредственной определенности,
так что сказанная непосредственность понижается через это тождество с собою
до некоторого опосредствованного, равно как тождество с собою также
опосредствовано лишь через эту внешность, но есть опосредствование
{\em с собою}; это —~рефлексия, определения которой
{\em суть}, но даны (sind) в этом бытии безоговорочно
лишь как моменты ее отрицательного единства. В мере качественное
количественно; определенность или различие дано (ist) в ней как
безразличное; тем самым оно есть такое различие, которое не есть различие;
оно снято; эта количественность как возвращение в себя, в котором она дана
(ist) как качественное составляет в-себе-и-для-себя-бытие, которое есть
{\em сущность}. Но мера есть сущность пока что лишь
{\em в себе} или, иначе говоря, в понятии; это
{\em понятие} меры еще не
{\em положено}. Мера, еще как таковая, сама есть
{\em сущее} единство качественного и количественного;
ее моменты суть, как некоторое наличное бытие, некоторое качество и
определенные количества этого качества, которые (качество и количество)
пока что лишь в себе неотделимы, но еще не имеют значения этого
рефлектированного определения. Развитие меры заключает в себе различение
этих моментов, но вместе с тем и их {\em соотнесение},
так что то тождество, которое они суть {\em в себе},
становится их взаимным отношением друг к другу,~т.~е. становится
{\em положенным}. Смыслом (die Bedeutung) этого
развития является реализация меры, в которой она полагает себя в отношении
к себе самой и тем самым полагает себя вместе с тем как момент; через это
опосредствование она определяется как снятая; ее непосредственность, как и
непосредственность ее моментов, исчезает; они оказываются
рефлектированными; таким образом, выступив как то, что она есть по своему
понятию, она перешла в {\em сущность}.

Мера есть прежде всего {\em непосредственное} единство
количественного и качественного, так что

{\em во-первых}, имеется {\em одно
определенное количество}, которое имеет качественное значение и выступает
{\em как мера}. Ее дальнейшее определение заключается в
том, что {\em в ней}, во {\em в
себе} определенном, выступает различие ее моментов, качественной и
количественной определенности. Эти моменты сами определяются далее в целые
меры, которые постольку имеют бытие как
{\em самостоятельные}; поскольку они по существу
соотносятся друг с другом, мера становится

{\em во-вторых}, {\em отношением}
специфических определенных количеств {\em как
самостоятельных мер}. Но их самостоятельность вместе с тем покоится по
существу на количественном отношении и различии по величине. Таким образом,
их самостоятельность становится переходом друг в друга. Мера тем самым идет
ко дну, погружается (geht zu Grunde) в {\em безмерном}.
— Но это потустороннее меры есть ее отрицательность лишь в себе самой;
поэтому

{\em в-третьих}, положена неразличенность (индиференция)
определений меры и, как реальная, мера с содержащейся в этой
неразличенности отрицательностью положена как
{\em обратное отношение мер}, которые как
самостоятельные качества существенно покоятся лишь на своем количестве и на
своем отрицательном соотношении друг с другом, и тем самым оказывается, что
они суть лишь моменты их истинно самосостоятельного единства, которое есть
их рефлексия в себя и полагание последней, —
{\em сущность}.

Развитие меры, как мы его попытались изложить в последующем, есть одна из
труднейших материй; начинаясь с непосредственной, внешней меры, оно должно
было бы, с одной стороны, перейти далее к абстрактному дальнейшему
определению количественного ({\em к математике
природы}), а, с другой стороны, вскрыть связь этого определения меры с
{\em качествами} вещей природы —~по крайней мере в
общем виде, ибо определенное доказательство проистекающей из понятия
конкретного предмета {\em связи} качественного и
количественного есть дело особых наук о конкретном (примеры таких
доказательств, касающиеся закона падения тел и закона свободного движения
небесных тел, смотри в Энциклопедии философских наук, изд. 3-е, § 267 и 270
и примечания к ним). При этом уместно заметить вообще, что различные формы,
в которых реализуется мера, принадлежат также различным
{\em сферам природной реальности}. Полное, абстрактное
безразличие развитой меры,~т.~е. ее {\em законов},
может иметь место только в сфере {\em механизма}, в
котором конкретно телесное есть лишь сама являющаяся абстрактной материя;
качественные различия материи имеют по существу своей определенностью
количественное; {\em пространство} и
{\em время} суть сами чистые внешности, а
{\em множество} (die Menge) материй, массы,
интенсивность {\em веса} точно так же суть внешние
определения, имеющие свою своеобразную определенность в количественном.
Напротив, такая определенность величины абстрактно-материального уже в
области {\em физики}, а еще больше в
{\em органической} природе, нарушается множественностью
и, значит, конфликтом качеств. Но здесь не только появляется конфликт между
качествами как таковыми, а мера подчиняется здесь более высоким отношениям,
и {\em имманентное развитие} меры сводится скорее к
простой форме непосредственной меры. Члены животного организма имеют меру,
которая как некоторое простое определенное количество находится в отношении
к другим определенным количествам других членов; пропорции человеческого
тела суть прочные отношения таких определенных количеств; естествознанию
еще предстоит задача проникнуть в связь таких величин с органическими
функциями, от которых они целиком зависят. Но ближайшим примером понижения
некоторой имманентной меры на степень исключительно внешним образом
детерминированной величины служит {\em движение}. В
небесных телах оно есть свободное, определяемое лишь понятием движение,
величины которого тем самым также находятся в зависимости только от понятия
(см. выше), но органическими существами оно понижается до
{\em произвольного} или механически-правильного,~т.~е.
вообще до абстрактного, формального движения.

Но еще в меньшей степени находит себе место своеобразное, свободное развитие
меры в царстве духа. Легко, например, усмотреть, что такой республиканский
государственный строй, как например афинский или строй аристократический,
смешанный с демократией, может иметь место лишь при известной величине
государства; что в развитом гражданском обществе количества индивидов,
занятых в различных промыслах, находятся между собою в известном отношении;
но это не дает ни законов мер, ни особых форм этого отношения. В области
духовного как такового мы встречаем различия
{\em интенсивности} характера,
{\em силы} воображения, чувств, представлений и~т.~п.;
но за пределы этой неопределенной характеристики
«{\em силы}» или «{\em слабости}»
определение не выходит. Какими тусклыми и совершенно пустыми оказываются
так называемые законы, устанавливаемые касательно отношения силы и слабости
ощущений, представлений и~т.~д., мы убеждаемся, обратившись к руководствам
по психологии, старающимся найти такого рода законы.
